\begin{SCn}
	\scnsectionheader{Предметная область и онтология многократно используемых компонентов интерфейсов ostis-систем}
	\scntext{введение}{Большое разнообразие \textit{интерфейсов} влечет за собой разработку большого числа компонентов. В качестве \textit{многократно используемых компонентов интерфейсов ostis-систем} могут выступать как уже спроектированные \textit{интерфейсы}, так и специфицированные \textit{компоненты интерфейсов}. Большое число \textit{многократно используемых компонентов интерфейсов ostis-систем} создает проблему их хранения и поиска. Чтобы решить эту проблему, в технологию включена \textit{библиотека многократно используемых компонентов пользовательских интерфейсов ostis-систем} и \textit{менеджер многократно используемых компонентов ostis-систем}.}
	\begin{scnindent}
		\scnrelfrom{смотрите}{Предметная область и онтология комплексной библиотеки многократно используемых семантически совместимых компонентов ostis-систем}
	\end{scnindent}
	
	\begin{scnsubstruct}
		\scnheader{Предметная область многократно используемых компонентов интерфейсов ostis-систем}
		\scniselement{предметная область}
		\scnhaselementrole{максимальный класс объектов исследования}{многократно используемый компонент пользовательских интерфейсов ostis-систем}
		\begin{scnhaselementrolelist}{класс объектов исследования}
			\scnitem{пользовательский интерфейс библиотеки многократно используемых компонентов интерфейсов ostis-систем}
			\scnitem{просмотрщик содержимого \textit{файлов ostis-системы}}
			\scnitem{редактор содержимого \textit{файлов ostis-системы}}
			\scnitem{транслятор содержимого \textit{файла ostis-системы} в \textit{SC-код}}
			\scnitem{транслятор из \textit{SC-кода} в содержимое \textit{файла ostis-системы}}
			\scnitem{транслятор \textit{базы знаний} во внешнее представление}
		\end{scnhaselementrolelist}
		
		\scnheader{многократно используемый компонент интерфейсов ostis-систем}
		\scnrelfrom{разбиение}{Типология многократно используемых компонентов интерфейсов ostis-систем по признаку решаемых задач}
		\begin{scnindent}
			\begin{scneqtoset}
				\scnitem{просмотрщик содержимого \textit{файлов ostis-системы}}
				\scnitem{редактор содержимого \textit{файлов ostis-системы}}
				\scnitem{транслятор содержимого \textit{файла ostis-системы} в \textit{SC-код}}
				\scnitem{транслятор из \textit{SC-кода} в содержимое \textit{файла ostis-системы}}
				\scnitem{транслятор \textit{базы знаний} во внешнее представление}
				\begin{scnindent}
					\begin{scnrelfromlist}{смотрите}
						\scnitem{\scncite{Koronchik2011}}
						\scnitem{\scncite{Koronchik2014}}
					\end{scnrelfromlist}
				\end{scnindent}
			\end{scneqtoset}
			\scntext{примечание}{\textit{Технология OSTIS} позволяет интегрировать в качестве компонентов редакторы и просмотрщики, разработанные с использованием других технологий (далее их будем называть \textit{платформенно-зависимыми многократно используемыми компонентами интерфейса ostis-систем}). В основном они используются для просмотра и редактирования содержимого \textit{файлов ostis-системы}. Это значительно позволяет сэкономить время при их разработке.}
		\end{scnindent}
		
		\scnheader{пользовательский интерфейс библиотеки многократно используемых компонентов интерфейсов ostis-систем}
		\scntext{примечание}{\textit{пользовательский интерфейс} \textit{библиотеки многократно используемых компонентов интерфейсов ostis-систем} строится на основе sc.g-интерфейса (комплекс информационно-программных средств обеспечивающих общение интеллектуальных систем с пользователями на основе \textit{SCg-кода}, как способа внешнего представления информации). Однако, это не исключает возможность использования других способов диалога пользователя с \textit{библиотекой многократно используемых компонентов интерфейсов ostis-систем}.}
		\scntext{примечание}{В рамках \textit{библиотеки многократно используемых компонентов интерфейсов ostis-систем} могут содержаться различные версии и модификации какого-либо компонента. К примеру, компонент просмотра \textit{sc.g-конструкций} может иметь модификации, для отображения в которых может использоваться двумерная, трехмерная или же многослойная визуализация, при этом каждая из модификаций компонента может иметь различные версии.}
		\scntext{примечание}{Использование \textit{библиотеки многократно используемых компонентов интерфейсов ostis-систем} при проектировании \textit{интерфейса} прикладной системы позволяет значительно сократить сроки проектирования, а также снизить требования, предъявляемые к начальной квалификации разработчика. Это достигается за счет проектирования \textit{интерфейса} из уже заранее подготовленных моделей интерфейса, что также позволяет повысить качество проектируемого \textit{интерфейса}.}
		
		\bigskip
	\end{scnsubstruct}
\end{SCn}
