\begin{SCn}
	\scnsectionheader{Предметная область и онтология действий и методик проектирования интерфейсов ostis-систем}
	\scntext{аннотация}{Проектирование \textit{интерфейса компьютерных систем} --- это один из наиболее важных этапов разработки любой системы. Пользователь при использовании \textit{интерфейса} должен представить себе, какая информация о выполняемой задаче у него существует, и в каком состоянии находятся средства, с помощью которых он будет решать данную задачу. Эффективность работы пользователя и его интерес обеспечивает правильно сформулированная \uline{методика разработки и проектирования пользовательского интерфейса}. В рамках предметной области и онтологии рассмотрены этапы проектирования \textit{пользовательских интерфейсов} и этапы проектирования \textit{адаптивных интеллектуальных мультимодальных пользовательских интерфейсов}.}
	\begin{scnrelfromvector}{введение}
		\scnfileitem{В настоящее время организация взаимодействия пользователя с компьютерной системой основана на парадигме \uline{грамотного пользователя}, который знает, как управлять системой и несет полную ответственность за качество взаимодействия с ней.}
		\scnfileitem{Многообразие форм и видов \textit{интерфейсов} приводит к необходимости пользователя адаптироваться к каждой конкретной системе, обучаться принципам взаимодействия с ней для решения необходимых ему задач.}
		\scnfileitem{Проектирование \textit{пользовательских интерфейсов} включает в себя ряд последовательных этапов.}
		\scnfileitem{Методика проектирования \textit{пользовательских интерфейсов} является важной частью \textit{Технологии OSTIS}, так как она описывает этапы проектирования \textit{пользовательских интерфейсов ostis-систем}, что позволяет ускорить процесс разработки, обеспечивает создание удобных \textit{пользовательских интерфейсов}, улучшает опыт использования \textit{интеллектуальной системы} и повышает эффективность работы пользователей, учитывая их потребности и предпочтения.}
	\end{scnrelfromvector}
	
	\begin{scnrelfromlist}{ключевое понятие}
		\scnitem{библиотека многократно используемых компонентов пользовательских интерфейсов ostis-систем}
		\scnitem{метод оценки пользовательских интерфейсов}
		\scnitem{качественный метод оценки пользовательских интерфейсов}
		\scnitem{количественный метод оценки пользовательских интерфейсов}
		\scnitem{тестирование удобства использования пользовательских интерфейсов}
		\scnitem{отслеживание движения глаз}
		\scnitem{экспертная оценка пользовательских интерфейсов}
		\scnitem{A/Б тестирование пользовательских интерфейсов}
		\scnitem{древовидное тестирование пользовательских интерфейсов}
	\end{scnrelfromlist}
	
	\begin{scnrelfromlist}{библиография}
		\scnitem{\scncite{Ehlert2003}}
		\scnitem{\scncite{Kong2011}}
		\scnitem{\scncite{Sadouski2022a}}
		\scnitem{\scncite{Ivory2001}}
		\scnitem{\scncite{Jeffries1991}}
		\scnitem{\scncite{ISO9241-161}}
		\scnitem{\scncite{Zeng2019}}
		\scnitem{\scncite{ISO16982}}
		\scnitem{\scncite{Koronchik2011}}
		\scnitem{\scncite{Koronchik2014}}
	\end{scnrelfromlist}
	
	\begin{scnsubstruct}
		\scnheader{Предметная область действий и методик проектирования интерфейсов ostis-систем}
		\scniselement{предметная область}
		\scnhaselementrole{максимальный класс объектов исследования}{проектирование адаптивных интеллектуальных мультимодальных пользовательских интерфейсов}
		\begin{scnhaselementrolelist}{класс объектов исследования}
			\scnitem{анализ методов оценки пользовательских интерфейсов}
		\end{scnhaselementrolelist}
		
		\scnheader{Существующие методики проектирования адаптивных интеллектуальных мультимодальных пользовательских интерфейсов}
		\scnhaselement{Проектирование адаптивных интеллектуальных мультимодальных пользовательских интерфейсов (Ehlert)}
		\scnhaselement{Проектирование адаптивных интеллектуальных мультимодальных пользовательских интерфейсов (Kong)}
		
		\scnheader{Проектирование адаптивных интеллектуальных мультимодальных пользовательских интерфейсов (Ehlert)}
		\begin{scnrelfromset}{этапы}
			\scnitem{анализ}
			\begin{scnindent}
				\scntext{примечание}{Этап анализа является, вероятно, самой важной фазой в любом процессе проектирования, в том числе в проектировании \textit{интерфейсов ostis-систем}. В процессе проектирования традиционного \textit{интерфейса} необходимо проанализировать, кто является обычным пользователем, какие задачи \textit{интерфейс} должен поддерживать.}
				\scnsuperset{функциональный анализ}
				\begin{scnindent}
					\scntext{примечание}{В рамках \textit{функционального анализа} необходимо дать ответ на вопрос: \scnqqi{каковы основные функции системы}.}
				\end{scnindent}
				\scnsuperset{анализ данных}
				\begin{scnindent}
					\scntext{примечание}{В рамках \textit{анализа данных} необходимо определить \uline{значение и структуру данных}, используемых в приложении.}
				\end{scnindent}
				\scnsuperset{анализ пользователей}
				\begin{scnindent}
					\scntext{примечание}{В рамках \textit{анализа пользователей} необходимо выделить \uline{типы пользователей и их возможности} в интеллектуальном и когнитивном плане.}
				\end{scnindent}
				\scnsuperset{анализ среды}
				\begin{scnindent}
					\scntext{примечание}{В рамках \textit{анализа среды} необходимо определить \uline{требования, предъявляемые к среде}, в которой будет работать система. Результатом данного этапа является \uline{cпецификация целей и информационных потребностей пользователя}, а также \uline{спецификация функций и информации}, которые требуются системе.}
				\end{scnindent}
			\end{scnindent}
			\scnitem{разработка \textit{интерфейса}}
			\begin{scnindent}
				\begin{scnrelfromset}{этапы}
					\scnfileitem{\textit{Создание модели интерфейса} в соответствии с этапом анализа.}
					\scnfileitem{Реализация модели \textit{интерфейса}.}
				\end{scnrelfromset}
				\scntext{результат}{Результатом данного этапа является \textit{пользовательский интерфейс}, который, по мнению разработчика, удовлетворяет требованиям пользователей и соответствует требованиям, сформулированным на этапе анализа.}
			\end{scnindent}
			\scnitem{оценка \textit{интерфейса}}
			\begin{scnindent}
				\begin{scnrelfromset}{предполагает}
					\scnfileitem{Требования, которые были сформулированы на этапе \textit{анализа}, должны быть удовлетворены.}
					\scnfileitem{Эффективность модели \textit{интерфейса} должна быть исследована.}
				\end{scnrelfromset}
				\scntext{примечание}{На этапе \textit{оценки интерфейса} необходимо вернуться к требованиям \textit{этапа анализа}. Требования, которые были сформулированы на \textit{этапе анализа}, должны быть выполнены, а также должна быть исследована эффективность модели \textit{интерфейса}. Чтобы определить эту эффективность, необходимо определить критерии эффективности.}
				\scntext{примечание}{Очень важным, но субъективным критерием является удовлетворенность пользователя. Поскольку пользователь должен работать с \textit{интерфейсом}, он имеет право голоса в вопросе о том, удобно ли работать с \textit{интерфейсом}, насколько привлекательным является интерфейс.}
				\begin{scnrelfromset}{критерии эффективности}
					\scnfileitem{Количество ошибок.}
					\scnfileitem{Время выполнения задачи.}
					\scnfileitem{Отношение пользователя к \textit{интерфейсу}.}
				\end{scnrelfromset}
			\end{scnindent}
			\scnitem{доработка и усовершенствование}
			\begin{scnindent}
				\scntext{примечание}{\textit{Доработка и усовершенствование} осуществляется на основе проблем, выявленных на этапе оценки. В рамках данного этапа вносится ряд улучшений в модель \textit{интерфейса}. Затем начинается новый цикл проектирования. Этот итеративный процесс будет продолжаться до тех пор, пока результат оценки не будет удовлетворять обозначенным требованиям.}
			\end{scnindent}
		\end{scnrelfromset}
		\begin{scnindent}
			\scnrelfrom{источник}{\scncite{Ehlert2003}}
		\end{scnindent}
		\scntext{примечание}{В \textit{пользовательском интерфейсе} часто нет среднего пользователя. В идеале, \textit{пользовательский интерфейс} должен быть способен адаптироваться к любому пользователю в любой среде. Поэтому используемая техника адаптации должна быть разработана таким образом, чтобы она могла поддерживать все типы пользователей.}
		
		\scnheader{Проектирование адаптивных интеллектуальных мультимодальных пользовательских интерфейсов (Kong)}
		\begin{scnrelfromset}{этапы}
			\scnfileitem{Моделирование \textit{пользовательского интерфейса} (описание абстрактного \textit{пользовательского интерфейса}).}
			\scnfileitem{Проектирование \textit{пользовательского интерфейса} по умолчанию (стандартная версия, конкретный \textit{пользовательский интерфейс}).}
			\scnfileitem{Разработка \textit{пользовательского интерфейса} (расширение или замена \textit{пользовательского интерфейса} по умолчанию) --- этот шаг опускается, когда система генерирует \textit{пользовательский интерфейс} по умолчанию автоматически.}
			\scnfileitem{Создание контекста использования (идентификация и создание контекста использования --- модели пользователя, модель устройства и модель среды/платформы).}
			\scnfileitem{Адаптация \textit{пользовательского интерфейса} --- автоматически (адаптация пользовательского интерфейса во время выполнения для соответствия конкретного контекста использования).}
			\scnfileitem{Кастомизация \textit{пользовательского интерфейса} --- настройка \textit{пользовательского интерфейса} самим пользователем (адаптируемость).}
		\end{scnrelfromset}
		\begin{scnindent}
			\scnrelfrom{источник}{\scncite{Kong2011}}
		\end{scnindent}
		
		\scnheader{Существующие методики проектирования адаптивных интеллектуальных мультимодальных пользовательских интерфейсов}
		\begin{scnrelfromset}{общие этапы}
			\scnitem{анализ контекста использования и задач пользователей}
			\scnitem{проектирование и разработка \textit{интерфейса}}
			\scnitem{оценка качества спроектированного \textit{интерфейса}}
		\end{scnrelfromset}
		\begin{scnrelfromset}{недостатки}
			\scnfileitem{Знания по каждому этапу проектирования находятся у разных специалистов в неформализорованном неунифицированном виде.}
			\scnfileitem{Отсутствие \textit{этапа формализованного документирования} этапов проектирования приводит в дальнейшем к необходимости создания отдельных help-систем для пользователей, разработчиков и так далее.}
		\end{scnrelfromset}
		
		\scnheader{Предлагаемая методика проектирования интерфейсов ostis-систем}
		\begin{scnrelfromvector}{этапы}
			\scnitem{анализ пользователя, его задач и целей, а также контекста использования}
			\begin{scnindent}
				\scntext{примечание}{Результаты первого этапа, такие как: модель конкретного пользователя, его потребности и контекст использования системы (устройство, окружающая среда) должны быть формализованы в рамках соответствующих онтологий \textit{базы знаний} \textit{интерфейса}. При этом в процессе формализации по необходимости должны быть переиспользованы \textit{компоненты базы знаний} из \textit{библиотеку многократно используемых компонентов ostis-систем}, а новые компоненты могут пополнить эту же библиотеку.}
			\end{scnindent}
			\scnitem{анализ требований к \textit{пользовательскому интерфейсу} и спецификация проектируемого \textit{пользовательского интерфейса}}
			\begin{scnindent}
				\scntext{примечание}{Результатом второго этапа являются конечные требования к \textit{интерфейсу}, которые должны быть сформулированы относительно модели пользователя и его цели, а также относительно контекста использования.}
				\scntext{примечание}{Спецификация включает в себя список задач решаемых интерфейсом, описание \textit{внешних языков представления знаний}.}
				\scntext{примечание}{Результаты должны быть формализованы, а в процессе выполнения могут быть использованы существующие компоненты из \textit{библиотеки многократно используемых компонентов интерфейсов ostis-систем}.}
			\end{scnindent}
			\scnitem{задачно-ориентированная декомпозиция \textit{пользовательского интерфейса}}
			\begin{scnindent}
				\scntext{примечание}{На этапе задачно-ориентированной декомпозиции \textit{пользовательского интерфейса} специфицированный интерфейс разбивается на интерфейсные подсистемы, которые могут разрабатываться параллельно. Это позволяет сократить сроки проектирования \textit{пользовательского интерфейса}. Целесообразно проводить разбиение таким образом, чтобы максимальное количество подсистем уже имелось в \textit{библиотеке многократно используемых компонентов пользовательских интерфейсов ostis-систем}.}
			\end{scnindent}
			\scnitem{проектирование \textit{пользовательского интерфейса} по умолчанию}
			\begin{scnindent}
				\scntext{примечание}{В соответствии с требованиями к \textit{пользовательскому интерфейсу}, строится модель \textit{адаптивного интеллектуального мультимодального пользовательского интерфейса}, которая является результатом третьего этапа.}
				\scntext{примечание}{Такая модель будет включать в себя формализованную модель \textit{базы знаний} и \textit{решателя задач}. При проектировании могут быть использованы компоненты \textit{интерфейса}, \textit{базы знаний} и \textit{решателя задач}. Компоненты будут записаны в унифицированном виде, что позволит обеспечить их автоматическую совместимость.}
			\end{scnindent}
			\scnitem{разработка \textit{пользовательского интерфейса}}
			\begin{scnindent}
				\scntext{примечание}{Результатом четвертого этапа является реализация спроектированного \textit{пользовательского интерфейса}.}
				\scntext{примечание}{После разработки \textit{пользовательского интерфейса} выделяются типовые фрагменты интерфейса. Специфицируя фрагменты интерфейса необходимым образом следует включать их в \textit{библиотеку многократно используемых компонентов пользовательских интерфейсов ostis-систем}.}
				\scntext{примечание}{При разработке пользовательского интерфейса также можно использовать готовые компоненты интерфейса из \textit{библиотеки многократно используемых компонентов пользовательских интерфейсов ostis-систем}.}
			\end{scnindent}
			\scnitem{анализ \textit{пользовательского интерфейса} и его адаптация}
			\begin{scnindent}
				\scntext{примечание}{На данном этапе используются готовые компоненты \textit{решателя задач}: \textit{sc-агенты анализа пользовательского интерфейса} и \textit{sc-агенты изменения модели пользовательского интерфейса на основе логических правил адаптации}.}
				\scntext{примечание}{Таким образом будет сформирована \textit{база знаний} проектируемого \textit{интерфейса}, которая автоматически может быть использована в качестве help-системы для пользователей, разработчиков и так далее.}
			\end{scnindent}
		\end{scnrelfromvector}
		\begin{scnindent}
			\scnrelfrom{источник}{\scncite{Sadouski2022a}}
			\scntext{примечание}{Поскольку знания о конкретном этапе обычно находятся у разных экспертов, особенностью предлагаемого подхода является обязательное формализованное документирование знаний в унифицированном виде и применение на каждом из этапов компонентного подхода.}
			\scntext{примечание}{Для применения компонентного подхода предлагается использовать \textit{библиотеку многократно используемых компонентов ostis-систем}.}
		\end{scnindent}
		
		\scnheader{анализ методов оценки пользовательских интерфейсов}
		\scntext{примечание}{Оценка \textit{пользовательского интерфейса} необходима для улучшения коммуникации между \textit{интеллектуальными системами} и их пользователями. Существует множество методов для оценки \textit{пользовательских интерфейсов}, направленных, в основном, на выявление проблем с использованием системы и минимизацию риска ошибок. Однако до сих пор отсутствует комплексный подход к оценке \textit{пользовательских интерфейсов интеллектуальных систем}.}
		\begin{scnindent}
			\begin{scnrelfromlist}{смотрите}
				\scnitem{\scncite{Ivory2001}}
				\scnitem{\scncite{Jeffries1991}}
			\end{scnrelfromlist}
		\end{scnindent}
		\scntext{примечание}{При оценке \textit{пользовательских интерфейсов} большую роль играет человеческий фактор, а основными участниками являются пользователи и эксперты. Очень сложно оценить правильность решения по адаптации \textit{пользовательского интерфейса интеллектуальной системы} и оценить его без участия человека.}
		\scntext{примечание}{Оценка \textit{пользовательских интерфейсов интеллектуальных систем} имеет свои особенности и требует специальных методов и инструментов.}
		
		\scnheader{пользовательский интерфейс}
		\begin{scnrelfromset}{правила построения}
			\scnfileitem{\textit{интерфейс} должен быть интуитивно понятен для конечного пользователя.}
			\scnfileitem{\textit{интерфейс} должен быть доступным для пользователей с ограниченными возможностями и пользователей, впервые сталкивающимися с информационными технологиями.}
			\scnfileitem{Достижение цели пользователем должно осуществляться наименее возможным количеством шагов.}
			\begin{scnindent}
				\scnrelfrom{источник}{\scncite{ISO9241-161}}
			\end{scnindent}
		\end{scnrelfromset}
		
		\scnheader{методы оценки пользовательских интерфейсов}
		\scnsuperset{оценка точности и полноты}
		\begin{scnindent}
			\scntext{примечание}{Это метод, при котором оценивается точность и полнота ответов системы на запросы пользователей.}
			\scntext{примечание}{Точность означает, насколько хорошо \textit{пользовательский интерфейс} отражает действительный опыт пользователя и предлагает ему наиболее подходящий контент в соответствии с его потребностями и предпочтениями. Оценить точность возможно путем анализа, насколько легко использовать интерфейс, насколько он отражает действительные возможности интеллектуальной системы и насколько он помогает пользователям достигать своих целей более эффективно.}
			\scntext{примечание}{Полнота, с другой стороны, означает, насколько хорошо \textit{пользовательский интерфейс} предоставляет всю необходимую информацию и функциональность, которые пользователь может потребовать в процессе взаимодействия с системой. Оценить полноту возможно путем анализа того, насколько полное и четкое описание функций, возможностей и ограничений системы предоставляется через \textit{пользовательский интерфейс}, в каком объеме пользователю доступна необходимая информация для принятия решений и насколько система может эффективно реагировать на запросы и потребности пользователя.}
		\end{scnindent}
		\scnsuperset{оценка персонализации}
		\begin{scnindent}
			\scntext{примечание}{Это метод, при котором оценивается способность системы адаптироваться к потребностям и предпочтениям каждого пользователя. Оценка персонализации выполняется с помощью анализа пользовательских данных, а также проведения опросов среди пользователей, чтобы определить, насколько хорошо система учитывает их потребности.}
			\begin{scnrelfromset}{параметры оценки}
				\scnfileitem{Работоспособность --- действительно ли \textit{пользовательский интерфейс} адаптивный и соответствует ли он потребностям пользователя?}
				\scnfileitem{Уникальность --- насколько уникальным и индивидуальным является персонализированный \textit{пользовательский интерфейс}?}
				\scnfileitem{Понятность --- насколько просто и легко пользователь может настроить и использовать персонализацию?}
				\scnfileitem{Эффективность --- насколько хорошо персонализированный \textit{пользовательский интерфейс} помогает пользователю в выполнении задач?}
				\scnfileitem{Удовлетворенность --- насколько положительным и удовлетворительным является \textit{пользовательский интерфейс}?}
			\end{scnrelfromset}
			\scntext{примечание}{Оценка персонализации \textit{адаптивных пользовательских интерфейсов} может быть выполнена различными методами, включая тестирование с использованием фокус-групп, опросы пользователей, анализ данных и другие. Важно отметить, что оценка персонализации должна проходить на всех этапах разработки для повышения пользовательского опыта и улучшения качества \textit{пользовательских интерфейсов интеллектуальных систем}.}
		\end{scnindent}
		\begin{scnrelfromset}{декомпозиция}
			\scnitem{качественный метод оценки пользовательских интерфейсов}
			\begin{scnindent}
				\scntext{задача}{Задача \textit{качественных методов оценки пользовательских интерфейсов} --- помочь понять мотивы поведения, потребности и логику пользователей.}
				\scntext{примечание}{\textit{качественные методы} нацелены на сбор данных, описывающих предмет изучения. Они позволяют углубиться в предметную область для получения представления о мотивации, мышлении и взглядах пользователей.}
				\scnsuperset{тестирование удобства использования пользовательских интерфейсов}
				\begin{scnindent}
					\scntext{примечание}{\textbf{\textit{тестирование удобства использования пользовательских интерфейсов}} является важной частью процесса проектирования и разработки. \textit{тестирование удобства использования} гарантирует, что эти интерфейсы удобны и полезны для потенциальных пользователей.}
					\begin{scnrelfromset}{этапы}
						\scnfileitem{Определение целевых пользователей. Это может потребовать проведения исследований пользователей для понимания потребностей и предпочтений потенциальных пользователей.}
						\scnfileitem{Разработка тестовых сценариев, которые отражают типичные случаи использования \textit{интерфейса}. Эти сценарии должны быть разработаны для проверки адаптивных возможностей \textit{интерфейса}, а также его удобства использования.}
						\scnfileitem{Проведение тестирования пользователей. Тестирование пользователей включает в себя наблюдение за пользователями во время взаимодействия с \textit{интерфейсом интеллектуальной системы}. Пользователи выполняют задачи, связанные с тестовыми сценариями. Тестирование пользователей может проводиться в контролируемой лабораторной среде или удаленно с использованием технологии совместного просмотра экрана.}
						\scnfileitem{Анализ результатов тестирования. Результаты тестирования пользователей анализируются для выявления проблем с удобством использования и областей для улучшения.}
						\scnfileitem{Итерация дизайна. На основе результатов тестирования пользователей и анализа результатов тестирования, \textit{интерфейс} может быть изменен для устранения проблем с удобством использования и улучшения общего пользовательского опыта.}
					\end{scnrelfromset}
				\end{scnindent}
				\scnsuperset{отслеживание движения глаз}
				\begin{scnindent}
					\scntext{примечание}{\textbf{\textit{отслеживание движения глаз}} --- это метод, который используется для анализа того, как пользователи взаимодействуют с \textit{пользовательским интерфейсом}. Отслеживание движения глаз определяет точки фиксации взгляда пользователя при взаимодействии с системой, а также переходы между ними.}
					\scntext{примечание}{При использовании метода определяется \textit{компонент пользовательского интерфейса}, на который пользователь смотрел дольше всего, сколько времени он уделяет каждому компоненту и как легко ему удается найти нужную информацию.}
					\scntext{примечание}{Метод выявляет \textit{компоненты интерфейса}, которым уделяется больше внимания, позволяет обнаружить области, вызывающие у пользователей затруднения.}
					\scntext{примечание}{Метод позволяет получить реалистичный образ отношения пользователя и \textit{интерфейса}, поскольку он фиксирует естественное движение глаз человека. Кроме того, \textit{отслеживание движения глаз} позволяет быстро найти проблемные места в \textit{пользовательском интерфейсе} и предложить улучшения, которые могут увеличить удобство использования \textit{интерфейса}.}
				\end{scnindent}
				\scnsuperset{экспертная оценка пользовательских интерфейсов}
				\begin{scnindent}
					\scntext{примечание}{Метод \textit{экспертной оценки пользовательских интерфейсов} заключается в исследовании \textit{пользовательского интерфейса} на соответствие заранее определенным правилам.}
					\begin{scnindent}
						\scnrelfrom{смотрите}{\scncite{ISO16982}}
					\end{scnindent}
					\scntext{примечание}{Достаточно часто метод \textit{экспертной оценки пользовательских интерфейсов} используется в тандеме с \textit{тестированием удобства использования}: \textit{экспертная оценка} используется для формирования гипотез о проблемах, а \textit{тестирование удобства использования} --- для их проверки.}
					\begin{scnrelfromvector}{этапы}
						\scnfileitem{Выявление экспертов.}
						\begin{scnindent}
							\scntext{примечание}{Первый этап --- поиск экспертов в предметной области, имеющих опыт работы как с \textit{интеллектуальными системами}, так и с моделью \textit{пользовательского интерфейса}. В число этих экспертов могут входить специалисты по взаимодействию человека с компьютером, специалисты по машинному обучению или эксперты в области, для которой разрабатывается интерфейс.}
							\scntext{примечание}{Экспертам предоставляется доступ к \textit{пользовательскому интерфейсу} и предлагается взаимодействовать с ним различными способами. Им может быть предложено выполнить задачи или сценарии, которые отражают типичные варианты использования интерфейса.}
						\end{scnindent}
						\scnfileitem{Проведение оценки.}
						\begin{scnindent}
							\scntext{примечание}{Эксперты оценивают \textit{пользовательский интерфейс} на основе набора установленных принципов удобства использования. Также может быть произведена оценка адаптивности \textit{интерфейса}.}
						\end{scnindent}
						\scnfileitem{Анализ результатов.}
						\begin{scnindent}
							\scntext{примечание}{Результаты оценки анализируются для выявления проблем с удобством использования и областей для улучшения. Эксперты могут давать рекомендации по улучшению интерфейса на основе своей оценки.}
						\end{scnindent}
					\end{scnrelfromvector}
				\end{scnindent}
			\end{scnindent}
			\scnitem{количественный метод оценки пользовательских интерфейсов}
			\begin{scnindent}
				\scntext{примечание}{\textit{количественные методы оценки пользовательских интерфейсов} позволяют выявить сложности или возможности, с которыми сталкиваются пользователи, и отделить реальные проблемы от предполагаемых. Такие методы измеряют числовые показатели.}
				\scntext{примечание}{\textit{количественными методы} формируют представление о том, чем занимаются пользователи.}
				\scnsuperset{A/Б тестирование пользовательских интерфейсов}
				\begin{scnindent}
					\scntext{примечание}{\textit{A/Б тестирование пользовательских интерфейсов} --- метод сравнения двух версий \textit{пользовательского интерфейса}. Результатом проведения метода является выявление версии \textit{интерфейса} наиболее подходящей для выполнения конкретной задачи.}
					\scntext{примечание}{Пользователи случайным образом разбиваются на два сегмента, каждый из которых видит только одну версию интерфейса.}
					\begin{scnrelfromvector}{этапы}
						\scnfileitem{Определение гипотезы.}
						\begin{scnindent}
							\scntext{примечание}{В первую очередь необходимо определить цель для проверки. В данном случае это могут быть как отдельные \textit{компоненты пользовательского интерфейса}, так и \textit{пользовательский интерфейс} в целом.}
							\scntext{примечание}{На основе анализа данных или личных предположений формулируется гипотеза, например, \scnqqi{если мы изменяем цвет и размер кнопки, пользователи будут чаще нажимать на нее} или \scnqqi{если мы добавим кнопку, которая будет видна на каждой странице, пользователи лучше будут понимать, как оставить отзыв}.}
						\end{scnindent}
						\scnfileitem{Определение размеров выборок.}
						\begin{scnindent}
							\scntext{примечание}{Чтобы получить репрезентативные результаты, нужно определить размер контрольной и экспериментальной групп.}
						\end{scnindent}
						\scnfileitem{Итерация интерфейса.}
						\begin{scnindent}
							\scntext{примечание}{На этом этапе вносятся изменения в интерфейс, которые соответствуют определенной гипотезе.}
						\end{scnindent}
						\scnfileitem{Наблюдение и сбор данных.}
						\begin{scnindent}
							\scntext{примечание}{Во время тестирования фиксируются действия пользователей, например, клики и время, проведенное на странице. Эти данные необходимы для определения изменений интерфейса, наиболее повлиявших на поведение пользователей.}
						\end{scnindent}
						\scnfileitem{Анализ результатов.}
						\begin{scnindent}
							\scntext{примечание}{После того, как тестирование завершено, проводится анализ данных для понимания, насколько значимы различия между контрольной и экспериментальной группами. Например, если пользователи, которые видели измененный интерфейс, кликали на кнопку в 2 раза чаще, чем пользователи, которые видели старый интерфейс, это говорит о том, что изменения были успешными и их необходимо внедрять в конечный \textit{пользовательский интерфейс}.}
						\end{scnindent}
					\end{scnrelfromvector}
				\end{scnindent}
				\scnsuperset{древовидное тестирование}
				\begin{scnindent}
					\scntext{примечание}{\textit{древовидное тестирование пользовательских интерфейсов} --- это метод оценки качества \textit{пользовательских интерфейсов}, который заключается в тестировании древовидной структуры навигации по системе.}
					\scntext{примечание}{Метод помогает определить, насколько эффективно пользователи могут находить нужную информацию и выполнять различные задачи.}
					\scntext{примечание}{В \textit{древовидном тестировании} объектом оценки является древовидная структура навигации по системе. Эта структура представляет собой дерево, в котором корневой элемент является главной страницей, а дочерние элементы --- подстраницы или разделы.}
					\scntext{цель}{Проверить, насколько легко пользователи могут навигировать по этой структуре и находить нужную информацию.}
					\begin{scnrelfromvector}{этапы}
						\scnfileitem{На основе структуры навигации создается тестовая среда, которая представляет собой виртуальный \textit{интерфейс}.}
						\scnfileitem{Респондентам предлагается выполнить задание, связанное с поиском нужной информации. Это может быть, например, поиск конкретной страницы или поиск информации по определенному тематическому разделу.}
						\scnfileitem{Респондентам дается доступ к тестовой среде, и они проходят по древовидной структуре навигации, пытаясь найти нужную им информацию.}
						\scnfileitem{В процессе прохождения теста респонденты записывают свои действия и комментарии о том, как они ориентируются в системе, какой путь выбирают для поиска нужной информации, какие ошибки и препятствия им приходится преодолевать и так далее.}
						\scnfileitem{По результатам тестирования делается анализ путей, выбранных респондентами, выявляются наиболее эффективные и неэффективные способы навигации, а также выделяются проблемные зоны \textit{интерфейса} системы.}
						\scnfileitem{На основе анализа результатов тестирования и выявленных проблем делается рекомендация по оптимизации древовидной структуры навигации или изменению \textit{пользовательского интерфейса} для улучшения пользовательского опыта.}
					\end{scnrelfromvector}
				\end{scnindent}
			\end{scnindent}
		\end{scnrelfromset}
		\scntext{примечание}{Представленные \textit{методы оценки пользовательских интерфейсов} предлагается применять для оценки \textit{пользовательских интерфейсов ostis-систем} в рамках рассмотренного ранее этапа \scnqqi{Анализ пользовательского интерфейса и его адаптация}. При этом особенностью проектирования \textit{пользовательских интерфейсов ostis-систем} является постоянная информационная поддержка пользователя на всех этапах проектирования \textit{интерфейса} за счет наличия в \textit{базе знаний} каждой \textit{ostis-системы} \textit{Предметной области методик проектирования пользовательских интерфейсов ostis-систем}, содержащей рассмотренные методики}
		
		\bigskip
	\end{scnsubstruct}
\end{SCn}
