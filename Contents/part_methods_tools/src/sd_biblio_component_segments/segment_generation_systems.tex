\begin{SCn}
	\scnsectionheader{Сегмент. Порождение ostis-систем с помощью библиотеки многократно используемых компонентов ostis-систем}
	
	\begin{scnsubstruct}
		\scnheader{компонентное проектирование компьютерных систем}
		\scntext{примечание}{Под компонентным проектированием компьютерных систем понимается не только расширение функционала уже созданной системы, но и создание целой системы \scnqq{с нуля}.}
		
		\scnheader{подсистема порождения ostis-систем}
		\scnrelfrom{назначение}{порождение ostis-систем}
		
		\scnheader{порождение ostis-систем}
		\scnidtf{создание ostis-систем}
		\begin{scnrelfromvector}{обобщённая последовательность действий пользователя}
			\scnitem{поиск ostis-платформы}
			\begin{scnindent}
				\scntext{примечание}{Различные ostis-платформы могут подходить для различных классов задач и компонентов, устанавливаемых в порождаемую систему.}
			\end{scnindent}
			\scnitem{установка ostis-платформы}
			\scnitem{поиск типовых подсистем}
			\begin{scnindent}
				\scntext{примечание}{В Библиотеке Метасистемы OSTIS находится множество подсистем, часто используемых в других ostis-системах. К типовым подсистемам относятся, например, подсистема коллективного проектирования ostis-систем, естественно-языковой интерфейс, обучающая подсистема, подсистема обеспечения безопасности и другие.}
				\scntext{примечание}{Устанавливая типовые подсистемы, можно значительно расширить функционал создаваемой ostis-системы.}
			\end{scnindent}
			\scnitem{установка типовых подсистем}
			\scnitem{поиск многократно используемых компонентов ostis-систем}
			\begin{scnindent}
				\scntext{примечание}{Благодаря широкому функционалу поиска многократно используемых компонентов ostis-систем, можно найти любые компоненты по различным критериям и их комбинациям.}
			\end{scnindent}
			\scnitem{установка многократно используемых компонентов}
			\scnitem{настройка ostis-системы}
			\begin{scnindent}
				\scnidtf{конфигурация ostis-системы}
				\scnsuperset{назначение ролей пользователей ostis-системы}
				\begin{scnindent}
					\scntext{примечание}{Возможность указать, какие пользователи являются администраторами, разработчиками, экспертами и пользователями создаваемой ostis-системы.}
				\end{scnindent}
			\end{scnindent}
		\end{scnrelfromvector}
		\begin{scnindent}
			\scntext{примечание}{Помимо действий пользователя при создании ostis-системы, подсистема порождения ostis-систем также регистрирует созданную ostis-систему в Метасистеме OSTIS. Таким образом, Метасистема OSTIS имеет возможность отслеживать и обновлять состояние компонентов этой системы.}
		\end{scnindent}
		
		
		\bigskip
	\end{scnsubstruct}
	\scnsourcecomment{Завершили \scnqqi{Сегмент. Порождение ostis-систем с помощью библиотеки многократно используемых компонентов ostis-систем}}
\end{SCn}