\begin{SCn}
	\scnsectionheader{Сегмент. Уточнение спецификации многократно используемого компонента ostis-систем}
	
	\begin{scnsubstruct}
		
	\scnheader{спецификация многократно используемого компонента ostis-систем}
	\scnsubset{спецификация}
	\scnidtf{описание многократно используемого компонента ostis-систем}
	\scnrelfrom{ключевой sc-элемент}{многократно используемый компонент ostis-систем}
	\scntext{примечание}{Каждый \textit{многократно используемый компонент ostis-систем} должен быть специфицирован в рамках библиотеки. Данные спецификации включают в себя основные знания о компоненте, которые позволяют обеспечить построение полной иерархии компонентов и их зависимостей, а также обеспечивают \uline{беспрепятственную} интеграцию компонентов в \scnkeyword{дочерние ostis-системы}. Для спецификации компонента используются как отношения, так и классы компонента.}
	\begin{scnindent}
		\begin{scnrelfromlist}{источник}
			\scnitem{\cite{Orlov2022a}}
			\scnitem{\cite{Davydenko2013}}
		\end{scnrelfromlist}
		\scntext{примечание}{Указание класса \scnkeyword{многократно используемый компонент ostis-систем} является обязательным.}
	\end{scnindent}
	\scntext{примечание}{Сам многократно используемый компонент в рамках спецификации является \textit{ключевым sc-элементом\scnrolesign}, а также может иметь множество своих ключевых sc-элементов.}
	\scnrelfrom{параметры, специфицирующие многократно используемый компонент ostis-систем}{параметр, заданный на многократно используемых компонентах ostis-систем\scnsupergroupsign}
	\scnrelfrom{классы отношений, специфицирующие многократно используемый компонент ostis-систем}{отношение, специфицирующее многократно используемый компонент ostis-систем\scnsupergroupsign}
	\scnrelfrom{описание примера}{\scnfileimage[40em]{Contents/part_methods_tools/src/images/sd_ostis_library/component_specification_example.png}}
	\scnheader{отношение, специфицирующее многократно используемый компонент ostis-систем\scnsupergroupsign}
	\scnidtf{отношение, которое используется при спецификации многократно используемого компонента ostis-систем}
	\begin{scnrelfromset}{разбиение}
		\scnitem{необходимое для установки отношение, специфицирующее многократно используемый компонент ostis-систем}
		\begin{scnindent}
			\scntext{примечание}{Чтобы многократно используемый компонент мог быть принят в библиотеку, нужно специфицировать его, используя каждое отношение из множества \textit{необходимое для установки отношение, специфицирующее многократно используемый компонент ostis-систем}. Здесь описана спецификация, общая для любых типов компонентов, однако в зависимости от типа компонента, спецификация может расширяться.}
			\scnhaselement{метод установки*}
			\scnhaselement{адрес хранилища*}
			\scnhaselement{зависимости компонента*}
		\end{scnindent}
		\scnitem{необязательное для установки отношение, специфицирующее многократно используемый компонент ostis-систем}
		\begin{scnindent}
			\scntext{примечание}{\textit{необязательное для установки отношение, специфицирующее многократно используемый компонент ostis-систем} помогает лучше понять суть компонента, упрощает поиск, но не является обязательным для того, чтобы компонент мог быть установлен в ostis-систему.}
			\scnhaselement{сопутствующие компоненты*}
			\scnhaselement{история изменений*}
			\scnhaselement{модификации компонентов*}
			\scnhaselement{авторы*}
			\scnhaselement{примечание*}
			\scnhaselement{пояснение*}
			\scnhaselement{идентификатор*}
			\scnhaselement{ключевой sc-элемент*}
			\scnhaselement{назначение*}
			\scnhaselement{требования полноты*}
			\scnhaselement{требования безошибочности*}
			\scnhaselement{преимущества*}
			\scnhaselement{недостатки*}
		\end{scnindent}
	\end{scnrelfromset}
	\scnheader{метод установки*}
	\scniselement{бинарное отношение}
	\scniselement{ориентированное отношение}
	\scntext{пояснение}{Пользователь может установить компонент вручную, а \scnkeyword{менеджер компонентов} - автоматически.}
	\scnrelfrom{первый домен}{многократно используемый компонент ostis-систем}
	\scnrelfrom{второй домен}{метод установки многократно используемого компонента}
	\begin{scnindent}
		\scnsubset{метод}
		\scnsuperset{метод установки динамически устанавливаемого многократно используемого компонента ostis-систем}
		\begin{scnindent}
			\scntext{примечание}{При динамической установке необходимо только скачать компонент и, при необходимости, его зависимые компоненты, и он сразу же функционирует в системе.}
			\scnrelfrom{описание примера}{\scnfileimage[35em]{Contents/part_methods_tools/src/images/sd_ostis_library/install_dynamic_method.png}}
			\begin{scnindent}
				\scniselement{sc.g-текст}
			\end{scnindent}
		\end{scnindent}
		\scnsuperset{метод установки многократно используемого компонента, при установке которого система требует перезапуска}
		\begin{scnindent}
			\scntext{примечание}{Установка таких компонентов происходит путём скачивания компонента и его трансляции в память системы.}
			\scnrelfrom{описание примера}{\scnfileimage[35em]{Contents/part_methods_tools/src/images/sd_ostis_library/install_with_reboot_method.png}}
			\begin{scnindent}
				\scniselement{sc.g-текст}
			\end{scnindent}
		\end{scnindent}
	\end{scnindent}
	\scnheader{адрес хранилища*}
	\scniselement{бинарное отношение}
	\scniselement{ориентированное отношение}
	\scntext{пояснение}{Связки отношения \textit{адрес хранилища*} связывают многократно используемый компонент, хранящийся в виде внешних файлов и файл, содержащий url-адрес многократно используемого компонента ostis-систем.}
	\scnrelfrom{первый домен}{многократно используемый компонент ostis-систем, хранящийся в виде внешних файлов}
	\scnrelfrom{второй домен}{файл, содержащий url-адрес многократно используемого компонента ostis-систем}
	\begin{scnindent}
		\scnsuperset{файл}
		\scnsubset{файл, содержащий url-адрес на GitHub многократно используемого компонента ostis-систем}
		\scnsubset{файл, содержащий url-адрес на Google Drive многократно используемого компонента ostis-систем}
		\scnsubset{файл, содержащий url-адрес на Docker Hub многократно используемого компонента ostis-систем}
	\end{scnindent}
	\scnheader{зависимости компонента*}
	\scniselement{квазибинарное отношение}
	\scniselement{ориентированное отношение}
	\scntext{пояснение}{Связки отношения \textit{зависимости компонента*} связывают многократно используемый компонент, и множество компонентов, без которых устанавливаемый компонент \uline{не может быть} встроен в \scnkeyword{дочернюю ostis-систему}.}
	\scnrelfrom{первый домен}{многократно используемый компонент ostis-систем}
	\scnrelfrom{второй домен}{множество многократно используемых компонентов ostis-систем}
	\scnheader{сопутствующие компоненты*}
	\scniselement{квазибинарное отношение}
	\scniselement{ориентированное отношение}
	\scntext{пояснение}{В некоторых случаях может оказаться, что для использования одного многократно используемого компонента OSTIS целесообразно или даже необходимо дополнительно использовать несколько других \textit{многократно используемых компонентов OSTIS}. Например, может оказаться целесообразным вместе с каким-либо sc-агентом информационного поиска использовать соответствующую команду интерфейса, которая представлена отдельным компонентом и позволит пользователю задавать вопрос для указанного sc-агента через интерфейс системы. В таких случаях для связи компонентов используется отношение \textit{сопутствующие компоненты*}. Наличие таких связей позволяет устранить возможные проблемы неполноты знаний и навыков в дочерней системе, из-за которых какие-либо из компонентов могут не выполнять свои функции. Связки отношения \textit{сопутствующий компонент*} связывают многократно используемые компоненты ostis-систем, которые целесообразно использовать в дочерней системе вместе. Каждая такая связка может дополнительно быть снабжена sc-комментарием или sc-пояснением, отражающим суть указываемой зависимости.}
	\scnrelfrom{первый домен}{многократно используемый компонент ostis-систем}
	\scnrelfrom{второй домен}{множество многократно используемых компонентов ostis-систем}
	\scnheader{история изменений*}
	\scniselement{бинарное отношение}
	\scniselement{ориентированное отношение}
	\scntext{пояснение}{Отношение \textit{история изменений*} позволяет специфицировать различные версии компонента и, при необходимости, устанавливать выбранную пользователем версию. Различные версии, как правило, отражают какие-либо улучшения или исправления ошибок.}
	\scnrelfrom{первый домен}{многократно используемый компонент ostis-систем}
	\scnrelfrom{второй домен}{история изменений}
	\scnheader{модификации компонентов*}
	\scniselement{бинарное отношение}
	\scniselement{ориентированное отношение}
	\scntext{пояснение}{\textit{модификации компонентов*} --- это функционально эквивалентные реализации одного и того же компонента, которые могут быть синтаксически эквивалентны (например, реализация одного и того же sc-агента на платформенно-зависимом и платформенно-независимом уровнях). Развитие \textit{Библиотеки Экосистемы OSTIS} происходит не только за счет ее пополнения новыми компонентами, но и за счет появления новых версий и модификаций уже существующих компонентов.}
	\scnrelfrom{первый домен}{многократно используемый компонент ostis-систем}
	\scnrelfrom{второй домен}{множество многократно используемых компонентов ostis-систем}
	\scnheader{авторы*}
	\scntext{пояснение}{Связки отношения \textit{авторы*} связывают многократно используемый компонент со множеством авторов этого компонента. Спецификация может также содержать дополнительную информацию об авторах при необходимости.}
	\scnheader{назначение*}
	\scntext{пояснение}{Отношение \textit{назначение*} позволяет описать ожидаемый сценарий, выделить рекомендации использования многократно используемого компонента. Требования полноты и безошибочности специфицируют возможные ограничения и ошибки компонента, область его использования.}
	
	\scnheader{параметр, заданный на многократно используемых компонентах ostis-систем\scnsupergroupsign}
	\scntext{примечание}{Для уточнения типа компонента могут использоваться другие классы, например дата публикации первой и последней версии компонента, качественно-количественные характеристики, такие как уровень семантической совместимости компонентов, сложность реализации компонента, уровень производительности компонента (для программ можно использовать O-нотацию), количество sc-элементов, входящих в состав многократно используемого компонента, количество ключевых узлов компонента, рейтинг компонента в рамках \textit{Экосистемы OSTIS}, количество скачиваний компонента и другие. Параметр \textit{лицензия многократно используемого компонента} используется для обозначения условий использования и распространения компонента. По умолчанию лицензия компонента открытая, если не указано иное.}
	\begin{scnindent}
		\scnrelfrom{источник}{\cite{Davydenko2013}}
	\end{scnindent}
		
		\bigskip
	\end{scnsubstruct}
	\scnsourcecomment{Завершили \scnqqi{Сегмент. Уточнение спецификации многократно используемого компонента ostis-систем}}
\end{SCn}