\begin{SCn}
	\scnsectionheader{Сегмент. Анализ библиотек многократно используемых компонентов}

\begin{scnsubstruct}
	\scnheader{библиотека многократно используемых компонентов}
	\scntext{примечание}{На данный момент не существует комплексной библиотеки многократно используемых семантически совместимых компонентов компьютерных систем в целом, не говоря об интеллектуальных. Существуют некоторые попытки создания библиотек типовых методов и программ для традиционных компьютерных систем, однако такие библиотеки не решают \scnkeyword{Проблемы в реализации компонентного проектирования интеллектуальных систем}.}
	
	\scnheader{библиотека подпрограмм}
	\scntext{примечание}{Термин \scnqq{библиотека подпрограмм}, одними из первых упомянули Уилкс М., Уиллер Д. и Гилл С. в качестве одной из форм организации вычислений на компьютере. Исходя из изложенного в их книге, под библиотекой понимался набор \scnqq{коротких, заранее заготовленных программ для отдельных, часто встречающихся (стандартных) вычислительных операций}. Стоит отметить, что компонентами библиотек являются не только программы, но и компоненты интерфейсов и баз знаний.}
	\begin{scnindent}
		\begin{scnrelfromlist}{источник}
			\scnitem{\cite{Wilkes1951}}
			\scnitem{\cite{Wilkes1953}}
			\scnitem{\cite{Volchenskova1984}}
		\end{scnrelfromlist}
	\end{scnindent}
	
	\scnheader{библиотека многократно используемых компонентов}
	\scntext{примечание}{К традиционным решениям относятся \scnkeyword{пакетные менеджеры} языков программирования и операционных систем, а также отдельные системы и платформы со встроенными компонентами и средствами для сохранения создаваемых компонентов.}
	\scntext{проблема текущего состояния}{Компоненты библиотеки могут быть реализованы на разных языках программирования (что приводит к тому, что	для каждого языка программирования разрабатываются свои библиотеки со своими решениями различных часто	встречаемых ситуаций), а также могут располагаться в разных местах, что приводит к тому, что в библиотеке необходимо средство для поиска компонентов и их установки.}
	
	\scnheader{пакетный менеджер}
	\begin{scnhaselementrolelist}{пример}
		\scnitem{pip}
		\scnitem{npm}
		\scnitem{poetry}
		\scnitem{maven}
		\scnitem{apt}
		\scnitem{pacman}
	\end{scnhaselementrolelist}
	\scntext{преимущество}{Решение конфликтов при установке зависимых компонентов.}
	\begin{scnrelfromset}{проблемы текущего состояния}
		\scnfileitem{\scnkeyword{пакетные менеджеры} не учитывают семантику компонентов, а только лишь устанавливают компоненты по идентификатору. Библиотеки таких компонентов являются только лишь хранилищем компонентов, никак не учитывающим назначение компонентов, их преимущества и недостатки, сферы применения, иерархию компонентов и другую информацию, необходимую для интеллектуализации компонентного проектирования компьютерных систем.}
		\scnfileitem{Поиск компонентов в библиотеках компонентов, соответствующих данных пакетным менеджерам сводится к поиску по идентификатору компонента. Современные библиотеки компонентов ориентированы только на какой-то определенный язык программирования, операционную систему или платформу.}
		\begin{scnindent}
			\scnrelfrom{источник}{\cite{Blahser2021}}
		\end{scnindent}
		\scnfileitem{Современные пакетные менеджеры являются лишь \scnqq{установщиками} без автоматической интеграции компонентов в систему.}
		\scnfileitem{Существенным недостатком современного подхода является	платформенная зависимость компонентов. Современные библиотеки компонентов ориентированы только на какой-то определенный язык программирования, операционную систему или платформу.}
	\end{scnrelfromset}
	\scntext{примечание}{Пакетные менеджеры языков программирования и операционных систем устроены по следующему принципу: существует хранилище компонентов (библиотека), которая представляет собой множество пакетов этого языка программирования или операционной системы и с которым взаимодействует менеджер компонентов.}
	
	\scnheader{pip}
	\scnrelto{пакетный менеджер}{Python}
	\scntext{примечание}{пакетный менеджер pip является системой управления пакетами, которая используется для установки пакетов из Python Package Index, который является некоторой библиотекой таких пакетов. Зачастую pip устанавливается вместе с Python.}
	\begin{scnrelfromset}{функциональные возможности}
		\scnfileitem{Установка пакета.}
		\scnfileitem{Установка пакета специализированной версии.}
		\scnfileitem{Удаление пакета.}
		\scnfileitem{Переустановка пакета.}
		\scnfileitem{Отображение установленных пакетов.}
		\scnfileitem{Поиск пакетов.}
		\scnfileitem{Верификация зависимостей пакетов.}
		\scnfileitem{Создание файла конфигурации со списком установленных пакетов и их версий.}
		\scnfileitem{Установка множества пакетов из файла конфигурации.}				
	\end{scnrelfromset}
	\scnrelfrom{рисунок}{\scnfileimage[35em]{Contents/part_methods_tools/src/images/sd_ostis_library/configuration_file.png}}
	\begin{scnindent}
		\scntext{пояснение}{Пример файла конфигурации пакетов pip.}
	\end{scnindent}
	\scntext{преимущество}{Хорошо работает с зависимостями, отображает безуспешно установленные пакеты, а также отображает информацию о требуемой версии пакета при конфликте с другим пакетом.}
	\scnrelfrom{альтернатива}{\scnkeyword{poetry}}
	
	\scnheader{poetry}
	\scntext{преимущество}{Автоматически работает с виртуальными окружениями, способен самостоятельно их находить и создавать.}
	\scntext{преимущество}{Файл конфигурации для пакетов poetry является более богатым, чем у pip, он хранит такие сведения, как имя проекта, версия проекта, его описание, лицензия, список авторов, URL проекта, его документации и сайта, список ключевых слов проекта и список PyPI классификаторов.}
	\scnrelfrom{рисунок}{\scnfileimage[35em]{Contents/part_methods_tools/src/images/sd_ostis_library/configuration_file2.png}}
	\begin{scnindent}
		\scntext{пояснение}{Пример файла конфигурации пакетов poetry}
		\scntext{примечание}{Такой вид спецификации не позволяет достичь совместимости между компонентами даже в рамках Python проектов и предназначена преимущественно только для чтения разработчиком.}
	\end{scnindent}
	\scntext{пояснение}{Автоматизировать проектирование компьютерных систем с помощью пакетного менеджера  \scnkeyword{poetry} или \scnkeyword{pip} невозможно, так как требуется вмешательство разработчика, который должен вручную совместить интерфейсы устанавливаемых пакетов.}
	
	\scnheader{Библиотека STL}
	\scnidtf{Библиотека стандартных шаблонов С++}
	\scniselement{библиотека подпрограмм языка программирования}
	\scniselement{C++}
	\scntext{пояснение}{Библиотека STL представляет собой набор согласованных обобщенных алгоритмов, контейнеров, средств доступа к их содержимому и различных вспомогательных функций в C++.}
	\begin{scnrelfromset}{включение}
		\scnitem{контейнер}
		\begin{scnindent}
			\scntext{назначение}{Хранение набора объектов в памяти.}
		\end{scnindent}
		\scnitem{итератор}
		\begin{scnindent}
			\scntext{назначение}{Обеспечение средств доступа к содержимому контейнера.}
		\end{scnindent}
		\scnitem{алгоритм}
		\begin{scnindent}
			\scntext{назначение}{Определение вычислительной процедуры.}
		\end{scnindent}
		\scnitem{адаптер}
		\begin{scnindent}
			\scntext{назначение}{Адаптация компонентов для обеспечения различного интерфейса.}
		\end{scnindent}
		\scnitem{функциональный объект}			
		\begin{scnindent}
			\scntext{назначение}{Сокрытие функции в объекте для использования другими компонентами.}
		\end{scnindent}
	\end{scnrelfromset}
	\scnrelfrom{рисунок}{\scnfileimage[35em]{Contents/part_methods_tools/src/images/sd_ostis_library/structure_stl.png}}
	\scntext{примечание}{Составляющие Библиотеки STL позволяют уменьшить количество создаваемых компонентов. Например, вместо написания отдельной функции поиска элемента для каждого типа контейнера обеспечивается единственная версия, которая работает с каждым из них, пока соблюдаются основные требования.}
	\scntext{примечание}{Совместимость компонентов (контейнеров) в Библиотеке STL обеспечивается общим интерфейсом использования этих компонентов.}
	
	\scnheader{компонентное проектирование компьютерных систем}
	\scntext{примечание}{Компонентный подход к проектированию компьютерных систем может реализовываться в рамках различных языков, платформ и приложений.}
	\begin{scnrelfromset}{примеры реализации}
		\scnitem{OWL}
		\begin{scnindent}
			\scntext{примечание}{Онтология, реализованная на языке \textit{OWL} (Web Ontology Language), представляет собой множество декларативных утверждений о сущностях словаря предметной области. \textit{OWL} предполагает концепцию \scnqq{открытого мира}, в соответствии с которой применимость описаний предметной области, помещенных в конкретном физическом документе, не ограничивается лишь рамками этого документа --- содержание онтологии может быть использовано и дополнено другими документами, добавляющими новые факты о тех же сущностях или описывающими другую предметную область в терминах данной. \scnqq{Открытость мира} достигается путем добавления URI каждому элементу онтологии, что позволяет воспринимать описанную на \textit{OWL} онтологию как часть всеобщего объединенного знания.}
			\begin{scnindent}
				\scnrelfrom{источник}{\cite{Hepp2008}}
			\end{scnindent}
		\end{scnindent}
		\scnitem{WebProtege}
		\begin{scnindent}
			\scntext{примечание}{\scnkeyword{WebProtege} представляет собой многопользовательский веб-интерфейс, позволяющий редактировать и хранить онтологии в формате \textit{OWL} в совместной среде. Данный проект позволяет не только создавать новые онтологии, но также загружать уже существующие онтологии, которые хранятся на сервере университета Стэнфорда. К преимуществу данного проекта можно отнести автоматическую проверку ошибок в процессе создания объектов онтологий. Данный проект является примером попытки решения проблемы накопления, систематизации и повторного использования уже существующих решений, однако, недостатком данного решения является обособленность разрабатываемых онтологий. Каждый разработанный компонент имеет свою иерархию понятий, подход к выделению классов и сущностей, которые зависят от разработчиков данных онтологий, так как в рамках данного подхода не существует универсальной модели представления знаний, а также формальной спецификации компонентов, представленных в виде онтологий. Следовательно, возникает проблема их семантической несовместимости, что, в свою очередь, приводит к невозможности повторного использования разработанных онтологий при проектировании баз знаний. Данный факт подтверждается наличием на сервере университета Стэнфорда многообразия различных онтологий на одни и те же темы.}
			\begin{scnindent}
				\scnrelfrom{источник}{\cite{Memduhoglu2018}}
			\end{scnindent}
		\end{scnindent}
		\scnitem{Modelica}
		\begin{scnindent}
			\scntext{примечание}{На основе языка \scnkeyword{Modelica} разработано большое число свободно доступных библиотек компонентов, одной из которых является библиотека Modelica\_StateGraph2, включающая компоненты для моделирования дискретных событий, реактивных и гибридных систем с помощью иерархических диаграмм состояния. Основным недостатком систем на базе языка \textit{Modelica} является отсутствие совместимости компонентов и достаточной документации, а также узкая направленность разрабатываемых компонентов.}
			\begin{scnindent}
				\scnrelfrom{источник}{\cite{Fritzson2014}}
			\end{scnindent}
		\end{scnindent}
		\scnitem{Microsoft Power Apps}
		\begin{scnindent}
			\scntext{примечание}{\scnkeyword{Microsoft Power Apps} --- это набор приложений, служб и соединителей, а также платформа данных, которая предоставляет среду разработки для эффективного создания пользовательских приложений для бизнеса. Платформа \textit{Microsoft Power Apps} имеет средства для создания библиотеки многократно используемых компонентов графического интерфейса, а также предварительно созданные модели распознавания текста (чтение визитных карточек или чеков) и средство обнаружения объектов, которые можно подключить к разрабатываемому приложению. Библиотека компонентов \textit{Microsoft Power Apps} представляет собой множество создаваемых пользователем компонентов, которые можно использовать в любых приложениях. Преимущество библиотеки в том, что компоненты могут настраивать свойства по умолчанию, которые можно гибко редактировать в любых приложениях, использующих компоненты. Недостаток в том, что отсутствует семантическая совместимость компонентов, спецификация компонентов, не решена проблема существования семантически эквивалентных компонентов, нет иерархии компонентов и средств поиска этих компонентов. Компоненты платформы \textit{Microsoft Power Apps} являются многократно используемыми только для однотипных приложений, которые создаются одним и тем же разработчиком.}
			\begin{scnindent}
				\scnrelfrom{источник}{\cite{Prakash2022}}
			\end{scnindent}
		\end{scnindent}
		\scnitem{Платформа IACPaaS}
		\begin{scnindent}
			\scntext{примечание}{\scnkeyword{Платформа IACPaaS} (Intelligent Applications, Control and Platform as a Service) --- облачная платформа для разработки, управления и удаленного использования интеллектуальных облачных сервисов. Она предназначена для обеспечения поддержки разработки, управления и удаленного использования прикладных и инструментальных мультиагентных облачных сервисов (прежде всего интеллектуальных) и их компонентов для различных предметных областей.}
			\begin{scnindent}
				\scnrelfrom{источник}{\cite{Moskalenko2016}}
			\end{scnindent}
			\begin{scnrelfromset}{предоставляет доступ}
				\scnfileitem{Прикладным пользователям (специалистам в различных предметных областях) --- к прикладным сервисам.}
				\scnfileitem{Разработчикам прикладных и инструментальных сервисов и их компонентов --- к инструментальным сервисам.}
				\scnfileitem{Управляющим интеллектуальными сервисами.}
				\scnfileitem{К сервисам управления.}
			\end{scnrelfromset}
			\begin{scnrelfromset}{поддерживает}
				\scnfileitem{Базовую технологию разработки прикладных и специализированных инструментальных (интеллектуальных) сервисов с использованием базовых инструментальных сервисов платформы, поддерживающих эту технологию.}
				\scnfileitem{Множество специализированных технологий разработки прикладных и специализированных инструментальных (интеллектуальных) сервисов, с использованием специализированных инструментальных сервисов платформы, поддерживающих эти технологии.}
			\end{scnrelfromset}
			\scntext{недостаток}{\textit{Платформа IACPaaS} не имеет средств для унифицированного представления компонентов интеллектуальных компьютерных систем и средств для их спецификации и автоматической интеграции компонентов.}
		\end{scnindent}
		\scntext{примечание}{На текущем состоянии развития информационных технологий \uline{не существует} комплексной библиотеки многократно используемых семантически совместимых компонентов компьютерных систем. Таким образом, предлагается комплексная библиотека многократно используемых семантически совместимых компонентов ostis-систем.}
	\end{scnrelfromset}
	 
	
		\bigskip
	\end{scnsubstruct}
	\scnsourcecomment{Завершили \scnqqi{Сегмент. Анализ библиотек многократно используемых компонентов}}
\end{SCn}