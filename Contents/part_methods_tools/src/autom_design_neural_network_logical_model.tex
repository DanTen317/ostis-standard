\usepackage{scn}\begin{SCn}
    \scnsectionheader{Логико-семантическая модель ostis-системы автоматизации проектирования искусственных нейронных сетей, семантически совместимых с базами знаний ostis-систем}
    \begin{scnsubstruct}
        \scntext{введение}{
            Наличия \textit{Языка представления нейросетевых методов в базах знаний} и его интерпретатора позволяет обеспечить интерпретацию \textit{нейросетевого метода} в памяти \textit{ostis-системы}. Наличие в единой памяти не только экземпляров методов, но и понятий, их описывающих, создает основу для автоматизации процесса построения нейросетевых методов. В памяти \textit{ostis-системы} хранятся знания о том, методы какого класса могут решить задачу заданного класса, но экземпляров класса этого метода может не быть представлено в системе. На этот случай система должна иметь возможность сообщить пользователю о возможности решения, для которого, однако, необходимо погрузить в систему определенный метод. Так как система хранит в единой памяти задачу и требования к методу ее решения, появляется возможность спроектировать необходимый метод. Для этого необходимо наличие среды проектирования методов соответствующих классов. В случае \textit{нейросетевого метода}, речь идет об интеллектуальной среде построения \textit{нейросетевых методов}.\\
            В основе интеллектуальной среды построения \textit{нейросетевых методов} лежат соответствующие другу другу иерархии действий, задач и методов построения \textit{и.н.с.} Наличие такой иерархии позволит описать язык представления методов построения \textit{и.н.с.} и разработать интерпретатор этого языка.\\
            Построение иерархии соответствующих действий построения \textit{и.н.с.} следует начать с изучения этапов проектирования и обучения \textit{и.н.с.}, которые, в общем случае, выполняют все разработчики и.н.с.
        }

        \scnsegmentheader{Логико-семантическая модель ostis-системы автоматизации проектирования искусственных нейронных сетей, семантически совместимых с базами знаний ostis-систем}
        \begin{scnsubstruct}

            \scnheader{Предметная область интеллектуальной среды построения нейросетевых методов}
            \scnidtf{Предметная область фреймворка нейросетей}
            \scniselement{предметная область}
            \begin{scnhaselementrole}{максимальный класс объектов исследования}
            {интеллектуальная среда построения нейросетевых методов}
            \end{scnhaselementrole}
            \begin{scnhaselementrolelist}{класс объектов исследования}
                \scnitem{интеллектуальная среда построения нейросетевых методов}
                \scnitem{постановка задачи}
                \scnitem{предобработка выборки}
                \scnitem{разбиение выборки на обучающую, валидационную и тестовую (контрольную)}
                \scnitem{выбор класса нейросетевых методов в соответствии со сформулированной задачей}
                \scnitem{формирование спецификации на входные и выходные данные}
                \scnitem{выбор метода оптимизации}
                \scnitem{выбор минимизируемой функции ошибки}
                \scnitem{начальная инициализация параметров нейронной сети}
                \scnitem{выбора гиперпараметров и.н.с.}
                \scnitem{обучение модели на обучающей выборке}
                \scnitem{оценка эффективности и.н.с}
                \scnitem{действие трансляции условия задачи}
                \scnitem{действие классификации задачи}
                \scnitem{действие поиска подходящей обучающей выборки}
                \scnitem{действие формирования требований к обучающей выборке}

            \end{scnhaselementrolelist}

            \scnheader{интеллектуальная среда построения нейросетевых методов}  %todo
            \begin{scnrelfromset}{этапы построения нейросетевых методов}
                \begin{scnindent}
                    \scnidtf{этапы построения и.н.с}
                    \scnitem{постановка задачи}
                    \scnitem{этапы, связанные с выборкой}
                    \begin{scnindent}
                        \begin{scnrelfromset}{разбиение}
                            \scnitem{предобработка выборки}
                            \scnitem{разбиение выборки на обучающую, валидационную и тестовую (контрольную)}
                        \end{scnrelfromset}
                    \end{scnindent}
                    \scnitem{этапы, связанные с импользуемыми моделями и.н.с}
                    \begin{scnindent}
                        \begin{scnrelfromset}{разбиение}
                            \scnitem{выбор класса нейросетевых методов в соответствии со сформулированной задачей}
                            \scnitem{формирование спецификации на входные и выходные данные}
                            \scnitem{выбор метода оптимизации}
                            \scnitem{выбор минимизируемой функции ошибки}
                            \scnitem{начальная инициализация параметров нейронной сети}
                            \scnitem{выбор гиперпараметров и.н.с.}
                            \scnitem{обучение модели на обучающей выборке} %todo: Ask(уже формализована с стандарте sd-ann, что делать) --- ответ(перенести из sd_ann и удалить соответственно в \item предметной области действий соответствующие объйкты)
                            \scnitem{оценка эффективности и.н.с}
                        \end{scnrelfromset}
                    \end{scnindent}
                \end{scnindent}
            \end{scnrelfromset}

            \scnheader{постановка задачи}
            \begin{scnrelfromset}{содержимое}
                \scnitem{входные данные}
                \begin{scnindent}
                    \scntext{пример}{изображения/видео, временные ряды, текст}
                \end{scnindent}
                \scnitem{выходные данные}
                \scnitem{требования к методу решения}
                \begin{scnindent}
                    \scntext{пример}{скорость, затраты по памяти и так далее}
                \end{scnindent}
                \scnitem{дополнительная информация}
                \begin{scnindent}
                    \scntext{пояснение}{информация, которая может помочь в построении метода решения задачи}
                \end{scnindent}
            \end{scnrelfromset}
            \begin{scnrelfromset}{декомпозиция}
                \scnitem{\textbf{действие трансляции условия задачи}}
                \scnitem{\textbf{действие классификации задачи}}
                \scnitem{\textbf{действие поиска подходящей обучающей выборки}}
                \scnitem{\textbf{действие формирования требований к обучающей выборке}}
            \end{scnrelfromset}


            \scnheader{действие трансляции условия задачи}
            \scniselement{действие}
            \scntext{пояснение}{Действие транслирует заданное с помощью \textit{интерфейса ostis-системы} (к примеру, естественно-языкового интерфейса) описание задачи в память ostis-системы. Действие необходимо в случае, когда условие задачи задается пользователем. Необходимо понимать, что описание задачи поступает в базу знаний не только от \textit{пользовательского интерфейса}. К примеру, задача может быть сформулирована самой системой в ходе ее жизнедеятельности.}
            \scntext{пояснение}{Данное действие является общим для всех ostis-систем, поэтому его рассмотрение выходит за рамки рассмотрения процесса построения интеллектуальной среды проектирования \textit{и.н.с.}}

            \scnheader{действие классификации задачи}
            \scniselement{действие}
            \scntext{пояснение}{Действие определяет класс задачи (задача регрессии, детекции, кластеризации и так далее), исходя из описания задачи в базе знаний.}

            \scnheader{действие поиска подходящей обучающей выборки}
            \scniselement{действие}
            \scntext{пояснение}{В базе знаний может храниться набор спецификаций выборок, к которым у ostis-системы есть доступ. Действие производит поиск выборок, которые могут быть использованы в качестве обучающей выборки.}

            \scnheader{действие формирования требований к обучающей выборке}
            \scniselement{действие}
            \scntext{пояснение}{Если обучающая выборка не была предоставлена и не была найдена, то необходимо сформировать описание требований к обучающей выборке, которое можно будет транслировать на язык пользовательского интерфейса и запросить необходимую выборку у пользователя.}

            \scnheader{предобработка выборки}
            \begin{scnrelfromset}{этапы}
                \scnitem{очистка} %1
                \begin{scnindent}
                    \begin{scnrelfromset}{содержимое}
                        \scnitem{обнаружение признаков, которые имеют в общем случае некорректные значения}
                        \begin{scnindent}
                            \scntext{пример}{для каких-то образов значение признака может иметь неопределенное значение, либо значение, не совпадающее по типу, либо аномально большое или очень маленькое значение, которое встречается в редком числе случаев}
                        \end{scnindent}
                        \scnitem{методы устранения признаков, имеющие неопределённые значения}
                        \begin{scnindent}
                            \scntext{пояснение}{значения могут быть заменены средним значением этого признака, рассчитанным по всем образам (для непоследовательных данных), либо они могут быть заменены средним значением по соседним образам (в случае временных рядов), либо каким-то фиксированным значением}
                            \scntext{радикальное решение}{удаление образов, имеющих неопределенные значения признаков из выборки}
                            \begin{scnindent}
                                \scntext{примечание}{однако его лучше применять, если образов с отсутствующими значениями признаков немного. Для выбросов и аномалий применяются схожие стратегии (но только в том случае, если задача не состоит в прогнозировании этих аномалий)}
                            \end{scnindent}
                        \end{scnindent}
                    \end{scnrelfromset}
                    \scntext{примечание}{в интеллектуальной среде проектирования данный этап соответствует выполнению \textbf{\textit{действия очистки выборки}}, которое выполняется в случае обработки выборки, которая ранее не была представлена в памяти системы (к примеру, была получена от пользователя). Реализация интерпретатора (агента) данного действия требует описания в памяти классификации стратегий очистки данных и реализации методов применения этих стратегий} %todo действия очистки выборки - непонятно где в стандарте, что формализовывать
                \end{scnindent}
                \scnitem{выявление содержательных признаков} %2
                \begin{scnindent}
                    \scntext{цель}{уменьшение размерности пространства признаков для снижения влияния эффекта переобучения на модель}
                    \begin{scnindent}
                        \scntext{реализация}{использование методов отбора признаков и выделения признаков}
                    \end{scnindent}

                    \scntext{задача}{инжиниринг признаков, состоящий в отборе признаков, влияющих на результат работы модели}
                    \begin{scnindent}
                        \scntext{примечание}{несодержательные признаки, которые никак не коррелируют с выходом модели, удаляются}
                        \begin{scnrelfromset}{содержимое}
                            \scnitem{формирование подмножества из исходных признаков (алгоритм последовательного обратного отбора, рекурсивный алгоритм обратного устранения признаков,  алгоритмы с использованием случайных лесов)}
                            \scnitem{извлечение информации для построения нового подпространства признаков (алгоритмы с использованием автоэнкодера)}
                        \end{scnrelfromset}
                    \end{scnindent}
%                    \begin{scnrelfromset}{содержимое} %todo возможно надо убрать тут scnrelfromset тк всего один элемент и как-то переделать
%                        \scnitem{инжиниринг признаков, состоящий в отборе признаков, влияющих на результат работы модели}
%                        \begin{scnindent}
%                            \scntext{примечание}{несодержательные признаки, которые никак не коррелируют с выходом модели, удаляются}
%                            \begin{scnrelfromset}{содержимое}
%                                \scnitem{формирование подмножества из исходных признаков (алгоритм последовательного обратного отбора, рекурсивный алгоритм обратного устранения признаков,  алгоритмы с использованием случайных лесов)}
%                                \scnitem{извлечение информации для построения нового подпространства признаков (алгоритмы с использованием автоэнкодера)}
%%                                \scnitem{}
%                            \end{scnrelfromset}
%                        \end{scnindent}
%                    \end{scnrelfromset}
                    \scntext{примечание}{в интеллектуальной среде проектирования данный этап соответствует выполнению \textbf{\textit{действия выявления содержательных признаков}}. Реализация интерпретатора (агента) данного действия требует описания в памяти классификации стратегий уменьшения размерности признакового пространства и реализации методов применения этих стратегий}
                \end{scnindent}
                \scnitem{трансформация} %3
                \begin{scnindent}
                    \scntext{задача}{подготовка данных к обучению}
                    \begin{scnindent}
                        \scntext{примечание}{следует уделить особое внимание наличию категориальных признаков, чаще всего заданных строковыми типами}
                        \begin{scnrelfromset}{содержимое}
                            \scnitem{категориальные признаки}
                            \begin{scnindent}
                                \begin{scnrelfromset}{разбиение}
                                    \scnitem{номинальные признаки}
                                    \begin{scnindent}
                                        \scntext{способ кодирования}{последовательное числовое кодирование}
                                        \begin{scnindent}
                                            \scntext{пример}{1,2,3,...}
                                        \end{scnindent}
                                    \end{scnindent}
                                    \scnitem{порядковые признаки}
                                    \begin{scnindent}
                                        \scntext{свойство}{равноправие}
                                        \begin{scnindent}
                                            \scntext{пояснение}{порядковые признаки не могут сравниваться по числовому коду}
                                            \begin{scnindent}
                                                \scntext{пример}{пол - 0/1}
                                            \end{scnindent}
                                        \end{scnindent}
                                        \scntext{способ кодирования}{прямое кодирование}
                                        \begin{scnindent}
                                            \scntext{пояснение}{создание и использование фиктивных признаков по количеству значений исходного}
                                            \scntext{пример}{признак пол(мужской, женский) преобразуется в два новых признака мужской и женский с соответствующими значениями для имеющихся образов}
                                        \end{scnindent}
                                    \end{scnindent}
                                \end{scnrelfromset}
                                \scntext{действие преобразования}{масштабирование}
                                \begin{scnindent}
                                    \scntext{описание}{приведение значений признаков к одному общему интервалу}
                                    \begin{scnindent}
                                        \scntext{пояснение}{это особенно актуально для признаков, имеющих несоразмерные выборочные средние значения по всем образам}
                                        \scntext{пример}{один признак в среднем имеет значение 10000 и 12}
                                        \scntext{примечание}{это может проявится в выполнении минимизации только по признаку с наибольшими значениями и плохой сходимости метода обучения}
                                    \end{scnindent}
                                    \begin{scnrelfromset}{способы реализации}
                                        \scnitem{нормализация на отрезок (min-max нормализация)}
                                        \begin{scnindent}
                                            \scntext{примечание}{чаще всего используется}
                                            \scnrelfrom{формула}{
                                                \begin{equation*}
                                                    x^i_{norm}=\frac{x^i-x_{min}}{x_{max}-x_{min}}
                                                \end{equation*}
                                                где \textit{$x^i$} --- значение признака для отдельно взятого образа i, \texit{$x_{min}$} --- наименьшее значение для признака, \textit{$x_{max}$} --- наибольшее значение для признака
                                            }
                                        \end{scnindent}
                                        \scnitem{применение стандартизации признаков}
                                        \begin{scnindent}
                                            \scnrelfrom{формула}{
                                                \begin{equation*}
                                                    x^i_{std}=\frac{x^i-\mu(x)}{\sigma(x)}
                                                \end{equation*}
                                                где \textit{$\mu(x)$} --- выборочное среднее отдельного признака, \texit{$\sigma(x)$} --- стандартное отклонение
                                            }
                                            \scntext{примечание}{стандартизация сохраняет полезную информацию о выбросах в исходных данных и делает алгоритм обучения менее чувствительным к ним}
                                        \end{scnindent}
                                        \scnitem{применение дискретизации}
                                        \begin{scnindent}
                                            \scntext{пояснение}{применяется для перехода от вещественного признака к порядковому за счет кодирования интервалов одним значением}
                                            \scntext{пример}{если признак отражает возраст человека, то может быть произведена дискретизация значений с выделением определенных возрастных групп, где каждая группа будет кодироваться одним целым числом}
                                        \end{scnindent}

                                    \end{scnrelfromset}
                                \end{scnindent}
                            \end{scnindent}
                        \end{scnrelfromset}
                    \end{scnindent}
                    \scntext{примечание}{в интеллектуальной среде проектирования данный этап соответствует выполнению \textbf{\textit{действия трансформации выборки}}. Реализация интерпретатора (агента) данного действия требует описания в памяти классификации методов масштабирования признаков и реализации методов применения этих стратегий}
                \end{scnindent}
            \end{scnrelfromset}

            \scnheader{разбиение выборки на обучающую, валидационную и тестовую (контрольную)}
            \scntext{пояснение}{производится разбиение всей выборки данных, на обучающую, тестовую и, в некоторых случаях, валидационную}
            \scntext{примечание}{валидационная выборка используется для оценки влияния изменения гиперпараметров на результат обучения и может применяться как дополнительный инструмент для этого наравне с сеточным поиском}
            \scntext{описание}{разбиение проводится в соотношении 3:1:1, в процентах (60/20/20), если валидационная выборка не используется, то 80/20}
            \scntext{примечание}{в интеллектуальной среде проектирования данный этап соответствует выполнению \textbf{\textit{действия разбиения выборки}}}


            %todo part 6

            \scnheader{начальная инициализация параметров нейронной сети}   %10
            \scntext{цель}{инициализация весовых коэффициентов и порогов нейронной сети}
            \begin{scnrelfromset}{способ}
                \scnitem{инициализация значениями из равномерного распределения на каком-то небольшом интервале}
                \begin{scnindent}
                    \scntext{пример}{[-0.1, 0.1]}
                \end{scnindent}
                \scnitem{инициализация значениями из стандартного нормального распределения}
                \scnitem{инициализация по методу Ксавье}
                \begin{scnindent}
                    \scntext{пояснение}{Применяется для предотвращения резкого уменьшения или увеличения значений выхода нейронных элементов после применения функции активации при прямом прохождении образа через глубокую нейронную сеть. Фактически инициализация этим методом осуществляется посредством выбора значений из равномерного распределения на отрезке $[- \sqrt{6} / \sqrt{n_i+n_{i+1}}, \sqrt{6} / \sqrt{n_i+n_{i+1}}]$, где $n_i$ --- это число входящих связей в данный слой, а $n_i$ --- число исходящих связей из данного слоя. Таким образом, инициализация этим методом проводится для разных слоев нейронной сети из разных отрезков}
                \end{scnindent}
                \scnitem{инициализация, полученная из предобученной модели}
                \begin{scnindent}
                    \scntext{пояснение}{Вариант инициализации, который предполагает использование в качестве \scnqq{стартовой} модели предобученной модели, взятой из некоторого репозитория предобученных моделей, обученную самим исследователем или в процессе работы интеллектуальной системы.}
                \end{scnindent}
                \scnitem{инициализация по методу Кайминга}
                \begin{scnindent}
                    \scntext{пояснение}{Данный метод инициализации применяется для решения проблемы \scnqq{затухающего} градиента и \scnqq{взрывающегося} градиента. Производится посредством выбора значений из равномерного распределения на отрезке $[-\sqrt{2} / \sqrt{(1+a^2)fan}, \sqrt{2} / \sqrt{(1+a^2)fan}]$, где \textit{a} --- угол наклона к оси абсцисс для отрицательной части области определения функции активации типа ReLU (для обычной ReLU функции этот параметр равен 0), $fan$ --- параметр режима работы, который для фазы прямого распространения равен количеству входящих связей (для устранения эффекта \scnqq{взрывающегося} градиента), а для фазы обратного распространения --- количеству выходящих (для устранения эффекта \scnqq{затухающего} градиента).}
                \end{scnindent}
            \end{scnrelfromset}
            \scntext{примечание}{В интеллектуальной среде проектирования данный этап соответствует выполнению действия начальной инициализации и.н.с..}

            \scnheader{выбор гиперпараметров и.н.с.}
            \scntext{пояснение}{На практике некоторые гиперпараметры (такие как количество слоев, их типы, количество нейронов в слое) часто определяются экспериментально, в процессе итеративного поиска лучшего варианта решения задачи. Хотя способы частично автоматизировать этот процесс существуют, они все же рассчитаны на наличие некоторых предусловий проведения эксперимента, в частности интервалов изменения параметра (например, скорости обучения).}
            \begin{scnrelfromset}{выбор}
                \scnitem{гиперпараметр}
                \begin{scnindent}
                    \begin{scnrelfromset}{декомпозиция}
                        \scnitem{параметры обучения и.н.с.}
                        \scnitem{архитектура модели и.н.с.}
                    \end{scnrelfromset}
                \end{scnindent}
            \end{scnrelfromset}
            \scntext{цель}{Нахождение оптимальных гиперпараметров}
            \begin{scnindent}
                \scntext{пример}{использование метода сеточного поиска, который позволяет проверить значения гиперпараметров, взятые с определенным шагом или из определенного интервала (кортежа). С помощью этого метода выбирается оптимальный набор гиперпараметров, который дает лучшие результаты, он используется для последующего дообучения. Или же, если полученные результаты являются приемлемыми, то процесс дальнейшего обучения вообще не проводится. Следует отметить затратность данного метода, так как фактически осуществляется перебор различных значений параметров обучения. Для снижения объема работы применяется метод случайного поиска.}
            \end{scnindent}
            \scntext{пояснение}{Для оптимизации архитектуры определяются типы слоев нейронной сети, количество нейронных элементов в каждом слое, их характеристики --- функция активации, для сверточных элементов --- размер ядра, а также параметры padding и шаг свертки (stride).}
            \scntext{примечаниея}{Здесь же может осуществляться оценка не только пользовательского варианта сети, но и предобученной архитектуры. Основное правило при выборе --- количество параметров модели не должно превышать размер обучающей выборки. Для предобученных архитектур это ограничение снимается.}
            \scntext{примечание}{В интеллектуальной среде проектирования данный этап соответствует выполнению действия выбора гипперпараметров и.н.с.. Действие использует классификацию и спецификации гиперпараметров и.н.с.}

            \scnheader{оценка эффективности и.н.с}
            \scntext{пояснение}{После выполнения обучения осуществляется оценка полученной модели с помощью метрик оценки качества.}
            \begin{scnrelfromset}{визуализация}
                \scnitem{матрица ошибок}
                \scnitem{ROC-кривая}
            \end{scnrelfromset}

            \scnheader{матрица ошибок}
            \scnsubset{матрица}
            \begin{scnrelfromset}{содержимое}
                \scnitem{число истинно-положительных предсказаний}
                \scnitem{число истинно-отрицательных предсказаний}
                \scnitem{число ложно-положительных предсказаний}
                \scnitem{число ложно-отрицательных предсказаний}
            \end{scnrelfromset}
            \scnrelfrom{схема}{\scnitem{\scnfileimage[20em]{Contents/part_methods_tools/src/images/autom_design_neural_network_logical_model/confusion_matrix.png}}}

            \scnheader{ROC-кривая}
            \scnsubset{график}
            \scnidtf{receiver operating characteristic}
            \scntext{пояснение}{\textbf{\textit{ROC-кривая}} --- это график, в котором, основываясь на заданном пороге решения классификатора, рассчитываются доли ложноположительных и истинно положительных исходов.}
            \scntext{примечание}{Основываясь на ROC-кривой, высчитывается AUC-показатель (площадь под кривой), которая используется в качестве характеристики качества модели.}
            \scntext{примечание}{В интеллектуальной среде проектирования данный этап соответствует выполнению действия оценки эффективности и.н.с..}

            \bigskip
        \end{scnsubstruct}

        \scnendsegmentcomment{Операционная семантика sc-моделей искусственных нейронных сетей, используемых в ostis-системах}

        \bigskip
    \end{scnsubstruct}

    \scnendcurrentsectioncomment


\end{SCn}