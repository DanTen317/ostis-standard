\usepackage{scn}
\begin{SCn}
    \scnsectionheader{Логико-семантическая модель ostis-системы автоматизации проектирования искусственных нейронных сетей, семантически совместимых с базами знаний ostis-систем}
%                    \begin{scnrelfromvector}{введение}
%                        \scnfileitem{Наличия \textit{Языка представления нейросетевых методов в базах знаний} и его интерпретатора позволяет обеспечить интерпретацию \textit{нейросетевого метода} в памяти \textit{ostis-системы}. Наличие в единой памяти не только экземпляров методов, но и понятий, их описывающих, создает основу для автоматизации процесса построения нейросетевых методов. В памяти \textit{ostis-системы} хранятся знания о том, методы какого класса могут решить задачу заданного класса, но экземпляров класса этого метода может не быть представлено в системе. На этот случай система должна иметь возможность сообщить пользователю о возможности решения, для которого, однако, необходимо погрузить в систему определенный метод. Так как система хранит в единой памяти задачу и требования к методу ее решения, появляется возможность спроектировать необходимый метод. Для этого необходимо наличие среды проектирования методов соответствующих классов. В случае \textit{нейросетевого метода}, речь идет об интеллектуальной среде построения \textit{нейросетевых методов}.}
%                        \scnfileitem{В основе интеллектуальной среды построения \textit{нейросетевых методов} лежат соответствующие другу другу иерархии действий, задач и методов построения \textit{и.н.с.} Наличие такой иерархии позволит описать язык представления методов построения \textit{и.н.с.} и разработать интерпретатор этого языка.}
%                        \scnfileitem{Построение иерархии соответствующих действий построения \textit{и.н.с.} следует начать с изучения этапов проектирования и обучения \textit{и.н.с.}, которые, в общем случае, выполняют все разработчики и.н.с.}
%                    \end{scnrelfromvector}
%
%                    \begin{scnrelfromlist}{ключевое понятие}
%                        \scnitem{действие трансляции условия задачи}
%                        \scnitem{действие классификации задачи}
%                        \scnitem{действие поиска подходящей обучающей выборки}
%                        \scnitem{действие формирования требований к обучающей выборке}
%                        \scnitem{действие очистки выборки}
%                        \scnitem{действие выявления содержательных признаков}
%                        \scnitem{действие трансформации выборки}
%                        \scnitem{действие разбиения выборки}
%                        \scnitem{действие выбора класса нейросетевых методов}
%                        \scnitem{действие формирования спецификации входов и выходов и.н.с.}
%                        \scnitem{действие выбора метода оптимизации}
%                        \scnitem{действие выбора минимизируемой функции ошибки}
%                        \scnitem{действие начальной инициализации и.н.с.}
%                        \scnitem{действие выбора гиперпараметров и.н.с.}
%                        \scnitem{метод обучения с учителем}
%                        \scnitem{метод обучения без учителя}
%                        \scnitem{действие обучения и.н.с.}
%                    \end{scnrelfromlist}
%
%                    \begin{scnsubstruct}
%                        %todo: Ask (возможно не как предметная область)
%                        \scnheader{Предметная область действий и методик проектирования автоматизации проектирования искусственных нейронных сетей, семантически совместимых с базами знаний ostis-систем}
%                        \scniselement{предментная область}
%                        \scnhaselementrole{максимальный класс объектов исследования}{}  %todo
%                        \begin{scnhaselementrolelist}{класс объектов исследования}
%                            \scnitem{}  %todo
%                        \end{scnhaselementrolelist}
%                    \end{scnsubstruct}
%
%                    \scnheader{Проектирование и обучения и.н.с.}
%                    \begin{scnrelfromset}{этапы}
%                        \scnitem{Постановка задачи}
%                        \begin{scnindent}
%
%                        \end{scnindent}
%                        \scnitem{}
%                    \end{scnrelfromset}
%
%                    \scnheader{}
%

    \begin{scnsubstruct}
        \scntext{введение}{
            Наличия \textit{Языка представления нейросетевых методов в базах знаний} и его интерпретатора позволяет обеспечить интерпретацию \textit{нейросетевого метода} в памяти \textit{ostis-системы}. Наличие в единой памяти не только экземпляров методов, но и понятий, их описывающих, создает основу для автоматизации процесса построения нейросетевых методов. В памяти \textit{ostis-системы} хранятся знания о том, методы какого класса могут решить задачу заданного класса, но экземпляров класса этого метода может не быть представлено в системе. На этот случай система должна иметь возможность сообщить пользователю о возможности решения, для которого, однако, необходимо погрузить в систему определенный метод. Так как система хранит в единой памяти задачу и требования к методу ее решения, появляется возможность спроектировать необходимый метод. Для этого необходимо наличие среды проектирования методов соответствующих классов. В случае \textit{нейросетевого метода}, речь идет об интеллектуальной среде построения \textit{нейросетевых методов}.\\
            В основе интеллектуальной среды построения \textit{нейросетевых методов} лежат соответствующие другу другу иерархии действий, задач и методов построения \textit{и.н.с.} Наличие такой иерархии позволит описать язык представления методов построения \textit{и.н.с.} и разработать интерпретатор этого языка.\\
            Построение иерархии соответствующих действий построения \textit{и.н.с.} следует начать с изучения этапов проектирования и обучения \textit{и.н.с.}, которые, в общем случае, выполняют все разработчики и.н.с.
        }

        \scnsegmentheader{not sure} %todo: Ask(как назвать)
        \begin{scnsubstruct}

            \scnheader{Предметная область интеллектуальной среды построения нейросетевых методов}
            \scnidtf{Предметная область фреймворка нейросетей}
            \scniselement{предметная область}
            \begin{scnhaselementrole}{максимальный класс объектов исследования}
            {интеллектуальная среда построения нейросетевых методов}
            \end{scnhaselementrole}
            \begin{scnhaselementrolelist}{класс объектов исследования}
                \scnitem{интеллектуальная среда построения нейросетевых методов}
                \scnitem{постановка задачи}
                \scnitem{предобработка выборки}
%                \scnitem{очистка}
%                \scnitem{выявление содержательных признаков}
%                \scnitem{трансформация}
                \scnitem{разбиение выборки на обучающую, валидационную и тестовую (контрольную)}
                \scnitem{выбор класса нейросетевых методов в соответствии со сформулированной задачей}
                \scnitem{формирование спецификации на входные и выходные данные}
                \scnitem{выбор метода оптимизации}
                \scnitem{выбор минимизируемой функции ошибки}
                \scnitem{начальная инициализация параметров нейронной сети}
                \scnitem{выбора гиперпараметров и.н.с.}
                \scnitem{обучение модели на обучающей выборке}
                \scnitem{оценка эффективности и.н.с}
            \end{scnhaselementrolelist}

            \scnheader{интеллектуальная среда построения нейросетевых методов}
            \begin{scnrelfromset}{этапы построения нейросетевых методов}
            \begin{scnindent}
                \scnidtf{этапы построения и.н.с}
                \scnitem{постановка задачи}
                \scnitem{предобработка выборки}
                \begin{scnindent}
                    \begin{scnrelfromset}{этапы}
                        \scnitem{очистка}
                        \scnitem{выявление содержательных признаков}
                        \scnitem{трансформация}
                    \end{scnrelfromset}
                \end{scnindent}
                \scnitem{разбиение выборки на обучающую, валидационную и тестовую (контрольную)}
                \scnitem{выбор класса нейросетевых методов в соответствии со сформулированной задачей}
                \scnitem{формирование спецификации на входные и выходные данные}
                \scnitem{выбор метода оптимизации}
                \scnitem{выбор минимизируемой функции ошибки}
                \scnitem{начальная инициализация параметров нейронной сети}
                \scnitem{выбора гиперпараметров и.н.с.}
                \scnitem{обучение модели на обучающей выборке} %todo Ask(уже формализована с стандарте sd-ann, что делать)
                \scnitem{оценка эффективности и.н.с}
            \end{scnindent}
            \end{scnrelfromset}

            %\scnheader{первый пункт(все как у меня в oper_sem), дальше пишешь все, как там}

            \bigskip
        \end{scnsubstruct}

        \scnendsegmentcomment{Операционная семантика sc-моделей искусственных нейронных сетей, используемых в ostis-системах}

        \bigskip
    \end{scnsubstruct}

    \scnendcurrentsectioncomment


\end{SCn}