\begin{SCn}
    \scnsectionheader{Предметная область и онтология комплексной библиотеки многократно используемых семантически совместимых компонентов ostis-систем}
    \begin{scnsubstruct}
        \scntext{эпиграф}{Всё, что можно сделать одинаково, нужно делать одинаково.}
        \scntext{аннотация}{Важнейшим этапом эволюции любой технологии является переход к компонентному проектированию на основе постоянно пополняемой библиотеки многократно используемых компонентов. Идея библиотеки компонентов не нова, но семантическая мощность \textbf{\textit{Библиотеки Экосистемы OSTIS}} значительно выше аналогов за счет того, что подавляющее большинство компонентов библиотеки --- компоненты \textit{базы знаний}, представленные на унифицированном языке смыслового представления знаний (\textit{SC-коде}). Таким образом, в Библиотеке Экосистемы OSTIS обеспечивается высокий уровень семантической совместимости компонентов, что приводит к высокому уровню семантической совместимости \textit{ostis-систем}, использующих комплексную библиотеку многократно используемых семантически совместимых компонентов ostis-систем.}
       
        \scnheader{Предметная область многократно используемых компонентов ostis-систем}
        \scniselement{предметная область}
        \begin{scnhaselementrolelist}{максимальный класс объектов исследования}
        	\scnitem{библиотека многократно используемых компонентов ostis-систем}
        	\scnitem{многократно используемый компонент ostis-систем}
        \end{scnhaselementrolelist}
        \begin{scnhaselementrolelist}{класс объектов исследования}
            \scnitem{многократно используемый компонент базы знаний}
            \scnitem{многократно используемый компонент решателя задач}
            \scnitem{многократно используемый компонент пользовательского интерфейса}
            \scnitem{атомарный многократно используемый компонент ostis-систем}
            \scnitem{неатомарный многократно используемый компонент ostis-систем}
            \scnitem{зависимый многократно используемый компонент ostis-систем}
            \scnitem{независимый многократно используемый компонент ostis-систем}
            \scnitem{платформенно независимый многократно используемый компонент ostis-систем}
            \scnitem{платформенно зависимый многократно используемый компонент ostis-систем}
            \scnitem{многократно используемый компонент ostis-систем, хранящийся в виде внешних файлов}
            \scnitem{многократно используемый компонент ostis-систем, хранящийся в виде файлов исходных текстов}
            \scnitem{многократно используемый компонент ostis-систем, хранящийся в виде скомпилированных файлов}
            \scnitem{многократно используемый компонент, хранящийся в виде sc-структуры}
            \scnitem{типовая подсистема ostis-систем;платформа интерпретации sc-моделей компьютерных систем}
            \scnitem{динамически устанавливаемый многократно используемый компонент ostis-систем}
            \scnitem{многократно используемый компонент, при установке которого система требует перезапуска}
            \scnitem{хранилище многократно используемого компонента ostis-систем, хранящегося в виде внешних файлов}
            \scnitem{хранилище многократно используемого компонента ostis-систем, хранящегося в виде файлов исходных текстов}
            \scnitem{хранилище многократно используемого компонента ostis-систем, хранящегося в виде скомпилированных файлов}
            \scnitem{спецификация многократно используемого компонента ostis-систем}
            \scnitem{отношение, специфицирующее многократно используемый компонент ostis-систем}
            \scnitem{параметр, заданный на многократно используемых компонентах ostis-систем\scnsupergroupsign}
            \scnitem{файл, содержащий url-адрес многократно используемого компонента ostis-систем}
            \scnitem{менеджер многократно используемых компонентов ostis-систем}
        \end{scnhaselementrolelist}
        \begin{scnhaselementrolelist}{исследуемое отношение}
            \scnitem{метод установки*}
            \scnitem{адрес хранилища*}
            \scnitem{зависимости компонента*}
            \scnitem{установленные компоненты*}
        \end{scnhaselementrolelist}
        \begin{scnhaselementrolelist}{исследуемый параметр}
            \scnitem{класс многократно используемого компонента ostis-систем\scnsupergroupsign}
        \end{scnhaselementrolelist}
        \begin{scnhaselementrolelist}{отношение, используемое в предметной области}
            \scnitem{автор*}
            \scnitem{ключевой sc-элемент*}
            \scnitem{пояснение*}
            \scnitem{sc-идентификатор*}
            \scnitem{история изменений*}
        \end{scnhaselementrolelist}
        \begin{scnrelfromset}{основные положения}
            \scnfileitem{Важнейшим этапом эволюции любой технологии является переход к компонентному проектированию на основе постоянно пополняемый библиотеки многократно используемых компонентов.}
            \begin{scnindent}
                \begin{scnrelfromset}{известные проблемы}
                    \scnfileitem{Отсутствие унификации в принципах представления различных видов знаний в рамках одной базы знаний, и, как следствие, отсутствие унификации в принципах выделения и спецификации многократно используемых компонентов приводит к несовместимости компонентов, разработанных в рамках разных проектов.}
                    \scnfileitem{Не ведётся разработка стандартов, обеспечивающих совместимость этих компонентов.}
                    \scnfileitem{Многие компоненты используют для идентификации язык разработчика (как правило, английский), и предполагается, что все пользователи будут использовать этот же язык. Однако для многих приложений это недопустимо  понятные только разработчику идентификаторы должны быть скрыты от конечных пользователей, которые должны быть в состоянии выбрать язык для идентификаторов, которые они видят}
                    \scnitem{Отсутствие средств поиска компонентов, удовлетворяющих заданных критериям.}
                \end{scnrelfromset}
                \begin{scnrelfromset}{необходимые требования}
                    \scnfileitem{Универсальный язык представления знаний.}
                    \scnfileitem{Универсальная процедура интеграции знаний в рамках указанного языка.}
                    \scnfileitem{Разработка стандарта, обеспечивающего семантическую совместимость интегрируемых знаний (таким стандартом является согласованная система используемых понятий).}
                \end{scnrelfromset}
            \end{scnindent}
            \scnfileitem{Повторное использование готовых компонентов широко применяется во многих отраслях, связанных с проектирование различного рода систем, поскольку позволяет уменьшить трудоемкость разработки и ее стоимость (путем минимизации количества труда за счет отсутствия необходимости разрабатывать какой-либо компонент), повысить качество создаваемого контента и снизить профессиональные требования к разработчикам компьютерных систем. Таким образом, осуществляется  переход от программирования компонентов или целых систем к их проектированию (дизайну, сборке) на основе готовых компонентов. \textbf{\textit{компонентное проектирование интеллектуальных компьютерных систем}} предполагает подбор существующих компонентов, способных решить поставленную задачу целиком или декомпозицию задачи на подзадачи с выделением компонентов для каждой из них.}
        \end{scnrelfromset}

\scnheader{семантически смежный раздел*}
\begin{scnhaselementset}
	\scnitem{Предметная область и онтология комплексной библиотеки многократно используемых семантически совместимых компонентов ostis-систем}
	\scnitem{Логико-семантическая модель Метасистемы OSTIS}
\end{scnhaselementset}
\begin{scnindent}
	\scntext{пояснение}{\textit{Метасистема OSTIS} ориентирована на разработку и практическое внедрение методов и средств компонентного проектирования семантически совместимых интеллектуальных систем, которая предоставляет возможность быстрого создания интеллектуальных приложений различного назначения. В состав Метасистемы OSTIS входит \scnkeyword{Библиотека Метасистемы OSTIS}. Сферы практического применения технологии компонентного проектирования семантически совместимых интеллектуальных систем ничем не ограничены.}
	\scntext{пояснение}{Основу для реализации компонентного подхода в рамках \textit{Технологии OSTIS} составляет \scnkeyword{Библиотека Метасистемы OSTIS}.}
\end{scnindent}
\scnheader{семантически смежный раздел*}
\begin{scnhaselementset}
	\scnitem{Предметная область и онтология комплексной библиотеки многократно используемых семантически совместимых компонентов ostis-систем}
	\scnitem{Предметная область и онтология комплексной технологии поддержки жизненного цикла интеллектуальных компьютерных систем нового поколения}
\end{scnhaselementset}
\begin{scnindent}
	\scntext{пояснение}{Основным требованием, предъявляемым к \textit{Технологии OSTIS}, является обеспечение возможности совместного использования в рамках ostis-систем различных \textit{видов знаний} и различных \textit{моделей решения задач} с возможностью \uline{неограниченного} расширения перечня используемых в ostis-системе видов знаний и моделей решения задач без существенных трудозатрат. Следствием данного требования является необходимость реализации компонентного подхода на всех уровнях, от простых компонентов баз знаний и решателей задач до целых встраиваемых ostis-систем.}
\end{scnindent}
\scnheader{дочерний раздел*}
\begin{scnhaselementvector}
	\scnitem{Предметная область и онтология комплексной библиотеки многократно используемых семантически совместимых компонентов ostis-систем}
	\scnitem{Предметная область и онтология многократно используемых компонентов баз знаний ostis-систем}
\end{scnhaselementvector}
\begin{scnindent}
	\scntext{пояснение}{На сегодняшний день разработано большое число \textit{баз знаний} по самым различным предметным областям. Однако в большинстве случаев каждая база знаний разрабатывается отдельно и независимо от других, в отсутствие единой унифицированной формальной основы для представления знаний, а также единых принципов формирования систем понятий для описываемой предметной области. В связи с этим разработанные базы оказываются, как правило, несовместимы между собой и не пригодны для повторного использования. Компонентный подход к разработке интеллектуальных компьютерных систем, реализуемый в виде \scnkeyword{библиотеки многокртно используемых компонентов ostis-систем}, позволяет решить описанные проблемы.}
\end{scnindent}
\scnheader{дочерний раздел*}
\begin{scnhaselementvector}
	\scnitem{Предметная область и онтология комплексной библиотеки многократно используемых семантически совместимых компонентов ostis-систем}
	\scnitem{Предметная область и онтология многократно используемых компонентов решателей задач ostis-систем}
\end{scnhaselementvector}
\begin{scnindent}
	\scntext{пояснение}{В области разработки \textit{решателей задач} существует большое количество конкретных реализаций, однако вопросы совместимости различных решателей при решении одной задачи практически не рассматриваются.}
\end{scnindent}
\scnheader{дочерний раздел*}
\begin{scnhaselementvector}
	\scnitem{Предметная область и онтология комплексной библиотеки многократно используемых семантически совместимых компонентов ostis-систем}
	\scnitem{Предметная область и онтология многократно используемых компонентов интерфейсов ostis-систем}
\end{scnhaselementvector}

\scnheader{компонентное проектирование интеллектуальных компьютерных систем}
		\scntext{назначение}{Позволяет уменьшить трудоемкость создания компьютерных систем и их стоимость (путем минимизации количества труда за счет отсутствия необходимости разрабатывать какой-либо компонент), повысить качество создаваемых компьютерных систем и снизить профессиональные требования к разработчикам этих систем.}      
		\scntext{пояснение}{Компонентное проектирование интеллектуальных компьютерных систем предполагает подбор существующих компонентов, способных решить поставленную задачу целиком или декомпозицию задачи на подзадачи с выделением компонентов для каждой из них.}  
		\scntext{преимущество}{Проектируемые системы по предлагаемой технологии обладают высоким уровнем гибкости, их разработка осуществляется поэтапно, переходя от одной целостной версии системы к другой. При этом стартовая версия системы может быть ядром соответствующего класса систем, входящим в библиотеку многократно используемых компонентов.}
		\scnheader{технология компонентного проектирования интеллектуальных компьютерных систем}		
		\scnhaselementrole{главный ключевой sc-элемент}{библиотека совместимых многократно используемых компонентов}
		\scntext{преимущество}{Позволяет проектировать интеллектуальные системы комбинируя уже существующие компоненты, выбирая нужные из соответствующих библиотек. Использование готовых компонентов предполагает, что распространяемый компонент верифицирован и документирован, а возможные ошибки и ограничения устранены либо специфицированы и известны. Создание \textit{библиотеки многократно используемых компонентов} не означает пересоздание всех уже существующих современных продуктов информационных технологий. Технология компонентного проектирования интеллектуальных компьютерных систем предполагает использование огромного опыта в разработке современных компьютерных систем, однако обязательным является \uline{спецификация} каждого компонента (как вновь созданного, так и интегрируемого существующего) для обеспечения возможности его установки и совместимости с другими компонентами и системами. Тем не менее эффективная технология компонентного проектирования появится только тогда, когда сформируется \scnqq{критическая масса} разработчиков прикладных систем, участвующих в пополнении \textit{библиотек многократно используемых компонентов} проектируемых систем.}
\scnrelfrom{проблемы текущего состояния}{Проблемы в реализации компонентного проектирования интеллектуальных компьютерных систем}
\begin{scnindent}
	\scntext{примечание}{Проблемы реализации компонентного подхода к проектированию интеллектуальных компьютерных систем наследуют проблемы современных \textit{технологий проектирования интеллектуальных систем}.}
	\begin{scneqtoset}
		\scnfileitem{\uline{Несовместимость} компонентов, разработанных в рамках разных проектов, вследствие отсутствия унификации в принципах представления различных видов знаний в рамках одной \textit{базы знаний}, и, как следствие, отсутствие унификации в принципах выделения и \textbf{\textit{спецификации многократно используемых компонентов}}.}
		\begin{scnindent}
			\scntext{примечание}{Большинство существующих систем создано как автономные программные продукты, которые не могут быть использованы в качестве компонентов других систем. Необходимо использовать либо целую систему, либо ничего. Небольшое число систем поддерживает компонентно-ориентированную архитектуру способную интегрироваться с другими системами. Однако, их интеграция возможна при условии использования одинаковых технологий и только при проектировании одной командой разработчиков.}
			\scntext{примечание}{Многократная повторная разработка уже имеющихся технических решений обусловлена либо тем, что известные технические решения \uline{плохо} интегрируются в разрабатываемую систему, либо тем, что эти технические решения трудно найти. Данная проблема актуальна как в целом в сфере разработки компьютерных систем, так и в сфере разработки систем, основанных на знаниях, поскольку в системах такого рода степень согласованности различных видов знаний влияет на возможность системы решать нетривиальные задачи.}  
		\end{scnindent}
		\scnfileitem{Невозможность автоматической интеграции компонентов в систему \uline{без} ручного вмешательства пользователя.}
		\scnfileitem{Автоматическое обновление компонентов приводит к рассогласованности как отдельных модулей компьютерных систем, так и самих систем между собой.}
		\scnfileitem{Отсутствие классификации компонентов на различных уровнях детализации.}
		\scnfileitem{Не проводится тестирование, верификация и анализ качества компонентов, не выделяются преимущества, недостатки, ограничения компонентов.}
		\scnfileitem{Многие компоненты используют для идентификации язык разработчика (как правило, английский), и предполагается, что все пользователи будут использовать этот же язык. Однако для многих приложений это недопустимо --- понятные только разработчику идентификаторы должны быть скрыты от конечных пользователей, которые должны быть в состоянии выбрать язык для идентификаторов, которые они видят.}
		\scnfileitem{Отсутствие средств поиска компонентов, удовлетворяющих заданным критериям.}
	\end{scneqtoset}
\end{scnindent}
\scntext{примечание}{\scnkeyword{компонентное проектирование интеллектуальных компьютерных систем} возможно только в том случае, если отбор компонентов будет осуществляться на основе тщательного анализа качества этих компонентов. Одним из важнейших критериев такого анализа является уровень семантической совместимости анализируемых компонентов со всеми компонентами, имеющимися в текущей версии библиотеки.}
\scnrelfrom{предъявляемые требования}{Требования к реализации компонентного проектирования интеллектуальных компьютерных систем}
\begin{scnindent}
	\begin{scneqtoset}
		\scnfileitem{Обеспечение совместимости (интегрируемости) компонентов интеллектуальных компьютерных систем на основе унификации представления этих компонентов.}
		\scnfileitem{Четкое разделение процесса разработки формальных описаний интеллектуальных компьютерных систем и процесса их реализации по этому описанию.}
		\scnfileitem{Четкое разделение разработки формального описания проектируемой интеллектуальной системы от разработки различных вариантов интерпретации таких формальных описаний систем.}
		\scnfileitem{Наличие онтологии компонентного проектирования интеллектуальных компьютерных систем, включающей (1) описание методов компонентного проектирования, (2) модель \textit{библиотеки многократно используемых компонентов}, (3) модель \textit{спецификации многократно используемых компонентов}, (4) полную \textit{классификацию многократно используемых компонентов}, (5) описание средств взаимодействия разрабатываемой интеллектуальной компьютерной системы с \textit{библиотеками многократно используемых компонентов}.}
		\scnfileitem{Наличие \textit{библиотек многократно используемых компонентов интеллектуальных компьютерных систем}, включающих спецификации компонентов.}
		\scnfileitem{Наличие средств взаимодействия разрабатываемой интеллектуальной компьютерной системы с библиотеками многократно используемых компонентов для установки любых видов компонентов и управления ими в создаваемой системе.}
		\begin{scnindent}
			\scnnote{Под установкой компонента понимается его транспортировка в систему (копирование sc-элементов и/или скачивание файлов компонента), а также выполнение вспомогательных действий для того, чтобы компонент мог функционировать в создаваемой системе.}
		\end{scnindent}
	\end{scneqtoset}
	\scntext{примечание}{Для того, чтобы решить возникшие проблемы при проектировании интеллектуальных систем и библиотек их многократно используемых компонентов, необходимо придерживаться общих принципов технологии проектирования интеллектуальных компьютерных систем, а также выполнить эти требования.}
\end{scnindent}

% сделать фрагментом
\scnheader{Предметная область и онтология комплексной библиотеки многократно используемых семантически совместимых компонентов ostis-систем}
\scnrelfrom{анализ}{Анализ библиотек многократно используемых компонентов}
\scnheader{Анализ библиотек многократно используемых компонентов}
\begin{scnsubstruct}
	\scnheader{библиотека многократно используемых компонентов}
	\scntext{примечание}{На данный момент не существует комплексной библиотеки многократно используемых семантически совместимых компонентов компьютерных систем в целом, не говоря об интеллектуальных. Существуют некоторые попытки создания библиотек типовых методов и программ для традиционных компьютерных систем, однако такие библиотеки не решают \scnkeyword{Проблемы в реализации компонентного проектирования интеллектуальных систем}.}
	
	\scnheader{библиотека подпрограмм}
	\scntext{примечание}{Термин "библиотека подпрограмм", одними из первых упомянули Уилкс М., Уиллер Д. и Гилл С. в качестве одной из форм организации вычислений на компьютере. Исходя из изложенного в их книге, под библиотекой понимался набор "коротких, заранее заготовленных программ для отдельных, часто встречающихся (стандартных) вычислительных операций". Стоит отметить, что компонентами библиотек являются не только программы, но и компоненты интерфейсов и баз знаний.}
	
	\scnheader{библиотека многократно используемых компонентов}
	\scntext{примечание}{К традиционным решениям относятся \scnkeyword{пакетные менеджеры} языков программирования и операционных систем, а также отдельные системы и платформы с встроенными компонентами и средствами для сохранения создаваемых компонентов.}
	\scntext{проблема текущего состояния}{Компоненты библиотеки могут быть реализованы на разных языках программирования (что приводит к тому, что	для каждого языка программирования разрабатываются свои библиотеки со своими решениями различных часто	встречаемых ситуаций), а также могут располагаться в разных местах, что приводит к тому, что в библиотеке необходимо средство для поиска компонентов и их установки.}
	
	\scnheader{пакетный менеджер}
	\begin{scnhaselementrolelist}{пример}
		\scnitem{pip}
		\scnitem{npm}
		\scnitem{poetry}
		\scnitem{maven}
		\scnitem{apt}
		\scnitem{pacman}
	\end{scnhaselementrolelist}
	\scntext{преимущество}{Решение конфликтов при установке зависимых компонентов}
	\begin{scnrelfromset}{проблемы текущего состояния}
		\scnfileitem{пакетные менеджеры не учитывают семантику компонентов, а только лишь устанавливают компоненты по идентификатору. Библиотеки таких компонентов являются только лишь хранилищем компонентов, никак не учитывающим назначение компонентов, их преимущества и недостатки, сферы применения, иерархию компонентов и другую информацию, необходимую для интеллектуализации компонентного проектирования компьютерных систем.}
		\scnfileitem{Поиск компонентов в библиотеках компонентов, соответствующих данных пакетным менеджерам сводится к поиску по идентификатору компонента. Современные библиотеки компонентов ориентированы только на какой-то определенный язык программирования, операционную систему или платформу.}
		\scnfileitem{Современные пакетные менеджеры являются лишь "установщиками" без автоматической интеграции компонентов в систему.}
		\scnfileitem{Существенным недостатком современного подхода является	платформенная зависимость компонентов. Современные библиотеки компонентов ориентированы только на какой-то определенный язык программирования, операционную систему или платформу.}
	\end{scnrelfromset}
	\scntext{примечание}{Пакетные менеджеры языков программирования и операционных систем устроены по следующему принципу: существует хранилище компонентов (библиотека), которая представляет собой множество пакетов этого языка программирования или операционной системы и с которым взаимодействует менеджер компонентов.}
	
	\scnheader{pip}
	\scnrelto{пакетный менеджер}{Python}
	\scntext{примечание}{пакетный менеджер pip является системой управления пакетами, которая используется для установки пакетов из Python Package Index, который является некоторой библиотекой таких пакетов. Зачастую pip устанавливается вместе с Python.}
	\begin{scnrelfromset}{функциональные возможности}
		\scnfileitem{Установка пакета.}
		\scnfileitem{Установка пакета специализированной версии.}
		\scnfileitem{Удаление пакета.}
		\scnfileitem{Переустановка пакета.}
		\scnfileitem{Отображение установленных пакетов.}
		\scnfileitem{Поиск пакетов.}
		\scnfileitem{Верификация зависимостей пакетов.}
		\scnfileitem{Создание файла конфигурации со списком установленных пакетов и их версий.}
		\scnfileitem{Установка множества пакетов из файла конфигурации.}				
	\end{scnrelfromset}
	\scnrelfrom{рисунок}{\scnfileimage[20em]{Contents/part_methods_tools/src/images/sd_ostis_library/configuration_file.png}}
	\begin{scnindent}
		\scntext{пояснение}{Пример файла конфигурации пакетов pip}
	\end{scnindent}
	\scntext{преимущество}{Хорошо работает с зависимостями, отображает безуспешно установленные пакеты, а также отображает информацию о требуемой версии пакета при конфликте с другим пакетом.}
	\scnrelfrom{альтернатива}{\scnkeyword{poetry}}
	
	\scnheader{poetry}
	\scntext{преимущество}{Автоматически работает с виртуальными окружениями, способен самостоятельно их находить и создавать.}
	\scntext{преимущество}{Файл конфигурации для пакетов poetry является более богатым, чем у pip, он хранит такие сведения, как имя проекта, версия проекта, его описание, лицензия, список авторов, URL проекта, его документации и сайта, список ключевых слов проекта и список PyPI классификаторов.}
	\scnrelfrom{рисунок}{\scnfileimage[20em]{Contents/part_methods_tools/src/images/sd_ostis_library/configuration_file2.png}}
	\begin{scnindent}
		\scntext{пояснение}{Пример файла конфигурации пакетов poetry}
		\scntext{примечание}{Такой вид спецификации не позволяет достичь совместимости между компонентами даже в рамках Python проектов и предназначена преимущественно только для чтения разработчиком.}
	\end{scnindent}
	\scntext{пояснение}{Автоматизировать проектирование компьютерных систем с помощью пакетного менеджера  \scnkeyword{poetry} или \scnkeyword{pip} невозможно, так как требуется вмешательство разработчика, который должен вручную совместить интерфейсы устанавливаемых пакетов.}
	
	\scnheader{Библиотека STL}
	\scnidtf{Библиотека стандартных шаблонов С++}
	\scniselement{библиотека подпрограмм языка программирования}
	\scniselement{C++}
	\scntext{пояснение}{Библиотека STL представляет собой набор согласованных обобщенных алгоритмов, контейнеров, средств доступа к их содержимому и различных вспомогательных функций в C++.}
	\begin{scnrelfromset}{включение}
		\scnitem{контейнер}
		\begin{scnindent}
			\scntext{назначение}{Хранение набора объектов в памяти.}
		\end{scnindent}
		\scnitem{итератор}
		\begin{scnindent}
			\scntext{назначение}{Обеспечение средств доступа к содержимому контейнера.}
		\end{scnindent}
		\scnitem{алгоритм}
		\begin{scnindent}
			\scntext{назначение}{Определение вычислительной процедуры.}
		\end{scnindent}
		\scnitem{адаптер}
		\begin{scnindent}
			\scntext{назначение}{Адаптация компонентов для обеспечения различного интерфейса.}
		\end{scnindent}
		\scnitem{функциональный объект}			
		\begin{scnindent}
			\scntext{назначение}{Сокрытие функции в объекте для использования другими компонентами.}
		\end{scnindent}
	\end{scnrelfromset}
	\scnrelfrom{рисунок}{\scnfileimage[20em]{Contents/part_methods_tools/src/images/sd_ostis_library/structure_stl.png}}
	\scntext{примечание}{Составляющие Библиотеки STL позволяют уменьшить количество создаваемых компонентов. Например, вместо написания отдельной функции поиска элемента для каждого типа контейнера обеспечивается единственная версия, которая работает с каждым из них, пока соблюдаются основные требования.}
	\scntext{примечание}{Совместимость компонентов (контейнеров) в Библиотеке STL обеспечивается общим интерфейсом использования этих компонентов.}
	
	\scnheader{компонентное проектирование компьютерных систем}
	\scntext{примечание}{Компонентный подход к проектированию компьютерных систем может реализовываться в рамках различных языков, платформ и приложений.}
	\begin{scnrelfromset}{примеры реализации}
		\scnitem{OWL}
		\begin{scnindent}
			\scntext{примечание}{Онтология, реализованная на языке \textit{OWL} (Web Ontology Language), представляет собой множество декларативных утверждений о сущностях словаря предметной области (подробнее рассматривается в работе. \textit{OWL} предполагает концепцию "открытого мира"{}, в соответствии с которой применимость описаний предметной области, помещенных в конкретном физическом документе, не ограничивается лишь рамками этого документа --- содержание онтологии может быть использовано и дополнено другими документами, добавляющими новые факты о тех же сущностях или описывающими другую предметную область в терминах данной. "Открытость мира"{} достигается путем добавления URI каждому элементу онтологии, что позволяет воспринимать описанную на \textit{OWL} онтологию как часть всеобщего объединенного знания.}
		\end{scnindent}
		\scnitem{WebProtege}
		\begin{scnindent}
			\scntext{примечание}{\scnkeyword{WebProtege} представляет собой многопользовательский веб-интерфейс, позволяющий редактировать и хранить онтологии в формате \textit{OWL} в совместной среде. Данный проект позволяет не только создавать новые онтологии, но также загружать уже существующие онтологии, которые хранятся на сервере университета Стэнфорда. К преимуществу данного проекта можно отнести автоматическую проверку ошибок в процессе создания объектов онтологий. Данный проект является примером попытки решения проблемы накопления, систематизации и повторного использования уже существующих решений, однако, недостатком данного решения является обособленность разрабатываемых онтологий. Каждый разработанный компонент имеет свою иерархию понятий, подход к выделению классов и сущностей, которые зависят от разработчиков данных онтологий, так как в рамках данного подхода не существует универсальной модели представления знаний, а также формальной спецификации компонентов, представленных в виде онтологий. Следовательно, возникает проблема их семантической несовместимости, что, в свою очередь, приводит к невозможности повторного использования разработанных онтологий при проектировании баз знаний. Данный факт подтверждается наличием на сервере университета Стэнфорда многообразия различных онтологий на одни и те же темы.}
		\end{scnindent}
		\scnitem{Modelica}
		\begin{scnindent}
			\scntext{примечание}{На основе языка \scnkeyword{Modelica} разработано большое число свободно доступных библиотек компонентов, одной из которых является библиотека Modelica\_StateGraph2, включающая компоненты для моделирования дискретных событий, реактивных и гибридных систем с помощью иерархических диаграмм состояния. Основным недостатком систем на базе языка \textit{Modelica} является отсутствие совместимости компонентов и достаточной документации, а также узкая направленность разрабатываемых компонентов.}
		\end{scnindent}
	\scnitem{Microsoft Power Apps}
	\begin{scnindent}
		\scntext{примечание}{\scnkeyword{Microsoft Power Apps} --- это набор приложений, служб и соединителей, а также платформа данных, которая предоставляет среду разработки для эффективного создания пользовательских приложений для бизнеса. Платформа \textit{Microsoft Power Apps} имеет средства для создания библиотеки многократно используемых компонентов графического интерфейса, а также предварительно созданные модели распознавания текста (чтение визитных карточек или чеков) и средство обнаружение объектов, которые можно подключить к разрабатываемому приложению. Библиотека компонентов \textit{Microsoft Power Apps} представляет собой множество создаваемых пользователем компонентов, которые можно использовать в любых приложениях. Преимущество библиотеки в том, что компоненты могут настраивать свойства по умолчанию, которые можно гибко редактировать в любых приложениях, использующих компоненты. Недостаток в том, что отсутствует семантическая совместимость компонентов, спецификация компонентов, не решена проблема существования семантически эквивалентных компонентов, нету иерархии компонентов и средств поиска этих компонентов. Компоненты платформы \textit{Microsoft Power Apps} являются многократно используемыми только для однотипных приложений, которые создаются одним и тем же разработчиком.}
	\end{scnindent}
	\scnitem{Платформа IACPaaS}
	\begin{scnindent}
		\scntext{примечание}{\scnkeyword{Платформа IACPaaS} (Intelligent Applications, Control and Platform as a Service) --- облачная платформа для разработки, управления и удаленного использования интеллектуальных облачных сервисов. Она предназначена для обеспечения поддержки разработки, управления и удаленного использования прикладных и инструментальных мультиагентных облачных сервисов (прежде всего интеллектуальных) и их компонентов для различных предметных областей.}
		\begin{scnrelfromset}{предоставляет доступ}
			\scnfileitem{Прикладным пользователям (специалистам в различных предметных областях) --- к прикладным сервисам.}
			\scnfileitem{Разработчикам прикладных и инструментальных сервисов и их компонентов --- к инструментальным сервисам.}
			\scnfileitem{Управляющим интеллектуальными сервисами.}
			\scnfileitem{К сервисам управления.}
		\end{scnrelfromset}
		\begin{scnrelfromset}{поддерживает}
			\scnfileitem{Базовую технологию разработки прикладных и специализированных инструментальных (интеллектуальных) сервисов с использованием базовых инструментальных сервисов платформы, поддерживающих эту технологию.}
			\scnfileitem{Множество специализированных технологий разработки прикладных и специализированных инструментальных (интеллектуальных) сервисов, с использованием специализированных инструментальных сервисов платформы, поддерживающих эти технологии.}
		\end{scnrelfromset}
		\scntext{недостаток}{\textit{Платформа IACPaaS} не имеет средств для унифицированного представления компонентов интеллектуальных компьютерных систем и средств для их спецификации и автоматической интеграции компонентов.}
	\end{scnindent}
	\scntext{примечание}{На текущем состоянии развития информационных технологий \uline{не существует} комплексной библиотеки многократно используемых семантически совместимых компонентов компьютерных систем. Таким образом, предлагается комплексная библиотека многократно используемых семантически совместимых компонентов ostis-систем.}
	\end{scnrelfromset}
	
\end{scnsubstruct}   
        
        \scnheader{библиотека многократно используемых компонентов ostis-систем}
        \scntext{часто используемый sc-идентификатор}{библиотека компонентов ostis-систем}
        \scntext{часто используемый sc-идентификатор}{библиотека компонентов}
        \scnidtf{библиотека совместимых многократно используемых компонентов}
        \scnidtf{комплексная библиотека многократно используемых семантически совместимых компонентов ostis-систем}
        \scnidtf{библиотека многократно используемых и совместимых компонентов интеллектуальных компьютерных систем нового поколения}
        \scnidtf{библиотека многократно используемых и совместимых компонентов интеллектуальных компьютерных систем нового поколения}
        \scnidtf{библиотека типовых компонентов ostis-систем}
        \scnidtf{библиотека многократно используемых компонентов OSTIS}
        \scnidtf{библиотека повторно используемых компонентов OSTIS}
        \scnidtf{библиотека intelligent property компонентов ostis-систем}
        \begin{scnindent}
        	\scntext{сокращение}{библиотека ip-компонентов ostis-систем}
        \end{scnindent}
        \scntext{примечание}{библиотека многократно используемых компонентов ostis-систем позволяет использовать проектный опыт по разработке и модернизации ostis-систем различного назначения.}
       \scnhaselementrole{типичный пример}{\scnkeyword{Библиотека Метасистемы OSTIS}}
       \begin{scnindent}
       	\scnidtf{Распределенная библиотека типовых (многократно используемых) компонентов ostis-систем в составе Метасистемы OSTIS}
       	\scnidtf{Библиотека многократно используемых компонентов ostis-систем в составе \textit{Метасистемы OSTIS}}
       \end{scnindent}
       \scnhaselementrole{типичный пример}{\scnkeyword{Библиотека Экосистемы OSTIS}}
        \begin{scnindent}
        	\scntext{часто используемый sc-идентификатор}{Библиотека OSTIS}
        	\scnidtf{Библиотека многократно используемых и совместимых компонентов интеллектуальных компьютерных систем нового поколения}
        	\scnidtf{Библиотека типовых компонентов интеллектуальных компьютерных систем нового поколения}
        	\scnidtf{Распределенная библиотека типовых (многократно используемых) компонентов ostis-систем в составе Экосистемы OSTIS}
        	\scnidtf{Библиотека многократно используемых компонентов ostis-систем в составе \textit{Экосистемы OSTIS}}
            \scntext{примечание}{Все библиотеки в рамках \textit{Экосистемы OSTIS} объединяются в \textit{Библиотеку Экосистемы OSTIS}.}
        \scntext{примечание}{Постоянно расширяемая Библиотека Экосистемы OSTIS существенно сокращает сроки разработки новых интеллектуальных компьютерных систем.}
        \scntext{назначение}{Основное назначение Библиотеки Экосистемы OSTIS --- создание условий для эффективного, осмысленного и массового проектирования ostis-систем и их компонентов путём создания среды для накопления и совместного использования компонентов ostis-систем.}
        \begin{scnindent}
            \scntext{примечание}{Такие условия осуществляются путём неограниченного расширения постоянно эволюционируемых семантически совместимых ostis-систем и их компонентов, входящих в \textit{Экосистему OSTIS}.}
        \end{scnindent}
        \scntext{примечание}{Различные \textit{многократно используемые компоненты ostis-систем} объединяются в \textit{библиотеки многократно используемых компонентов ostis-систем}. Разработчики \uline{любой} \textit{ostis-системы} могут включить в ее состав библиотеку, которая позволит им накапливать и распространять результаты своей деятельности среди других участников \textit{Экосистемы OSTIS} в виде \scnkeyword{многократно используемых компонентов}. Решение о включении компонента в библиотеку принимается экспертным сообществом разработчиков, ответственным за качество этой библиотеки. Верификацию компонентов можно автоматизировать путем проверки наличия обязательной части их спецификации, а также тестированием корректности автоматической установки, интеграции и функционирования компонентов.}
    	\end{scnindent}
        \scntext{примечание}{В рамках \textit{Экосистемы OSTIS} существует множество библиотек многократно используемых компонентов ostis-систем, являющихся подсистемами соответствующих материнских ostis-систем. Главной библиотекой многократно используемых компонентов ostis-систем является \textit{Библиотека Метасистемы OSTIS}. \textit{Метасистема OSTIS} выступает \scnkeyword{материнской системой} для всех разрабатываемых ostis-систем, поскольку содержит все базовые компоненты.}
        \begin{scnindent}
            \scnrelfrom{описание примера}{\scnfileimage[20em]{Contents/part_methods_tools/src/images/sd_ostis_library/ecosystem_architecture.png}}
            \begin{scnindent}
                \scnrelfrom{смотрите}{менеджер многократно используемых компонентов ostis-систем}
            \end{scnindent}
        \end{scnindent}
        \begin{scnrelfromset}{функциональные возможности}
            \scnfileitem{Хранение многократно используемых компонентов ostis-систем и их спецификаций.}
            \begin{scnindent}
                \scntext{примечание}{При этом часть компонентов, специфицированных в рамках библиотеки, могут физически храниться в другом месте ввиду особенностей их  технической реализации (например, исходные тексты платформы интерпретации sc-моделей компьютерных систем могут физически храниться в каком-либо отдельном репозитории, но специфицированы как компонент будут в соответствующей библиотеке). В этом случае спецификация компонента в рамках библиотеки должна также включать описание (1) того где располагается компонент и (2) сценария его автоматической или хотя бы ручной установки в дочернюю ostis-систему.}
                \begin{scnindent}
                    \scnrelfrom{смотрите}{менеджер многократно используемых компонентов ostis-систем}
                \end{scnindent}
            \end{scnindent}
            \scnfileitem{Просмотр имеющихся компонентов и их спецификаций, а также поиск компонентов по фрагментам их спецификации.}
            \scnfileitem{Хранение сведений о том, в каких ostis-системах-потребителях какие из компонентов библиотеки и какой версии используются (были скачаны). Это необходимо как минимум для учета востребованности того или иного компонента, оценки его важности и популярности.}
            \scnfileitem{Систематизация многократно используемых компонентов ostis-систем.}
            \scnfileitem{Обеспечение версионирования многократно используемых компонентов ostis-систем.}
            \scnfileitem{Поиск зависимостей между многократно используемыми компонентами в рамках библиотеки компонентов.}
            \scnfileitem{Обеспечение автоматического обновления компонентов, заимствованных в дочерние ostis-системы. Данная функция может включаться и отключаться по желанию разработчиков дочерней ostis-системы.}
            \begin{scnindent}
                \scntext{примечание}{Одновременное обновление одних и тех же компонентов во всех системах, его использующих, не должно ни в каком контексте приводить к несогласованности между этими системами. Это требование может оказаться довольно сложным, но без него работа Экосистемы невозможна.}
            \end{scnindent}
        \end{scnrelfromset}
        \scntext{примечание}{библиотека многократно используемых компонентов ostis-систем позволяет избавиться от дублирования семантически эквивалентных информационных компонентов. А также от многообразия форм технической реализации используемых моделей решения задач.}
        \scntext{примечание}{Проблема интеграции многократно используемых компонентов ostis-систем решается путем взаимодействия компонентов через общую базу знаний. Интеграция многократно используемых компонентов ostis-систем сводится к отождествлению (склеиванию) ключевых узлов по различным признакам и устранению возможных дублирований и противоречий исходя из спецификации компонента и его содержания. Такой способ интеграции компонентов позволяет разрабатывать их параллельно и независимо друг от друга, что значительно сокращает сроки проектирования. Отождествление sc-элементов происходит в ходе выполнения \scnkeyword{действие. отождествить два указанных sc-элемента}. Автоматическая интеграция компонентов интеллектуальных систем представляет широкие возможности для существенного сокращения сроков проектирования интеллектуальных систем, поскольку позволяет использовать опыт прошлых разработок. Интеграция любых компонентов ostis-систем происходит автоматически, без вмешательства разработчика. Это достигается за счет использования SC-кода и его преимуществ, унификации спецификации многократно используемых компонентов и тщательного отбора компонентов в библиотеках экспертным сообществом, ответственным	за эту библиотеку.}
        \begin{scnindent}
            \scntext{примечание}{Это достигается за счёт использования \textit{SC-кода} и его преимуществ, унификации спецификации многократно используемых компонентов и тщательного отбора компонентов в библиотеках экспертным сообществом, ответственным за эту библиотеку.}
        \end{scnindent}
        \scnheader{ostis-система}
        \scnsuperset{материнская ostis-система}
        \begin{scnindent}
            \scntext{пояснение}{ostis-система, имеющая в своем составе библиотеку многократно используемых компонентов.}
            \scnhaselement{Метасистема OSTIS}
            \scntext{примечание}{материнская ostis-система в свою очередь может являться дочерней ostis-системой для какой-либо другой ostis-системы, заимствуя компоненты из библиотеки, входящей в состав этой другой ostis-системы.}
            \scntext{примечание}{материнская ostis-система отвечает за какой-то класс компонентов и является САПРом для этого класса, например, содержит методики разработки таких компонентов, их классификацию, подробные пояснения ко всем подклассам компонентов. Таким образом, формируется иерархия \scnkeyword{материнских ostis-систем}.}
        \end{scnindent}
        \scnsuperset{дочерняя ostis-система}
        \begin{scnindent}
            \scntext{пояснение}{ostis-система, в составе которой имеется компонент, заимствованный из какой-либо библиотеки многократно используемых компонентов.}
        \end{scnindent}
        \scnheader{библиотека многократно используемых компонентов ostis-систем}
        \begin{scnreltoset}{объединение}
        	\scnitem{библиотека многократно используемых компонентов баз знаний ostis-систем}
        	\scnitem{библиотека многократно используемых компонентов решателей задач ostis-систем}
        	\scnitem{библиотека многократно используемых компонентов интерфейсов ostis-систем}
        	\scnitem{библиотека встраиваемых ostis-систем}
        	\scnitem{библиотека ostis-платформ}
        \end{scnreltoset}
        \scntext{примечание}{библиотека многократно используемых компонентов ostis-систем является подсистемой ostis-систем, которая имеет свою базу знаний, свой решатель задач и свой интерфейс. Однако не каждая ostis-система обязана иметь библиотеку компонентов.}
        \begin{scnrelfromset}{обобщенная декомпозиция}
            \scnitem{база знаний библиотеки многократно используемых компонентов ostis-систем}
            \begin{scnindent}
                \scntext{примечание}{база знаний библиотеки многократно используемых компонентов ostis-систем представляет собой иерархию многократно используемых компонентов ostis-систем и их спецификаций.}
            \end{scnindent}
            \scnitem{решатель задач библиотеки многократно используемых компонентов ostis-систем}
            \begin{scnindent}
                \scntext{примечание}{решатель задач библиотеки многократно используемых компонентов ostis-систем реализует функциональные возможности библиотеки ostis-систем.}
            \end{scnindent}
            \scnitem{интерфейс библиотеки многократно используемых компонентов ostis-систем}
            \begin{scnindent}
                \scntext{примечание}{Интерфейс обеспечивает доступ к многократно используемым компонентам и возможностям библиотеки ostis-систем для пользователя и других систем.}
                \begin{scnrelfromset}{декомпозиция}
                    \scnitem{минимальный интерфейс библиотеки многократно используемых компонентов ostis-систем}
                    \begin{scnindent}
                        \scntext{примечание}{Данный вид интерфейса позволяет \textit{менеджеру многократно используемых компонентов ostis-систем}, входящему в состав какой-либо дочерней ostis-системы, подключиться к библиотеке многократно используемых компонентов ostis-систем и использовать ее функциональные возможности, то есть, например, получить доступ к спецификации компонентов и установить выбранные компоненты в дочернюю ostis-систему, получить сведения до доступных версиях компонента, его зависимостях и так далее.}
                    \end{scnindent}
                    \scnitem{расширенный интерфейс библиотеки многократно используемых компонентов ostis-систем}
                    \begin{scnindent}
                        \scnidtf{графический интерфейс библиотеки многократно используемых компонентов ostis-систем}
                        \scntext{примечание}{В частном случае у библиотеки может быть расширенный пользовательский интерфейс, который, в отличие от минимального интерфейса, позволяет не только получить доступ к компонентам для дальнейшей работы с ними, но и просматривать существующую структуру библиотеки, а также компоненты и их элементы в удобном и интуитивно понятном для пользователя виде.}
                    \end{scnindent}
                \end{scnrelfromset}
            \end{scnindent}
        \end{scnrelfromset}

        \scnheader{многократно используемый компонент ostis-систем}
        \scnidtf{типовой компонент ostis-систем}
        \scnidtf{повторно используемый компонент ostis-систем}
        \scnidtf{многократно используемый компонент OSTIS}
        \scnidtf{ip-компонент ostis-систем}
        \scnidtftext{часто используемый sc-идентификатор}{многократно используемый компонент}
        \scnrelfrom{аббревиатура}{\scnfilelong{МИК ostis-систем}}
        \scnsubset{sc-структура}
        \scnsubset{компонент ostis-системы}
        \begin{scnindent}
            \scntext{пояснение}{Целостная часть ostis-системы, которая содержит все те (и только те) sc-элементы, которые необходимы для её функционирования в ostis-системе.}
        \end{scnindent}
        \scntext{определение}{многократно используемый компонент ostis-систем --- компонент некоторой ostis-системы, который может быть использован в рамках другой ostis-системы.}
        \scntext{пояснение}{Компонент ostis-системы, который может быть использован в других ostis-системах (\scnkeyword{дочерних ostis-системах}).}
        \scntext{пояснение}{Компонент некоторой \scnkeyword{материнской ostis-системы}, который может быть использован в некоторой \scnkeyword{дочерней ostis-системе}.}
        \scntext{пояснение}{многократно используемый компонент ostis-систем — это общий компонент (общая часть) для некоторого множества ostis-систем, который многократно используется, дублируется и входит в состав некоторого множества ostis-систем.}
        \scntext{примечание}{Для включения многократно используемого компонента в некоторую систему, его необходимо установить в эту систему, то есть скопировать в нее все sc-элементы компонента и, при необходимости, вспомогательные файлы, такие как исходные или скомпилированные файлы компонента.}
        \scntext{примечание}{многократно используемый компонент ostis-систем должен иметь унифицированную спецификацию и иерархию для поддержки \uline{совместимости} с другими компонентами.}
        \scntext{примечание}{Совместимость многократно используемых компонентов приводит систему к новому качеству, к дополнительному расширению множества решаемых задач при интеграции компонентов.}
        \begin{scnrelfromset}{необходимые требования}
            \scnfileitem{Существует техническая возможность встроить многократно используемый компонент в \scnkeyword{дочернюю ostis-систему}.}
            \scnfileitem{Полнота многократно используемого компонента: компонент должен в полной мере выполнять свои функции, соответствовать своему назначению.}
            \scnfileitem{Связность многократно используемого компонента: компонент должен логически выполнять только одну задачу из предметной области, для которой он предназначен. Многократно используемый компонент должен выполнять свои функции наиболее общим образом, чтобы круг возможных систем, в которые он может быть встроен, был наиболее широким.}
            \scnfileitem{Совместимость многократно используемого компонента: компонент должен стремиться повышать уровень \uline{договороспособности} ostis-систем, в которые он встроен и иметь возможность \uline{автоматической} интеграции в другие системы.}
            \scnfileitem{Самодостаточность компонентов (или групп компонентов) технологии, то есть способности их функционировать отдельно от других компонентов без утраты целесообразности их использования.}
        \end{scnrelfromset}
        \scnheader{следует отличать*}
        \begin{scnhaselementset}
            \scnitem{многократно используемый компонент ostis-систем}
            \scnitem{компонент ostis-системы}
        \end{scnhaselementset}
        \scntext{отличие}{многократно используемый компонент ostis-систем имеет \uline{спецификацию, достаточную для установки} этого компонента в \scnkeyword{дочернюю ostis-систему}. Спецификация является частью базы знаний \scnkeyword{библиотеки многократно используемых компонентов} соответствующей \scnkeyword{материнской ostis-системы}. Есть техническая возможность встроить многократно используемый компонент в дочернюю ostis-систему.}
        \scnheader{параметр, заданный на многократно используемых компонентах ostis-систем\scnsupergroupsign}
        \scnsubset{параметр}
        \scnhaselement{класс многократно используемого компонента ostis-систем\scnsupergroupsign}
        \begin{scnindent}
            \scntext{примечание}{класс многократно используемого компонента ostis-систем является важной частью спецификации компонента, позволяющей лучше понять назначение и область применения данного компонента, а также класс многократно используемого компонента является важнейшим критерием поиска компонентов в библиотеке ostis-систем.}
        \end{scnindent}
        \scnhaselement{начало\scnsupergroupsign}
        \scnhaselement{завершение\scnsupergroupsign}
        \scnheader{многократно используемый компонент ostis-систем}
        \scntext{примечание}{Интеллектуальная система, спроектированная по \textit{Технологии OSTIS}, представляет собой интеграцию \textit{многократно используемых компонентов баз знаний}, \textit{многократно используемых компонентов решателей задач} и \textit{многократно используемых компонентов интерфейсов}.}
        \scntext{примечание}{Основной признак классификации многократно используемых компонентов является признак предметной области, к которой относится компонент. Здесь структура может быть довольно богатой в соответствии с иерархией областей человеческой деятельности. Существует также множество предметно-независимых многократно используемых компонентов, которые могут использоваться в любой	предметной области.}
        \begin{scnrelfromset}{разбиение}
            \scnitem{многократно используемый компонент базы знаний ostis-систем}
            \begin{scnindent}
            	\scnidtf{многократно используемый компонент базы знаний}
                \scniselement{класс многократно используемого компонента ostis-систем\scnsupergroupsign}
                \scntext{примечание}{Важнейшим признаком классификации многократно используемых компонентов баз знаний является вид знаний.}
                \begin{scnindent}
                    \scnrelfrom{смотрите}{вид знаний}
                \end{scnindent}
                \scnsuperset{семантическая окрестность}
                \begin{scnindent}
                	\scnhaselement{Семантическая окрестность города Минска}
                	\scnhaselement{Семантическая окрестность понятия множество}
                \end{scnindent}
                \scnsuperset{предметная область и онтология}
                \begin{scnindent}
                	\scnhaselement{Предметная область и онтология треугольников}
                \end{scnindent}
                \scnsuperset{база знаний}
                \scnsuperset{шаблон типового компонента ostis-систем}
                \begin{scnindent}
                    \scnhaselement{Шаблон описания предметной области}
                    \scnhaselement{Шаблон описания отношения}
                \end{scnindent}
            \end{scnindent}
            \scnitem{многократно используемый компонент решателя задач ostis-систем}
            \begin{scnindent}
            	\scnidtf{многократно используемый компонент решателя задач}
                \scniselement{класс многократно используемого компонента ostis-систем\scnsupergroupsign}
                \scntext{примечание}{Важнейшим признаком классификации многократно используемых компонентов баз решателя задач является используемая модель решения задачи.}
                \scnsuperset{атомарный абстрактный sc-агент}
                \begin{scnindent}
                	\scnhaselement{Абстрактный sc-агент подсчета мощности множества}
                \end{scnindent}
                \scnsuperset{программа обработки знаний}
                \scnsuperset{scp-машина}
                \scnsuperset{scl-машина}
            \end{scnindent}
            \scnitem{многократно используемый компонент интерфейса ostis-систем}
            \begin{scnindent}
            	\scnidtf{многократно используемый компонент интерфейса}
                \scniselement{класс многократно используемого компонента ostis-систем\scnsupergroupsign}
                \scntext{примечание}{Важнейшим признаком классификации многократно используемых компонентов баз решателя задач является вид интерфейса в соответствии с классификацией интерфейсов.}
                \scnsuperset{многократно используемый компонент пользовательских интерфейсов ostis-систем}
            \end{scnindent}
        \end{scnrelfromset}
        \scnrelfrom{разбиение}{\scnkeyword{Типология компонентов ostis-систем по атомарности\scnsupergroupsign}}
        \begin{scnindent}
            \scnsubset{класс многократно используемого компонента ostis-систем\scnsupergroupsign}
            \begin{scneqtoset}
                \scnitem{атомарный многократно используемый компонент ostis-систем}
                \begin{scnindent}
                	\scnhaselement{Абстрактный sc-агент подсчета мощности множества}
                    \scntext{пояснение}{Многократно используемый компонент, который в текущем состоянии библиотеки ostis-систем рассматривается как неделимый, то есть не содержит в своем составе других компонентов.}
                    \scntext{примечание}{Принадлежность МИК ostis-систем классу атомарных компонентов зависит от того, как специфицирован этот компонент и от существующих на данный момент компонентов в библиотеке.}
                    \begin{scnindent}
                        \scntext{примечание}{В библиотеку ostis-систем нельзя опубликовать многократно используемый компонент как атомарный, в составе которого есть какой-либо другой известный библиотеке ostis-систем компонент.}
                    \end{scnindent}
                    \scntext{примечание}{В общем случае атомарный компонент может перейти в разряд неатомарных в случае, если будет принято решение выделить какую-то из его частей в качестве отдельного компонента. Все вышесказанное, однако, не означает, что даже в случае его платформенной независимости, атомарный компонент всегда хранится в sc-памяти как сформированная sc-структура. Например, платформенно-независимая реализация sc-агента всегда будет представлена набором \textit{scp-программ}, но будет \textit{атомарным многократно используемым компонентом OSTIS} в случае, если ни одна из этих программ не будет представлять интереса как самостоятельный компонент.}
                \end{scnindent}
                \scnitem{неатомарный многократно используемый компонент ostis-систем}
                \begin{scnindent}
                	\scnidtf{составной многократно используемый компонент ostis-систем}
                	\scnhaselement{Решатель задач по геометрии}
                    \scntext{пояснение}{Многократно используемый компонент, который в текущем состоянии библиотеки ostis-систем содержит в своем составе другие атомарные или неатомарные компоненты.}
                    \scntext{примечание}{Неатомарный многократно используемый компонент не зависит от своих частей. Без какой-либо части неатомарного компонента его назначение сужается.}
                    \scntext{примечание}{В общем случае неатомарный компонент может перейти в разряд атомарных в случае, если будет принято решение по каким-либо причинам исключить все его части из рассмотрения в качестве отдельных компонентов. Следует отметить, что неатомарный компонент необязательно должен складываться \uline{полностью} из других компонентов, в его состав могут также входить и части, не являющиеся самостоятельными компонентами. Например, в состав реализованного на \textit{Языке SCP} \textit{sc-агента}, являющего \textit{неатомарным многократно используемым компонентом} могут входить как \textit{scp-программы}, которые могут являться многократно используемыми компонентами (а могут и не являться), а также агентная \textit{scp-программа}, которая не имеет смысла как многократно используемый компонент.}
                    \scntext{примечание}{Спецификация неатомарного многократно используемого компонента должна содержать информацию о том, из каких компонентов он состоит, используя отношение декомпозиция*. При этом sc-структура, обозначающая неатомарный компонент не обязана содержать все sc-элементы компонентов, на которые она декомпозируется, достаточно, чтобы неатомарному компоненту принадлежали знаки всех тех компонентов, из которых он состоит должно быть полное перечисление составных компонентов).}
                \end{scnindent}
            \end{scneqtoset}
        \end{scnindent}
        \scnrelfrom{разбиение}{\scnkeyword{Типология компонентов ostis-систем по зависимости\scnsupergroupsign}}
        \begin{scnindent}
            \scnsubset{класс многократно используемого компонента ostis-систем\scnsupergroupsign}
            \begin{scneqtoset}
                \scnitem{зависимый многократно используемый компонент ostis-систем}
                \begin{scnindent}
                	\scnhaselement{визуальный редактор системы по химии}
                    \scntext{пояснение}{Многократно используемый компонент, который зависит хотя бы от одного другого компонента библиотеки ostis-систем, то есть не может быть встроен в дочернюю ostis-систему без компонентов, от которых он зависит.}
                \end{scnindent}
                \scnitem{независимый многократно используемый компонент ostis-систем}
                \begin{scnindent}
                	\scnhaselement{Предметная область множеств}
                    \scntext{пояснение}{Многократно используемый компонент, который не зависит ни от одного другого компонента библиотеки ostis-систем.}
                \end{scnindent}
            \end{scneqtoset}
        \end{scnindent}
        \scnrelfrom{разбиение}{\scnkeyword{Типология компонентов ostis-систем по способу их хранения\scnsupergroupsign}}
        \begin{scnindent}
            \scnsubset{класс многократно используемого компонента ostis-систем\scnsupergroupsign}
            \begin{scneqtoset}
                \scnitem{многократно используемый компонент ostis-систем, хранящийся в виде внешних файлов}
                \begin{scnindent}
                    \begin{scnrelfromset}{разбиение}
                        \scnitem{многократно используемый компонент ostis-систем, хранящийся в виде файлов исходных текстов}
                        \scnitem{многократно используемый компонент ostis-систем, хранящийся в виде скомпилированных файлов}
                    \end{scnrelfromset}
                \end{scnindent}
                \scnitem{многократно используемый компонент, хранящийся в виде sc-структуры}
            \end{scneqtoset}
            \scntext{примечание}{На данном этапе развития \textit{Технологии OSTIS} более удобным является хранение компонентов в виде исходных текстов.}
        \end{scnindent}
        \scnrelfrom{разбиение}{\scnkeyword{Типология компонентов ostis-систем по зависимости от платформы\scnsupergroupsign}}
        \begin{scnindent}
            \scnsubset{класс многократно используемого компонента ostis-систем\scnsupergroupsign}
            \begin{scneqtoset}
                \scnitem{платформенно зависимый многократно используемый компонент ostis-систем}
                \begin{scnindent}
                    \scntext{пояснение}{Под платформенно-зависимым многократно используемым компонентом OSTIS понимается компонент, частично или полностью реализованный при помощи каких-либо сторонних с точки зрения \textit{Технологии OSTIS} средств.}
                    \scntext{недостаток}{Интеграция таких компонентов в интеллектуальные системы может сопровождаться дополнительными трудностями, зависящими от конкретных средств реализации компонента.}
                    \scntext{преимущество}{В качестве возможного преимущества платформенно-зависимых многократно используемых компонентов ostis-систем можно выделить их, как правило, более высокую производительность за счет реализации их на более приближенном к платформе уровне.}
                    \scntext{примечание}{С точки зрения \textit{Технологии OSTIS} любая платформа интерпретации sc-моделей компьютерных систем является платформенно-зависимым многократно используемым компонентом.}
                    \scntext{примечание}{В общем случае платформенно-зависимый многократно используемый компонент ostis-систем может поставляться как в виде набора исходных кодов, так и в скомпилированном виде. Процесс интеграции платформенно-зависимого многократно используемого компонента ostis-систем в дочернюю систему, разработанную по \textit{Технологии OSTIS}, сильно зависит от технологий реализации данного компонента и в каждом конкретном случае может состоять из различных этапов. Каждый платформенно-зависимый многократно используемый компонент ostis-систем должен иметь соответствующую подробную, корректную и понятную инструкцию по его установке и внедрению в дочернюю систему.}
                    \scnsuperset{ostis-платформа}
                    \scnsuperset{абстрактный sc-агент, не реализуемый на Языке SCP}
                \end{scnindent}
                \scnitem{платформенно-независимый многократно используемый компонент ostis-систем}
                \begin{scnindent}
                    \scntext{пояснение}{Под платформенно-независимым многократно используемым компонентом ostis-систем понимается компонент, который целиком и полностью представлен на \textit{SC-коде}.}
                    \scnsuperset{многократно используемый компонент базы знаний}
                    \scnsuperset{платформенно-независимый scp-агент}
                    \scnsuperset{scp-программа}
                    \scntext{примечание}{В случае \textit{неатомарного многократно используемого компонента} это означает, что \uline{все} более простые компоненты, входящие в его состав также обязаны быть платформенно-независимыми многократно используемыми компонентами ostis-систем.}
                    \scntext{примечание}{Процесс интеграции платформенно-зависимого многократно используемого компонента ostis-систем в дочернюю систему, разработанную по Технологии OSTIS, существенно упрощается за счет использования общей унифицированной формальной основы представления и обработки знаний.}
                    \scntext{примечание}{Наиболее ценными являются платформенно-независимые многократно используемые компоненты ostis-систем.}
                \end{scnindent}
            \end{scneqtoset}
        \end{scnindent}
        \scnrelfrom{разбиение}{\scnkeyword{Типология компонентов ostis-систем по динамичности их установки\scnsupergroupsign}}
        \begin{scnindent}
            \scnsubset{класс многократно используемого компонента ostis-систем\scnsupergroupsign}
            \begin{scneqtoset}
                \scnitem{динамически устанавливаемый многократно используемый компонент ostis-систем}
                \begin{scnindent}
                    \scnidtf{многократно используемый компонент, при установке которого система не требует перезапуска}
                    \begin{scnrelfromset}{декомпозиция}
                        \scnitem{многократно используемый компонент, хранящийся в виде скомпилированных файлов}
                        \scnitem{многократно используемый компонент базы знаний}
                    \end{scnrelfromset}
                \end{scnindent}
                \scnitem{многократно используемый компонент, при установке которого система требует перезапуска}
            \end{scneqtoset}
            \scntext{примечание}{Процесс интеграции компонентов разных видов на разных этапах жизненного цикла osits-систем бывает разным. Наиболее ценными являются компоненты, которые могут быть интегрированы в работающую систему \uline{без} прекращения её функционирования. Некоторые системы, особенно системы управления, нельзя останавливать, а устанавливать и обновлять компоненты нужно.}
        \end{scnindent}
        \scnsuperset{встраиваемая ostis-систем}
        \begin{scnindent}
        	\scnidtf{типовая подсистема ostis-систем}
        	\scnsubset{ostis-система}
        	\scnsubset{неатомарный многократно используемый компонент ostis-систем}
        	\scntext{пояснение}{\scnkeyword{встраиваемая ostis-система} --- это \textit{неатомарный многократно используемый компонент}, который состоит из \textit{базы знаний}, \textit{решателя задач} и \textit{интерфейса}.}
        	\begin{scnrelfromset}{декомпозиция}
        		\scnitem{многократно используемый компонент баз знаний ostis-систем}
        		\scnitem{многократно используемый компонент решателей задач ostis-систем}
        		\scnitem{многократно используемый компонент интерфейсов ostis-систем}
        	\end{scnrelfromset}
            \scniselement{класс многократно используемого компонента ostis-систем\scnsupergroupsign}
            \scnhaselement{Среда коллективной разработки баз знаний ostis-систем}
            \scnhaselement{Визуальный web-ориентированный редактор sc.g-текстов}
            \scnhaselement{Естественно-языковой интерфейс ostis-системы}
            \scnsuperset{менеджер многократно используемых компонентов ostis-систем}
            \scnsuperset{интеллектуальная обучающая ostis-система}
            \scnsuperset{система тестирования и верификации ostis-систем}
            \scntext{примечание}{Особенность \textit{встраиваемых ostis-систем} в том, что интеграция целых интеллектуальных систем предполагает интеграцию баз знаний этих систем, интеграцию их решателей задач и интеграцию их интеллектуальных интерфейсов. При интеграции встраиваемых ostis-систем база знаний встраиваемой системы становится частью базы знаний системы, в которую она встраивается. Решатель задач встраиваемой ostis-системы становится частью решателя задач системы, в которую она встраивается. И интерфейс встраиваемой ostis-системы становится частью интерфейса системы, в которую она встраивается. При этом встраиваемая система является целостной и может функционировать отдельно от других ostis-систем, в отличие от других многократно используемых компонентов.}
            \scntext{примечание}{\textit{встраиваемые ostis-системы} зачастую являются предметно-независимыми многократно используемыми компонентами. Таким образом, например, встраиваемая ostis-система в виде среды проектирования баз знаний может быть встроена как в систему из предметной области по геометрии, так и в систему управления аграрными объектами.}
            \scntext{примечание}{\textit{встраиваемая ostis-система}, как и любой другой многократно используемый компонент ostis-систем, должна поддерживать семантическую совместимость ostis-систем. Как сама встраиваемая ostis-система, так и все ее компоненты должны быть специфицированы и согласованы. Компоненты встраиваемых ostis-систем могут быть заменены на другие, имеющие то же назначение, например, естественно-языковой интерфейс может иметь различные варианты базы знаний в зависимости от естественного языка, поддерживаемого системой, различные варианты интерфейса, в зависимости от требований и удобства пользователей и также различные варианты реализации решателя задач для обработки естественного языка, которые могут использовать различные модели, однако решать одну и ту же задачу. Встраиваемая ostis-система связывается с системой, в которую она встроена с помощью отношения \textbf{\textit{встроенная ostis-система*}}, которое является подмножеством отношения \textit{встроенная кибернетическая система*}.}
        \end{scnindent}
        \scnheader{хранилище многократно используемых компонентов ostis-систем, хранящихся в виде внешних файлов}
        \scntext{примечание}{Для того, чтобы хранить \textit{многократно используемые компоненты ostis-систем}, необходимо некоторое хранилище. Таким хранилищем может выступать как какая-либо ostis-система, так и стороннее хранилище, например, облачное. Помимо исходных файлов компонента в хранилище должна находиться его \uline{спецификация}.}
        \scnsuperset{хранилище многократно используемого компонента ostis-систем, хранящегося в виде файлов исходных текстов}
        \begin{scnindent}
            \scntext{пояснение}{Место хранения файлов исходных текстов многократно используемого компонента.}
            \scnsuperset{хранилище на основе системы контроля версий Git}
            \begin{scnindent}
                \scnsuperset{репозиторий GitHub}
                \scntext{примечание}{На данном этапе в рамках \textit{Технологии OSTIS} (в силу открытости технологии, а также хранения компонентов в виде файлов исходных текстов) для хранения компонентов чаще всего используются хранилища на основе системы контроля версий Git.}
            \end{scnindent}
        \end{scnindent}
        \scnsuperset{хранилище многократно используемого компонента ostis-систем, хранящегося в виде скомпилированных файлов}
        \begin{scnindent}
            \scntext{пояснение}{Место хранения скомпилированных файлов многократно используемого компонента.}
        \end{scnindent}
        \scntext{примечание}{Помимо внешних файлов компонента в хранилище должна находиться его \uline{спецификация}.}
        
        \scnheader{спецификация многократно используемого компонента ostis-систем}
        \scnsubset{спецификация}
        \scnidtf{описание многократно используемого компонента ostis-систем}
        \scnrelfrom{ключевой sc-элемент}{многократно используемый компонент ostis-систем}
        \scntext{примечание}{Каждый \textit{многократно используемый компонент ostis-систем} должен быть специфицирован в рамках библиотеки. Данные спецификации включают в себя основные знания о компоненте, которые позволяют обеспечить построение полной иерархии компонентов и их зависимостей, а также обеспечивают \uline{беспрепятственную} интеграцию компонентов в \scnkeyword{дочерние ostis-системы}. Для спецификации компонента используются как отношения, так и классы компонента.}
        \begin{scnindent}
        	\scntext{примечание}{Указание класса \scnkeyword{многократно используемый компонент ostis-систем} является обязательным.}
        \end{scnindent}
        \scntext{примечание}{Сам многократно используемый компонент в рамках спецификации является \textit{ключевым sc-элементом}, а также может иметь множество своих ключевых sc-элементов.}
        \scnrelfrom{параметры, специфицирующие многократно используемый компонент ostis-систем}{параметр, заданный на многократно используемых компонентах ostis-систем\scnsupergroupsign}
        \scnrelfrom{классы отношений, специфицирующие многократно используемый компонент ostis-систем}{отношение, специфицирующее многократно используемый компонент ostis-систем\scnsupergroupsign}
        \scnrelfrom{описание примера}{\scnfileimage[20em]{Contents/part_methods_tools/src/images/sd_ostis_library/component_specification_example.png}}
        \scnheader{отношение, специфицирующее многократно используемый компонент ostis-систем\scnsupergroupsign}
        \scnidtf{отношение, которое используется при спецификации многократно используемого компонента ostis-систем}
        \begin{scnrelfromset}{разбиение}
            \scnitem{необходимое для установки отношение, специфицирующее многократно используемый компонент ostis-систем}
            \begin{scnindent}
                \scntext{примечание}{Чтобы многократно используемый компонент мог быть принят в библиотеку, нужно специфицировать его используя каждое отношение из множества \textit{необходимое для установки отношение, специфицирующее многократно используемый компонент ostis-систем}. Здесь описана спецификация, общая для любых типов компонентов, однако в зависимости от типа компонента, спецификация может расширяться}
                \scnhaselement{метод установки*}
                \scnhaselement{адрес хранилища*}
                \scnhaselement{зависимости компонента*}
            \end{scnindent}
            \scnitem{необязательное для установки отношение, специфицирующее многократно используемый компонент ostis-систем}
            \begin{scnindent}
                \scntext{примечание}{\textit{необязательное для установки отношение, специфицирующее многократно используемый компонент ostis-систем} помогает лучше понять суть компонента, упрощает поиск, но не является обязательным для того, чтобы компонент мог быть установлен в ostis-систему.}
                \scnhaselement{сопутствующие компоненты*}
                \scnhaselement{история изменений*}
                \scnhaselement{модификации компонентов*}
                \scnhaselement{авторы*}
                \scnhaselement{примечание*}
                \scnhaselement{пояснение*}
                \scnhaselement{идентификатор*}
                \scnhaselement{ключевой sc-элемент*}
                \scnhaselement{назначение*}
                \scnhaselement{требования полноты*}
                \scnhaselement{требования безошибочности*}
                \scnhaselement{преимущества*}
                \scnhaselement{недостатки*}
            \end{scnindent}
        \end{scnrelfromset}
        \scnheader{метод установки*}
        \scniselement{бинарное отношение}
        \scniselement{ориентированное отношение}
        \scntext{пояснение}{Пользователь может установить компонент вручную, а \scnkeyword{менеджер компонентов} - автоматически.}
        \scnrelfrom{первый домен}{многократно используемый компонент ostis-систем}
        \scnrelfrom{второй домен}{метод установки многократно используемого компонента}
        \begin{scnindent}
            \scnsubset{метод}
            \scnsuperset{метод установки динамически устанавливаемого многократно используемого компонента ostis-систем}
            \begin{scnindent}
                \scntext{примечание}{При динамической установке необходимо только скачать компонент и, при необходимости, его зависимые компоненты, и он сразу же функционирует в системе.}
                \scnrelfrom{описание примера}{\scnfileimage[20em]{Contents/part_methods_tools/src/images/sd_ostis_library/install_dynamic_method.png}}
                \begin{scnindent}
                    \scniselement{sc.g-текст}
                \end{scnindent}
            \end{scnindent}
            \scnsuperset{метод установки многократно используемого компонента, при установке которого система требует перезапуска}
            \begin{scnindent}
                \scntext{примечание}{Установка таких компонентов происходит путём скачивания компонента и его трансляции в память системы.}
                \scnrelfrom{описание примера}{\scnfileimage[20em]{Contents/part_methods_tools/src/images/sd_ostis_library/install_with_reboot_method.png}}
                \begin{scnindent}
                    \scniselement{sc.g-текст}
                \end{scnindent}
            \end{scnindent}
        \end{scnindent}
        \scnheader{адрес хранилища*}
        \scniselement{бинарное отношение}
        \scniselement{ориентированное отношение}
        \scntext{пояснение}{Связки отношения \textit{адрес хранилища*} связывают многократно используемый компонент, хранящийся в виде внешних файлов и файл, содержащий url-адрес многократно используемого компонента ostis-систем.}
        \scnrelfrom{первый домен}{многократно используемый компонент ostis-систем, хранящийся в виде внешних файлов}
        \scnrelfrom{второй домен}{файл, содержащий url-адрес многократно используемого компонента ostis-систем}
        \begin{scnindent}
            \scnsuperset{файл}
            \scnsubset{файл, содержащий url-адрес на GitHub многократно используемого компонента ostis-систем}
            \scnsubset{файл, содержащий url-адрес на Google Drive многократно используемого компонента ostis-систем}
            \scnsubset{файл, содержащий url-адрес на Docker Hub многократно используемого компонента ostis-систем}
        \end{scnindent}
        \scnheader{зависимости компонента*}
        \scniselement{квазибинарное отношение}
        \scniselement{ориентированное отношение}
        \scntext{пояснение}{Связки отношения \textit{зависимости компонента*} связывают многократно используемый компонент, и множество компонентов, без которых устанавливаемый компонент \uline{не может быть} встроен в \scnkeyword{дочернюю ostis-систему}.}
        \scnrelfrom{первый домен}{многократно используемый компонент ostis-систем}
        \scnrelfrom{второй домен}{множество многократно используемых компонентов ostis-систем}
        \scnheader{сопутствующие компоненты*}
        \scniselement{квазибинарное отношение}
        \scniselement{ориентированное отношение}
        \scntext{пояснение}{В некоторых случаях может оказаться, что для использования одного многократно используемого компонента OSTIS целесообразно или даже необходимо дополнительно использовать несколько других \textit{многократно используемых компонентов OSTIS}. Например, может оказаться целесообразным вместе с каким либо sc-агентом информационного поиска использовать соответствующую команду интерфейса, которая представлена отдельным компонентом и позволит пользователю задавать вопрос для указанного sc-агента через интерфейс системы. В таких случаях для связи компонентов используется отношение \textit{сопутствующие компоненты*}. Наличие таких связей позволяет устранить возможные проблемы неполноты знаний и навыков в дочерней системе, из-за которых какие-либо из компонентов могут не выполнять свои функции. Связки отношения \textit{сопутствующий компонент*} связывают многократно используемые компоненты ostis-систем, которые целесообразно использовать в дочерней системе вместе. Каждая такая связка может дополнительно быть снабжена sc-комментарием или sc-пояснением, отражающим суть указываемой зависимости.}
        \scnrelfrom{первый домен}{многократно используемый компонент ostis-систем}
        \scnrelfrom{второй домен}{множество многократно используемых компонентов ostis-систем}
        \scnheader{история изменений*}
        \scniselement{бинарное отношение}
        \scniselement{ориентированное отношение}
        \scntext{пояснение}{Отношение \textit{история изменений*} позволяет специфицировать различные версии компонента и, при необходимости, устанавливать выбранную пользователем версию. Различные версии, как правило, отражают какие-либо улучшения или исправления ошибок.}
        \scnrelfrom{первый домен}{многократно используемый компонент ostis-систем}
        \scnrelfrom{второй домен}{история изменений}
        \scnheader{модификации компонентов*}
        \scniselement{бинарное отношение}
        \scniselement{ориентированное отношение}
        \scntext{пояснение}{\textit{модификации компонентов*} --- это функционально эквивалентные реализации одного и того же компонента, которые могут быть синтаксически эквивалентны (например, реализация одного и того же sc-агента на платформенно-зависимом и платформенно-независимом уровнях). Развитие \textit{Библиотеки Экосистемы OSTIS} происходит не только за счет ее пополнения новыми компонентами, но и за счет появления новых версий и модификаций уже существующих компонентов.}
        \scnrelfrom{первый домен}{многократно используемый компонент ostis-систем}
        \scnrelfrom{второй домен}{множество многократно используемых компонентов ostis-систем}
        \scnheader{авторы*}
        \scntext{пояснение}{Связки отношения \textit{авторы*} связывают многократно используемый компонент со множеством авторов этого компонента. Спецификация может также содержать дополнительную информацию об авторах при необходимости.}
        \scnheader{назначение*}
        \scntext{пояснение}{Отношение \textit{назначение*} позволяет описать ожидаемый сценарий, выделить рекомендации использования многократно используемого компонента. Требования полноты и безошибочности специфицируют возможные ограничения и ошибки компонента, область его использования.}
        
        \scnheader{параметр, заданный на многократно используемых компонентах ostis-систем\scnsupergroupsign}
        \scntext{примечание}{Для уточнения типа компонента могут использоваться другие классы, например дата публикации первой и последней версии компонента, качественно-количественные характеристики, такие как уровень семантической совместимости компонентов, сложность реализации компонента, уровень производительности компонента (для программ можно использовать O-нотацию), количество sc-элементов, входящих в состав многократно используемого компонента, количество ключевых узлов компонента, рейтинг компонента в рамках \textit{Экосистемы OSTIS}, количество скачиваний компонента и другие. Параметр \textit{лицензия многократно используемого компонента} используется для обозначения условий использования и распространения компонента. По умолчанию лицензия компонента открытая, если не указано иное.}
        
        \scnheader{менеджер многократно используемых компонентов ostis-систем}
        \scnidtftext{часто используемый sc-идентификатор}{менеджер многократно используемых компонентов}
        \scnidtftext{часто используемый sc-идентификатор}{менеджер компонентов}
        \scnsubset{платформенно-зависимый многократно используемый компонент ostis-систем}
        \scntext{пояснение}{менеджер многократно используемых компонентов ostis-систем --- подсистема ostis-системы, с помощью которой происходит взаимодействие с библиотекой компонентов ostis-систем.}
        \scnhaselement{Реализация менеджера многократно используемых компонентов ostis-систем}
        \begin{scnindent}
        	\scntext{адрес компонента}{https://github.com/ostis-ai/sc-component-manager}
        \end{scnindent}
        \begin{scnrelfromset}{обобщенная декомпозиция}
            \scnitem{база знаний менеджера многократно используемых компонентов ostis-систем}
            \begin{scnindent}
                \scntext{примечание}{база знаний менеджера компонентов содержит все те знания, которые необходимы для установки многократно используемого компонента в \scnkeyword{дочернюю ostis-систему}. К таким знаниям относятся знания о спецификации многократно используемых компонентов, методы установки компонентов,знание о  библиотеках ostis-систем, с которыми происходит взаимодействие. \textit{Классификация компонентов} и другие.}
            \end{scnindent}
            \scnitem{решатель задач менеджера многократно используемых компонентов ostis-систем}
            \begin{scnindent}
                \scntext{примечание}{решатель задач менеджера компонентов взаимодействует с библиотекой ostis-систем и позволяет установить и интегрировать многократно используемые компоненты в \scnkeyword{дочернюю ostis-систему}, также выполнять поиск, обновление, публикацию, удаление компонентов и другие операции с ними.}
                \begin{scnrelfromset}{декомпозиция абстрактного sc-агента}
                	\scnitem{Абстрактный sc-агент поиска многократно используемых компонентов ostis-систем}
                	\scnitem{Абстрактный sc-агент установки многократно используемых компонентов ostis-систем}
                	\scnitem{Абстрактный sc-агент управления отслеживаемых менеджером компонентов библиотек}
                	\begin{scnindent}
                		\begin{scnrelfromset}{декомпозиция абстрактного sc-агента}
                			\scnitem{Абстрактный sc-агент добавления отслеживаемой менеджером компонентов библиотеки}
                			\scnitem{Абстрактный sc-агент удаления отслеживаемой менеджером компонентов библиотеки}
                		\end{scnrelfromset}
                	\end{scnindent}
                \end{scnrelfromset}
            \end{scnindent}
            \scnitem{интерфейс менеджера многократно используемых компонентов ostis-систем}
            \begin{scnindent}
                \scntext{примечание}{интерфейс менеджера многократно используемых компонентов обеспечивает удобное для пользователя и других систем использование менеджера компонентов.}
            \end{scnindent}
        \end{scnrelfromset}
        \scnrelfrom{минимальные функциональные возможности}{Минимальные функциональные возможности менеджера компонентов}
        \begin{scnindent}
        	\scntext{примечание}{Используя минимальные функциональные возможности менеджер компонентов может установить компоненты, которые будут расширять его же функционал.}
        	\begin{scneqtoset}
        		\scnfileitem{\textbf{Поиск многократно используемых компонентов ostis-систем.} Множество возможных критериев поиска соответствует спецификации многократно используемых компонентов. Такими критериями могут быть классы компонента, его авторы, идентификатор, фрагмент примечания, назначение, принадлежность какой-либо предметной области, вид знаний компонента и другие.}
        		\scnfileitem{\textbf{Установка многократно используемого компонента ostis-систем.} Установка многократно используемого компонента происходит вне зависимости от типологии, способа установки и местоположения компонента. Необходимое условие для возможности установки многократно используемого компонента --- наличие \textbf{\textit{спецификации многократно используемого компонента ostis-систем}}. Перед установкой многократно используемого компонента в дочернюю систему необходимо установить все зависимые компоненты. Также для платформенно-зависимых компонентов может быть необходимо установить иные зависимости, которые не являются компонентами какой-либо библиотеки ostis-систем. После успешной установки компонента в базе знаний дочерней системы генерируется информационная конструкция, обозначающая факт установки компонента в систему с помощью отношения \textit{установленные компоненты*}.}
        		\scnfileitem{\textbf{Добавление и удаление отслеживаемых менеджером компонентов библиотек.} Менеджер компонентов содержит информацию о множестве источников для установки компонентов, перечень которых можно дополнять. По умолчанию менеджер компонентов отслеживает \textit{Библиотеку Метасистемы OSTIS}, однако можно создавать и добавлять дополнительные библиотеки ostis-систем.}
        	\end{scneqtoset}
        \end{scnindent}
        \scnrelfrom{расширенные функциональные возможности}{Расширенные функциональные возможности менеджера компонентов}
        \begin{scnindent}
        	\scntext{примечание}{Компоненты, реализующие расширенный функционал менеджера компонентов являются частью \textit{Библиотеки Метасистемы OSTIS}.}
        	\begin{scneqtoset}
        		\scnfileitem{\textbf{Спецификация} многократно используемого компонента ostis-систем. Менеджер компонентов позволяет специфицировать компоненты, входящие в состав библиотеки ostis-систем, а также специфицировать новые создаваемые компоненты, которые будут публиковаться в библиотеку ostis-систем. При этом спецификация может происходить как автоматически, так и вручную. Например, менеджер компонентов может обновить спецификацию используемого компонента в соответствии с тем, в какие новые ostis-системы его установили, обновить спецификацию авторства компонента при его редактировании в библиотеке ostis-систем, спецификацию ошибок, выявленных в процессе эксплуатации компонента и так далее.}
        		\scnfileitem{\textbf{Формирование} многократно используемого компонента ostis-систем по шаблону с заданными параметрами. При установке шаблона многократно используемого компонента ostis-систем менеджер компонентов позволяет сформировать по нему конкретный компонент. Для этого пользователю предлагается определить значения всех sc-переменных в шаблоне для формирования конкретного компонента из некоторой предметной области. Например, для формирования многократно используемого компонента баз знаний, представляющего собой семантическую окрестность некоторого отношения (см. рисунок \textit{\nameref{fig:relation_template}}), нужно определить значения всех переменных, кроме переменной, являющейся ключевым sc-элементом данной структуры.}
        		\scnfileitem{\textbf{Публикация} многократно используемого компонента ostis-систем в библиотеку ostis-систем. При публикации компонента в библиотеку ostis-систем происходит верификация на основе спецификации компонента. Публикация компонента может сопровождаться сборкой неатомарного компонента из существующих атомарных. Также существует возможность обновления версии опубликованного компонента сообществом его разработчиков.}
        		\scnfileitem{\textbf{Обновление} установленного многократно используемого компонента ostis-систем.}
        		\scnfileitem{\textbf{Удаление} установленного многократно используемого компонента. Как и в случае установки после удаления многократно используемого компонента из ostis-системы в базе знаний системы устанавливается факт удаления компонента. Эта информация является важной частью \uline{истории эксплуатации} ostis-системы.}
        		\scnfileitem{\textbf{Редактирование} многократно используемого компонента в библиотеке ostis-систем.}
        		\scnfileitem{\textbf{Сравнение} многократно используемых компонентов ostis-систем.}
        	\end{scneqtoset}
        \end{scnindent}
        \scntext{примечание}{Для того, чтобы создать новую ostis-систему "с нуля"{}, используя \textit{ostis-платформу}, необходимо установить некоторый \textit{Программный вариант реализации ostis-платформы} с помощью сторонних средств. Такими средствами могут быть (1) хранилища исходного кода платформы, например, облачные хранилища, такие как GitHub репозиторий, с соответствующим набором инструкций по установке платформы или (2) средства установки заранее скомпилированной программной реализации платформы, например, средство установки программного обеспечения apt. Далее установка многократно используемых компонентов в ostis-систему (независимо от типа компонентов) осуществляется с помощью менеджера компонентов. При установке платформенно-зависимых компонентов менеджер компонентов должен управлять соответствующими средствами сборки таких компонентов (CMake, Ninja, npm, grunt и другие).}
        \scntext{примечание}{Компонент находится в некотором хранилище --- (1) \textit{библиотеке компонентов ostis-систем} или (2) в виде файлов в некотором облачном хранилище. В случае, когда компонент хранится в библиотеке, для его установки менеджер компонентов копирует все sc-элементы, которые представляют собой компонент, в дочернюю ostis-систему. В случае, когда компонент хранится в виде файлов в облачном хранилище, менеджер компонентов скачивает файлы компонента и устанавливает их в соответствии со спецификацией. Адреса хранилищ спецификаций компонентов должны храниться в базе знаний менеджера компонентов, чтобы иметь доступ к спецификациям компонентов для их последующего использования (поиска, установки и так далее).}
        \scntext{примечание}{\textit{менеджер многократно используемых компонентов ostis-систем} является \uline{необязательной} подсистемой \textit{ostis-платформы}. Однако система, имеющая менеджер компонентов, может устанавливать компоненты не только сама в себя, но и в другие системы при наличии доступа. Таким образом, одна система может заменить \textit{ostis-платформу} другой системы, оставив при этом \textit{sc-модель кибернетической системы}. Таким же образом некоторая ostis-система может порождать другие ostis-системы, используя компонентный подход.}
        \scntext{примечание}{Включение компонента в \textit{дочернюю ostis-систему} в общем случае состоит из следующих этапов:
        	\begin{itemize}
        		\item поиск подходящего компонента во множестве доступных библиотек;
        		\item выделение компонента в виде, удобном для транспортировки в дочернюю ostis-систему с указанием версии и модификации при необходимости (например, выбор доступного хранилища компонента, выбор оптимального варианта реализации компонента с учетом состава дочерней системы);
        		\item установка многократно используемого компонента и его зависимостей (с указанием версии и модификации при необходимости);
        		\item интеграция компонента в дочернюю систему;
        		\item поиск и устранение ошибок и противоречий в дочерней системе.
        \end{itemize}}
        
        \scnheader{установленные компоненты*}
        \scniselement{квазибинарное отношение}
        \scniselement{ориентированное отношение}
        \scntext{пояснение}{Квазибинарное отношение, связывающее некоторую ostis-систему и компоненты, которые установлены в ней.}
        \scnrelfrom{первый домен}{ostis-система}
        \scnrelfrom{второй домен}{множество многократно используемых компонентов ostis-систем}
        \begin{scnindent}
            \scntext{пояснение}{множество многократно используемых компонентов ostis-систем --- это множество, все элементы которого являются многократно используемыми компонентами ostis-систем.}
        \end{scnindent}
        \scntext{примечание}{Данное отношение позволяет хранить сведения о системах и компонентах, которые установлены в них, тем самым предоставляя возможность анализировать функциональные возможности системы.}
        \scntext{примечание}{Данное отношение позволяет оценивать частоту скачивания компонентов, то есть их использования в \scnkeyword{дочерних ostis-системах}.}
        
        \scnheader{Предметная область и онтология комплексной библиотеки многократно используемых семантически совместимых компонентов ostis-систем}
        \scnrelfrom{заключение}{Заключение Предметной области и онтологии комплексной библиотеки многократно используемых семантически совместимых компонентов ostis-систем}
        \scnheader{Заключение Предметной области и онтологии комплексной библиотеки многократно используемых семантически совместимых компонентов ostis-систем}
        \begin{scnsubstruct}
        	\scnheader{компонентное проектирование интеллектуальных компьютерных систем}  
        	\scntext{примечание}{Компонентный подход является ключевым в технологии проектирования интеллектуальных компьютерных систем. Вместе с этим, технология компонентного проектирования тесно связана с остальными составляющими \textit{технологии проектирования интеллектуальных компьютерных систем} и обеспечивает их совместимость, производя мощнейший синергетический эффект при использовании всего комплекса частных технологий проектирования интеллектуальных систем. Важнейшим принципом в реализации компонентного подхода является семантическая совместимость многократно используемых компонентов, что позволяет минимизировать участие программистов в создании новых компьютерных систем и в совершенствовании существующих.}
        	\scntext{примечание}{Для реализации компонентного подхода предлогается \textit{библиотека многократно используемых совместимых компонентов интеллектуальных компьютерных систем на основе Технологии OSTIS}, введена классификация и спецификация многократно используемых компонентов ostis-систем, предложена модель менеджера компонентов, позволяющего взаимодействовать \textit{ostis-системам} с \textit{библиотеками многократно используемых компонентов} и управлять компонентами в системе, рассмотрена архитектура экосистемы \textit{интеллектуальных компьютерных систем} с точки зрения использования библиотеки многократно используемых компонентов.}
        	\scntext{примечание}{Полученные результаты позволят повысить эффективность проектирования интеллектуальных систем и средств автоматизации разработки таких систем, а также обеспечить возможность не только разработчику, но и интеллектуальной системе автоматически дополнять систему новыми знаниями и навыками.}
        \end{scnsubstruct}
        
        \bigskip
    \end{scnsubstruct}
\end{SCn}
