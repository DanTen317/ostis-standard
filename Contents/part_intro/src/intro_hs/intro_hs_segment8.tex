\scnsegmentheader{Комплекс свойств, определяющих качество многоагентной системы}
\begin{scnsubstruct}
    \begin{scnrelfromlist}{рассматриваемый вопрос}
        \scnitem{Почему переход от отдельных кибернетических систем к их коллективным, многоагентным системам,
            агентами которых являются заданные кибернетические системы, является одним из направлений
            повышения уровня интеллекта исходных кибернетических систем}
        \scnitem{Всегда ли объединение кибернетических систем в коллектив этих систем приводит у повышению уровня
            интеллекта.}
        \scnitem{Почему не каждое соединение достаточно качественных кибернетических систем порождает
            качественную многоагентную систему.}
    \end{scnrelfromlist}

    \scnheader{многоагентная система}
    \scntext{пояснение}{Переход от \textit{кибернетических систем} к коллективам
        взаимодействующих между собой \textit{кибернетических систем}, т.е. к
        социальной организации кибернетических систем, является важнейшим фактором
        эволюции \textit{кибернетических систем}.}
        \begin{scnindent}
            \scntext{источник}{\scncite{Dorri2018}}
        \end{scnindent}
    \scnsubset{кибернетическая система}
    \begin{scnsubdividing}

        \scnitem{моногенная многоагентная система}
        \begin{scnindent}    
            \scnidtf{однородная \textit{многоагентная система}, состоящая из однотипных \textit{агентов}}
        \end{scnindent}
        \scnitem{гетерогенная многоагентная система}
        \begin{scnindent}
            \scnidtf{неоднородная \textit{многоагентная система}, состоящая из \textit{агентов} разного типа}
        \end{scnindent}

    \end{scnsubdividing}
    \begin{scnsubdividing}

        \scnitem{одноуровневая многоагентная система}
        \begin{scnindent}    
            \scnidtf{многоагентная система, \textit{агенты} которой не являются \textit{многоагентными системами}}
            \scnidtf{многоагентная система, которая реализует либо одну модель параллельного (распределенного) решения
            задач соответствующего класса, либо комбинацию фиксированного числа разных и параллельно реализованных
            \textit{моделей решения задач}}
        \end{scnindent}
        \scnitem{иерархическая многоагентная система}
        \begin{scnindent}
            \scnidtf{многоагентная система, некоторые или все \textit{агенты} которой являются \textit{многоагентнымисистемами}}
            \scnidtf{многоагентная система, состоящая из агентов, которые могут быть \textit{индивидуальными кибернетическими системами}, \textit{коллективами индивидуальных кибернетических систем}, а также
            \textit{коллективами, состоящими из индивидуальных кибернетических систем и коллективов индивидуальных кибернетических систем}}
        \end{scnindent}

    \end{scnsubdividing}
    \begin{scnsubdividing}

        \scnitem{многоагентная система с централизованным управлением}
        \scnitem{многоагентная система с децентрализованным управлением}

    \end{scnsubdividing}
    \scntext{примечание}{Агенты \textit{многоагентной системы} могут (но вовсе не
        обязательно должны) быть \textit{интеллектуальными системами}. Так, например,
        агенты интеллектуального решателя задач, имеющего агентно-ориентированную
        архитектуру, не являются интеллектуальными системами.}
        \begin{scnindent}
            \scntext{источник}{\scncite{Ferber2003}}
        \end{scnindent}
        
    \scnheader{агент*}
    \scnidtf{быть агентом данной многоагентной системы*}
    \scnidtf{быть кибернетической системой, входящей в состав данной многоагентной
        системы*}
    \scntext{примечание}{Агентом иерархической многоагентной системы может быть другая
        многоагентная система}\scnsuperset{член многоагентной системы*}
    \begin{scnindent}
        \scnidtf{агент многоагентной системы, не являющийся агентом другого агента этой системы*}
        \scnidtf{непосредственный (ближайший) агент многоагентной системы*}
    \end{scnindent}

    \scnheader{кибернетическая система}
    \begin{scnsubdividing}
        \scnitem{индивидуальная кибернетическая система}
        \begin{scnindent}
            \scnidtftext{пояснение}{минимальная целостная \textit{кибернетическая система} обладающая достаточно высоким уровнем самостоятельности и способности выживать  в своей \textit{внешней среде}}
        \end{scnindent}
        \scnitem{кибернетическая система, являющаяся минимальным компонентом
            индивидуальной кибернетической системы}
        \begin{scnindent}
            \scntext{пояснение}{Это такой компонент, в состав которого не входят \textit{кибернетические системы}}
        \end{scnindent}
        \scnitem{кибернетическая система, являющаяся комплексом компонентов
            соответствующей индивидуальной кибернетической системы}
        \scnitem{сообщество индивидуальных кибернетических систем}
        \begin{scnindent}    
        \begin{scnsubdividing}
                \scnitem{простое сообщество индивидуальных кибернетических систем}
                \scnitem{иерархическое сообщество индивидуальных кибернетических систем}
            \end{scnsubdividing}
        \end{scnindent}
    \end{scnsubdividing}

    \scnheader{теория многоагентных систем}
    \scntext{пояснение}{Теория многоагентных систем --- это теория перехода количества кибернетических систем в кибернетическую
        систему нового качества, --- это выявление принципов и методов, позволяющие множество кибернетических
        систем соединить в коллективную кибернетическую систему, обладающую существенно более высоким уровнем
        качества (в том числе интеллекта) по сравнению с качеством кибернетической системы, входящих в этот
        коллектив.}

    \scnheader{многоагентная система}
    \scnidtf{коллектив взаимодействующих автономных кибернетических систем, имеющих общую среду обитания, жизнедеятельности}
    \begin{scnindent}
        \scntext{источник}{\scncite{Hadzic2009}}
    \end{scnindent}
    \begin{scnsubdividing}

        \scnitem{сообщество индивидуальных кибернетических систем}
        \scnitem{индивидуальная кибернетическая система, реализованная в виде многоагентной системы}
            \begin{scnindent}
                \scnsubset{кибернетическая система, являющаяся комплексом компонентов
                    соответствующей индивидуальной кибернетической системы}
                \scntext{пояснение}{Такая внутренняя	\textit{многоагентная система} в
                    индивидуальной кибернетической системе появляется, когда на определенном этапе
                    её эволюции \textit{решатель задач} \textit{индивидуальной кибернетической системы} переходит  на
                    \textit{агентно-ориентированную модель обработки информации} в памяти \textit{индивидуальной компьютерной системы}}
            \end{scnindent}

    \end{scnsubdividing}
    \scnidtf{кибернетическая система, представляющая собой коллектив
        взаимодействующих кибернетических систем, обладающих определенной степенью
        самостоятельности (самодостаточности, свободы выбора)}

    \scnheader{многоагентная система с централизованным управлением}
    \scnidtf{многоагентная система, в которой специально выделяются агенты, которые
        принимают решения в определенной области деятельности многоагентной системы и
        обеспечивают выполнение этих решений  путем управления деятельностью остальных
        агентов, входящих в состав этой системы}
    \scnsubset{многоагентная система}
    
    \scnheader{сообщество интеллектуальных систем с децентрализованным управлением}
    \scnidtf{многоагентная система с децентрализованным управлением, агентами
        которой являются интеллектуальные системы}
    \scnidtf{многоагентная система, в которой решения принимаются коллегиально и
        автоматически  (\uline{решения} о признании новой кем-то предложенной
        информации --- в том числе, об инициировании некоторой задачи, \uline{решения} о
        коррекции (уточнении) уже ранее признанной (одобренной, согласованной)
        информации) \uline{на основе} четко продуманной и постоянно совершенствуемой
        методики, а также \uline{на основе} активного участия всех агентов в
        формировании новых предложений, подлежащих признанию (одобрению, согласованию)}
        \begin{scnindent}
            \scntext{источник}{\scncite{Balaji2010}}
        \end{scnindent}
    \scnsubset{многоагентная система}
    \scntext{примечание}{В такой многоагентной системе все агенты участвуют в управлении
        этой системы}\scnhaselement{Экосистема OSTIS}
    \scntext{пояснение}{В такой многоагентной системе отсутствуют специально
        назначенные  агенты, которые обязаны  принимать решения о том, какую
        коллективно решаемую задачу надо инициировать, и о том, как распределить между
        агентами подзадачи указанной инициированной задачи.}
    \scnsubset{многоагентная система с децентрализованным управлением}
    \scnsubset{сообщество интеллектуальных систем}
    \scntext{примечание}{Примером такой системы является оркестр, способный играть без
        дирижера. При этом подчеркнем, что каждый музыкант такого
        оркестра:\begin{scnitemize}

            \item должен иметь квалификацию не только музыканта, но и дирижера и даже
            композитора
            \item должен быть договороспособным --- уметь согласовывать свои действия с
            действиями коллег\end{scnitemize}
        Аналогичным примером децентрализованной многоагентной системы является
        строительная бригада, способная построить дом без бригадира, прораба,
        архитектора.}
    
    \scnheader{синергетическая кибернетическая система}
    \scnidtf{эволюционная многоагентная система}
    \scnidtf{многоагентная система, состоящая из когнитивных агентов}
    \scnidtf{многоагентная система, обладающая высоким уровнем коллективного
        интеллекта, атомарными агентами которой являются индивидуальные
        интеллектуальные системы, имеющие высокий уровень интероперабельности}
    \scnrelfrom{пояснение}{Ярушкина.Н.Г.НечетГС-2007кн.-стр.88-101}
    \begin{scnindent}
        \scnrelto{цитата}{\scncite{YarushinaHS}}
    \end{scnindent}
    \begin{scnrelfromlist}{пояснение}
        \scnitem{\scncite{Lopes2022}}
        \scnitem{\scncite{Tarasov1997}}
        \scnitem{\scncite{Tarasov1998}}
        \scnitem{\scncite{Hamilton2006}}
    \end{scnrelfromlist}
    \scntext{примечание}{Очевидным примером синергетической кибернетической системы
        является творческий коллектив, реализующий сложный наукоемкий проект. Огромная
        сложность создания таких коллективов является главной причиной медленного
        развития целого ряда весьма актуальных научно-технических проектов, таких как
        создание принципиально нового технологического уровня автоматизации
        человеческой деятельности на основе интеллектуальных семантически совместимых
        компьютерных систем, способных самостоятельно взаимодействовать друг с
        другом.}
        
    \scnheader{многоагентная система}
    \scntext{примечание}{Переход к \textit{многоагентным системам} является важнейшим
        фактором повышения \textit{качества} (и, в частности, уровня
        \textit{интеллекта}) \textit{кибернетических систем}, т.к. уровень интеллекта
        \textit{многоагентной системы} может быть значительно выше уровня интеллекта
        каждого входящего в неё агента. Но это бывает далеко не всегда, поскольку
        важнейшим фактором качества многоагентных систем является не только качество
        входящих в неё агентов, но и организация взаимодействия агентов и, в частности,
        переход от централизованного к децентрализованному управлению. Количество
        далеко не всегда переходит в новое качество.
        \\Повышение уровня интеллекта многоагентной системы
        обеспечивается\begin{scnitemize}

            \item не только повышением уровня интеллекта и, в первую очередь, уровня
            \textit{интероперабельности} ее агентов;
            \item не только переходом от централизованного к децентрализованного управлению
            деятельности управлению деятельностью агентов;
            \item но и качеством общей базы знаний всей многоагентной
            системы.\end{scnitemize}
    }
    
    

    \scnheader{интероперабельность кибернетической системы}
    \scntext{примечание}{Когда мы говорим о \textit{интероперабельности кибернетической системы},
        речь идет только об \textit{индивидуальных кибернетических системах}, т.е. о
        \textit{кибернетических  системах}, достигших некоторого уровня целостности и
        автономности и способных входить в состав различных коллективов. Соответственно
        этому, качество \textit{индивидуальных кибернетических систем} определяется,
        кроме всего прочего тем, насколько большой вклад \textit{индивидуальная
            кибернетическая система} вносит в повышение качества тех коллективов, в состав
        которых она входит. Указанное свойство \textit{индивидуальных кибернетических
            систем} будем называть уровнем их \textit{интероперабельности}. Прежде, чем
        детализировать это свойство, целесообразно рассмотреть то, чем определяется
        качество коллектива кибернетических систем, например, качество творческого
        сообщества компьютерных систем и людей.}
        \begin{scnindent}
            \scntext{источник}{\scncite{Ouksel1999}}
            \scntext{детализация}{\nameref{sd_ostis_tech}}
        \end{scnindent}
        
    \scnheader{качество сообщества компьютерных систем и людей}
    \scntext{пояснение}{Эффективность творческого коллектива (например в области
        научно-технической деятельности) определяется:\begin{scnitemize}

            \item согласованностью мотивации (целевой установки) всего коллектива и каждого
            его члена:\begin{scnitemizeii}

                \item не должно быть синдрома лебедя, рака и щуки;
                \item не должно быть противоречий между целью коллектива и творческой
                самореализацией каждого его члена;\end{scnitemizeii}

            \item эффективной организацией децентрализованного управления деятельностью
            членов сообщества;
            \item четкой, оперативной и доступной всем фиксацией документации текущего
            состояния содеянного и направлений его дальнейшего развития;
            \item уровнем трудоемкости оперативности фиксации индивидуальных результатов в
            рамках коллективно создаваемого общего результата;
            \item уровнем структурированности и, прежде всего, стратифицированности
            обобщенной документации  (базы знаний);
            \item эффективностью ассоциативного доступа к фрагментам документации;
            \item гибкостью коллективно создаваемой базы;
            \item автоматизацией анализа содеянного и управления проектом.
        \end{scnitemize}
    }
    
    \scnheader{качество многоагентной системы}
    \begin{scnrelfromlist}{свойство-предпосылка}

        \scnitem{средний уровень интеллекта членов многоагентной системы}
        \scnitem{средний уровень интероперабельности членов многоагентной системы}
        \scnitem{минимальный уровень интероперабельности членов многоагентной системы}
            \begin{scnindent}
                \scntext{примечание}{Члены многоагентной системы, имеющие низкий уровень
                интероперабельности, существенно снижают качество системы.}
            \end{scnindent}
        \scnitem{качество организации взаимодействия членов многоагентной системы}
            \begin{scnindent}
            \scntext{примечание}{Высший уровень качества организации взаимодействия агентов
                многоагентной системы обеспечивается:
                \begin{scnitemize}
                    \item введением дополнительного специального (корпоративного) агента,
                    выполняющего функцию хранителя интегратора общих (корпоративных) знаний
                    многоагентной системы
                    \item реализацией децентрализованного взаимодействия агентов, управляемого
                    текущим состоянием информации, хранимой в памяти корпоративного агента.
                \end{scnitemize}}
            \end{scnindent}

    \end{scnrelfromlist}
    \bigskip
    \begin{scnset}
        \scnheader{ostis-система}
        \begin{scnindent}
            \scnsubset{многоагентная система, управляемая общей базой знаний}
        \end{scnindent}
    \end{scnset}
    \scntext{примечание}{Агенты \textit{ostis-системы} (sc-системы) являются
        \uline{специализированными} \textit{кибернетическими системами},
        \uline{действия} каждой из которых (кроме \textit{сенсорных sc-агентов})
        инициализируются определенного вида ситуациями и/или событиями в памяти
        \textit{ostis-системы} и \uline{заключаются} (за исключением
        \textit{эффекторных sc-агентов}) в преобразовании текущего состояния
        информации, хранимой в этой памяти. Таким образом, sc-агенты не являются
        интеллектуальными системами.}
    \bigskip
\end{scnsubstruct}
    \scnsourcecomment{Завершили Сегмент \scnqqi{Комплекс свойств, определяющих качество многоагентной системы}}
