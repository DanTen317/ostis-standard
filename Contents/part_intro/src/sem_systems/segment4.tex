\scnsegmentheader{Достоинства предлагаемых принципов, лежащие в основе интеллектуальных компьютерных систем нового поколения}

\begin{scnsubstruct}
    \scnheader{принципы, лежащие в основе интеллектуальных компьютерных систем нового поколения}
    \scntext{достоинство}{\textbf{смысловое представление информации} в памяти \textit{интеллектуальных компьютерных систем} обеспечивает
        устранение дублирования информации, хранимой в памяти \textit{интеллектуальной компьютерной системы}, то есть
        устранение многообразия форм представления одной и той же информации, запрещение появления в одной памяти 
        \textit{семантически эквивалентных информационных конструкций} и, в том числе, синонимичных \textit{знаков}. Это
        существенно снижает сложность и повышает качество:
    \begin{scnitemize}
        \item разработки различных \textit{моделей обработки знаний} (так как нет необходимости учитывать многообразие форм
            представления одного и того же знания);
        \item \textit{семантического анализа} и \textit{понимания} информации, поступающей (передаваемой) от различных внешних
            субъектов (от пользователей, от разработчиков, от других \textit{интеллектуальных компьютерных систем});
        \item \textit{конвергенции} и \textit{интеграции} различных видов знаний в рамках каждой \textit{интеллектуальной компьютерной
            системы};
        \item обеспечения \textit{семантической совместимости} и \textit{взаимопонимания} между различными \textit{интеллектуальными
            компьютерными системами}, а также между \textit{интеллектуальными компьютерными системами} и их пользователями
    \end{scnitemize}}

    \scntext{достоинство}{Понятие \textit{семантической сети} нами рассматривается не как красивая метафора сложноструктурированных
        \textit{знаковых конструкций}, а как формальное уточнение понятия \textit{смыслового представления информации}, как принцип
        представления информации, лежащей в основе нового поколения \textit{компьютерных языков} и самих \textit{компьютерных 
        систем} --- \textit{графовых языков} и \textit{графовых компьютеров}. \textit{Семантическая сеть} --- это нелинейная (графовая)
        \textit{знаковая конструкция}, обладающая следующими свойствами:
        \begin{scnitemize}
            \item все элементы (то есть синтаксически элементарные фрагменты) этой \textit{графовой структуры} (узлы и связки)
                являются знаками описываемых сущностей и, в частности, \textit{знаками связей} между этими сущностями;
            \item все знаки, входящие в эту \textit{графовую структуру}, не имеют \textit{синонимов} в рамках этой структуры;
            \item \scnqq{внутреннюю} структуру (строение) \textit{знаков}, входящих в семантическую сеть не требуется учитывать при ее
                \textit{семантическом анализе} (понимании);
            \item смысл \textit{семантической сети} определяется денотационной семантикой всех входящих в нее знаков и конфигурацией
                \textit{связей инцидентности} этих знаков;
            \item из двух \textit{инцидентных знаков}, входящих в \textit{семантическую сеть}, по крайней мере один является знаком связи.
        \end{scnitemize}}

    \scntext{достоинство}{\textit{рафинированная семантическая сеть} --- это \textit{семантическая сеть}, имеющая максимально простую
        \textit{синтаксическую структуру}, в которой, в частности,
        \begin{scnitemize}
            \item используется \uline{конечный} \textit{алфавит} элементов \textit{семантической сети}, то есть конечное число синтаксически
                выделяемых типов (синтаксических меток), приписываемых этим элемента;
            \item внешние идентификаторы (в частности, имена), приписываемые элементам \textit{семантической сети} используются
                \uline{только} для ввода/вывода информации
        \end{scnitemize}}

    \scntext{достоинство}{\textit{агентно-ориентированная модель обработки информации} в сочетании с \textit{децентрализованным ситуационным
        управлением процессом обработки информации}, а также со \textit{смысловым представлением информации} в памяти
        \textit{интеллектуальной компьютерной системы} существенно снижает сложность и повышает качество интеграции
        \begin{scnitemize}
        \item беспечивает автоматизацию решения сложных комплексных задач, для которых требуется создание
            временных или постоянных \uline{коллективов};
        \item превращает \textit{интеллектуальные компьютерные системы} в \uline{самостоятельные} активные \textit{субъекты}, способные
            инициировать различные комплексные задачи и, собственно, инициировать для этого работо-
            способные коллективы, состоящие из людей и \textit{интероперабельных интеллектуальных компьютерных
            систем} требуемой квалификации
        \end{scnitemize}}

    \scntext{достоинство}{Высокий уровень семантической гибкости информации, хранимой в памяти интеллектуальной компьютерной
        системы нового поколения, обеспечивается тем, что каждое удаление или добавление синтаксически элементарного
        фрагмента хранимой информации, а также удаление или добавление каждой связи инцидентности между такими
        элементами имеет четкую семантическую интерпретацию.}

    \scntext{достоинство}{Высокий уровень стратифицированности информации, хранимой в памяти интеллектуальной компьютерной
        системы нового поколения, обеспечивается онтологически ориентированной структуризацией базы знаний интеллектуальной
        компьютерной системы нового поколения.}

    \scntext{достоинство}{Высокий уровень индивидуальной обучаемости интеллектуальных компьютерных систем нового поколения (то
        есть их способности к быстрому расширению своих знаний и навыков) обеспечивается:
        \begin{scnitemize}
            \item семантической гибкостью информации, хранимой в их памяти;
            \item стратифицированностью этой информации;
            \item рефлексивностью интеллектуальных компьютерных систем нового поколения.
        \end{scnitemize}}

    \scntext{достоинство}{Высокий уровень коллективной обучаемости интеллектуальных компьютерных систем нового поколения
        обеспечивается высоким уровнем их интероперабельности (их социализации, способности к эффективному участию в
        деятельности различных коллективов, состоящих из интеллектуальных компьютерных систем нового поколения и
        людей) и, прежде всего, высоким уровнем их взаимопонимания.}

    \scntext{достоинство}{Высокий уровень интероперабельности интеллектуальных компьютерных систем нового поколения
        принципиально меняет характер взаимодействия компьютерных систем с людьми, автоматизацию деятельности которых они
        осуществляют, --- от управления этими средствами автоматизации к равноправным партнерским осмысленным
        взаимоотношениям}

    \scntext{достоинство}{Каждая интеллектуальная компьютерная система нового поколения способна:
        \begin{scnitemize}
            \item самостоятельно или по приглашению войти в состав коллектива, состоящего из интеллектуальных компьютерных
                систем нового поколения и/или людей. Такие коллективы создаются на временной или постоянной основе
                для коллективного решения сложных задач;
            \item участвовать в распределении (в том числе в согласовании распределения) задач --- как \scnqq{разовых} задач, так и
                долгосрочных задач (обязанностей);
            \item мониторить состояние всего процесса коллективной деятельности и координировать свою деятельность с
                деятельностью других членов коллектива при возможных непредсказуемых изменениях условий (состояния)
                соответствующей среды.
        \end{scnitemize}}

    \scntext{достоинство}{Высокий уровень интеллекта интеллектуальных компьютерных систем нового поколения и, соответственно,
        высокий уровень их самостоятельности и целенаправленности позволяет им быть полноправными членами самых
        различных сообществ, в рамках которых интеллектуальные компьютерные системы нового поколения получают
        права самостоятельно инициировать (на основе детального анализа текущего положения дел и, в том числе,
        текущего состояния плана действий сообщества) широкий спектр действий (задач), выполняемых другими членами
        сообщества, и тем самым участвовать в согласовании и координации деятельности членов сообщества. Способность
        интеллектуальной компьютерной системы нового поколения согласовывать свою деятельность с другими
        подобными системами, а также корректировать деятельность всего коллектива интеллектуальных компьютерных
        систем нового поколения, адаптируясь к различного вида изменениям среды (условий), в которой эта деятельность
        осуществляется, позволяет существенно автоматизировать деятельность системного интегратора как на этапе
        создания коллектива интеллектуальных компьютерных систем нового поколения, так и на этапе его обновления
        (реинжиниринга)}
    
    \scntext{примечание}{Достоинства интеллектуальных компьютерных систем нового поколения обеспечиваются:
        \begin{scnitemize}
            \item достоинствами языка внутреннего смыслового кодирования информации, хранимой в памяти этих систем;
            \item достоинствами организации графодинамической ассоциативной смысловой памяти интеллектуальных
                компьютерных систем нового поколения;
            \item достоинствами смыслового представления баз знаний интеллектуальных компьютерных систем нового
                поколения и средствами онтологической структуризации баз знаний этих систем;
            \item достоинствами агентно-ориентированных моделей решения задач, используемых в интеллектуальных
                компьютерных системах нового поколения в сочетании с децентрализованным управлением процессом обработки
                информации.
        \end{scnitemize}}

    \begin{scnrelfromlist}{основные положения}
    \scnfileitem{основным практически значимым направлением развития современных интеллектуальных компьютерных
        систем является переход к интероперабельным интеллектуальным компьютерным системам, способным к эффективному
        взаимодействию между собой и с пользователями, что:
        \begin{scnitemize}
            \item обеспечивает автоматизацию решения сложных комплексных задач, для которых требуется создание
                временных или постоянных \uline{коллективов};
            \item превращает интеллектуальные компьютерные системы в \uline{самостоятельные} активные субъекты, способные
                инициировать различные комплексные задачи и, собственно, инициировать для этого работоспособные
                коллективы, состоящие из людей и интероперабельных интеллектуальных компьютерных систем требуе-
                мой квалификации.
        \end{scnitemize}}
    \scnfileitem{коллективы, состоящие из самостоятельных \textit{интероперабельных интеллектуальных компьютерных систем}
        и людей, имеют хорошие перспективы стать \textit{синергетическими} системами}
    \scnfileitem{\textit{интероперабельность интеллектуальных компьютерных систем} обеспечивается:
        \begin{scnitemize}
            \item высоким уровнем взаимопонимания и, соответственно, семантической совместимостью;
            \item высоким уровнем договороспособности, то есть способности предварительно согласовывать свои действия
                с действиями других субъектов;
            \item высоким уровнем способности оперативно координировать свои действия с действиями других субъектов
                в ходе их выполнения
        \end{scnitemize}}
    \scnfileitem{к числу принципов, лежащих в основе построения \textit{интероперабельных интеллектуальных компьютерных
        систем}, относятся:
        \begin{scnitemize}
            \item смысловое представление знаний в памяти \textit{интеллектуальных компьютерных систем} в виде рафинированных
                семантических сетей;
            \item использование универсального языка внутреннего смыслового представления знаний;
            \item графодинамическая организация обработки знаний;
            \item агентно-ориентированные модели решения задач;
            \item структуризация и стратификация баз знаний в виде иерархической системы формальных онтологий;
            \item семантически дружественный пользовательский интерфейс.
        \end{scnitemize}}
    \scnfileitem{для разработки большого количества интероперабельных семантически совместимых \textit{интеллектуальных
        компьютерных систем}, обеспечивающих переход на принципиально новый уровень автоматизации \textit{человеческой
        деятельности}, необходимо создание технологии, обеспечивающей массовое производство таких \textit{интеллектуальных
        компьютерных систем}, участие в котором доступно широкому контингенту разработчиков (в том
        числе разработчиков средней квалификации и начинающих разработчиков). Основными положениями такой
        технологии являются
        \begin{scnitemize}
            \item стандартизация \textit{интероперабельных интеллектуальных компьютерных систем};
            \item широкое использование \textit{компонентного проектирования} на основе мощной библиотеки семантически
                совместимых многократно используемых (типовых) компонентов \textit{интероперабельных интеллектуальных
                компьютерных систем}
        \end{scnitemize}}
    \scnfileitem{эффективная эксплуатация \textit{интероперабельных интеллектуальных компьютерных систем} требует создания
        не только \textit{технологии проектирования} таких систем, но также и семейства технологий поддержки всех
        остальных этапов их жизненного цикла. Особенно это касается технологии перманентной поддержки \textit{семантической
        совместимости} всех взаимодействующих \textit{интероперабельных интеллектуальных компьютерных систем} в ходе их эксплуатации}
    \end{scnrelfromlist}
\end{scnsubstruct}
