\scnsegmentheader{Принципы, лежащие в основе онтологических моделей мультимодальных интерфейсов интеллектуальных компьютерных систем нового поколения}

\begin{scnsubstruct}
    \begin{scnrelfromlist}{ключевое понятие}
    	\scnitem{мультимодальный интерфейс}
        \scnitem{вербальный интерфейс}
        \scnitem{естественно-языковой интерфейс}
        \scnitem{внешний язык}
    	\begin{scnindent}
    		\scnidtf{язык обмена сообщениями}
    	\end{scnindent}
        \scnitem{внутренний язык}
    	\begin{scnindent}
    		\scnidtf{язык представления информации в памяти кибернетической системы}
    	\end{scnindent}
        \scnitem{синтаксис внешнего языка}
        \scnitem{денотационная семантика внешнего языка}
        \scnitem{интерфейсная задача}
        \scnitem{понимание сообщения}
        \scnitem{синтез сообщения}
        \scnitem{невербальный интерфейс}
        \scnitem{сенсор}
    	\begin{scnindent}
    		\scnidtf{рецептор}
    	\end{scnindent}
        \scnitem{сенсорная подсистема}
        \scnitem{мультисенсорная подсистема}
        \scnitem{сенсорная информация}
        \scnitem{эффектор}
        \scnitem{мультиэффекторная подсистема}
        \scnitem{сенсо-моторная координация}
    \end{scnrelfromlist}
   
    \begin{scnrelfromlist}{ключевое знание}
    	\scnitem{Принципы, лежащие в основе интерфейсов интеллектуальных компьютерных систем нового
            поколения}
    \end{scnrelfromlist}

    \scnheader{интерфейс интеллектуальной компьютерной системы нового поколения}
    \begin{scnrelfromlist}{принципы, лежащие в основе}
        \scnfileitem{интерфейс \textit{интеллектуальной компьютерной системы нового поколения} рассматривается как решатель
        задач частного вида --- \textit{интерфейсных задач}}
        \begin{scnrelfromlist}{основные зачачи}
            \scnfileitem{задачи понимания вербальной информации, приобретаемой интеллектуальной компьютерной системой 
                (синтаксический анализ, семантический анализ и погружение в базу знаний интеллектуальной
                компьютерной системы)}
            \scnfileitem{задачи понимания невербальной информации, воспринимаемой сенсорными подсистемами
                интеллектуальной компьютерной системы (анализ изображений, анализ аудио-сигналов, погружение 
                результатов анализа в базу знаний интеллектуальной компьютерной системы)}
            \scnfileitem{задачи синтеза сообщений, адресуемых внешним субъектам (кибернетическим системам)}
        \end{scnrelfromlist}
        \scnfileitem{тот факт, что интерфейс \textit{интеллектуальной компьютерной системы нового поколения} является 
            решателем частного вида \textit{задач интеллектуальной компьютерной системы нового поколения}, свойства,
            лежащие в основе решателей \textit{задач интеллектуальной компьютерной систем нового поколения}, наследуются
            интерфейсами \textit{интеллектуальной компьютерной систем нового поколения}}
        \begin{scnrelfromlist}{принципы, лежащие в основе}
            \scnfileitem{смысловое представление накапливаемых (приобретаемых знаний)}
            \scnfileitem{трактовка семантического анализа приобретаемой вербальной информации как процесса перевода
                этой информации на внутренний язык смыслового представления знаний с последующим погружением
                (вводом, интеграцией) результата этого перевода в состав текущего состояния базы знаний
                \textit{интеллектуальной компьютерной системы нового поколения}}
            \scnfileitem{трактовка синтеза сообщений, адресуемых внешними субъектами как процесса обратного перевода
                некоторого фрагмента базы знаний с внутреннего языка смыслового представления информации на
                внешний язык, используемый для общения с заданным субъектом}
            \scnfileitem{агентно-ориентированная организация решения интерфейсных задач, реализуемая соответствующим
                коллективов внутренних агентов \textit{интерфейса интеллектуальных компьютерных систем нового 
                поколения}, взаимодействующих через общедоступную для них базу знаний \textit{интеллектуальной
                компьютерной системы нового поколения}}
        \end{scnrelfromlist}
        \scnfileitem{интерфейс \textit{интеллектуальной компьютерной системы нового поколения} трактуется как специализированная
            встроенная \textit{интеллектуальная компьютерная система нового поколения}, входящая в состав
            указанной выше интеллектуальной компьютерной системы, база знаний которой включает в себя:
            \begin{scnitemize}
                \item онтологию синтаксиса внутреннего языка смыслового преставления информации
                \item онтологию денотационной семантики внутреннего языка смыслового представления информации
                \item онтологию синтаксиса всех внешних языков, используемых для общения с внешними субъектами
                \item онтологии денотационной семантики всех внешних языков, используемых для общения с внешними субъектами (каждая такая онтология с формальной точки зрения является описанием соответствия между текстами внешних языков и семантически эквивалентными им текстами внутреннего языка смыслового представления информации)
            \end{scnitemize}}
        \scntext{примечание}{Подчеркнем при этом, что все указанные онтологии, входящие в состав базы знаний интерфейса
            интеллектуальных компьютерных систем нового поколения, как и вся остальная информация, входящая в
            состав этой базы знаний, представляется на внутреннем языке смыслового представления информации,
            который, соответственно используется в данном случае как метаязык}
    \end{scnrelfromlist}

    \scnheader{интерфейс индивидуальной интеллектуальной компьютерной системы нового поколения}
    \begin{scnrelfromlist}{принципы, лежащие в основе}
        \scnfileitem{интерфейс индивидуальной интеллектуальной компьютерной системы нового поколения является
            специализированным компонентом решателя задач интеллектуальной компьютерной системы нового поколения,
            то есть специализированной \uline{встроенной} (в индивидуальную интеллектуальную компьютерную систему
            нового поколения) интеллектуальной компьютерной системой нового поколения, ориентированной на
            решение интерфейсных задач, к которым относятся:
            \begin{scnitemize}
                \item понимание принятых сообщений (их перевод на язык внутреннего смыслового представления информации
                    и погружения в текущее состояние базы знаний)
                \item синтез передаваемых сообщений (перевод сформированного сообщения с внутреннего языка смыслового
                    представления на используемый внешний язык)
                \item первичный анализ приобретаемой сенсорной информации, предполагающий распознавание некоторого
                    семейства первичных образов и сцен
                \end{scnitemize}}
        \scnfileitem{сенсомоторная координация действий, выполняемых эффекторами интеллектуальной компьютерной системы}
        \scnfileitem{мультимодальный характер интерфейса --- многообразие внешних языков, видов сенсоров и эффекторов}
        \scnfileitem{формальное онтологическое описание на языке внутреннего смыслового представления информации}
        \begin{scnrelfromlist}{виды информации}
            \scnfileitem{синтаксиса и денотационной семантики всех используемых внешних языков}
            \scnfileitem{первичных образов и сцен (ситуаций), являющихся результатом первичного анализа приобретаемой
                сенсорной информации}
            \scnfileitem{методов низкого уровня, непосредственно интерпретируемых эффекторами интеллектуальной компьютерной системы}
        \end{scnrelfromlist}
    \end{scnrelfromlist}
    \scntext{примечание}{Разговоры о дружественном и, в частности, адаптивном \textit{пользовательском интерфейсе} ведутся давно, но это, чаще
        всего, касается формы (\scnqq{синтаксической} стороны) \textit{пользовательского интерфейса}, а не смыслового содержания
        взаимодействия с пользователями. В настоящее время \textit{пользовательские интерфейсы} компьютерных систем (в
        том числе и \textit{интеллектуальных компьютерных систем}) для широкого контингента пользователей не являются
        семантически (содержательно) дружественными (семантически комфортными). Организация взаимодействия
        пользователей с компьютерными системами (в том числе и с \textit{интеллектуальными компьютерными системами})
        является \scnqq{узким местом}, оказывающим существенное влияние на эффективность \textit{автоматизации человеческой
        деятельности}. В основе современной организации взаимодействия пользователя с компьютерной системой лежит
        парадигма \uline{грамотного} пользователя, который знает, чего он хочет от используемого им инструмента и несет полную
        ответственность за качество взаимодействия с этим инструментом. Эта парадигма лежит в основе деятельности
        лесоруба во взаимодействии с топором, всадника во взаимодействии с лошадью, автоводителя, летчика во взаимодействии
        с соответствующим транспортным средством, оператора атомной электростанции, железнодорожного диспетчера и так далее.}
    \scntext{примечание}{На современном этапе развития \textit{Искусственного интеллекта} для повышения эффективности взаимодействия
        необходим переход \uline{от парадигмы грамотного управления} используемым инструментом \uline{к парадигме равноправного
        сотрудничества}, партнерскому взаимодействию интеллектуальной компьютерной системы со своим пользователем.
        \textit{Интеллектуальная компьютерная система} должна повернуться \scnqq{лицом} к пользователю. Семантическая дружественность 
        пользовательского интерфейса должна заключаться в адаптивности к особенностям и квалификации пользователя, исключении 
        любых проблем для пользователя в процессе диалога с \textit{интеллектуальной компьютерной системой}, в перманентной заботе о 
        совершенствовании коммуникационных навыков пользователя.}
    \scntext{примечание}{При организации взаимодействия пользователя с \textit{Глобальной сетью} компьютерным системам необходимо перейти
        от парадигмы \scnqq{многооконного} интерфейса, в каждом \scnqq{окне} которого свои \scnqq{правила игры}, к парадигме \scnqq{одного
        окна}. Пользователь не должен знать, какое \scnqq{окно} ему надо \scnqq{открыть} (в какую систему ему надо войти) для
        удовлетворения той или иной его потребности.
        Пользователь не должен знать, какая конкретно система будет решать его задачу. Пользователь должен уметь с
        помощью \uline{универсальных} средств сформулировать свою задачу, а соответствующая компьютерная система, входящая 
        в \textit{Глобальную сеть} и способная решить эту задачу, должна сама инициироваться, реагируя на факт появления
        указанной задачи. Таким образом пользовательский интерфейс должен быть интерфейсом пользователя не с 
        конкретной компьютерной системой, а в целом со всей \textit{Глобальной сетью компьютерных систем}.}
\end{scnsubstruct}
