\begin{SCn}
    \scnsectionheader{Предметная область и онтология базовых интерпретаторов логико-семантических моделей ostis-систем}
    \begin{scnsubstruct}
    	\scniselement{раздел базы знаний}
        \scnidtf{Предметная область и онтология платформ интерпретации sc-моделей компьютерных систем}
        \scnidtf{Предметная область и онтология аппаратных и программных платформ реализации интеллектуальных компьютерных систем нового поколения}
        \scnidtf{Предметная область и онтология платформ для массовой разработки семантически совместимых интероперабельных интеллектуальных компьютерных систем нового поколения}
        \scnidtf{Предметная область и онтология платформ для ostis-систем}
        \scntext{аннотация}{Уточнение понятия платформенной независимости интеллектуальных компьютерных систем нового поколения. Требования, предъявляемые к платформам интерпретации sc-моделей ostis-систем. Принципы и способы реализации аппаратной платформы для ostis-систем (ассоциативного семантического компьютера). Спецификация и существующие аналоги программного варианта реализации платформы для ostis-систем.}
        \scntext{аннотация}{В предметной области и онтологии рассматривается подход к решению проблемы платформенной независимости компьютерных систем, предполагающий унификацию принципов реализации таких систем и обеспечения их семантической совместимости на основе Технологии OSTIS. Приводится формализованная система понятий, определяющая принципы реализации данного подхода.}
        \begin{scnrelfromlist}{библиографическая ссылка}
        	\scnitem{\scncite{Komarcova2004}}
        	\scnitem{\scncite{USB_Accelerator}}
        	\scnitem{\scncite{Kolesnikov2001}}
        	\scnitem{\scncite{TuringMachine}}
        	\scnitem{\scncite{Neumann1993}}
        	\scnitem{\scncite{NeumanMachine}}
        \end{scnrelfromlist}
        \begin{scnrelfromlist}{дочерний раздел}
            \scnitem{Предметная область и онтология программных вариантов реализации базового интерпретатора логико-семантических моделей ostis-систем на современных компьютерах}
            \scnitem{Предметная область и онтология ассоциативных семантических компьютеров для ostis-систем}
        \end{scnrelfromlist}
        
        \begin{scnreltovector}{конкатенация сегментов}
        	\scnitem{Сегмент. Уточнение понятия платформенной независимости и анализ современных подходов к ее обеспечению}
        	\scnitem{Сегмент. Уточнение понятия ostis-платформы}
        \end{scnreltovector}
        
        \begin{SCn}
	\scnsectionheader{Сегмент. Уточнение понятия платформенной независимости и анализ современных подходов к ее обеспечению}
	
	\begin{scnsubstruct}
		\scnheader{производство интеллектуальных компьютерных систем}
		\scnrelfrom{этапы}{Этапы производства интеллектуальных компьютерных систем}
		\begin{scnindent}
			\begin{scneqtoset}
				\scnfileitem{Этап проектирования, то есть построения формальной модели системы, достаточной для понимания принципов ее устройства и выполнения последующего этапа ее реализации.}
				\scnfileitem{Этап реализации, то есть непосредственно воплощения разработанной модели с использованием конкретных средств (инструментов, материалов, комплектующих и так далее). В случае компьютерных систем выполнение данного этапа обычно предполагает выбор конкретных языков программирования, библиотек, сторонних средств, таких как с.у.б.д. и различные сервисы, а также собственно программирование и отладку системы с использованием выбранных средств.}
			\end{scneqtoset}
			\scntext{примечание}{Для каждого из указанных этапов могут существовать свои методики, а также средства автоматизации соответствующих процессов.}
			\scntext{отличие}{Если этап проектирования компьютерной системы как правило требует участия высококвалифицированных специалистов и экспертов в предметных областях, в которых осуществляется автоматизация, то этап реализации, с одной стороны, как правило является более простым (при условии качественного выполнения этапа проектирования), а с другой стороны требует значительных ресурсов.}
			\begin{scnindent}
				\scntext{причина}{Одной из причин этого является необходимость работы компьютерной системы на различных платформах (устройствах), каждое из которых в общем случае может иметь свои особенности и ограничения, которые необходимо учитывать на этапе реализации. Решением данной проблемы является обеспечение платформенной независимости (или кроссплатформенности) разрабатываемых компьютерных систем.}
			\end{scnindent}
		\end{scnindent}
		
		\scnheader{платформенная независимость компьютерных систем}
		\scntext{примечание}{Сама по себе идея обеспечения платформенной независимости давно и широко используется в современных компьютерных системах.}
		\scnsuperset{платформенная независимость на уровне операционных систем}
		\begin{scnindent}
			\scntext{пояснение}{Проблема обеспечения возможности работы программной системы в разных операционных системах.}
		\end{scnindent}
		\scnsuperset{платформенная независимость на уровне аппаратных архитектур}
		\begin{scnindent}
			\scntext{пояснение}{Проблема обеспечения совместимости операционной системы с различными аппаратными архитектурами. Для решения этой проблемы могут существовать разные сборки ядра операционной системы для разных аппаратных архитектур, как это делается для операционных систем семейства Linux. При этом следует отметить, что речь в подавляющем большинстве случаев идет не о принципиально разных архитектурах, а о вариантах реализации базовой архитектуры фон Неймана.}
		\end{scnindent}
	\scntext{примечание}{В случае, когда разрабатываемая компьютерная система проектируется на более низком уровне, чем операционная система как таковая (например, при программировании контроллеров управления различными устройствами), проблема обеспечения платформенной независимости значительно усугубляется и чаще всего может быть решена только для набора аппаратных средств определенного класса, для которого стандартизируется интерфейс доступа, то есть, по сути, набор низкоуровневых команд обработки информации.}
	
	\scnheader{платформенная независимость на уровне операционных систем}
	\scntext{примечание}{Большее внимание при проектировании современных компьютерных систем на данный момент уделяется платформенной независимости на уровне операционных систем.}
	\begin{scnrelfromlist}{достигается}
		\scnfileitem{Использование кроссплатформенных языков программирования, которые, в свою очередь, можно разделить на \scnqqi{полностью} интерпретируемые языки (Python, JavaScript и языки на его основе, PHP и другие) и языки, использующие компиляцию в сохраняющий независимость от платформы низкоуровневый байт-код, с его возможной последующей компиляцией в машинный код непосредственно в процессе исполнения (Just-in-time компиляция или JIT-компиляция). К языкам второго класса относятся, например, Java и C Sharp. Реализация такого подхода требует установки на целевой компьютер с операционной системой интерпретатора соответствующего языка программирования или байт-кода.}
		\begin{scnindent}
			\begin{scnrelfromlist}{ограничения}
				\scnfileitem{В среднем производительность интерпретируемых программ ниже, чем компилируемых. Одним из подходов к решению данной проблемы и является JIT-компиляция.}
				\scnfileitem{Строго говоря, кроссплатформенность при таком варианте обеспечивается не для всех операционных систем, а для класса операционных систем и соответствующего класса устройств, например, операционных систем, предназначенных для персональных компьютеров. Так, например, приложение, написанное на языке Java для персонального компьютера не может быть напрямую перенесено на мобильное устройство, поскольку при разработке мобильных приложений учитываются другие принципы работы пользователя с интерфейсом системы, отсутствие многооконности и многое другое.}
			\end{scnrelfromlist}
			\scntext{примечание}{Важно также отметить, что даже для интерпретируемых языков программирования существует проблема зависимости приложения от используемого набора библиотек и фреймворков. Так, при разработке интерфейса web-приложения могут использоваться популярные фреймворки AngularJS и ReactJS, при этом после выбора одного из них быстрый перевод приложения на другой фреймворк невозможен.}
		\end{scnindent}
		
		\scnfileitem{Реализация системы в виде web-приложения, работа с которым осуществляется через web-браузер и интерфейс которого, таким образом, реализуется на базе общепринятых стандартов Всемирной паутины (HTML, CSS, JavaScript и языки и библиотеки на его основе). Такой вариант обеспечивает возможность работы с приложением с любого устройства, имеющего web-браузер, в том числе, мобильного.}
		\begin{scnindent}
			\begin{scnrelfromlist}{недостатки}
				\scnfileitem{Как правило, высокая требовательность к производительности конечного устройства. Современный web-браузер является одним из самых ресурсоёмких приложений почти на любом устройстве.}
				\scnfileitem{Остается за кадром проблема обеспечения платформенной независимости серверной части web-приложения, которая должна решаться каким-то другим способом.}
				\scnfileitem{Несмотря на стандартизацию, разработчикам часто приходится учитывать особенности конкретных web-браузеров и тестировать работоспособность приложений для каждого из них.}
				\scnfileitem{Потенциально одним и тем же web-приложением можно пользоваться на любом устройстве, однако для обеспечения удобства и наглядности как правило приходится разрабатывать отдельные версии web-приложения, адаптированные под разные устройства, имеющие, например разные размеры экрана.}
			\end{scnrelfromlist}
		\end{scnindent}
		\scnfileitem{Виртуализация (контейнеризация, эмуляция). Перечисленные термины не являются полностью синонимичными, но в целом обозначают подход, при котором в рамках операционной системы создается некоторое изолированное локальное окружение (виртуальная машина, контейнер, среда эмуляции), содержащее все необходимые для работы приложения настройки и гарантирующее его работу на любых операционных системах и устройствах, где может интерпретироваться соответствующая виртуальная машина или контейнер. Соответственно, запуск таких окружений требует установки на конечное устройство соответствующего интерпретатора или эмулятора.}
		\begin{scnindent}
			\scntext{преимущества}{Данный подход бурно развивается и набирает популярность в настоящее время, поскольку позволяет решить не только проблему кроссплатформенности, но и избавить потребителя от установки большого числа зависимостей и выполнения настройки приложения на конечном устройстве.}
			\scntext{примеры}{Среди популярных средств, реализующих данный подход можно указать средства виртуализации (VirtualBox, DosBox, VMWare Workstation), контейнеризации (Docker), эмуляции приложений Android для настольных операционных систем (Genymotion, Bluestacks, Anbox) и многие другие.}
			\scntext{недостатки}{К недостаткам такого подхода можно отнести его ресурсоемкость и снижение производительности, а также ограниченность применения (как правило, соответствующие интерпретаторы разрабатываются только для наиболее популярных и востребованных операционных систем). Кроме того, возникает проблема следующего уровня, связанная уже с зависимостью от выбранного средства виртуализации (контейнеризации).}
		\end{scnindent}
	\end{scnrelfromlist}
	
	\scnheader{платформенная независимость компьютерных систем}
	\scntext{примечание}{Проблеме обеспечения платформенной независимости в современных компьютерных системах уделяется достаточно много внимания, однако в полной мере она не решена. В то же время, существует большое количество успешных частных решений, которые, однако, обладают серьезными ограничениями, связанными, в первую очередь, с отсутствием унификации современных подходов к разработке компьютерных систем.}
	\scntext{примечание}{Еще более актуальной проблема обеспечения платформенной независимости становится в контексте разработки \textit{интеллектуальных компьютерных систем}.}
	\begin{scnindent}
		\begin{scnrelfromlist}{обусловлено}
			\scnfileitem{Значительно более сложная по сравнению с традиционными компьютерными системами структура представляемой информации и, соответственно, многообразие форм ее представления, хранение и обработка которых на разных платформах могут быть организованы совершенно по-разному.}
			\scnfileitem{Высокие требования к производительности для некоторых классов систем, в частности, систем, использующих машинное обучение, что приводит к созданию специализированных аппаратных архитектур, таких как, например, нейрокомпьютеры.}
			\begin{scnindent}
				\begin{scnrelfromlist}{источник}
					\scnitem{\scncite{Komarcova2004}}
					\scnitem{\scncite{USB_Accelerator}}
				\end{scnrelfromlist}
			\end{scnindent}
			\scnfileitem{Многообразие моделей решения задач, которые в общем случае реализуются по-разному в разных системах.}
			\scnfileitem{Актуальность разработки гибридных интеллектуальных систем, в рамках которых интегрируются различные виды знаний и различные модели решения задач. В виду отсутствия на настоящий момент общепринятой унифицированной основы для их интеграции такие системы создаются в основном с ориентацией на какую-то определенную платформу и трудно переносимы на другие платформы.}
			\begin{scnindent}
				\begin{scnrelfromlist}{источник}
					\scnitem{\scncite{Kolesnikov2001}}
				\end{scnrelfromlist}
			\end{scnindent}
		\end{scnrelfromlist}
	\end{scnindent} 
	
	\scnheader{платформенная независимость интеллектуальных компьютерных систем}
	\scnsubset{платформенная независимость компьютерных систем}
	\scntext{примечание}{Проблема обеспечения платформенной независимости для интеллектуальных систем обусловлена во многом отсутствием семантической совместимости компонентов таких систем между собой, что, в свою очередь, создает препятствия даже для реализации подходов к обеспечению платформенной независимости, реализуемых в процессе разработки традиционных компьютерных систем. То есть, для решения проблемы обеспечения платформенной независимости интеллектуальных систем, требуется вначале обеспечить семантическую совместимость компонентов таких систем между собой.}
	\begin{scnindent}
		\scnrelfrom{предполагает}{Принципы семантической совместимости компонентов интеллектуальных компьютерных систем}
	\end{scnindent}
	
	\scnheader{Принципы семантической совместимости компонентов интеллектуальных компьютерных систем}
	\begin{scneqtoset}
		\scnfileitem{Унификация представления различного рода информации, хранимой в базах знаний таких систем.}
		\scnfileitem{Унификация базовых моделей обработки информации, хранимой в базах знаний таких систем, то есть выделение универсального низкоуровневого языка программирования, позволяющего осуществлять обработку информации, хранимой в унифицированном виде.}
		\scnfileitem{Унификация принципов реализации различных моделей решения задач и, как следствие, возможность их интеграции в рамках гибридных интеллектуальных систем.}
		\scnfileitem{Унификация принципов разработки интерфейсов компьютерных систем, которая бы позволила реализовать в рамках одной интеллектуальной системы возможность взаимодействия с другими системами и пользователями таких систем на разных внешних языках, включая естественные языки.}
	\end{scneqtoset}
	\scntext{примечание}{Указанные принципы реализуются в рамках \textit{Технологии OSTIS}, которая, таким образом, может стать основой для решения проблемы обеспечения семантической совместимости компонентов интеллектуальных компьютерных систем в целом и обеспечения платформенной независимости таких систем. С одной стороны, принципы, лежащие в основе \textit{Технологии OSTIS}, обеспечивают принципиальную возможность реализации платформенной независимости компьютерных систем, разрабатываемых на ее основе ostis-систем. С другой стороны, благодаря своей универсальности \textit{Технология OSTIS} позволяет преобразовать любую современную компьютерную систему в ostis-систему, которая будет функционально эквивалентна исходной компьютерной системе, но при этом будет обладать всеми перечисленными выше свойствами, создающими предпосылки для решения проблемы платформенной независимости.}
	\scntext{примечание}{Для реализации данного подхода в рамках Технологии OSTIS требуется разработать семейство онтологий, обеспечивающих уточнение таких понятий, как ostis-система, ostis-платформа, их структуры, типологии и предъявляемых к ним требований. Рассмотрению указанных понятий и посвящена данная онтология.}
	
	\scnheader{платформенная независимость на уровне операционных систем}
	\scntext{примечание}{Что касается обозначенной проблемы зависимости компьютерных систем от конкретных фрейморков, то аналогичная проблема может возникнуть и при дальнейшем развитии \textit{Технологии OSTIS}, в ситуации, когда соответствующие библиотеки будут содержать достаточно большое количество функционально эквивалентных компонентов. Однако, благодаря принципам, лежащим в основе \textit{Технологии OSTIS}, в частности, смысловому представлению информации и семантической совместимости компонентов, данная проблема будет значительно менее острой.}
	\begin{scnindent}
		\begin{scnrelfromset}{обоснование}
			\scnfileitem{Число функционально эквивалентных компонентов будет значительно ниже, чем в традиционных информационных технологиях, нет необходимости создавать синтаксически разные компоненты, отличия будут только на семантическом уровне.}
			\scnfileitem{Сами по себе компоненты будут являться более универсальными, то есть смогут быть использованы в значительно большем количестве систем.}
			\scnfileitem{Есть возможность автоматически выявить близкие компоненты, их сходства, различия, потенциальные конфликты и зависимости компонентов.}
			\scnfileitem{Есть возможность построения достаточно простых (по сравнению с традиционными технологиями) процедур перехода от одного фреймворка к другому, поскольку все компоненты и фреймворки имеют общую формальную смысловую основу, более высокоуровневую, чем в традиционных технологиях.}
		\end{scnrelfromset}
	\end{scnindent}
	
	\bigskip
	\end{scnsubstruct}
\scnsourcecomment{Завершили \scnqqi{Сегмент. Уточнение понятия платформенной независимости и анализ современных подходов к ее обеспечению}}
\end{SCn}
\begin{SCn}
	\scnsectionheader{Сегмент. Уточнение понятия ostis-платформы}
	\begin{scnsubstruct}
		
	\begin{scnrelfromlist}{ключевое понятие}
		\scnitem{ostis-платформа}
		\scnitem{базовая ostis-платформа}
		\scnitem{расширенная ostis-платформа}
		\scnitem{специализированная ostis-платформа}
		\scnitem{реализация sc-памяти}
		\scnitem{реализация файловой памяти sc-машины}
		\scnitem{scp-интерпретатор}
		\scnitem{базовая подсистема взаимодействия ostis-системы с внешней средой}
		\scnitem{подсистема обеспечения жизнедеятельности ostis-системы}
		\scnitem{специализированная платформенно-зависимая машина обработки знаний}
		\scnitem{минимальная конфигурация ostis-системы}
		\scnitem{однопользовательская ostis-платформа}
		\scnitem{многопользовательская ostis-платформа}
		\scnitem{программный вариант ostis-платформы}
		\scnitem{ассоциативный семантический компьютер}
	\end{scnrelfromlist}
	
	\scnheader{ostis-платформа}
	\scnidtf{платформа интерпретации sc-моделей компьютерных систем}
	\scnidtf{интерпретатор sc-моделей кибернетических систем}
	\scnidtf{интерпретатор унифицированных логико-семантических моделей компьютерных систем}
	\scnidtf{платформа реализации sc-моделей компьютерных систем}
	\scnidtf{\scnqq{пустая} ostis-система}
	\scnidtf{реализация sc-машины}
	\scnsubset{платформенно-зависимый многократно используемый компонент ostis-систем}
	\scnidtf{базовый интерпретатор логико-семантических моделей ostis-систем}
	\scnidtf{семейство платформ интерпретации sc-моделей компьютерных систем}
	\scnidtftext{часто используемый sc-идентификатор}{универсальный интерпретатор sc-моделей компьютерных систем}
	\scnidtf{универсальный интерпретатор унифицированных логико-семантических моделей компьютерных систем}
	\scnsubset{встроенная ostis-система}
	\scnidtf{встроенная \scnqq{пустая} ostis-система}
	\scnidtf{универсальный интерпретатор sc-моделей ostis-систем}
	\scnidtf{универсальная базовая ostis-система, обеспечивающая имитацию любой ostis-системы путем интерпретации sc-модели имитируемой ostis-системы}
	\scntext{примечание}{соотношение между имитируемой и универсальной ostis-системой в известной мере аналогично соотношению между машиной Тьюринга и универсальной машиной Тьюринга}
	\scntext{пояснение}{Под \textbf{\textit{ostis-платформой}} понимается реализация платформы интерпретации sc-моделей, которая в общем случае включает в себя: Реализация \textit{ostis-платформы} (\textit{универсального интерпретатора sc-моделей компьютерных систем}) может иметь большое число вариантов --- как программно, так и аппаратно реализованных. Логическая архитектура \textit{ostis-платформы} обеспечивает независимость проектируемых компьютерных систем от многообразия вариантов реализации интерпретатора их моделей и в общем случае включает в себя:
		\begin{scnitemize}
			\item хранилище \textit{sc-текстов} (\textit{sc-хранилище}, хранилище знаковых конструкций, представленных SC-коде);
			\item файловую память \textit{sc-машины};
			\item средства, обеспечивающие взаимодействие \textit{sc-агентов} над общей памятью (sc-памятью);
			\item базовые средства интерфейса для взаимодействия системы с внешним миром (пользователем или другими системами). Указанные средства включают в себя, как минимум, редактор, транслятор (в sc-память и из нее) и визуализатор для одного из базовых универсальных вариантов представления \textit{SC-кода} (\textit{SCg-код}, \textit{SCs-код}, \textit{SCn-код}), средства, позволяющие задавать системе вопросы из некоторого универсального класса (например, запрос семантической окрестности некоторого объекта);
			\item реализацию \textit{Абстрактной scp-машины}, то есть интерпретатор \textit{scp-программ} (программ Языка SCP).
		\end{scnitemize}
		При необходимости, в \textbf{\textit{ostis-платформу}} могут быть заранее на платформенно-зависимом уровне включены какие-либо компоненты машин обработки знаний или баз знаний, например, с целью упрощения создания первой версии \textit{прикладной ostis-системы}. Реализация платформы может осуществляться на основе произвольного набора существующих технологий, включая аппаратную реализацию каких-либо ее частей. С точки зрения компонентного подхода любая \textbf{\textit{ostis-платформа}} является \textbf{\textit{платформенно-зависимым многократно используемым компонентом}}.}
	\begin{scnrelfromset}{разбиение}
		\scnitem{базовая ostis-платформа}
		\scnitem{расширенная ostis-платформа}
		\scnitem{специализированная ostis-платформа}
	\end{scnrelfromset}
	
	\scnheader{базовая ostis-платформа}
	\scnidtf{базовый интерпретатор логико-семантических моделей ostis-систем}
	\scnidtf{минимальная универсальная ostis-платформа, обеспечивающая интерпретацию sc-модели любой \textit{ostis-системы} и включающая интерпретатор базового языка программирования \textit{ostis-систем} (Языка SCP)}
	\scnidtf{универсальный интерпретатор sc-моделей ostis-систем}
	\scnidtf{универсальная базовая ostis-система, обеспечивающая имитацию любой \textit{ostis-системы} путем интерпретации sc-модели имитируемой ostis-системы}
	\scntext{примечание}{Понятие \textit{базовой ostis-платформы} является ключевым с точки зрения обеспечения платформенной независимости \textit{ostis-систем}. Универсальность \textit{базовой ostis-платформы} подразумевает возможность интерпретации на ее основе любой \textit{sc-модели кибернетической системы}. Это достигается за счет наличия в рамках \textit{Технологии OSTIS} средств, позволяющих описывать на уровне sc-модели \textit{базу знаний}, \textit{решатель задач} и \textit{интерфейс кибернетической системы}, а также наличия Базового универсального языка программирования для \textit{ostis-систем} (\textit{Языка SCP}). \textit{Язык SCP} в таком случае выступает в роли базового низкоуровневого стандарта (ассемблера) обработки конструкций \textit{SC-кода}, гарантирующего полноту с точки зрения обработки, то есть, обеспечивающего возможность осуществить любое преобразование любого фрагмента \textit{SC-кода} при условии сохранения синтаксической корректности этого фрагмента. Следует отметить, что в общем случае таких функционально эквивалентных ассемблеров может быть несколько (и, как следствие, соответствующих им \textit{scp-машин}), но для обеспечения совместимости в рамках \textit{Технологии OSTIS} один из таких вариантов выбирается в качестве стандарта и описывается в \textit{Предметной области и онтологии решателей задач ostis-систем}.}
	\scntext{предъявляемое требование}{Основным и \uline{единственным требованием}, предъявляемым ко всем \textit{базовым ostis-платформам} для обеспечения их универсальности, является необходимость обеспечения интерпретации \textit{Языка SCP}, стандартизированного в рамках \textit{Технологии OSTIS}. При этом важно отметить, что все \textit{базовые ostis-платформы} обязаны быть \uline{функционально эквивалентными}, поскольку интерпретируют один и тот же стандарт \textit{Языка SCP}.}
	\begin{scnrelfromset}{обобщенная декомпозиция}
		\scnitem{реализация sc-памяти}
		\begin{scnindent}
			\scnrelfrom{обобщенная часть}{реализация файловой памяти sc-машины}
		\end{scnindent}
		\scnitem{scp-интерпретатор}
		\scnitem{базовая подсистема взаимодействия \textit{ostis-системы} с внешней средой}
		\begin{scnindent}
			\scntext{примечание}{Реализация базового набора \textit{рецепторных sc-агентов} и \textit{эффекторных sc-агентов}, обеспечивающих минимально необходимый обмен информацией между \textit{ostis-системой} и внешней средой. Конкретный перечень таких агентов требует уточнения, однако можно сказать, что в общем случае они могут быть реализованы как в составе \textit{scp-интерпретатора} (в этом случае им будут соответствовать определенные классы \textit{scp-операторов}), так и отдельно от него в составе платформы.}
		\end{scnindent}
		\scnitem{подсистема обеспечения жизнедеятельности ostis-системы}
		\begin{scnindent}
			\scntext{примечание}{Реализацию набора sc-агентов, обеспечивающих базовые функции \textit{ostis-системы}, связанные с обеспечением ее жизнедеятельности, которые принципиально не могут быть реализованы на платформенно-независимом уровне. К таким функциям относятся, например, запуск системы, загрузка базы знаний в память системы, запуск \textit{scp-интерпретатора} и так далее.}
		\end{scnindent}
	\end{scnrelfromset}
	
	\scnheader{расширенная ostis-платформа}
	\scnidtf{ostis-платформа, содержащая дополнительные компоненты, реализованные на уровне платформы}
	\scnidtf{базовая ostis-платформа и множество компонентов, реализованных на уровне платформы}
	\scntext{примечание}{\textit{расширенная ostis-платформа} представляет собой \textit{базовую ostis-платформу}, дополненную каким-либо множеством компонентов (хотя бы одним), реализованных на уровне платформы, при условии сохранения при этом всех возможностей \textit{базовой ostis-платформы}. Таким образом, \textit{расширенная ostis-платформа} по сути представляет собой \textit{базовую ostis-платформу}, адаптированную для более эффективного решения задач определенных классов в рамках конкретного класса \textit{ostis-систем}. Компонент, реализуемый на уровне платформы, становится частью этой платформы и, таким образом,  преобразует \textit{базовую ostis-платформу} в \textit{расширенную ostis-платформу}.}
	\scntext{примечание}{Введение понятия \textit{расширенной ostis-платформы} позволяет сформулировать ряд дополнительных принципов реализации \textit{ostis-систем}:
		\begin{scnitemize}
			\item Может существовать произвольное количество ostis-систем, каждая из которых будет иметь свою уникальную \textit{расширенную ostis-платформу}, но при этом все они будут основаны на одном и том же варианте \textit{базовой ostis-платформы}.
			\item Для каждого варианта \textit{базовой ostis-платформы} может существовать своя \textit{библиотека многократно используемых компонентов ostis-платформ} (см. Предметная область и онтология комплексной библиотеки многократно используемых семантически совместимых компонентов ostis-систем).
	\end{scnitemize}}
	
	\scnheader{специализированная ostis-платформа}
	\scnidtf{ostis-платформа, не содержащая реализацию интерпретатора Языка SCP}
	\scnidtf{неуниверсальная ostis-платформа}
	\scntext{примечание}{\textbf{\textit{специализированная ostis-платформа}} представляет собой ограниченный вариант реализации \textit{ostis-платформы}, не содержащий \textit{scp-интерпретатора}. Таким образом, \uline{все} \textit{sc-агенты}, в рамках \textit{ostis-системы}, основанной на \textit{специализированной ostis-платформе} должны быть реализованы на платформенно-зависимом уровне. Такая \textit{специализированная ostis-платформа} является аналогом специализированного компьютера, реализованного для конкретной компьютерной системы. Таким образом, в общем случае каждая \textit{ostis-система}, реализуемая на \textit{специализированной ostis-платформе} будет иметь свою \uline{уникальную} \textit{специализированную ostis-платформу}.}
	\scntext{примечание}{\textbf{\textit{специализированная ostis-платформа}} может быть получена из \textit{базовой ostis-платформы} путем исключения из нее реализации  \textit{scp-интерпретатора} и реализации всех необходимых \textit{sc-агентов} на уровне платформы (или заимствования всех или части агентов из соответствующей данному варианту \textit{базовой ostis-платформы} \textit{библиотеки многократно используемых компонентов ostis-платформ}).}
	\begin{scnrelfromset}{обобщенная декомпозиция}
		\scnitem{реализация sc-памяти}
		\begin{scnindent}
			\scnrelfrom{обобщенная часть}{реализация файловой памяти sc-машины}
		\end{scnindent}
		\scnitem{базовая подсистема взаимодействия ostis-системы с внешней средой}
		\scnitem{подсистема обеспечения жизнедеятельности ostis-системы}
		\scnitem{специализированная платформенно-зависимая машина обработки знаний}
		\begin{scnindent}
			\scnidtf{sc-агент, как правило неатомарный, обеспечивающий выполнение всех функций некоторой специализированной ostis-платформы, связанных с обработкой знаний}
			\scnsubset{платформенно-зависимый sc-агент}
		\end{scnindent}
	\end{scnrelfromset}
	\scntext{примечание}{Применение \textit{специализированных ostis-платформ} может быть целесообразным на стартовом этапе развития \textit{Технологии OSTIS}, а также с целью повышения производительности конкретных наиболее высоконагруженных \textit{ostis-систем}, однако активное развитие таких \textit{специализированных ostis-платформ} и их компонентов с точки зрения \textit{Технологии OSTIS} является нецелесообразным, поскольку:
		\begin{scnitemize}
			\item если какой-либо компонент разработан с ориентацией на конкретную платформу, то нет гарантий возможности его повторного использования в других вариантах реализации \textit{ostis-платформы} (как минимум, компоненты, разработанные для \textit{программного варианта реализации ostis-платформы} не смогут быть использованы в рамках \textit{ассоциативного семантического компьютера});
			\item наличие большого числа платформенно-зависимых компонентов требует развития и сопровождения отдельной инфраструктуры библиотек для хранения и повторного использования таких компонентов. Чем больше будет вариантов \textit{ostis-платформ} и чем больше будет число платформенно-зависимых компонентов, тем более сложной и громоздкой будет такая инфраструктура. Как минимум, необходимо будет отслеживать совместимость компонентов с разными версиями разных вариантов реализации \textit{ostis-платформ};
			\item изменения в \textit{специализированной ostis-платформе}, например, связанные с переходом на более новую и эффективную версию \textit{базовой ostis-платформы}, на основе которой построена данная \textit{специализированная ostis-платформа} в общем случае могут привести к необходимости внесения изменений в компоненты, зависящие от данного варианта реализации \textit{ostis-платформы}. Чем больше таких платформенно-зависимых компонентов, тем больше потенциальных изменений может потребоваться и, соответственно, тем сложнее будет осуществляться эволюция платформы при условии сохранения работоспособности \textit{ostis-систем}, в которых она используется.
		\end{scnitemize} 
		
		Перечисленные тезисы справедливы и для \textit{расширенных ostis-платформ}, однако в случае \textit{расширенной ostis-платформы} проблемы, связанные с переходом на более новую версию платформы и изменениями в соответствующих компонентах всегда могут быть решены путем временной замены платформенно-зависимых компонентов на их платформенно-независимые версии с соответствующим снижением производительности, но зато с сохранением функциональной целостности системы.}
	
	\scnheader{минимальная конфигурация ostis-системы}
	\begin{scnrelfromset}{обобщенная декомпозиция}
		\scnitem{sc-модель базы знаний}
		\scnitem{специализированная ostis-платформа}
	\end{scnrelfromset}
	\begin{scnrelfromlist}{предъявляемые требования}
		\scnfileitem{Использование \textit{SC-кода} как базового языка кодирования информации в базе знаний, и, соответственно, наличие памяти, хранящей конструкции \textit{SC-кода}.}
		\scnfileitem{Наличие \textit{базы знаний}, определяющей денотационную семантику понятий, используемых системой.}
		\scnfileitem{Наличие хотя бы одного внутреннего sc-агента, осуществляющего обработку знаний в памяти ostis-системы. Этот sc-агент может быть реализован на уровне платформы, соответственно база знаний такой системы может не содержать процедурных знаний (методов).}
	\end{scnrelfromlist}
	\scntext{примечание}{Такой вариант \textit{минимальной конфигурации ostis-системы} обладает только \textit{внутренним sc-агентом} и, соответственно, не имеет возможности общаться с внешним миром (можно сказать, что такая \textit{ostis-система} не обладает \scnqq{органами чувств}). Для того, чтобы система имела возможность общаться с внешним миром, необходимо добавить к \textit{минимальной конфигурации ostis-системы} хотя бы один \textit{рецепторный sc-агент} и хотя бы один \textit{эффекторный sc-агент}.}
	\scntext{примечание}{Важно отметить, что, как видно из представленного описания \textit{минимальной конфигурации ostis-системы}, в общем случае \textit{ostis-система} не обязана по умолчанию быть \textit{интеллектуальной системой}. Применение \textit{Технологии OSTIS} для разработки компьютерных систем не делает их автоматически интеллектуальными, оно позволяет обеспечить возможность последующей \uline{неограниченной интеллектуализации} таких систем с минимальными накладными расходами при условии соблюдения при их разработке всех принципов \textit{Технологии OSTIS}.}
	
	\scnheader{ostis-платформа}
	\begin{scnsubdividing}
		\scnitem{однопользовательская ostis-платформа}
		\begin{scnindent}
			\scnidtf{вариант реализации ostis-платформы, рассчитанный на то, что с конкретной ostis-системой взаимодействует только один пользователь (владелец)}
			\scntext{примечание}{При таком варианте реализации платформы оказывается невозможным реализовать некоторые важные принципы \textit{Технологии OSTIS}, например, коллективную согласованную разработку базы знаний системы в процессе ее эксплуатации. При этом могут использоваться различные сторонние средства, например, для разработки базы знаний на уровне исходных текстов.}
		\end{scnindent}
		\scnitem{многопользовательская ostis-платформа}
		\begin{scnindent}
			\scnidtf{вариант реализации ostis-платформы, рассчитанный на то, что с конкретной ostis-системой одновременно или в разное время могут взаимодействовать разные пользователи, в общем случае обладающие разными правами, сферами ответственности, уровнем опыта, и имеющие свою конфиденциальную часть хранимой в базе знаний информации}
		\end{scnindent}
	\end{scnsubdividing}
	\scnheader{платформа интерпретации sc-моделей компьютерных систем}
	\begin{scnsubdividing}
		\scnitem{программный вариант реализации платформы интерпретации sc-моделей компьютерных систем}
		\begin{scnindent}
			\scnidtf{программная платформа интерпретации sc-моделей ostis-систем}
			\scnidtf{программный базовый интерпретатор sc-моделей ostis-систем}
		\end{scnindent}
		\scnitem{семантический ассоциативный компьютер}
		\begin{scnindent}
			\scnidtf{аппаратная платформа интерпретации sc-моделей ostis-систем}
			\scnidtf{аппаратно реализованный базовый интерпретатор sc-моделей ostis-систем}
		\end{scnindent}
	\end{scnsubdividing}
	
	\scnheader{ostis-платформа}
	\begin{scnrelfromset}{разбиение}
		\scnitem{программный вариант ostis-платформы}
		\begin{scnindent}
			\scnidtf{платформа интерпретации sc-моделей ostis-систем, реализованная в виде программной системы на базе традиционной компьютерной архитектуры}
			\scnidtf{программная платформа интерпретации sc-моделей ostis-систем}
			\scnidtf{программный интерпретатор sc-моделей ostis-систем}
			\scntext{примечание}{Целесообразность разработки \textit{программных вариантов ostis-платформы} на настоящий момент обусловлена очевидной распространенностью фон-неймановской архитектуры и, соответственно, необходимостью реализации \textit{ostis-систем} на современных компьютерах различного вида. В то же время очевидно, что разработка специализированных \textit{ассоциативных семантических компьютеров} позволит существенно повысить эффективность работы \textit{ostis-систем}, а четкое разделение \textit{sc-модели кибернетической системы} и платформы ее интерпретации позволит осуществить перевод уже работающих \textit{ostis-систем} с традиционных архитектур на \textit{ассоциативные семантические компьютеры} с минимальными накладными расходами.}
		\end{scnindent}
		\scnitem{ассоциативный семантический компьютер}
		\begin{scnindent}
			\scnidtf{аппаратная платформа интерпретации sc-моделей ostis-систем}
			\scnidtf{аппаратно реализованный базовый интерпретатор sc-моделей ostis-систем}
		\end{scnindent}
	\end{scnrelfromset}
	\scntext{примечание}{Важно отметить, что в любом варианте реализации \textit{ostis-платформы} всегда присутствует как программная, так и аппаратная часть. Так, любой \textit{программный вариант ostis-платформы} предполагает его последующую интерпретацию на какой-либо аппаратной основе, например, на персональном компьютере с традиционной архитектурой. В то же время, разработка \textit{ostis-платформы} в виде \textit{ассоциативного семантического компьютера} предполагает разработку набора микропрограмм, реализующих базовые операции поиска и преобразования sc-конструкций, хранящихся в \textit{sc-памяти}.}
	\begin{scnindent}
		\scntext{примечание}{Таким образом, разделение множества возможных реализаций \textit{ostis-платформы} на программный и аппаратный варианты скорее отражает вариант аппаратной архитектуры, на которую в конечном итоге ориентирован тот или иной вариант реализации платформы --- либо на традиционную фон-неймановскую архитектуру, либо на специализированную архитектуру \textit{ассоциативного семантического компьютера} со структурно-перестраиваемой (графодинамической) памятью. \textit{Программный вариант ostis-платформы} по сути является моделью (виртуальной машиной) \textit{ассоциативного семантического компьютера}, построенной на базе традиционной фон-неймановской архитектуры, а \textit{Язык SCP} выступает в роли ассемблера для \textit{ассоциативного семантического компьютера} и также может интерпретироваться как в рамках аппаратной реализации такого компьютера, так и в рамках его программной модели. }
	\end{scnindent}
	\scntext{примечание}{Каждой конкретной \textit{ostis-системе} однозначно соответствует конкретная \textit{ostis-платформа}, которая может относиться к разному набору классов \textit{ostis-платформ}. В то же время очевидно, что на этапе разработки платформы проектируется и реализуется некоторый вариант \textit{ostis-платформы}, который затем тиражируется в разные \textit{ostis-системы}. Впоследствии в каждой \textit{ostis-системе} в этот вариант \textit{ostis-платформы} могут быть внесены изменения, но в общем случае в большом количестве \textit{ostis-систем} могут использоваться полностью эквивалентные \textit{ostis-платформы}. Таким образом, целесообразно говорить о \textit{типовых ostis-платформах}, которые:
		\begin{scnitemize}
			\item Являются объектом разработки для разработчиков \textit{ostis-платформ}.
			\item Являются \textit{многократно используемым компонентом ostis-систем} и специфицируются в рамках соответствующих библиотек.
			\item Являются образцом для тиражирования (копирования) при создании новых \textit{ostis-систем}.
	\end{scnitemize}}
	
	\bigskip
	\end{scnsubstruct}
\scnsourcecomment{Завершили \scnqqi{Сегмент. Уточнение понятия ostis-платформы}}
\end{SCn}
        
        \bigskip
    \end{scnsubstruct}
\end{SCn}
