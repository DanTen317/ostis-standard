\scnsegmentheader{Сегмент. Современное состояние работ в области разработки компьютеров для интеллектуальных систем}
\begin{scnsubstruct}
   	\scntext{введение}{Подавляющее большинство современных программно-аппаратных платформ, применяемых при разработке современных компьютерных систем, и, в частности, интеллектуальных компьютерных систем, основаны на принципах абстрактной \textit{машины фон-Неймана} или архитектуры фон-Неймана. Рассмотрим основные принципы, лежащие в основе \textit{машины фон-Неймана}.}
   	\begin{scnindent}
   		\begin{scnrelfromlist}{источник}
   			\scnitem{\scncite{Neumann1993}}
   			\scnitem{\scncite{NeumanMachine}}
   		\end{scnrelfromlist}
	\end{scnindent}
   	
	\scnheader{машина фон-Неймана}
	\scnidtf{абстрактная машина фон-Неймана}
	\scniselement{абстрактная машина обработки информации}
	\begin{scnrelfromvector}{принципы, лежащие в основе}
		\scnfileitem{Информация в памяти представляется в виде последовательности строк символов в бинарном алфавите (0 или 1).}
		\scnfileitem{Память машины представляет собой последовательность \uline{адресуемых} ячеек памяти.}
		\scnfileitem{В каждую ячейку может быть записана любая строка символов в бинарном алфавите. При этом длина строк для всех адресуемых ячеек одинакова (в текущем стандарте ячеек, называемых байтами, равна 8 бит).}
		\scnfileitem{Каждой ячейке памяти взаимно однозначно соответствует битовая строка, обозначающая эту ячейку и являющаяся ее адресом.}
		\scnfileitem{Каждому типу элементарных действий (операций), выполняемых в памяти машины фон-Неймана, взаимно однозначно ставится ее идентификатор, который в памяти представляется также в виде битовой строки.}
		\scnfileitem{Каждая конкретная операция (команда), выполняемая в памяти, представляется (специфицируется) в памяти в виде строки, состоящей
			\begin{scnitemize}
				\item из кода соответствущего типа операции;
				\item из последовательности адресов фрагментов памяти, в которых находятся операнды, над которыми выполняются операции --- исходные аргументы и результаты. Любой такой фрагмент задается адресом первого байта и количеством байт. Количество операндов \uline{однозначно} задается кодом типа операции;
			\end{scnitemize}}
		\scnfileitem{Программа, выполняемая в памяти, хранится в памяти в виде последовательности спецификаций конкретных операций (команд).}
		\scnfileitem{Таким образом, и обрабатываемые данные, и программы для обработки этих данных хранятся в одной и той же памяти (в отличие, например, от Гарвардской архитектуры) и кодируются одинаковым образом.}
	\end{scnrelfromvector}
	\scnrelfrom{особенности логической организации}{Особенности логической организации фон-Неймановской архитектуры}
	\begin{scnindent}
		\begin{scneqtoset}
			\scnfileitem{Последовательная обработка, ограничивающая эффективность компьютеров физическими возможностями элементной базы.}
			\scnfileitem{Низкий уровень доступа к памяти, то есть сложность и громоздкость выполнения процедуры ассоциативного поиска нужного фрагмента знаний.}
			\scnfileitem{Линейная организация памяти и чрезвычайно простой вид конструктивных объектов, непосредственно хранимых в памяти. Это приводит к тому, что в интеллектуальных системах, построенных на базе современных компьютеров, манипулирование знаниями осуществляется с большим трудом. Во-первых, приходится оперировать не самими структурами, а их громоздкими линейными представлениями (списками, матрицами смежности, матрицами инцидентности); во-вторых, линеаризация сложных структур разрушает локальность их преобразований.}
			\scnfileitem{Представление информации в памяти современных компьютеров имеет уровень весьма далекий от смыслового, что делает переработку знаний довольно громоздкой, требующей учета большого количества деталей, касающейся не смысла перерабатываемой информации, а способа ее представления в памяти.}
			\scnfileitem{В современных компьютерах имеет место весьма низкий уровень аппаратно реализуемых операций над нечисловыми данными и полностью отсутствует аппаратная поддержка логических операций над фрагментами знаний, имеющих сложную структуру, что делает манипулирование такими фрагментами весьма сложным.}
		\end{scneqtoset}
		\scntext{примечание}{Данные особенности архитектуры существенно затруднят эффективную реализацию \textit{ostis-систем} на ее основе.}
	\end{scnindent}
	\scnrelfrom{альтернативные подходы}{Альтернативные фон-Неймановским подходы организации ЭВМ}
	\begin{scnindent}
		\scntext{примечание}{Попытки преодоления ограничений традиционных фон-Неймановских ЭВМ привели к возникновению множества подходов, связанных с отдельными изменениями принципов логической организации ЭВМ, прежде всего в зависимости от классов задач и предметных областей, на которые ориентируется тот или иной класс ЭВМ. Все эти тенденции, рассмотренные в совокупности, позволяют очертить некоторые ключевые принципы логической организации ЭВМ, ориентированных на переработку знаний (машин переработки знаний --- МПЗ).}
		\begin{scneqtoset}
			\scnfileitem{Переход к нелинейной организации памяти и аппаратная интерпретация сложных структур данных.}
			\begin{scnindent}
				\begin{scnrelfromlist}{источник}
					\scnitem{\scncite{Glushkov1974}}
					\scnitem{\scncite{Ajlif1973}}
					\scnitem{\scncite{Moldovan1992}}
					\scnitem{\scncite{Chu1976}}
					\scnitem{\scncite{Kalynychenko1990}}
				\end{scnrelfromlist}
			\end{scnindent}
			\scnfileitem{Аппаратная реализация ассоциативного доступа к информации.}
			\begin{scnindent}
				\begin{scnrelfromlist}{источник}
					\scnitem{\scncite{Martin1980}}
					\scnitem{\scncite{Ozkarahan1989}}
					\scnitem{\scncite{Glushkov1974}}
					\scnitem{\scncite{Kohonen1980}}
					\scnitem{\scncite{Ignatushhenko1981}}
					\scnitem{\scncite{Berkovich1975}}
					\scnitem{\scncite{Ajzerman1977}}
				\end{scnrelfromlist}
			\end{scnindent}
			\scnfileitem{Реализация параллельных асинхронных процессов над памятью и, в частности, разработка вычислительных машин, управляемых потоком данных.}
			\begin{scnindent}
				\begin{scnrelfromlist}{источник}
					\scnitem{\scncite{Glushkov1974}}
					\scnitem{\scncite{Marchuk1978}}
					\scnitem{\scncite{Prangishvili1981}}
					\scnitem{\scncite{Zatuliver1981}}
					\scnitem{\scncite{Ackerman1979}}
					\scnitem{\scncite{Majers1985}}
				\end{scnrelfromlist}
			\end{scnindent}
			\scnfileitem{Аппаратная интерпретация языков высокого уровня.}
			\begin{scnindent}
				\begin{scnrelfromlist}{источник}
					\scnitem{\scncite{Glushkov1980}}
					\scnitem{\scncite{Glushkov1978}}
					\scnitem{\scncite{Rabinovich1995}}
				\end{scnrelfromlist}
			\end{scnindent}
			\scnfileitem{Разработка аппаратных средств ведения баз данных (процессоров баз данных).}
			\begin{scnindent}
				\begin{scnrelfromlist}{источник}
					\scnitem{\scncite{Zadyhajlo1979}}
					\scnitem{\scncite{Schuster1979}}
					\scnitem{\scncite{Suvorov1985}}
				\end{scnrelfromlist}
			\end{scnindent}
		\end{scneqtoset}
	\end{scnindent}
	
	\scnheader{классы вычислительных устройств}
	\scntext{примечание}{На пересечении \scnkeyword{Альтернативных фон-Неймановским подходов организации ЭВМ} в разное время возникали различные классы вычислительных устройств}
	\scnhaselement{машины с аппаратной интерпретацией сложных структур данных}
	\begin{scnindent}
		\begin{scnrelfromlist}{источник}
			\scnitem{\scncite{Brukle1978}}
			\scnitem{\scncite{Chu1976}}
			\scnitem{\scncite{Chu1977}}
		\end{scnrelfromlist}
	\end{scnindent}
	\scnhaselement{машины с развитой ассоциативной памятью}
	\begin{scnindent}
		\begin{scnrelfromlist}{источник}
			\scnitem{\scncite{Kohonen1980}}
			\scnitem{\scncite{Kohonen1982}}
			\scnitem{\scncite{Foster1981}}
		\end{scnrelfromlist}
	\end{scnindent}
	\scnhaselement{ассоциативные параллельные матричные процессоры}
	\begin{scnindent}
		\scnrelfrom{источник}{\scncite{Ershov1982}}
	\end{scnindent}
	\scnhaselement{однородные параллельные структуры для решения комбинаторно-логических задач на графах и гиперграфах}
	\begin{scnindent}
		\scnrelfrom{источник}{\scncite{Bershtejn1975}}
	\end{scnindent}
	\scnhaselement{устройства переработки графов}
	\begin{scnindent}
	\begin{scnrelfromlist}{источник}
		\scnitem{\scncite{Vasilev1987}}
		\scnitem{\scncite{Sapatyj1984}}
		\scnitem{\scncite{Popov2019}}
		\scnitem{\scncite{Popov2020}}
	\end{scnrelfromlist}
	\scnsuperset{устройства переработки графов на основе FPGA}
	\begin{scnindent}
	\begin{scnrelfromlist}{источник}
		\scnitem{\scncite{Zhang2017}}
		\scnitem{\scncite{Hu2021}}
		\scnitem{\scncite{Song2016}}
	\end{scnrelfromlist}
	\end{scnindent}
	\scnsuperset{устройства переработки графов на основе векторных процессоров}
	\begin{scnindent}
		\scnrelfrom{источник}{\scncite{Afanasyev2021}}
	\end{scnindent}
	\end{scnindent}
	\scnhaselement{системы, осуществляющие переработку информации непосредственно в памяти путем равномерного распределения функциональных средств по памяти и, в частности, предложенная М. Н. Вайнцвайгом процессоро-память, ориентированная на решение задач Искусственного интеллекта}
	\begin{scnindent}
		\begin{scnrelfromlist}{источник}
			\scnitem{\scncite{Vajncvajg1980}}
			\scnitem{\scncite{Vajncvajg1987}}
		\end{scnrelfromlist}
	\end{scnindent}
	\scnhaselement{машины, управляемые потоком данных}
	\begin{scnindent}
		\begin{scnrelfromlist}{источник}
			\scnitem{\scncite{Ershov1982}}
			\scnitem{\scncite{Prangishvili1981}}
			\scnitem{\scncite{Majers1985}}
		\end{scnrelfromlist}
		\scnsuperset{процессоры, реконфигурируемые с учетом семантики входного потока данных}
		\begin{scnindent}
			\scnrelfrom{источник}{\scncite{Somsubhra2006}}
		\end{scnindent}
	\end{scnindent}
	\scnhaselement{рекурсивные вычислительные машины}
	\begin{scnindent}
		\scnrelfrom{источник}{\scncite{Glushkov1974}}
	\end{scnindent}
	\scnhaselement{процессоры реляционных баз данных}
	\begin{scnindent}
		\begin{scnrelfromlist}{источник}
			\scnitem{\scncite{Zadyhajlo1979}}
			\scnitem{\scncite{Majers1985}}
			\scnitem{\scncite{Ozkarahan1989}}
		\end{scnrelfromlist}
	\end{scnindent}
	\scnhaselement{вычислительные машины со структурно-перестраиваемой памятью}
	\begin{scnindent}
		\begin{scnrelfromlist}{источник}
			\scnitem{\scncite{Rabinovich1979a}}
			\scnitem{\scncite{Rabinovich1979b}}
			\scnitem{\scncite{Gladun1977}}
			\scnitem{\scncite{Gladun1987}}
		\end{scnrelfromlist}
	\end{scnindent}
	\scnhaselement{активные семантические сети}
	\begin{scnindent}
		\scnidtf{М-сети}
		\scnrelfrom{источник}{\scncite{Amosov1973}}
	\end{scnindent}
	\scnhaselement{ассоциативные однородные среды}
	\begin{scnindent}
		\scnrelfrom{источник}{\scncite{Zolotov1982}}
	\end{scnindent}
	\scnhaselement{нейроподобные структуры}
	\begin{scnindent}
		\begin{scnrelfromlist}{источник}
			\scnitem{\scncite{Galushkin1997}}
			\scnitem{\scncite{Heht-Nilsen1998}}
		\end{scnrelfromlist}
		\scntext{примечание}{В последние годы активное развитие теории искусственных нейронных сетей привело к развитию различных подходов к построению высокопроизводительных компьютеров, предназначенных для обучения и интерпретации искусственных нейронных сетей и их внедрению в различные программно-аппаратные комплексы.}
		\begin{scnindent}
			\begin{scnrelfromlist}{источник}
				\scnitem{\scncite{Komarcova2004}}
				\scnitem{\scncite{USB_Accelerator}}
				\scnitem{\scncite{Moussa2013}}
			\end{scnrelfromlist}
		\end{scnindent}
		\scntext{примечание}{В отдельное направление выделены так называемые нейроморфные процессоры, отличающиеся высокой производительностью и низким уровнем энергопотребления.}
		\begin{scnindent}
			\scnrelfrom{источник}{\scncite{Altay}}
		\end{scnindent}
	\end{scnindent}
	\scnhaselement{машины для интерпретации логических правил}
	\begin{scnindent}
		\scnrelfrom{источник}{\scncite{Allen1989}}
	\end{scnindent}
	\scntext{примечание}{Развитие графических процессоров (graphics processing unit, GPU) привело к возможности организации параллельных вычислений непосредственно на GPU, для чего разрабатываются специализированные программно-аппаратные архитектуры, например CUDA. Преимуществом GPU в данном случае выступает наличие в рамках одного GPU большого (по сравнению с центральным процессором) числа ядер, что позволяет эффективно решать на такой архитектуре задачи, обладающие естественным параллелизмом (например, операции с матрицами). Развиваются также работы, посвященные принципам обработки графовых структур на GPU.}
	\begin{scnindent}
		\begin{scnrelfromlist}{источник}
			\scnitem{\scncite{CUDA}}
			\scnitem{\scncite{OpenCL}}
			\scnitem{\scncite{Tran2018}}
			\scnitem{\scncite{Shi2018}}
			\scnitem{\scncite{Lu2021}}
		\end{scnrelfromlist}
	\end{scnindent}
	\scntext{примечание}{Благодаря повышению производительности современных ЭВМ число разработок специализированных аппаратных решений в последние десятилетия снизилось, поскольку многие сложные вычислительные задачи в настоящее время за приемлемое время могут быть решены и на традиционных универсальных архитектурах. Как показано выше, исключение составляют в основном специализированные компьютеры для обработки искусственных нейронных сетей и других графовых моделей, что обусловлено большой востребованностью таких моделей и их сложностью.}
	\scntext{примечание}{Большинство перечисленных подходов (даже если они достаточно далеко отходят от предложенных фон-Нейманом базовых принципов организации вычислительных машин) неявно сохраняют точку зрения на компьютер как на большой арифмометр и тем самым сохраняют ее ориентацию на задачи числового характера. Работы же направленные на разработку аппаратных архитектур, предназначенных для обработки информации, представленной в более сложных формах, чем в традиционных архитектурах не получили широкого распространения и применения, по причине, во-первых, частности предлагаемых решений, и во-вторых из-за отсутствия общего универсального и унифицированного языка кодирования любой информации, в роли которого в рамках \textit{Технологии OSTIS} выступает \textit{SC-код}, а также соответствующей апробированной технологии разработки программных систем для таких аппаратных архитектур. Таким образом, зачастую разработчики подобных архитектур сталкиваются необходимостью разработки специализированного программного обеспечения для этих архитектур, что в конечном итоге приводит к сильному ограничению сфер применения таких архитектур, поскольку их применение оказывается целесообразным только в случае, если трудоемкость разработки специализированного программного обеспечения оправдывает себя с учетом низкой эффективности решения соответствующих задач на более традиционных архитектурах.}
	
	\scnheader{ассоциативный семантический компьютер}
	\scntext{примечание}{\textit{SC-код}, являющийся формальной основой \textit{Технологии OSTIS} изначально разрабатывался как язык кодирования информации в памяти \textit{ассоциативных семантических компьютеров}, таким образом в нем изначально заложены такие принципы, как универсальность (возможность представить знания любого рода) и унификация (единообразие) представления, а также минимизация \textit{Алфавита SC-кода}, которая, в свою очередь, позволяет облегчить создание аппаратной платформы, позволяющей хранить и обрабатывать тексты \textit{SC-кода}.}
	\scntext{примечание}{Основная методологическая особенность предлагаемого подхода к разработке средств аппаратной реализации поддержки интеллектуальных систем заключается в том, что такие средства должны разрабатываться не до, а \uline{после того, как} основные положения соответствующей \uline{технологии} проектирования и эксплуатации интеллектуальных систем будут апробированы на современных технических средствах. Более того, в рамках \textit{\textit{Технологии OSTIS}} четко продумана методика перехода на новые аппаратные средства, которая затрагивает только самый нижний уровень технологии --- уровень реализации базовой машины обработки семантических сетей (интерпретатора \textit{Языка SCP}).}
	\begin{scnrelfromvector}{основные этапы истории}
		\scnfileitem{1984 год --- в Московском институте электронной техники В.В. Голенковым защищена кандидатского диссертация на тему \scnqq{Структурная организация и переработка информации в электронных математических машинах, управляемых потоком сложноструктурированных данных}, в которой были сформулированы и рассмотрены основные принципы семантических ассоциативных компьютеров.}
		\begin{scnindent}
			\scnrelfrom{источник}{\scncite{Golenkov1984}}
		\end{scnindent}
		\scnfileitem{1993 год --- комиссия Госкомпрома провела успешные испытания прототипа \textit{ассоциативного семантического компьютера}, разработанного на базе транспьютеров в рамках научно-исследовательского проекта \scnqq{Параллельная графовая вычислительная система, ориентированная на решение задач искусственного интеллекта}.}
			\begin{scnindent}
			\begin{scnrelfromlist}{источник}
				\scnitem{\scncite{Golenkov1994f}}
				\scnitem{\scncite{Golenkov1994g}}
			\end{scnrelfromlist}
		\end{scnindent}
		\scnfileitem{1996 год --- В.В. Голенковым защищена докторская диссертация на тему \scnqq{Графодинамические модели и методы параллельной асинхронной переработки информации в интеллектуальных системах}.}
		\begin{scnindent}
			\scnrelfrom{источник}{\scncite{Golenkov1996}}
		\end{scnindent}
		\scnfileitem{2000 год --- в Институте проблем управления РАН П.А. Гапоновым защищена кандидатская диссертация на тему \scnqq{Модели и методы параллельной асинхронной переработки информации в графодинамической ассоциативной памяти}.}
		\begin{scnindent}
			\scnrelfrom{источник}{\scncite{Gaponov2000}}
		\end{scnindent}
		\scnfileitem{2000 год --- в Институте программных систем РАН В.М. Кузьмицким защищена кандидатская диссертация на тему \scnqq{Принципы построения графодинамического параллельного компьютера, ориентированного на решение задач искусственного интеллекта}.}
		\begin{scnindent}
			\scnrelfrom{источник}{\scncite{Kuzmickij2000}}
		\end{scnindent}
		\scnfileitem{2004 год --- в Белорусском государственном университете информатики и радиоэлектроники Р.Е. Сердюковым защищена кандидатская диссертация на тему \scnqq{Базовые алгоритмы и инструментальные средства обработки информации в графодинамических ассоциативных машинах}, в которой было рассмотрено базовое программное обеспечение семантических ассоциативных компьютеров.}
		\begin{scnindent}
			\scnrelfrom{источник}{\scncite{Serdiukov2004}}
		\end{scnindent}
	\end{scnrelfromvector}
	\scntext{примечание}{В тоже время, несмотря на наличие действующего прототипа \textit{ассоциативного семантического компьютера} на базе транспьютеров, основное внимание в рамках соответствующего проекта и других перечисленных работ уделялось принципам организации распределенной параллельной обработки конструкций SC-кода, в частности, был разработан \textit{SCD-код} (Semantic Code Distributed) для распределенного хранения конструкций SC-кода и \textit{Язык SCPD} для распределенной параллельной их обработки. Однако, общие принципы хранения информации и общая архитектура каждого из процессорных элементов (транспьютеров) оставались фон-Неймановскими. В частности, для кодирования конструкций SC-кода в традиционной адресной памяти были разработаны соответствующие структуры данных.}
	\begin{scnindent}
		\scnrelfrom{смотрите}{Предметная область и онтология программных вариантов реализации базового интерпретатора логико-семантических моделей ostis-систем на современных компьютерах}
	\end{scnindent}
	\scntext{примечание}{Обоснованность и необходимость разработки \textit{ассоциативного семантического компьютера}, а также компетентность авторов в данной области подтверждается более чем 30-летним опытом работы и рядом успешных проектов в данном направлении, однако, в тоже время, в предшествующих работах в полной мере не устранены все недостатки фон-Неймановской архитектуры, рассмотренные выше, и разработка и реализация проекта \textit{ассоциативного семантического компьютера}, устраняющего перечисленные недостатки, остается актуальной.}
\bigskip
\end{scnsubstruct}
\scnsourcecomment{Завершили \scnqqi{Сегмент. Современное состояние работ в области разработки компьютеров для интеллектуальных систем}}
