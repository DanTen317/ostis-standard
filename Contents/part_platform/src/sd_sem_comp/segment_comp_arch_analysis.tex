\scnsegmentheader{Сегмент. Анализ существующих архитектур вычислительных систем}
\begin{scnsubstruct}
	
	\scnheader{Альтернативные фон-Неймановским подходы организации ЭВМ}
	\scntext{примечание}{Для того, чтобы преодолеть недостатки существующих архитектур вычислительных систем, включая фон-Неймановскую, было предложено множество различных подходов. При разработке новых архитектур и, в частности, архитектуры \textit{ассоциативного семантического компьютера}, целесообразно в виде соответствующей онтологии выделить основные признаки классификации и соответствующие им классы (виды) архитектур вычислительных систем.}
	\begin{scnindent}
		\begin{scnrelfromlist}{источник}
			\scnitem{\scncite{Ivashenko2021OSTIS}}
			\scnitem{\scncite{Ivashenko2016Tatur}}
			\scnitem{\scncite{Ivashenko2015Tatur}}
			\scnitem{\scncite{Rasheed2019}}
			\scnitem{\scncite{Dubrovin2020}}
			\scnitem{\scncite{Wolfram2002}}
		\end{scnrelfromlist}
	\end{scnindent}
	
	\scnheader{архитектура вычислительной системы}
	\begin{scnsubdividing}
		\scnitem{архитектура вычислительной системы с глобальной оперативной памятью}
		\begin{scnindent}
			\scnnote{Архитектура, в которой все узлы (процессоры или машины) имеют доступ к глобальной оперативной памяти.}
			\begin{scnsubdividing}
				\scnitem{архитектура вычислительной системы с глобальной оперативной памятью данных}
				\scnitem{архитектура вычислительной системы с глобальной оперативной памятью программ}
				\scnitem{архитектура вычислительной системы с глобальной оперативной памятью программ и данных}
				\begin{scnindent}
					\scnnote{Примером такой архитектуры вычислительной системы является архитектура фон-Неймана.}
				\end{scnindent}
			\end{scnsubdividing}
		\end{scnindent}
		\scnitem{архитектура вычислительной системы без глобальной оперативной памяти}
	\end{scnsubdividing}
	\begin{scnsubdividing}
		\scnitem{архитектура вычислительной системы с единственной глобальной внутренней памятью}
		\scnitem{архитектура вычислительной системы со множественной глобальной внутренней памятью}
	\end{scnsubdividing}
	\begin{scnsubdividing}
		\scnitem{архитектура вычислительной системы со структурно перестраиваемыми межпроцессорными связями}
		\scnitem{архитектура вычислительной системы без структурно перестраиваемых междпроцессорных связей}
	\end{scnsubdividing}
	\begin{scnsubdividing}
		\scnitem{архитектура вычислительной системы со структурно становящейся памятью}
		\scnitem{архитектура вычислительной системы без структурно становящейся памяти}
	\end{scnsubdividing}
	\begin{scnsubdividing}
		\scnitem{архитектура вычислительной системы с ассоциативным доступом к глобальной (внутренней) памяти}
		\begin{scnindent}
			\scnnote{Ассоциативный характер доступа важен в системах, ориентированных на хранение данных со сложной структурой и ориентированных на масштабируемые (в том числе локальные) механизмы обработки информации.}
		\end{scnindent}
		\scnitem{архитектура вычислительной системы без ассоциативного доступа к глобальной (внутренней) памяти}
	\end{scnsubdividing}
	\begin{scnsubdividing}
		\scnitem{архитектура вычислительной системы с адресным доступом к глобальной памяти с линейным адресным пространством}
		\begin{scnindent}
			\scnnote{Примерами таких архитектур является большинство используемых на настоящий момент, включая архитектуру фон-Неймана.}
			\begin{scnindent}
				\scnrelfrom{источник}{\scncite{VonNeuman1971}}
			\end{scnindent}
		\end{scnindent}
		\scnitem{архитектура вычислительной системы без адресного доступа к глобальной памяти с линейным адресным пространством}
	\end{scnsubdividing}
	\begin{scnsubdividing}
		\scnitem{архитектура вычислительной системы с системой команд регистровой обработки данных}
		\begin{scnindent}
			\scnnote{Большинство используемых на настоящий момент архитектур являются примерами архитектур данного класса, включая архитектуру фон-Неймана. Архитектуры с системой команд регистровой обработки данных удобны для задач управления данными как для систем обработки образов в задачах пользовательского интерфейса, так и для задач машинного обучения на основе аппарата линейной алгебры.}
		\end{scnindent}
		\scnitem{архитектура вычислительной системы без системы команд с регистровой обработкой данных}
	\end{scnsubdividing}
	\begin{scnsubdividing}
		\scnitem{архитектура вычислительной системы с системой команд стековой обработки данных}
		\begin{scnindent}
			\scnnote{Примерами применения такой архитектуры являются LISP-машины.}
			\begin{scnindent}
				\begin{scnrelfromlist}{источник}
					\scnitem{\scncite{Moon1987}}
					\scnitem{\scncite{Smith1984}}
					\scnitem{\scncite{Steele2011}}
					\scnitem{\scncite{McJones2015}}
					\scnitem{\scncite{VanderLeun2017}}
				\end{scnrelfromlist}
			\end{scnindent}
		\end{scnindent}
		\scnitem{архитектура вычислительной системы без системы команд стековой обработки данных}
	\end{scnsubdividing}
	\begin{scnsubdividing}
		\scnitem{архитектура вычислительной системы с поддержкой системы команд обработки обобщенных строк}
		\begin{scnindent}
			\scnnote{Примером такой архитектуры является архитектура вычислительной системы с поддержкой системы команд обработки списков и обобщенных строк. Эта модель позволяет эффективно производить операции не только над строками и списками, но и работать с отношениями вида \scnqq{ключ-значение} с целью их интеграции в системы, управляемые знаниями. Программная реализация этой модели использует B-деревья.}
			\begin{scnindent}
				\scnrelfrom{источник}{\scncite{Ivashenko2020String}}
			\end{scnindent}
		\end{scnindent}
		\scnitem{архитектура вычислительной системы без системы поддержки команд обработки обобщенных строк}
	\end{scnsubdividing}
	\begin{scnsubdividing}
		\scnitem{архитектура вычислительной системы с поддержкой системы команд обработки графовых структур}
		\begin{scnindent}
			\scnnote{Примером такой архитектуры является архитектура компьютера Leonhard. Этот компьютер ориентирован на обработку графовых и гиперграфовых структур различных видов, включая иерархические графы. Поддерживается представление в виде строк и списка смежных вершин, упорядоченных локальных списков инцидентных ребер, глобального упорядоченного списка инцидентных ребер.}
			\begin{scnindent}
				\begin{scnrelfromlist}{источник}
					\scnitem{\scncite{Rasheed2019}}
					\scnitem{\scncite{Dubrovin2020}}
				\end{scnrelfromlist}
			\end{scnindent}
		\end{scnindent}
		\scnitem{архитектура вычислительной системы без системы поддержки команд обработки графовых структур}
	\end{scnsubdividing}
	\begin{scnsubdividing}
		\scnitem{архитектура вычислительной системы с системой команд (аппаратной) обработки знаний}
		\begin{scnindent}
			\scnnote{Примером такой архитектуры является архитектура компьютера Leonhard. Компьютер Leonhard поддерживает системы команд DISC (Discrete Instruction Set Computing). Система команд DISC включает следующие команды: создание целочисленного отношения со схемой, являющегося множеством объектов формального контекста (первым доменом бинарного отношения), тогда как соответствующим множеством образов является множество неотрицательных целых чисел (второй домен бинарного отношения); добавление пары в формальный контекст, содержащей добавляемый объект (ключ), который добавляется как кортеж через добавление элементов этого кортежа, вместе с целочисленным образом (значением) для этого объекта; получение следующего или предыдущего объекта в линейно (лексикографически) упорядоченным списке объектов; получение следующего большего или предыдущего меньшего объекта в линейно (лексикографически) упорядоченном списке объектов; получение минимального или максимального объекта в линейно (лексикографически) упорядоченном списке объектов; получение количества (мощности множества) образов для заданного объекта (кортежа-ключа); поиск пар по ключу; удаление пар; удаление всех пар формального контекста, включая объекты (ключи) и образы (значения); срез (подмножество) формальных контекстов контекста; объединение,пересечение и дополнение формальных контекстов. Для представления обрабатываемых данных используются B+-деревья. Другие архитектуры рассматривают реализацию операций обработки знаний, используя логическую модель представления знаний, LISP-структуры, обобщенные формальные языки. В последнем случае для развития системы команд обработки знаний рассматривается переход от обработки знаний к обработки метазнаний (на базе семантики становления актуального и неактуального), результатом которого является система метаопераций. Рассмотрение подобных архитектур важно для создания систем, управляемых знаниями.}
			\begin{scnindent}
				\begin{scnrelfromlist}{источник}
					\scnitem{\scncite{Rasheed2019}}
					\scnitem{\scncite{Dubrovin2020}}
					\scnitem{\scncite{Hewitt2009}}
					\scnitem{\scncite{Moon1987}}
					\scnitem{\scncite{Smith1984}}
					\scnitem{\scncite{Ivashenko2020String}}
					\scnitem{\scncite{Ivashenko2020}}
					\scnitem{\scncite{Ivashenko2020}}
					\scnitem{\scncite{Ivashenko2016BSUIR}}
				\end{scnrelfromlist}
			\end{scnindent}
		\end{scnindent}
		\scnitem{архитектура вычислительной системы без системы команд (аппаратной) обработки знаний}
	\end{scnsubdividing}
	\begin{scnsubdividing}
		\scnitem{архитектура вычислительной системы с адаптивным распределением данных}
		\begin{scnindent}
			\scnnote{Адаптивное распределение данных (включая как частный случай виртуальное адресное пространство) важно для целей управления данными (и знаниями) и задач виртуализации для многозадачных и многопользовательских систем, а также тесно связано с возможностями масштабируемости системы.}	
		\end{scnindent}
		\scnitem{архитектура вычислительной системы без адаптивного распределения данных}
	\end{scnsubdividing}
	\begin{scnsubdividing}
		\scnitem{архитектура вычислительной системы с системой команд локальной обработки информации}
		\begin{scnindent}
			\scnnote{Примером такой архитектуры является клеточный автомат. Элементарные клеточные (двоичные) автоматы разделяются на: быстро переходящие в однородное состояние (состояние только из нулей или единиц); быстро переходящие в устойчивое или циклическое состояние; остающиеся в хаотическом (случайном) состоянии; образующие как области с устойчивым или циклическим состоянием, так и области, в которых проявляются сложные взаимодействия элементов состояний, вплоть до Тьюринг-полных.\\
			Обработка информации с помощью клеточных автоматов позволяет строить вычислительные системы в том числе с перестраиваемой (в том числе фрактало-подобной) структурой на основе локальных параллельно (конкурентно) выполняемых несложных правил. Существуют разновидности клеточных автоматов, поддерживающих необратимые, обратимые, детерминированные, недетерминированные, специализированные, универсальные (в том числе Тьюринг-полные) вычисления. Работа клеточных автоматов напоминает волновые процессы распространяющиеся в среде элементов состояния клеточного автомата.}
			\begin{scnindent}
				\scnrelfrom{источник}{\scncite{Wolfram2002}}
			\end{scnindent}
		\end{scnindent}
		\scnitem{архитектура вычислительной системы без системы команд локальной обработки информации}
	\end{scnsubdividing}
	\begin{scnsubdividing}
		\scnitem{архитектура вычислительной системы исключительно с двоичным представлением данных в оперативной памяти}
		\begin{scnindent}
			\scnnote{Большинство современных архитектур цифровых вычислительных систем, включая реализации архитектуры фон-Неймана, использует именно двоичное представление.}
		\end{scnindent}
		\scnitem{архитектура вычислительной системы не исключительно с двоичным представлением данных в оперативной памяти}
	\end{scnsubdividing}
	\begin{scnsubdividing}
		\scnitem{архитектура вычислительной системы с исключительно дискретным представлением данных}
		\begin{scnindent}
			\scnnote{Примером такой архитектуры является архитектура фон-Неймана.}
		\end{scnindent}
		\scnitem{архитектура вычислительной системы без исключительно дискретного представления данных}
	\end{scnsubdividing}
	\begin{scnsubdividing}
		\scnitem{архитектура вычислительной системы с дискретным представлением данных}
		\begin{scnindent}
			\scnsuperset{архитектура вычислительной системы с исключительно дискретным представлением данных}
		\end{scnindent}
		\scnitem{архитектура вычислительной системы без дискретного представления данных}
	\end{scnsubdividing}
	\begin{scnsubdividing}
		\scnitem{архитектура вычислительной системы с управлением от потока данных}
		\begin{scnindent}
			\scnnote{Архитектуры вычислительных систем с управлением от потока данных видятся более естественными при решении многих задач Искусственного интеллекта. Варианты таких архитектур рассмотрены в работах. Архитектуру клеточных автоматов можно рассматривать как архитектуру вычислительной системы с управлением от потока данных.}
		\end{scnindent}
		\scnitem{архитектура вычислительной системы без управления от потока данных}
	\end{scnsubdividing}
	\begin{scnsubdividing}
		\scnitem{архитектура вычислительной системы с управлением от потока команд}
		\begin{scnindent}
			\scnnote{Примером такой архитектуры является архитектура фон-Неймана.}
		\end{scnindent}
		\scnitem{архитектура вычислительной системы без управления от потока команд}
	\end{scnsubdividing}
	\begin{scnsubdividing}
		\scnitem{архитектура вычислительной системы с процессором с арифметико-логическим устройством}
		\begin{scnindent}
			\scnnote{Примером такой архитектуры является архитектура фон-Неймана.}
		\end{scnindent}
		\scnitem{архитектура вычислительной системы без процессора с арифметико-логическим устройством}
	\end{scnsubdividing}
	\begin{scnsubdividing}
		\scnitem{архитектура вычислительной системы с управляющим блоком со счетчиком инструкций}
		\begin{scnindent}
			\scnnote{Примером такой архитектуры является архитектура фон-Неймана.}
		\end{scnindent}
		\scnitem{архитектура вычислительной системы без управляющего блока со счетчиком иструкций}
	\end{scnsubdividing}
	\begin{scnsubdividing}
		\scnitem{архитектура вычислительной системы с управляющим блоком с регистром команд}
		\begin{scnindent}
			\scnnote{Примером такой архитектуры является архитектура фон-Неймана.}
		\end{scnindent}
		\scnitem{архитектура вычислительной системы без управляющего блока с регистром команд}
	\end{scnsubdividing}
	\begin{scnsubdividing}
		\scnitem{архитектура вычислительной системы с устройством ввода-вывода}
		\begin{scnindent}
			\scnnote{Примером такой архитектуры является архитектура фон-Неймана.}
		\end{scnindent}
		\scnitem{архитектура вычислительной системы без устройства ввода-вывода}
	\end{scnsubdividing}
	\begin{scnsubdividing}
		\scnitem{архитектура вычислительной системы с доступом к внешнему (оперативному) запоминающему устройству}
		\begin{scnindent}
			\scnnote{Примером такой архитектуры является архитектура фон-Неймана.}
		\end{scnindent}
		\scnitem{архитектура вычислительной системы без доступа к внешнему (оперативному) запоминающему устройству}
	\end{scnsubdividing}
	\begin{scnsubdividing}
		\scnitem{архитектура вычислительной системы с масштабируемой (модульной) глобальной памятью}
		\begin{scnindent}
			\scnnote{Масштабируемость как свойство архитектуры важно для систем, ориентированных на обучение (самообучение) с целью решения широкого класса задач. Подобные архитектуры могут быть ориентированы на обработку структур знаний, интегрированных в единое смысловое пространство.}
			\begin{scnindent}
				\begin{scnrelfromlist}{источник}
					\scnitem{\scncite{Ivashenko2019InfiniteMemory}}
					\scnitem{\scncite{Ivashenko2016BSUIR}}
				\end{scnrelfromlist}
			\end{scnindent}
		\end{scnindent}
		\scnitem{архитектура вычислительной системы без масштабируемой глобальной памяти}
	\end{scnsubdividing}
	\begin{scnsubdividing}
		\scnitem{архитектура вычислительной системы с поддержкой модели активной графовой памяти}
		\begin{scnindent}
			\scnnote{Модель активной графовой памяти в архитектурах вычисленных систем важна для эффективной и согласованной (конвергентной) реализации параллельных процессов обработки знаний, включая механизмы возбуждения и торможения процессов обработки знаний. Модель активной графовой памяти ориентирована на реализацию представления знаний в смысловом пространстве и реализацию систем, управляемых знаниями.}
			\begin{scnindent}
				\begin{scnrelfromlist}{источник}
					\scnitem{\scncite{Ivashenko2021OSTIS}}
					\scnitem{\scncite{Ivashenko2022}}
					\scnitem{\scncite{Ivashenko2016BSUIR}}
				\end{scnrelfromlist}
			\end{scnindent}
		\end{scnindent}
		\scnitem{архитектура вычислительной системы без поддержки модели активной графовой памяти}
	\end{scnsubdividing}
	\begin{scnsubdividing}
		\scnitem{архитектура вычислительной системы с поддержкой параллельной обработки информации}		
		\begin{scnindent}
			\scnnote{Архитектуры вычислительных систем с поддержкой параллельной обработки знаний важны для эффективной реализации процессов обработки знаний, повышения производительности и масштабируемости систем обработки знаний, включая многоагентные системы в виде интеллектуальных компьютерных систем и коллективов интеллектуальных компьютерных систем.}
			\begin{scnindent}
				\begin{scnrelfromlist}{источник}
					\scnitem{\scncite{Ivashenko2020ReductionScheme}}
					\scnitem{\scncite{Ivashenko2020}}
					\scnitem{\scncite{Kuzmickij2000}}
				\end{scnrelfromlist}
			\end{scnindent}
		\end{scnindent}
		\scnitem{архитектура вычислительной системы с последовательной обработкой информации}
	\end{scnsubdividing}
	\begin{scnsubdividing}
		\scnitem{архитектура вычислительной системы с поддержкой последовательной модели консистентности глобальной оперативной памяти}
		\begin{scnindent}
			\scnnote{Архитектуры с поддержкой моделей консистентности ориентированы на решение задач управления взаимодействующими процессами, включая их синхронизацию и синхронные и асинхронные механизмы исполнения алгоритмов обработки знаний. Целью поддержки последовательной модели консистентности является обеспечение существования глобальных состояний базы знаний как структур единого смыслового пространства в интеллектуальных компьютерных системах.}
			\begin{scnindent}
				\scnrelfrom{смотрите}{Предметная область и онтология решателей задач ostis-систем}
				\begin{scnrelfromlist}{источник}
					\scnitem{\scncite{Ivashenko2020}}
					\scnitem{\scncite{Ivashenko2021PRIP}}
					\scnitem{\scncite{Gaponov2000}}
					\scnitem{\scncite{Serdiukov2004}}
				\end{scnrelfromlist}
			\end{scnindent}
		\end{scnindent}
		\scnitem{архитектура вычислительной системы без поддержки последовательной модели консистентности глобальной оперативной памяти}
	\end{scnsubdividing}
	\begin{scnsubdividing}
		\scnitem{архитектура с поддержкой причинной модели консистентности памяти}
		\begin{scnindent}
			\scnnote{Архитектуры с поддержкой моделей консистентности ориентированы на решение задач управления взаимодействующими процессами, включая их синхронизацию и синхронные и асинхронные механизмы исполнения алгоритмов обработки знаний. Целью поддержки причинной модели консистентности является обеспечение интероперабельности и конвергенции в едином смысловом пространстве структур знаний агентов коллективов интеллектуальных компьютерных системах. Для обеспечение той или иной модели консистентности могут использоваться различные механизмы.}
			\begin{scnindent}
				\scnrelfrom{смотрите}{Предметная область и онтология решателей задач ostis-систем}
				\begin{scnrelfromlist}{источник}
					\scnitem{\scncite{Ivashenko2020}}
					\scnitem{\scncite{Gaponov2000}}
					\scnitem{\scncite{Serdiukov2004}}
				\end{scnrelfromlist}
			\end{scnindent}
		\end{scnindent}
		\scnitem{архитектура без поддержки причинной модели консистентности памяти}
	\end{scnsubdividing}
	\begin{scnsubdividing}
		\scnitem{архитектура вычислительной системы, обладающая асимметрией}
		\begin{scnindent}
			\scnnote{Архитектуры вычислительных систем, обладающие асимметрией важны для эволюции многоагентных систем, интеллектуальных компьютерных систем и их коллективов, в которых асимметрия рассматривается в широком смысле, в том числе и как неоднородность таких систем или коллективов. Частным случаем случаем неоднородности является разнородность и гетерогенность архитектуры, которая позволяет реализовывать как интегрированное, так и гибридные модели обработки знаний, в рамках интеллектуальных компьютерных систем и их коллективов.}
		\end{scnindent}
		\scnitem{архитектура вычислительной системы, обладающая симметрией}
	\end{scnsubdividing}
	
	\scnheader{ассоциативный семантический компьютер}
	\scntext{примечание}{Для определения архитектуры \textit{ассоциативных семантических компьютеров}, в соответствии с выявленными классами и признаками, а также выработанными общими принципами, лежащими в основе таких архитектур, необходимо в рамках соответствующего признакового пространства рассмотреть конкретные множества архитектур и провести сравнительный анализ элементов этих множеств с целью обоснования выбора (оптимальных) вариантов архитектуры \textit{ассоциативных семантических компьютеров}.}
\bigskip
\end{scnsubstruct}
\scnsourcecomment{Завершили \scnqqi{Сегмент. Анализ существующих архитектур вычислительных систем}}

