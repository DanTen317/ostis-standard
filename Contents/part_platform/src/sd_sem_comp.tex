\begin{SCn}
    \scnsectionheader{Предметная область и онтология ассоциативных семантических компьютеров для ostis-систем}
    \begin{scnsubstruct}
    	\scntext{аннотация}{В главе рассмотрены принципы реализации аппаратной платформы для реализации систем, построенных на основе Технологии OSTIS, --- ассоциативного семантического компьютера.}
    	\begin{scnrelfromlist}{ключевое понятие}
    		\scnitem{машина фон-Неймана}	
    		\scnitem{архитектура вычислительной системы}
    		\scnitem{ассоциативный семантический компьютер}
    		\scnitem{scp-компьютер}
    		\scnitem{процессорный модуль}
    		\scnitem{накопительный модуль}
    		\scnitem{терминальный модуль}
    		\scnitem{процессорный элемент}
    		\scnitem{физический канал связи}
    		\scnitem{логический канал связи}
    		\scnitem{волновая микропрограмма}
    		\scnitem{волновой язык программирования}
    	\end{scnrelfromlist}
    	\begin{scnrelfromlist}{библиографическая ссылка}
    		\scnitem{\scncite{Neumann1993}}
    		\scnitem{\scncite{NeumanMachine}}
    		\scnitem{\scncite{Glushkov1974}}
    		\scnitem{\scncite{Ajlif1973}}
    		\scnitem{\scncite{Moldovan1992}}
    		\scnitem{\scncite{Chu1976}}
    		\scnitem{\scncite{Kalynychenko1990}}
    		\scnitem{\scncite{Martin1980}}
    		\scnitem{\scncite{Ozkarahan1989}}
    		\scnitem{\scncite{Kohonen1980}}
    		\scnitem{\scncite{Ignatushhenko1981}}
    		\scnitem{\scncite{Berkovich1975}}
    		\scnitem{\scncite{Ajzerman1977}}
    		\scnitem{\scncite{Marchuk1978}}
    		\scnitem{\scncite{Prangishvili1981}}
    		\scnitem{\scncite{Zatuliver1981}}
    		\scnitem{\scncite{Ackerman1979}}
    		\scnitem{\scncite{Majers1985}}
    		\scnitem{\scncite{Glushkov1980}}
    		\scnitem{\scncite{Glushkov1978}}
    		\scnitem{\scncite{Rabinovich1995}}
    		\scnitem{\scncite{Zadyhajlo1979}}
    		\scnitem{\scncite{Schuster1979}}
    		\scnitem{\scncite{Suvorov1985}}
    		\scnitem{\scncite{Brukle1978}}
    		\scnitem{\scncite{Chu1977}}
    		\scnitem{\scncite{Kohonen1982}}
    		\scnitem{\scncite{Foster1981}}
    		\scnitem{\scncite{Ershov1982}}
    		\scnitem{\scncite{Bershtejn1975}}
    		\scnitem{\scncite{Vasilev1987}}
    		\scnitem{\scncite{Sapatyj1984}}
    		\scnitem{\scncite{Popov2019}}
    		\scnitem{\scncite{Popov2020}}
    		\scnitem{\scncite{Zhang2017}}
    		\scnitem{\scncite{Hu2021}}
    		\scnitem{\scncite{Song2016}}
    		\scnitem{\scncite{Afanasyev2021}}
    		\scnitem{\scncite{Vajncvajg1980}}
    		\scnitem{\scncite{Vajncvajg1987}}
    		\scnitem{\scncite{Somsubhra2006}}
    		\scnitem{\scncite{Rabinovich1979a}}
    		\scnitem{\scncite{Rabinovich1979b}}
    		\scnitem{\scncite{Gladun1977}}
    		\scnitem{\scncite{Gladun1987}}
    		\scnitem{\scncite{Amosov1973}}
    		\scnitem{\scncite{Zolotov1982}}
    		\scnitem{\scncite{Galushkin1997}}
    		\scnitem{\scncite{Heht-Nilsen1998}}
    		\scnitem{\scncite{Komarcova2004}}
    		\scnitem{\scncite{USB_Accelerator}}
    		\scnitem{\scncite{Moussa2013}}
    		\scnitem{\scncite{Altay}}
    		\scnitem{\scncite{Allen1989}}
    		\scnitem{\scncite{CUDA}}
    		\scnitem{\scncite{OpenCL}}
    		\scnitem{\scncite{Tran2018}}
    		\scnitem{\scncite{Shi2018}}
    		\scnitem{\scncite{Lu2021}}
    		\scnitem{\scncite{Golenkov1984}}
    		\scnitem{\scncite{Golenkov1994f}}
    		\scnitem{\scncite{Golenkov1994g}}
    		\scnitem{\scncite{Golenkov1996}}
    		\scnitem{\scncite{Gaponov2000}}
    		\scnitem{\scncite{Kuzmickij2000}}
    		\scnitem{\scncite{Serdiukov2004}}
    		\scnitem{\scncite{Ivashenko2021OSTIS}}
    		\scnitem{\scncite{Ivashenko2016Tatur}}
    		\scnitem{\scncite{Ivashenko2015Tatur}}
    		\scnitem{\scncite{Rasheed2019}}
    		\scnitem{\scncite{Dubrovin2020}}
    		\scnitem{\scncite{Wolfram2002}}
    		\scnitem{\scncite{VonNeuman1971}}
    		\scnitem{\scncite{Moon1987}}
    		\scnitem{\scncite{Smith1984}}
    		\scnitem{\scncite{Steele2011}}
    		\scnitem{\scncite{McJones2015}}
    		\scnitem{\scncite{VanderLeun2017}}
    		\scnitem{\scncite{Ivashenko2020String}}
    		\scnitem{\scncite{Hewitt2009}}
    		\scnitem{\scncite{Ivashenko2020}}
    		\scnitem{\scncite{Ivashenko2016BSUIR}}
    		\scnitem{\scncite{Ivashenko2019InfiniteMemory}}
    		\scnitem{\scncite{Ivashenko2022}}
    		\scnitem{\scncite{Ivashenko2020ReductionScheme}}
    		\scnitem{\scncite{Ivashenko2021PRIP}}
    		\scnitem{\scncite{LegUp}}
    		\scnitem{\scncite{VHDPlus}}
    		\scnitem{\scncite{SystemC}}
    		\scnitem{\scncite{MyHDL}}
    		\scnitem{\scncite{Sapatyj1986}}
    		\scnitem{\scncite{Moldovan1985}}
    		\scnitem{\scncite{Letichevskij2003}}
    		\scnitem{\scncite{Letichevskij2012}}
    		\scnitem{\scncite{Backus1978}}
    		\scnitem{\scncite{Kotov1966}}
    	\end{scnrelfromlist}
    	\scntext{введение}{Применение для разработки \textit{ostis-систем} современных программно-аппаратных платформ, ориентированных на адресный доступ к хранящимся в памяти данным, не всегда оказывается эффективным, поскольку при разработке интеллектуальных систем фактически приходится моделировать нелинейную память на базе линейной. Повышение эффективности решения задач интеллектуальными системами требует разработки специализированных платформ, в том числе аппаратных, ориентированных на унифицированные семантические модели представления и обработки информации. Таким образом, основной целью создания \textit{ассоциативных семантических компьютеров} является повышение производительности ostis-систем.}
    	
        \scnheader{Предметная область семантических ассоциативных компьютеров для ostis-систем}
        \scniselement{предметная область}
        \begin{scnhaselementrolelist}{класс объектов исследования}
            \scnitem{семантический ассоциативный компьютер}
        \end{scnhaselementrolelist}
        \begin{scnhaselementrolelist}{класс объектов исследования}
            \scnitem{информационно-логическая задача}
            \scnitem{машина, ориентированная на решение информационно-логических задачи}
        \end{scnhaselementrolelist}
        
       	\begin{scnreltovector}{конкатенация сегментов}
        	\scnitem{Сегмент. Современное состояние работ в области разработки компьютеров для интеллектуальных систем}
        \end{scnreltovector}
        
        \scnsegmentheader{Сегмент. Современное состояние работ в области разработки компьютеров для интеллектуальных систем}
\begin{scnsubstruct}
   	\scntext{введение}{Подавляющее большинство современных программно-аппаратных платформ, применяемых при разработке современных компьютерных систем, и, в частности, интеллектуальных компьютерных систем, основаны на принципах абстрактной \textit{машины фон-Неймана} или архитектуры фон-Неймана. Рассмотрим основные принципы, лежащие в основе \textit{машины фон-Неймана}.}
   	\begin{scnindent}
   		\begin{scnrelfromlist}{источник}
   			\scnitem{\scncite{Neumann1993}}
   			\scnitem{\scncite{NeumanMachine}}
   		\end{scnrelfromlist}
	\end{scnindent}
   	
	\scnheader{машина фон-Неймана}
	\scnidtf{абстрактная машина фон-Неймана}
	\scniselement{абстрактная машина обработки информации}
	\begin{scnrelfromvector}{принципы, лежащие в основе}
		\scnfileitem{Информация в памяти представляется в виде последовательности строк символов в бинарном алфавите (0 или 1).}
		\scnfileitem{Память машины представляет собой последовательность \uline{адресуемых} ячеек памяти.}
		\scnfileitem{В каждую ячейку может быть записана любая строка символов в бинарном алфавите. При этом длина строк для всех адресуемых ячеек одинакова (в текущем стандарте ячеек, называемых байтами, равна 8 бит).}
		\scnfileitem{Каждой ячейке памяти взаимно однозначно соответствует битовая строка, обозначающая эту ячейку и являющаяся ее адресом.}
		\scnfileitem{Каждому типу элементарных действий (операций), выполняемых в памяти машины фон-Неймана, взаимно однозначно ставится ее идентификатор, который в памяти представляется также в виде битовой строки.}
		\scnfileitem{Каждая конкретная операция (команда), выполняемая в памяти, представляется (специфицируется) в памяти в виде строки, состоящей
			\begin{scnitemize}
				\item из кода соответствущего типа операции;
				\item из последовательности адресов фрагментов памяти, в которых находятся операнды, над которыми выполняются операции --- исходные аргументы и результаты. Любой такой фрагмент задается адресом первого байта и количеством байт. Количество операндов \uline{однозначно} задается кодом типа операции;
			\end{scnitemize}}
		\scnfileitem{Программа, выполняемая в памяти, хранится в памяти в виде последовательности спецификаций конкретных операций (команд).}
		\scnfileitem{Таким образом, и обрабатываемые данные, и программы для обработки этих данных хранятся в одной и той же памяти (в отличие, например, от Гарвардской архитектуры) и кодируются одинаковым образом.}
	\end{scnrelfromvector}
	\scnrelfrom{особенности логической организации}{Особенности логической организации фон-Неймановской архитектуры}
	\begin{scnindent}
		\begin{scneqtoset}
			\scnfileitem{Последовательная обработка, ограничивающая эффективность компьютеров физическими возможностями элементной базы.}
			\scnfileitem{Низкий уровень доступа к памяти, то есть сложность и громоздкость выполнения процедуры ассоциативного поиска нужного фрагмента знаний.}
			\scnfileitem{Линейная организация памяти и чрезвычайно простой вид конструктивных объектов, непосредственно хранимых в памяти. Это приводит к тому, что в интеллектуальных системах, построенных на базе современных компьютеров, манипулирование знаниями осуществляется с большим трудом. Во-первых, приходится оперировать не самими структурами, а их громоздкими линейными представлениями (списками, матрицами смежности, матрицами инцидентности); во-вторых, линеаризация сложных структур разрушает локальность их преобразований.}
			\scnfileitem{Представление информации в памяти современных компьютеров имеет уровень весьма далекий от смыслового, что делает переработку знаний довольно громоздкой, требующей учета большого количества деталей, касающейся не смысла перерабатываемой информации, а способа ее представления в памяти.}
			\scnfileitem{В современных компьютерах имеет место весьма низкий уровень аппаратно реализуемых операций над нечисловыми данными и полностью отсутствует аппаратная поддержка логических операций над фрагментами знаний, имеющих сложную структуру, что делает манипулирование такими фрагментами весьма сложным.}
		\end{scneqtoset}
		\scntext{примечание}{Данные особенности архитектуры существенно затруднят эффективную реализацию \textit{ostis-систем} на ее основе.}
	\end{scnindent}
	\scnrelfrom{альтернативные подходы}{Альтернативные фон-Неймановским подходы организации ЭВМ}
	\begin{scnindent}
		\scntext{примечание}{Попытки преодоления ограничений традиционных фон-Неймановских ЭВМ привели к возникновению множества подходов, связанных с отдельными изменениями принципов логической организации ЭВМ, прежде всего в зависимости от классов задач и предметных областей, на которые ориентируется тот или иной класс ЭВМ. Все эти тенденции, рассмотренные в совокупности, позволяют очертить некоторые ключевые принципы логической организации ЭВМ, ориентированных на переработку знаний (машин переработки знаний --- МПЗ).}
		\begin{scneqtoset}
			\scnfileitem{Переход к нелинейной организации памяти и аппаратная интерпретация сложных структур данных.}
			\begin{scnindent}
				\begin{scnrelfromlist}{источник}
					\scnitem{\scncite{Glushkov1974}}
					\scnitem{\scncite{Ajlif1973}}
					\scnitem{\scncite{Moldovan1992}}
					\scnitem{\scncite{Chu1976}}
					\scnitem{\scncite{Kalynychenko1990}}
				\end{scnrelfromlist}
			\end{scnindent}
			\scnfileitem{Аппаратная реализация ассоциативного доступа к информации.}
			\begin{scnindent}
				\begin{scnrelfromlist}{источник}
					\scnitem{\scncite{Martin1980}}
					\scnitem{\scncite{Ozkarahan1989}}
					\scnitem{\scncite{Glushkov1974}}
					\scnitem{\scncite{Kohonen1980}}
					\scnitem{\scncite{Ignatushhenko1981}}
					\scnitem{\scncite{Berkovich1975}}
					\scnitem{\scncite{Ajzerman1977}}
				\end{scnrelfromlist}
			\end{scnindent}
			\scnfileitem{Реализация параллельных асинхронных процессов над памятью и, в частности, разработка вычислительных машин, управляемых потоком данных.}
			\begin{scnindent}
				\begin{scnrelfromlist}{источник}
					\scnitem{\scncite{Glushkov1974}}
					\scnitem{\scncite{Marchuk1978}}
					\scnitem{\scncite{Prangishvili1981}}
					\scnitem{\scncite{Zatuliver1981}}
					\scnitem{\scncite{Ackerman1979}}
					\scnitem{\scncite{Majers1985}}
				\end{scnrelfromlist}
			\end{scnindent}
			\scnfileitem{Аппаратная интерпретация языков высокого уровня.}
			\begin{scnindent}
				\begin{scnrelfromlist}{источник}
					\scnitem{\scncite{Glushkov1980}}
					\scnitem{\scncite{Glushkov1978}}
					\scnitem{\scncite{Rabinovich1995}}
				\end{scnrelfromlist}
			\end{scnindent}
			\scnfileitem{Разработка аппаратных средств ведения баз данных (процессоров баз данных).}
			\begin{scnindent}
				\begin{scnrelfromlist}{источник}
					\scnitem{\scncite{Zadyhajlo1979}}
					\scnitem{\scncite{Schuster1979}}
					\scnitem{\scncite{Suvorov1985}}
				\end{scnrelfromlist}
			\end{scnindent}
		\end{scneqtoset}
	\end{scnindent}
	
	\scnheader{классы вычислительных устройств}
	\scntext{примечание}{На пересечении \scnkeyword{Альтернативных фон-Неймановским подходов организации ЭВМ} в разное время возникали различные классы вычислительных устройств}
	\scnhaselement{машины с аппаратной интерпретацией сложных структур данных}
	\begin{scnindent}
		\begin{scnrelfromlist}{источник}
			\scnitem{\scncite{Brukle1978}}
			\scnitem{\scncite{Chu1976}}
			\scnitem{\scncite{Chu1977}}
		\end{scnrelfromlist}
	\end{scnindent}
	\scnhaselement{машины с развитой ассоциативной памятью}
	\begin{scnindent}
		\begin{scnrelfromlist}{источник}
			\scnitem{\scncite{Kohonen1980}}
			\scnitem{\scncite{Kohonen1982}}
			\scnitem{\scncite{Foster1981}}
		\end{scnrelfromlist}
	\end{scnindent}
	\scnhaselement{ассоциативные параллельные матричные процессоры}
	\begin{scnindent}
		\scnrelfrom{источник}{\scncite{Ershov1982}}
	\end{scnindent}
	\scnhaselement{однородные параллельные структуры для решения комбинаторно-логических задач на графах и гиперграфах}
	\begin{scnindent}
		\scnrelfrom{источник}{\scncite{Bershtejn1975}}
	\end{scnindent}
	\scnhaselement{устройства переработки графов}
	\begin{scnindent}
	\begin{scnrelfromlist}{источник}
		\scnitem{\scncite{Vasilev1987}}
		\scnitem{\scncite{Sapatyj1984}}
		\scnitem{\scncite{Popov2019}}
		\scnitem{\scncite{Popov2020}}
	\end{scnrelfromlist}
	\scnsuperset{устройства переработки графов на основе FPGA}
	\begin{scnindent}
	\begin{scnrelfromlist}{источник}
		\scnitem{\scncite{Zhang2017}}
		\scnitem{\scncite{Hu2021}}
		\scnitem{\scncite{Song2016}}
	\end{scnrelfromlist}
	\end{scnindent}
	\scnsuperset{устройства переработки графов на основе векторных процессоров}
	\begin{scnindent}
		\scnrelfrom{источник}{\scncite{Afanasyev2021}}
	\end{scnindent}
	\end{scnindent}
	\scnhaselement{системы, осуществляющие переработку информации непосредственно в памяти путем равномерного распределения функциональных средств по памяти и, в частности, предложенная М. Н. Вайнцвайгом процессоро-память, ориентированная на решение задач Искусственного интеллекта}
	\begin{scnindent}
		\begin{scnrelfromlist}{источник}
			\scnitem{\scncite{Vajncvajg1980}}
			\scnitem{\scncite{Vajncvajg1987}}
		\end{scnrelfromlist}
	\end{scnindent}
	\scnhaselement{машины, управляемые потоком данных}
	\begin{scnindent}
		\begin{scnrelfromlist}{источник}
			\scnitem{\scncite{Ershov1982}}
			\scnitem{\scncite{Prangishvili1981}}
			\scnitem{\scncite{Majers1985}}
		\end{scnrelfromlist}
		\scnsuperset{процессоры, реконфигурируемые с учетом семантики входного потока данных}
		\begin{scnindent}
			\scnrelfrom{источник}{\scncite{Somsubhra2006}}
		\end{scnindent}
	\end{scnindent}
	\scnhaselement{рекурсивные вычислительные машины}
	\begin{scnindent}
		\scnrelfrom{источник}{\scncite{Glushkov1974}}
	\end{scnindent}
	\scnhaselement{процессоры реляционных баз данных}
	\begin{scnindent}
		\begin{scnrelfromlist}{источник}
			\scnitem{\scncite{Zadyhajlo1979}}
			\scnitem{\scncite{Majers1985}}
			\scnitem{\scncite{Ozkarahan1989}}
		\end{scnrelfromlist}
	\end{scnindent}
	\scnhaselement{вычислительные машины со структурно-перестраиваемой памятью}
	\begin{scnindent}
		\begin{scnrelfromlist}{источник}
			\scnitem{\scncite{Rabinovich1979a}}
			\scnitem{\scncite{Rabinovich1979b}}
			\scnitem{\scncite{Gladun1977}}
			\scnitem{\scncite{Gladun1987}}
		\end{scnrelfromlist}
	\end{scnindent}
	\scnhaselement{активные семантические сети}
	\begin{scnindent}
		\scnidtf{М-сети}
		\scnrelfrom{источник}{\scncite{Amosov1973}}
	\end{scnindent}
	\scnhaselement{ассоциативные однородные среды}
	\begin{scnindent}
		\scnrelfrom{источник}{\scncite{Zolotov1982}}
	\end{scnindent}
	\scnhaselement{нейроподобные структуры}
	\begin{scnindent}
		\begin{scnrelfromlist}{источник}
			\scnitem{\scncite{Galushkin1997}}
			\scnitem{\scncite{Heht-Nilsen1998}}
		\end{scnrelfromlist}
		\scntext{примечание}{В последние годы активное развитие теории искусственных нейронных сетей привело к развитию различных подходов к построению высокопроизводительных компьютеров, предназначенных для обучения и интерпретации искусственных нейронных сетей и их внедрению в различные программно-аппаратные комплексы.}
		\begin{scnindent}
			\begin{scnrelfromlist}{источник}
				\scnitem{\scncite{Komarcova2004}}
				\scnitem{\scncite{USB_Accelerator}}
				\scnitem{\scncite{Moussa2013}}
			\end{scnrelfromlist}
		\end{scnindent}
		\scntext{примечание}{В отдельное направление выделены так называемые нейроморфные процессоры, отличающиеся высокой производительностью и низким уровнем энергопотребления.}
		\begin{scnindent}
			\scnrelfrom{источник}{\scncite{Altay}}
		\end{scnindent}
	\end{scnindent}
	\scnhaselement{машины для интерпретации логических правил}
	\begin{scnindent}
		\scnrelfrom{источник}{\scncite{Allen1989}}
	\end{scnindent}
	\scntext{примечание}{Развитие графических процессоров (graphics processing unit, GPU) привело к возможности организации параллельных вычислений непосредственно на GPU, для чего разрабатываются специализированные программно-аппаратные архитектуры, например CUDA. Преимуществом GPU в данном случае выступает наличие в рамках одного GPU большого (по сравнению с центральным процессором) числа ядер, что позволяет эффективно решать на такой архитектуре задачи, обладающие естественным параллелизмом (например, операции с матрицами). Развиваются также работы, посвященные принципам обработки графовых структур на GPU.}
	\begin{scnindent}
		\begin{scnrelfromlist}{источник}
			\scnitem{\scncite{CUDA}}
			\scnitem{\scncite{OpenCL}}
			\scnitem{\scncite{Tran2018}}
			\scnitem{\scncite{Shi2018}}
			\scnitem{\scncite{Lu2021}}
		\end{scnrelfromlist}
	\end{scnindent}
	\scntext{примечание}{Благодаря повышению производительности современных ЭВМ число разработок специализированных аппаратных решений в последние десятилетия снизилось, поскольку многие сложные вычислительные задачи в настоящее время за приемлемое время могут быть решены и на традиционных универсальных архитектурах. Как показано выше, исключение составляют в основном специализированные компьютеры для обработки искусственных нейронных сетей и других графовых моделей, что обусловлено большой востребованностью таких моделей и их сложностью.}
	\scntext{примечание}{Большинство перечисленных подходов (даже если они достаточно далеко отходят от предложенных фон-Нейманом базовых принципов организации вычислительных машин) неявно сохраняют точку зрения на компьютер как на большой арифмометр и тем самым сохраняют ее ориентацию на задачи числового характера. Работы же направленные на разработку аппаратных архитектур, предназначенных для обработки информации, представленной в более сложных формах, чем в традиционных архитектурах не получили широкого распространения и применения, по причине, во-первых, частности предлагаемых решений, и во-вторых из-за отсутствия общего универсального и унифицированного языка кодирования любой информации, в роли которого в рамках \textit{Технологии OSTIS} выступает \textit{SC-код}, а также соответствующей апробированной технологии разработки программных систем для таких аппаратных архитектур. Таким образом, зачастую разработчики подобных архитектур сталкиваются необходимостью разработки специализированного программного обеспечения для этих архитектур, что в конечном итоге приводит к сильному ограничению сфер применения таких архитектур, поскольку их применение оказывается целесообразным только в случае, если трудоемкость разработки специализированного программного обеспечения оправдывает себя с учетом низкой эффективности решения соответствующих задач на более традиционных архитектурах.}
	
	\scnheader{ассоциативный семантический компьютер}
	\scntext{примечание}{\textit{SC-код}, являющийся формальной основой \textit{Технологии OSTIS} изначально разрабатывался как язык кодирования информации в памяти \textit{ассоциативных семантических компьютеров}, таким образом в нем изначально заложены такие принципы, как универсальность (возможность представить знания любого рода) и унификация (единообразие) представления, а также минимизация \textit{Алфавита SC-кода}, которая, в свою очередь, позволяет облегчить создание аппаратной платформы, позволяющей хранить и обрабатывать тексты \textit{SC-кода}.}
	\scntext{примечание}{Основная методологическая особенность предлагаемого подхода к разработке средств аппаратной реализации поддержки интеллектуальных систем заключается в том, что такие средства должны разрабатываться не до, а \uline{после того, как} основные положения соответствующей \uline{технологии} проектирования и эксплуатации интеллектуальных систем будут апробированы на современных технических средствах. Более того, в рамках \textit{\textit{Технологии OSTIS}} четко продумана методика перехода на новые аппаратные средства, которая затрагивает только самый нижний уровень технологии --- уровень реализации базовой машины обработки семантических сетей (интерпретатора \textit{Языка SCP}).}
	\begin{scnrelfromvector}{основные этапы истории}
		\scnfileitem{1984 год --- в Московском институте электронной техники В.В. Голенковым защищена кандидатского диссертация на тему \scnqq{Структурная организация и переработка информации в электронных математических машинах, управляемых потоком сложноструктурированных данных}, в которой были сформулированы и рассмотрены основные принципы семантических ассоциативных компьютеров.}
		\begin{scnindent}
			\scnrelfrom{источник}{\scncite{Golenkov1984}}
		\end{scnindent}
		\scnfileitem{1993 год --- комиссия Госкомпрома провела успешные испытания прототипа \textit{ассоциативного семантического компьютера}, разработанного на базе транспьютеров в рамках научно-исследовательского проекта \scnqq{Параллельная графовая вычислительная система, ориентированная на решение задач искусственного интеллекта}.}
			\begin{scnindent}
			\begin{scnrelfromlist}{источник}
				\scnitem{\scncite{Golenkov1994f}}
				\scnitem{\scncite{Golenkov1994g}}
			\end{scnrelfromlist}
		\end{scnindent}
		\scnfileitem{1996 год --- В.В. Голенковым защищена докторская диссертация на тему \scnqq{Графодинамические модели и методы параллельной асинхронной переработки информации в интеллектуальных системах}.}
		\begin{scnindent}
			\scnrelfrom{источник}{\scncite{Golenkov1996}}
		\end{scnindent}
		\scnfileitem{2000 год --- в Институте проблем управления РАН П.А. Гапоновым защищена кандидатская диссертация на тему \scnqq{Модели и методы параллельной асинхронной переработки информации в графодинамической ассоциативной памяти}.}
		\begin{scnindent}
			\scnrelfrom{источник}{\scncite{Gaponov2000}}
		\end{scnindent}
		\scnfileitem{2000 год --- в Институте программных систем РАН В.М. Кузьмицким защищена кандидатская диссертация на тему \scnqq{Принципы построения графодинамического параллельного компьютера, ориентированного на решение задач искусственного интеллекта}.}
		\begin{scnindent}
			\scnrelfrom{источник}{\scncite{Kuzmickij2000}}
		\end{scnindent}
		\scnfileitem{2004 год --- в Белорусском государственном университете информатики и радиоэлектроники Р.Е. Сердюковым защищена кандидатская диссертация на тему \scnqq{Базовые алгоритмы и инструментальные средства обработки информации в графодинамических ассоциативных машинах}, в которой было рассмотрено базовое программное обеспечение семантических ассоциативных компьютеров.}
		\begin{scnindent}
			\scnrelfrom{источник}{\scncite{Serdiukov2004}}
		\end{scnindent}
	\end{scnrelfromvector}
	\scntext{примечание}{В тоже время, несмотря на наличие действующего прототипа \textit{ассоциативного семантического компьютера} на базе транспьютеров, основное внимание в рамках соответствующего проекта и других перечисленных работ уделялось принципам организации распределенной параллельной обработки конструкций SC-кода, в частности, был разработан \textit{SCD-код} (Semantic Code Distributed) для распределенного хранения конструкций SC-кода и \textit{Язык SCPD} для распределенной параллельной их обработки. Однако, общие принципы хранения информации и общая архитектура каждого из процессорных элементов (транспьютеров) оставались фон-Неймановскими. В частности, для кодирования конструкций SC-кода в традиционной адресной памяти были разработаны соответствующие структуры данных.}
	\begin{scnindent}
		\scnrelfrom{смотрите}{Предметная область и онтология программных вариантов реализации базового интерпретатора логико-семантических моделей ostis-систем на современных компьютерах}
	\end{scnindent}
	\scntext{примечание}{Обоснованность и необходимость разработки \textit{ассоциативного семантического компьютера}, а также компетентность авторов в данной области подтверждается более чем 30-летним опытом работы и рядом успешных проектов в данном направлении, однако, в тоже время, в предшествующих работах в полной мере не устранены все недостатки фон-Неймановской архитектуры, рассмотренные выше, и разработка и реализация проекта \textit{ассоциативного семантического компьютера}, устраняющего перечисленные недостатки, остается актуальной.}
\bigskip
\end{scnsubstruct}
\scnsourcecomment{Завершили \scnqqi{Сегмент. Современное состояние работ в области разработки компьютеров для интеллектуальных систем}}

        
        \scnheader{семантический ассоциативный компьютер}
        \scnidtf{аппаратно реализованный интерпретатор семантических моделей (sc-моделей) компьютерных систем}
        \scnidtf{семантический ассоциативный компьютер, управляемый знаниями}
        \scnidtf{компьютер с нелинейной структурно перестраиваемой (графодинамической) ассоциативной памятью, переработка информации в которой сводится не к изменению состояния элементов памяти, а к изменению конфигурации связей между ними}
        \scnidtf{sc-компьютер}
        \scnidtf{scp-компьютер}
        \scnidtf{универсальный компьютер нового поколения, специально предназначенный для реализации семантически совместимых гибридных интеллектуальных компьютерных систем}
        \scnidtf{универсальный компьютер нового поколения, ориентированный на аппаратную интерпретацию логико-семантических моделей интеллектуальных компьютерных систем}
        \scnidtf{универсальный компьютер нового поколения, ориентированный на аппаратную интерпретацию ostis-систем}
        \scnidtf{ostis-компьютер}
        \scnidtf{компьютер для реализации ostis-систем}
        \scnidtf{компьютер, управляемый знаниями, представленными в SC-коде}
        \scnidtf{компьютер, ориентированный на обработку текстов SC-кода}
        \begin{scnrelfromset}{принципы, лежащие в основе}
            \scnfileitem{нелинейная память --- каждый элементарный фрагмент хранимого в памяти текста может быть инцидентен неограниченному числу других элементарных фрагментов этого текста}
            \scnfileitem{структурно перестраиваемая (реконфигурируемая) память --- процесс отработки хранимой в памяти информации сводится не только к изменению состояния элементов, но и к реконфигурации связей между ними}
            \scnfileitem{в качестве внутреннего способа кодирования знаний, хранимых в памяти семантического ассоциативного компьютера, используется универсальный (!) способ нелинейного (графоподобного) смыслового представления знаний, названный нами SC-кодом (семантическим, смысловым компьютерным кодом)}
            \scnfileitem{обработка информации осуществляется коллективом агентов, работающих над общей памятью. Каждый из них реагирует на соответствующую ему ситуацию или событие в памяти (компьютер, управляемый хранимыми знаниями)}
            \scnfileitem{есть программно реализуемые агенты, поведение которых описывается хранимыми в памяти агентно-ориентированными программами, которые интерпретируются соответствующими коллективами агентов}
            \scnfileitem{есть базовые агенты, которые не могут быть реализованы программно (в частности, это агенты интерпретации агентных программ, базовые рецепторные агенты-датчики, базовые эффекторные агенты)}
            \scnfileitem{все агенты работают над общей памятью одновременно. Более того, если для какого-либо агента в некоторый момент времени в различных частях памяти возникает сразу несколько условий его применения, разные акты указанного агента в разных частях памяти могут выполняться одновременно (акт агента --- это неделимый, целостный процесс деятельности агента)}
            \scnfileitem{для того, чтобы акты агентов, параллельно выполняемые в общей памяти не мешали друг другу, для каждого акта в памяти фиксируется и постоянно актуализируется его текущее состояние. То есть каждый акт сообщает всем остальным о своих намерениях и пожеланиях, которым остальные агенты не должны препятствовать (например, это различного рода блокировки используемых элементов семантической памяти)}
            \scnfileitem{кроме того, агенты (точнее, выполняемые ими акты) должны соблюдать этику, стараясь не в ущерб себе создавать максимально благоприятные условия для других агентов (актов), например, не жадничать, быстрее возвращать, не захватывать (не блокировать) лишние элементы памяти, как можно скорее освобождать (деблокировать) заблокированные элементы памяти}
            \scnfileitem{процессор и память семантического ассоциативного компьютера глубоко интегрированы и составляют единую процессоро-память. Процессор семантического ассоциативного компьютера равномерно распределен по его памяти так, что процессорные элементы одновременно являются и элементами памяти компьютера. Обработка информации в семантическом ассоциативном компьютере сводится к реконфигурации каналов связи между процессорными элементами,  следовательно память такого компьютера есть не что иное, как \uline{коммутатор} (!) указанных каналов связи. Таким образом, текущее состояние конфигурации этих каналов связи и есть текущее состояние обрабатываемой информации}
        \end{scnrelfromset}
        \scntext{основная цель}{Повысить производительность компьютерных систем, основанных на знаниях, представленных в виде семантических сетей}
        \scntext{проблема создания}{Традиционная архитектура компьютерных систем не позволяет эффективно использовать сложно-структурированные знания в реальном масштабе времени.Решение проблемы функционирования компьютерных систем обработки знаний в реальном масштабе времени трудно достижимо без использования принципов параллельной обработки и ассоциативного доступа к структурам данных. Разработка подобных систем чрезвычайно трудоемка, поэтому для обеспечения ее экономической целесообразности жизненный цикл функционирования подобных систем должен быть достаточно продолжительным (от десятка лет и более). Сменяемость аппаратных и программных средств вычислительной техники в настоящее время достигает от нескольких лет до нескольких месяцев. Это обуславливает необходимость наличия у \textit{интеллектуальных компьютерных систем} таких качеств, как:
            \begin{scnitemize}
                \item открытость (в плане модифицируемости и добавления новых методов представления и переработки знаний);
                \item интегрируемость (в смысле возможности функционирования в составе комплексов различных вычислительных и исполнительных средств).
            \end{scnitemize}
            Анализ современных систем обработки знаний показывает, что ни одна из архитектур, на базе которых они реализованы, не обладает в совокупности всеми указанными выше свойствами.}
        \scntext{пояснение}{Особенности логической организации вычислительных систем, которые существенно затрудняют создание машин переработки знаний на основе традиционных ЭВМ:
            \begin{scnitemize}
                \item последовательная обработка, ограничивающая эффективность ЭВМ физическими возможностями элементной базы;
                \item низкий уровень доступа к памяти, т.е. сложность и громоздкость выполнения процедуры ассоциативного поиска нужного фрагмента знаний.
                \item линейная организация памяти и чрезвычайно простой вид конструктивных объектов, непосредственно хранимых в памяти. Это приводит к тому, что в интеллектуальных системах, построенных на базе современных ЭВМ, манипулирование знаниями осуществляется с большим трудом. Во-первых, приходится оперировать не самими структурами, а их громоздкими линейными представлениями (списками, матрицами смежности, матрицами инцидентности); во-вторых, линеаризация сложных структур разрушает локальность их преобразований;
                \item представление информации в памяти современных ЭВМ имеет уровень весьма далекий от семантического, что делает переработку знаний довольно громоздкой, требующей учета большого количества деталей, касающейся не смысла перерабатываемой информации, а способа ее представления в памяти;
                \item в современных ЭВМ имеет место весьма низкий уровень аппаратно реализуемых операций над нечисловыми данными и полностью отсутствует аппаратная поддержка логических операций над фрагментами знаний, имеющих сложную структуру, что делает манипулирование такими фрагментами весьма сложным.
            \end{scnitemize}
            Основным проявлением перечисленных недостатков традиционных ЭВМ является гипертрофированное развитие программного обеспечения, сложность его создания и невысокая надежность.Трудности, связанные с использованием универсальных многопроцессорных ЭВМ, послужили толчком к развитию теоретических исследований в области параллельного программирования.}
        \scnrelfrom{формальная основа решения проблемы}{\scnkeyword{SC-код}
        }
        \scnidtf{Предлагаемое в рамках \textit{Технологии OSTIS} уточнение понятия \textit{семантической сети}}
        \scnidtf{Semantic Computer Code}
        \scntext{примечание}{\textit{SC-код} обладает фиксированными синтаксисом и денотационной семантикой, но нефиксированной операционой семантикой. Следовательно, позволяет на его основе создать семейство языков представления знаний различного вида.\textit{SC-код} изначально разрабатывался как язык кодирования информации в памяти \textit{семантических ассоциативных компьютеров}, таким образом в нем изначально заложены такие принципы, как универсальность (возможность представить знания любого рода) и унификация (единообразие) представления, а также минимизацию \textit{Алфавита SC-кода}, которая, в свою очередь, позволяет облегчить создание аппаратной платформы, позволяющей хранить и обрабатывать тексты \textit{SC-кода}.}
        \scnheader{информационно-логическая задача}
        \scnidtf{информационно-логическая или комбинаторная задачи по переработке сложноструктурированных баз данных}
        \scnidtf{класс задач, предполагающих переработку нечисловой сложноструктурированной информации и допускающих при этом отсутствие точного алгоритма их решения}
        \scntext{примечание}{Понятие \textit{информационно-логических задач} фактически совпадает по смыслу с широко используемым в последнее время понятием задач искусственного интеллекта (\textit{интеллектуальных задач}), что позволяет использовать оба эти термина.}
        \scntext{примечание}{Разработка средств решения задач того или иного класса в настоящее время обычно осуществляется путем создания \textit{языка программирования высокого уровня}, ориентированного на этот класс задач, и путем реализации такого языка на современных компьютерах, т.е. путем создания транслятора. Поскольку логика решения задач искусственного интеллекта плохо согласуется с современными языками программирования, более целесообразной является разработка принципиально новых языков, отражающих на уровне их элементарных операций основные логические механизмы решения задач рассматриваемого класса. Такие языки программирования обычно называют языками сверх-высокого уровня или непроцедурными языками. Реализация языков сверх-высокого уровня на современных компьютерах представляется весьма сложной в силу большого разрыва между языками этого класса и внутренними языками современных компьютеров, для преодоления которого создание эффективного транслятора оказывается практически невозможным.}
        \scntext{пояснение}{Таким образом, состояние проблемы автоматизации решения задач искусственного интеллекта (информационно-логических задач) в настоящее время можно охарактеризовать тем, что эта проблема входит в противоречие c принципами логической организации современных компьютеров и, в первую очередь, с используемыми в современных компьютерах внутренними языками. Причины плохой приспособленности современных компьютеров к решению информационно-логических задач:
            \begin{scnitemize}
                \item в современных компьютерах при работе со сложноструктурированными базами данных время, затрачиваемое на информационный поиск, на 2-3 порядка превышает время, затрачиваемое на собственно переработку;
                \item в современных компьютерах имеет место весьма низкий уровень аппаратно реализуемых операций над нечисловыми данными;
                \item представление информации в памяти современных компьютеров имеет уровень весьма далекий от семантического, что делает переработку информации довольно громоздкой, требующей учета большого количества деталей, касающихся не смысла перерабатываемой информации, a способа ее представления в памяти;
                \item в современных компьютерах громоздко реализуются даже простейшие процедуры логического вывода.
            \end{scnitemize}
            Перечисленные причины, по существу, не устраняются также и в развиваемых в настоящее время подходах к построению нетрадиционных высокопроизводительных компьтеров, ибо, в основном, все эти подходы (даже если они достаточно далеко отходят от предложенных фон Нейманом принципов организации вычислительных машин) неявно сохраняют точку зрения на компьютер как на большой арифмометр и тем самым сохраняют ее ориентацию на задачи числового характера. Очевидно, что эффективность этих машин будет прежде всего определяться степенью близости их внутреннего языка к языкам непроцецурного типа (к языкам сверх-высокого уровня). Учитывая указанное назначение таких машин, их естественно назвать не вычислительными машинами, a математическими машинами или даже мыслящими машинами.}
        \scnheader{машина, ориентированная на решение информационно-логических задач}
        \scnrelfrom{класс решаемых задач}{информационно-логическая задача}
        \scnidtf{машина, реализующая стремление так организовать процесс переработки информации, чтобы он был наиболее близок к семантическому (содержательному) уровню}
        \begin{scnrelfromset}{задачи разработки}
            \scnfileitem{разработка семантического способа представления перерабатываемой информации в памяти машины}
            \scnfileitem{разработка такого внутреннего языка, запись программ на котором была бы максимально близка к тому, что называют записью алгоритма на содержательном уровне}
            \scnfileitem{разработка и исследование способов семантического представления информации различного вида}
            \scnfileitem{разработка и исследование принципов организации развитой ассоциативной памяти для непосредственного хранения семантического представления информации}
            \scnfileitem{разработка и исследование языка программирования высокого уровня, который (I) ориентирован на решение информационно-логических задач; (2) обеспечивает непосредственную реализацию простых процедур логического вывода. (3) согласован c выбранным способом семантического представления перерабатываемой информации, (4) приспособлен к использованию в качестве внутреннего языкё параллельной однородной структуры, имеющей распределенную ассоциативную память для хранения сложноструктурированных данных}
            \scnfileitem{разработка и исследование принципов построения и принципов параллельного взаимодействия функциональных средств, обеспечивающих непосредственную переработку семантического представления информации в распределенной ассоциативной памяти и реализующих управление потоком словноструктурированных данных}
            \scnfileitem{экспериментальная проверка полученных результатов}
        \end{scnrelfromset}
        \scntext{примечание}{Исследования по системам искусственного интеллекта убедительно показали, что способ представления знаний в их памяти, точнее степень его близости к семантическому, является фактором, во многом определяющим эффективность таких систем.}
        \begin{scnrelfromset}{принципы построения}
            \scnfileitem{такие машины целесообразно строить как машины, манипулирующие графовыми структурами непосредственно на физическом уровне}
            \begin{scnindent}
                \scntext{пояснение}{Предполагается создание структурно-перестраиваемой (графовой) памяти. Такая память состоит из ячеек, однозначно соответствующих вершинам хранимого в памяти графа, но, в отличие от обычной памяти, эти ячейки связываются не фиксированными условными связями, задающими фиксированную последовательность (порядок) ячеек в памяти, a реально (физически) проводимыми связями произвольной конфигурации. Эти связи соответствуют дугам, ребрам, гиперребрам записанного в памяти графа. Очевидно, что в ходе переработки информации в структурно-перестраиваемой памяти меняются на только и даже не столько состояния ячеек памяти, как это имеет место в обычной памяти, сколько конфигурация связей между этими ячейками. Т.е. в структурно-перестраиваемой памяти в ходе переработки информации не только перераспределяются метки на вершинах записанного в памяти графа, но и меняется структура самого этого графа.}
            \end{scnindent}
            \scnfileitem{в качестве внутреннего языка таких машин целесообразно использовать язык типа PROLOG}
            \begin{scnindent}
                \scntext{пояснение}{Разработка языка типа PROLOG, предназначенного к использованию в качестве внутреннего языка программирования для машин co структурно-перестраиваемой памятью требует решения нетривиальной задачи согласования графового способа представления данных в структурно-перестраиваемой памяти и способа записи в этой же памяти самих программ, описывающих переработку этих данных. Переход на графовый способ кодирования программ и данных в структурно-перестраиваемой памяти обеспечивает компактность их представления и существенно упрощает аппаратурную реализацию операций над сложными структурами. Говоря об аппаратурной интерпретации языка типа PROLOG, необходимо подчеркнуть следующее. На уровне любого языка типа PROLOG, т.е. на уровне абстрактной PROLOG --- машины, естественным образом реализуется эффективное распараллеливание процесса переработки сложных структур, организованное по принципу управления потоком запросов или управления потоком перерабатываемых сложноструктурированных данных. Управление потоком сложноструктурированных данных при этом основывается на использовании развитой формы ассоциативного доступа, a именно, доступа к произвольному фрагменту перерабатываемых данных (фрагменту, имеющему произвольный размер и произвольную структуру). Из вышесказанного следует, что создание машины, аппаратурно интерпретирующей язык типа PROLOG, есть не что иное, как создание параллельной машины, управляемой потоком сложноструктурированных данных и имеющей развитую ассоциативную память.}
            \end{scnindent}
            \scnfileitem{принцип организации переработки информации непосредственно в памяти}
            \begin{scnindent}
                \scntext{пояснение}{Этот принцип обеспечивает значительное ускорение переработки информации благодаря исключению этапов передачи информации из памяти в процессор и обратно, но оплачивается ценой большой избыточности функциональных (процессорных) средств, равномерно распределяемых по памяти. При распределении функциональных средств по структурно-перестраиваемой памяти каждая ячейка дополняется функциональным (процессорным) элементом, a перестраиваемые связи между ячейками становятся коммутируемыми каналами связи между функциональными элементами. Каждый функциональный элемент при этом имеет свою специальную внутреннюю регистровую память, отражающую важные для данного функционального элемента аспекты текущего состояния процесса выполнения элементарных операций внутреннего языка.}
            \end{scnindent}
        \end{scnrelfromset}
        \begin{scnrelfromset}{формальная основа и направления исследований}
            \scnfileitem{сочетание теории множеств, теории алгебраических систем, теории графов, математической логики, теории вычислений, методов машинного моделирования}
            \scnfileitem{теория наиболее общего вида структур данных (квази-графов), используемых в задачах искусственного интеллекта и являющихся обобщением классических алгебраических моделей}
            \scnfileitem{ориентированный на аппаратурную интерпретацию способ кодирования сложноструктурированных данных (квази-графв) в виде однородных семантических сетей специального вида}
            \scnfileitem{новый логический язык, являющийся модификацией классического и обеспечивающий описание квази-графов}
            \scnfileitem{графовые варианты логического языка описания квази-графов и, в частности, ориентированный на аппаратурную интерпретацию способ кодирования логических формул в виде однородных семантических сетей}
            \scnfileitem{непроцецурный язык программирования типа PROLOG, отличающийся удобством работы со сложными структурами данных (квази-графами) и развитостью средств управления вычислительным процессом}
            \scnfileitem{предлагаемый в качестве внутреннего аппаратурно интерпретируемого языка программирования способ записи непроцедурных программ в виде однородных семантических сетей}
            \scnfileitem{архитектура и принципы организации однородной ассоциативной параллельной структуры, ориентированной на переработку семантических сетей и обеспечивающей аппаратурную интерпретацию непроцецурного языка программирования типа PROLOG}
        \end{scnrelfromset}
        \scntext{задачи исследований}{Исследовать пути построения параллельных машин, управляемых потоком сложноструктурированных данных. В качестве памяти таких машин рассмотреть структурно-перестраиваемые запоминающие среды, обеспечивающие непосредственное хранение графовых структур и манипулирование ими, a также обеспечивающие ассоциативный доступ к произвольным фрагментам перерабатываемых графовых структур (фрагментам, имеющим произвольный вид и произвольный размер). Исследовать пути и принципы аппаратной интерпретации непроцецурных языков программирования типа PROLOG. B качестве интерпретируемого (внутреннего) языка исследовать язык программирования графового типа, являющийся способом записи программ в виде однородных семантических сетей.}
        \scntext{достоинства}{Разработка \textit{машин, ориентированных на решение информационно-логических задач} позволит:
            \begin{scnitemize}
                \item существенно расширить класс аппаратурно интерпретируемых (непосредственно перарабатываемых) структур данных;
                \item обеспечить высокую скорость переработки сложноструктурированных данных, благодаря (1) укрупнению аппаратно реализуемых операций преобразования структур данных, (2) глубокому распараллеливанию процесса переработки сложных структур как на программном, так и на микропрограммном уровне, (3) организации переработки информации непосредственно в памяти;
                \item существенно расширить логические возможности компьютеров благодаря использованию логического языка в качестве основы внутреннего языка программирования;
                \item обеспечить достаточно высокую технологичность компьютеров рассматриваемого класса благодаря их организации как однородных структур.
            \end{scnitemize}
        }
        \bigskip
    \end{scnsubstruct}
    \scnendcurrentsectioncomment
\end{SCn}
