\begin{SCn}
	\scnsectionheader{Сегмент. Уточнение понятия ostis-платформы}
	\begin{scnsubstruct}
		
	\begin{scnrelfromlist}{ключевое понятие}
		\scnitem{ostis-платформа}
		\scnitem{базовая ostis-платформа}
		\scnitem{расширенная ostis-платформа}
		\scnitem{специализированная ostis-платформа}
		\scnitem{реализация sc-памяти}
		\scnitem{реализация файловой памяти sc-машины}
		\scnitem{scp-интерпретатор}
		\scnitem{базовая подсистема взаимодействия ostis-системы с внешней средой}
		\scnitem{подсистема обеспечения жизнедеятельности ostis-системы}
		\scnitem{специализированная платформенно-зависимая машина обработки знаний}
		\scnitem{минимальная конфигурация ostis-системы}
		\scnitem{однопользовательская ostis-платформа}
		\scnitem{многопользовательская ostis-платформа}
		\scnitem{программный вариант ostis-платформы}
		\scnitem{ассоциативный семантический компьютер}
	\end{scnrelfromlist}
	
	\scnheader{ostis-платформа}
	\scnidtf{платформа интерпретации sc-моделей компьютерных систем}
	\scnidtf{интерпретатор sc-моделей кибернетических систем}
	\scnidtf{интерпретатор унифицированных логико-семантических моделей компьютерных систем}
	\scnidtf{платформа реализации sc-моделей компьютерных систем}
	\scnidtf{\scnqq{пустая} ostis-система}
	\scnidtf{реализация sc-машины}
	\scnsubset{платформенно-зависимый многократно используемый компонент ostis-систем}
	\scnidtf{базовый интерпретатор логико-семантических моделей ostis-систем}
	\scnidtf{семейство платформ интерпретации sc-моделей компьютерных систем}
	\scnidtftext{часто используемый sc-идентификатор}{универсальный интерпретатор sc-моделей компьютерных систем}
	\scnidtf{универсальный интерпретатор унифицированных логико-семантических моделей компьютерных систем}
	\scnsubset{встроенная ostis-система}
	\scnidtf{встроенная \scnqq{пустая} ostis-система}
	\scnidtf{универсальный интерпретатор sc-моделей ostis-систем}
	\scnidtf{универсальная базовая ostis-система, обеспечивающая имитацию любой ostis-системы путем интерпретации sc-модели имитируемой ostis-системы}
	\scntext{примечание}{соотношение между имитируемой и универсальной ostis-системой в известной мере аналогично соотношению между машиной Тьюринга и универсальной машиной Тьюринга}
	\scntext{пояснение}{Под \textbf{\textit{ostis-платформой}} понимается реализация платформы интерпретации sc-моделей, которая в общем случае включает в себя: Реализация \textit{ostis-платформы} (\textit{универсального интерпретатора sc-моделей компьютерных систем}) может иметь большое число вариантов --- как программно, так и аппаратно реализованных. Логическая архитектура \textit{ostis-платформы} обеспечивает независимость проектируемых компьютерных систем от многообразия вариантов реализации интерпретатора их моделей и в общем случае включает в себя:
		\begin{scnitemize}
			\item хранилище \textit{sc-текстов} (\textit{sc-хранилище}, хранилище знаковых конструкций, представленных SC-коде);
			\item файловую память \textit{sc-машины};
			\item средства, обеспечивающие взаимодействие \textit{sc-агентов} над общей памятью (sc-памятью);
			\item базовые средства интерфейса для взаимодействия системы с внешним миром (пользователем или другими системами). Указанные средства включают в себя, как минимум, редактор, транслятор (в sc-память и из нее) и визуализатор для одного из базовых универсальных вариантов представления \textit{SC-кода} (\textit{SCg-код}, \textit{SCs-код}, \textit{SCn-код}), средства, позволяющие задавать системе вопросы из некоторого универсального класса (например, запрос семантической окрестности некоторого объекта);
			\item реализацию \textit{Абстрактной scp-машины}, то есть интерпретатор \textit{scp-программ} (программ Языка SCP).
		\end{scnitemize}
		При необходимости, в \textbf{\textit{ostis-платформу}} могут быть заранее на платформенно-зависимом уровне включены какие-либо компоненты машин обработки знаний или баз знаний, например, с целью упрощения создания первой версии \textit{прикладной ostis-системы}. Реализация платформы может осуществляться на основе произвольного набора существующих технологий, включая аппаратную реализацию каких-либо ее частей. С точки зрения компонентного подхода любая \textbf{\textit{ostis-платформа}} является \textbf{\textit{платформенно-зависимым многократно используемым компонентом}}.}
	\begin{scnrelfromset}{разбиение}
		\scnitem{базовая ostis-платформа}
		\scnitem{расширенная ostis-платформа}
		\scnitem{специализированная ostis-платформа}
	\end{scnrelfromset}
	
	\scnheader{базовая ostis-платформа}
	\scnidtf{базовый интерпретатор логико-семантических моделей ostis-систем}
	\scnidtf{минимальная универсальная ostis-платформа, обеспечивающая интерпретацию sc-модели любой \textit{ostis-системы} и включающая интерпретатор базового языка программирования \textit{ostis-систем} (Языка SCP)}
	\scnidtf{универсальный интерпретатор sc-моделей ostis-систем}
	\scnidtf{универсальная базовая ostis-система, обеспечивающая имитацию любой \textit{ostis-системы} путем интерпретации sc-модели имитируемой ostis-системы}
	\scntext{примечание}{Понятие \textit{базовой ostis-платформы} является ключевым с точки зрения обеспечения платформенной независимости \textit{ostis-систем}. Универсальность \textit{базовой ostis-платформы} подразумевает возможность интерпретации на ее основе любой \textit{sc-модели кибернетической системы}. Это достигается за счет наличия в рамках \textit{Технологии OSTIS} средств, позволяющих описывать на уровне sc-модели \textit{базу знаний}, \textit{решатель задач} и \textit{интерфейс кибернетической системы}, а также наличия Базового универсального языка программирования для \textit{ostis-систем} (\textit{Языка SCP}). \textit{Язык SCP} в таком случае выступает в роли базового низкоуровневого стандарта (ассемблера) обработки конструкций \textit{SC-кода}, гарантирующего полноту с точки зрения обработки, то есть, обеспечивающего возможность осуществить любое преобразование любого фрагмента \textit{SC-кода} при условии сохранения синтаксической корректности этого фрагмента. Следует отметить, что в общем случае таких функционально эквивалентных ассемблеров может быть несколько (и, как следствие, соответствующих им \textit{scp-машин}), но для обеспечения совместимости в рамках \textit{Технологии OSTIS} один из таких вариантов выбирается в качестве стандарта и описывается в \textit{Предметной области и онтологии решателей задач ostis-систем}.}
	\scntext{предъявляемое требование}{Основным и \uline{единственным требованием}, предъявляемым ко всем \textit{базовым ostis-платформам} для обеспечения их универсальности, является необходимость обеспечения интерпретации \textit{Языка SCP}, стандартизированного в рамках \textit{Технологии OSTIS}. При этом важно отметить, что все \textit{базовые ostis-платформы} обязаны быть \uline{функционально эквивалентными}, поскольку интерпретируют один и тот же стандарт \textit{Языка SCP}.}
	\begin{scnrelfromset}{обобщенная декомпозиция}
		\scnitem{реализация sc-памяти}
		\begin{scnindent}
			\scnrelfrom{обобщенная часть}{реализация файловой памяти sc-машины}
		\end{scnindent}
		\scnitem{scp-интерпретатор}
		\scnitem{базовая подсистема взаимодействия \textit{ostis-системы} с внешней средой}
		\begin{scnindent}
			\scntext{примечание}{Реализация базового набора \textit{рецепторных sc-агентов} и \textit{эффекторных sc-агентов}, обеспечивающих минимально необходимый обмен информацией между \textit{ostis-системой} и внешней средой. Конкретный перечень таких агентов требует уточнения, однако можно сказать, что в общем случае они могут быть реализованы как в составе \textit{scp-интерпретатора} (в этом случае им будут соответствовать определенные классы \textit{scp-операторов}), так и отдельно от него в составе платформы.}
		\end{scnindent}
		\scnitem{подсистема обеспечения жизнедеятельности ostis-системы}
		\begin{scnindent}
			\scntext{примечание}{Реализацию набора sc-агентов, обеспечивающих базовые функции \textit{ostis-системы}, связанные с обеспечением ее жизнедеятельности, которые принципиально не могут быть реализованы на платформенно-независимом уровне. К таким функциям относятся, например, запуск системы, загрузка базы знаний в память системы, запуск \textit{scp-интерпретатора} и так далее.}
		\end{scnindent}
	\end{scnrelfromset}
	
	\scnheader{расширенная ostis-платформа}
	\scnidtf{ostis-платформа, содержащая дополнительные компоненты, реализованные на уровне платформы}
	\scnidtf{базовая ostis-платформа и множество компонентов, реализованных на уровне платформы}
	\scntext{примечание}{\textit{расширенная ostis-платформа} представляет собой \textit{базовую ostis-платформу}, дополненную каким-либо множеством компонентов (хотя бы одним), реализованных на уровне платформы, при условии сохранения при этом всех возможностей \textit{базовой ostis-платформы}. Таким образом, \textit{расширенная ostis-платформа} по сути представляет собой \textit{базовую ostis-платформу}, адаптированную для более эффективного решения задач определенных классов в рамках конкретного класса \textit{ostis-систем}. Компонент, реализуемый на уровне платформы, становится частью этой платформы и, таким образом,  преобразует \textit{базовую ostis-платформу} в \textit{расширенную ostis-платформу}.}
	\scntext{примечание}{Введение понятия \textit{расширенной ostis-платформы} позволяет сформулировать ряд дополнительных принципов реализации \textit{ostis-систем}:
		\begin{scnitemize}
			\item{Может существовать произвольное количество ostis-систем, каждая из которых будет иметь свою уникальную \textit{расширенную ostis-платформу}, но при этом все они будут основаны на одном и том же варианте \textit{базовой ostis-платформы}.}
			\item{Для каждого варианта \textit{базовой ostis-платформы} может существовать своя \textit{библиотека многократно используемых компонентов ostis-платформ} (см. Предметная область и онтология комплексной библиотеки многократно используемых семантически совместимых компонентов ostis-систем)}.
	\end{scnitemize}}
	
	\scnheader{специализированная ostis-платформа}
	\scnidtf{ostis-платформа, не содержащая реализацию интерпретатора Языка SCP}
	\scnidtf{неуниверсальная ostis-платформа}
	\scntext{примечание}{\textbf{\textit{специализированная ostis-платформа}} представляет собой ограниченный вариант реализации \textit{ostis-платформы}, не содержащий \textit{scp-интерпретатора}. Таким образом, \uline{все} \textit{sc-агенты}, в рамках \textit{ostis-системы}, основанной на \textit{специализированной ostis-платформе} должны быть реализованы на платформенно-зависимом уровне. Такая \textit{специализированная ostis-платформа} является аналогом специализированного компьютера, реализованного для конкретной компьютерной системы. Таким образом, в общем случае каждая \textit{ostis-система}, реализуемая на \textit{специализированной ostis-платформе} будет иметь свою \uline{уникальную} \textit{специализированную ostis-платформу}.}
	\scntext{примечание}{\textbf{\textit{специализированная ostis-платформа}} может быть получена из \textit{базовой ostis-платформы} путем исключения из нее реализации  \textit{scp-интерпретатора} и реализации всех необходимых \textit{sc-агентов} на уровне платформы (или заимствования всех или части агентов из соответствующей данному варианту \textit{базовой ostis-платформы} \textit{библиотеки многократно используемых компонентов ostis-платформ}).}
	\begin{scnrelfromset}{обобщенная декомпозиция}
		\scnitem{реализация sc-памяти}
		\begin{scnindent}
			\scnrelfrom{обобщенная часть}{реализация файловой памяти sc-машины}
		\end{scnindent}
		\scnitem{базовая подсистема взаимодействия ostis-системы с внешней средой}
		\scnitem{подсистема обеспечения жизнедеятельности ostis-системы}
		\scnitem{специализированная платформенно-зависимая машина обработки знаний}
		\begin{scnindent}
			\scnidtf{sc-агент, как правило неатомарный, обеспечивающий выполнение всех функций некоторой специализированной ostis-платформы, связанных с обработкой знаний}
			\scnsubset{платформенно-зависимый sc-агент}
		\end{scnindent}
	\end{scnrelfromset}
	\scntext{примечание}{Применение \textit{специализированных ostis-платформ} может быть целесообразным на стартовом этапе развития \textit{Технологии OSTIS}, а также с целью повышения производительности конкретных наиболее высоконагруженных \textit{ostis-систем}, однако активное развитие таких \textit{специализированных ostis-платформ} и их компонентов с точки зрения \textit{Технологии OSTIS} является нецелесообразным, поскольку:
		\begin{scnitemize}
			\item если какой-либо компонент разработан с ориентацией на конкретную платформу, то нет гарантий возможности его повторного использования в других вариантах реализации \textit{ostis-платформы} (как минимум, компоненты, разработанные для \textit{программного варианта реализации ostis-платформы} не смогут быть использованы в рамках \textit{ассоциативного семантического компьютера});
			\item наличие большого числа платформенно-зависимых компонентов требует развития и сопровождения отдельной инфраструктуры библиотек для хранения и повторного использования таких компонентов. Чем больше будет вариантов \textit{ostis-платформ} и чем больше будет число платформенно-зависимых компонентов, тем более сложной и громоздкой будет такая инфраструктура. Как минимум, необходимо будет отслеживать совместимость компонентов с разными версиями разных вариантов реализации \textit{ostis-платформ};
			\item изменения в \textit{специализированной ostis-платформе}, например, связанные с переходом на более новую и эффективную версию \textit{базовой ostis-платформы}, на основе которой построена данная \textit{специализированная ostis-платформа} в общем случае могут привести к необходимости внесения изменений в компоненты, зависящие от данного варианта реализации \textit{ostis-платформы}. Чем больше таких платформенно-зависимых компонентов, тем больше потенциальных изменений может потребоваться и, соответственно, тем сложнее будет осуществляться эволюция платформы при условии сохранения работоспособности \textit{ostis-систем}, в которых она используется.
		\end{scnitemize} 
		
		Перечисленные тезисы справедливы и для \textit{расширенных ostis-платформ}, однако в случае \textit{расширенной ostis-платформы} проблемы, связанные с переходом на более новую версию платформы и изменениями в соответствующих компонентах всегда могут быть решены путем временной замены платформенно-зависимых компонентов на их платформенно-независимые версии с соответствующим снижением производительности, но зато с сохранением функциональной целостности системы.}
	
	\scnheader{минимальная конфигурация ostis-системы}
	\begin{scnrelfromset}{обобщенная декомпозиция}
		\scnitem{sc-модель базы знаний}
		\scnitem{специализированная ostis-платформа}
	\end{scnrelfromset}
	\begin{scnrelfromlist}{предъявляемые требования}
		\scnfileitem{Использование \textit{SC-кода} как базового языка кодирования информации в базе знаний, и, соответственно, наличие памяти, хранящей конструкции \textit{SC-кода}.}
		\scnfileitem{Наличие \textit{базы знаний}, определяющей денотационную семантику понятий, используемых системой.}
		\scnfileitem{Наличие хотя бы одного внутреннего sc-агента, осуществляющего обработку знаний в памяти ostis-системы. Этот sc-агент может быть реализован на уровне платформы, соответственно база знаний такой системы может не содержать процедурных знаний (методов).}
	\end{scnrelfromlist}
	\scntext{примечание}{Такой вариант \textit{минимальной конфигурации ostis-системы} обладает только \textit{внутренним sc-агентом} и, соответственно, не имеет возможности общаться с внешним миром (можно сказать, что такая \textit{ostis-система} не обладает \scnqq{органами чувств}). Для того, чтобы система имела возможность общаться с внешним миром, необходимо добавить к \textit{минимальной конфигурации ostis-системы} хотя бы один \textit{рецепторный sc-агент} и хотя бы один \textit{эффекторный sc-агент}.}
	\scntext{примечание}{Важно отметить, что, как видно из представленного описания \textit{минимальной конфигурации ostis-системы}, в общем случае \textit{ostis-система} не обязана по умолчанию быть \textit{интеллектуальной системой}. Применение \textit{Технологии OSTIS} для разработки компьютерных систем не делает их автоматически интеллектуальными, оно позволяет обеспечить возможность последующей \uline{неограниченной интеллектуализации} таких систем с минимальными накладными расходами при условии соблюдения при их разработке всех принципов \textit{Технологии OSTIS}.}
	
	\scnheader{ostis-платформа}
	\begin{scnsubdividing}
		\scnitem{однопользовательская ostis-платформа}
		\begin{scnindent}
			\scnidtf{вариант реализации ostis-платформы, рассчитанный на то, что с конкретной ostis-системой взаимодействует только один пользователь (владелец)}
			\scntext{примечание}{При таком варианте реализации платформы оказывается невозможным реализовать некоторые важные принципы \textit{Технологии OSTIS}, например, коллективную согласованную разработку базы знаний системы в процессе ее эксплуатации. При этом могут использоваться различные сторонние средства, например, для разработки базы знаний на уровне исходных текстов.}
		\end{scnindent}
		\scnitem{многопользовательская ostis-платформа}
		\begin{scnindent}
			\scnidtf{вариант реализации ostis-платформы, рассчитанный на то, что с конкретной ostis-системой одновременно или в разное время могут взаимодействовать разные пользователи, в общем случае обладающие разными правами, сферами ответственности, уровнем опыта, и имеющие свою конфиденциальную часть хранимой в базе знаний информации}
		\end{scnindent}
	\end{scnsubdividing}
	\scnheader{платформа интерпретации sc-моделей компьютерных систем}
	\begin{scnsubdividing}
		\scnitem{программный вариант реализации платформы интерпретации sc-моделей компьютерных систем}
		\begin{scnindent}
			\scnidtf{программная платформа интерпретации sc-моделей ostis-систем}
			\scnidtf{программный базовый интерпретатор sc-моделей ostis-систем}
		\end{scnindent}
		\scnitem{семантический ассоциативный компьютер}
		\begin{scnindent}
			\scnidtf{аппаратная платформа интерпретации sc-моделей ostis-систем}
			\scnidtf{аппаратно реализованный базовый интерпретатор sc-моделей ostis-систем}
		\end{scnindent}
	\end{scnsubdividing}
	
	\scnheader{ostis-платформа}
	\begin{scnrelfromset}{разбиение}
		\scnitem{программный вариант ostis-платформы}
		\begin{scnindent}
			\scnidtf{платформа интерпретации sc-моделей ostis-систем, реализованная в виде программной системы на базе традиционной компьютерной архитектуры}
			\scnidtf{программная платформа интерпретации sc-моделей ostis-систем}
			\scnidtf{программный интерпретатор sc-моделей ostis-систем}
			\scntext{примечание}{Целесообразность разработки \textit{программных вариантов ostis-платформы} на настоящий момент обусловлена очевидной распространенностью фон-неймановской архитектуры и, соответственно, необходимостью реализации \textit{ostis-систем} на современных компьютерах различного вида. В то же время очевидно, что разработка специализированных \textit{ассоциативных семантических компьютеров} позволит существенно повысить эффективность работы \textit{ostis-систем}, а четкое разделение \textit{sc-модели кибернетической системы} и платформы ее интерпретации позволит осуществить перевод уже работающих \textit{ostis-систем} с традиционных архитектур на \textit{ассоциативные семантические компьютеры} с минимальными накладными расходами.}
		\end{scnindent}
		\scnitem{ассоциативный семантический компьютер}
		\begin{scnindent}
			\scnidtf{аппаратная платформа интерпретации sc-моделей ostis-систем}
			\scnidtf{аппаратно реализованный базовый интерпретатор sc-моделей ostis-систем}
		\end{scnindent}
	\end{scnrelfromset}
	\scntext{примечание}{Важно отметить, что в любом варианте реализации \textit{ostis-платформы} всегда присутствует как программная, так и аппаратная часть. Так, любой \textit{программный вариант ostis-платформы} предполагает его последующую интерпретацию на какой-либо аппаратной основе, например, на персональном компьютере с традиционной архитектурой. В то же время, разработка \textit{ostis-платформы} в виде \textit{ассоциативного семантического компьютера} предполагает разработку набора микропрограмм, реализующих базовые операции поиска и преобразования sc-конструкций, хранящихся в \textit{sc-памяти}.}
	\begin{scnindent}
		\scntext{примечание}{Таким образом, разделение множества возможных реализаций \textit{ostis-платформы} на программный и аппаратный варианты скорее отражает вариант аппаратной архитектуры, на которую в конечном итоге ориентирован тот или иной вариант реализации платформы --- либо на традиционную фон-неймановскую архитектуру, либо на специализированную архитектуру \textit{ассоциативного семантического компьютера} со структурно-перестраиваемой (графодинамической) памятью. \textit{Программный вариант ostis-платформы} по сути является моделью (виртуальной машиной) \textit{ассоциативного семантического компьютера}, построенной на базе традиционной фон-неймановской архитектуры, а \textit{Язык SCP} выступает в роли ассемблера для \textit{ассоциативного семантического компьютера} и также может интерпретироваться как в рамках аппаратной реализации такого компьютера, так и в рамках его программной модели. }
	\end{scnindent}
	\scntext{примечание}{Каждой конкретной \textit{ostis-системе} однозначно соответствует конкретная \textit{ostis-платформа}, которая может относиться к разному набору классов \textit{ostis-платформ}. В то же время очевидно, что на этапе разработки платформы проектируется и реализуется некоторый вариант \textit{ostis-платформы}, который затем тиражируется в разные \textit{ostis-системы}. Впоследствии в каждой \textit{ostis-системе} в этот вариант \textit{ostis-платформы} могут быть внесены изменения, но в общем случае в большом количестве \textit{ostis-систем} могут использоваться полностью эквивалентные \textit{ostis-платформы}. Таким образом, целесообразно говорить о \textit{типовых ostis-платформах}, которые:
		\begin{scnitemize}
			\item{Являются объектом разработки для разработчиков \textit{ostis-платформ}.}
			\item{Являются \textit{многократно используемым компонентом ostis-систем} и специфицируются в рамках соответствующих библиотек.}
			\item{Являются образцом для тиражирования (копирования) при создании новых \textit{ostis-систем}.}
	\end{scnitemize}}
	
	\bigskip
	\end{scnsubstruct}
\scnsourcecomment{Завершили \scnqqi{Сегмент. Уточнение понятия ostis-платформы}}
\end{SCn}