\begin{SCn}
\scnsectionheader{Предметная область и онтология сообщений, входящих в ostis-систему и выходящих из неё}
\begin{scnsubstruct}
\scnrelfrom{соавтор}{Садовский М.Е.}

\scnheader{сообщение}
\scntext{определение}{\scnkeyword{сообщение} --- дискретная информационная конструкция, используемая в процессе передачи от отправителя к получателю.}
\begin{scnrelfromset}{разбиение}
	\scnitem{сообщение пользователя системы}
	\begin{scnindent}
		\scnsuperset{сообщение пользователя ostis-системы}
	\end{scnindent}
	\scnitem{сообщение системы}
	\begin{scnindent}
		\scnsuperset{сообщение ostis-системы}
		\begin{scnindent}
			\scnsuperset{эффекторное сообщение ostis-системы}
			\begin{scnindent}
				\scntext{определение}{\scnkeyword{эффекторное сообщение ostis-системы} --- сообщение ostis-системы, формируемое самой ostis-системой при возникновении некоторых ситуаций.}
				\scntext{примечание}{К ситуациям, инициирующим возникновение эффекторных сообщений, можно отнести:
					\begin{scnitemize}
						\item{ситуации, возникающие при анализе деятельности самого пользователя. Например, задание аргументов, не соответствующих типу инициируемого действия или появление подсказок при использовании компонентов пользовательского интерфейса;}
						\item{ситуации, возникающие при анализе синтаксиса текстов внешних языков. Например, неполнота сформированного предложения на внешнем языке или использование конструкций, нехарактерных или некорректно использованных в контексте отдельно взятого внешнего языка.}
					\end{scnitemize}
				}
			\end{scnindent}
			\scnsuperset{рецепторное сообщение ostis-системы}
			\begin{scnindent}
				\scntext{определение}{\scnkeyword{рецепторное сообщение ostis-системы} --- сообщение ostis-системы, являющееся реакцией на императивное сообщение (сообщение, побуждающее к какому-либо действию).}
				\scntext{примечание}{Возможными реакциями ostis-системы на императивное сообщение пользователя являются:
					\begin{scnitemize}
						\item{указание факта завершения выполнения некоторой задачи, что, например, характерно для поведенческих действий;}
						\item{получение ответа на поставленную задачу, формируемого либо в результате анализа базы знаний пользовательского интерфейса, либо в результате анализа предметной части базы знаний самой ostis-системы.}
					\end{scnitemize}
				}
			\end{scnindent}
		\end{scnindent}
	\end{scnindent}
\end{scnrelfromset}
\begin{scnrelfromset}{разбиение}
	\scnitem{атомарное сообщение}
	\scnitem{неатомарное сообщение}
\end{scnrelfromset}
\begin{scnrelfromset}{разбиение}
	\scnitem{сообщение на естественном языке}
	\scnitem{сообщение на искусственном языке}
\end{scnrelfromset}
\scnsuperset{графическое сообщение}
\begin{scnindent}
	\scnidtf{сообщение, содержащее графическую информацию}
	\scnsuperset{видео-сообщение}
	\begin{scnindent}
		\scnidtf{сообщение, содержащее видео-информацию}
	\end{scnindent}
\end{scnindent}
\scnsuperset{аудио-сообщение}
\begin{scnindent}
	\scnidtf{сообщение, представленное в звуковом формате}
\end{scnindent}
\scnsuperset{обонятельное сообщение}
\begin{scnindent}
	\scnidtf{сообщение, содержащее информацию о запахах}
\end{scnindent}
\scnsuperset{текстовое сообщение}
\begin{scnindent}
	\scnidtf{сообщение, содержащее текстовую информацию}
\end{scnindent}
\scnsuperset{сообщение, требующее трансляции}
\begin{scnindent}
	\scnidtf{сообщение, которое необходимо сформировать системой для дальнейшей передачи его пользователю}
\end{scnindent}
\scnsuperset{протранслированное сообщение}
\begin{scnindent}
	\scnidtf{сообщение, которое было сформировано системой для дальнейшей передачи его пользователю}
\end{scnindent}

\bigskip
\end{scnsubstruct}
\scnendcurrentsectioncomment
\end{SCn}