\begin{SCn}
\scnsectionheader{Предметная область и онтология действий и внутренних агентов пользовательского интерфейса ostis-системы}
\begin{scnsubstruct}
\scnrelfrom{соавтор}{Садовский М.Е.}

\scnheader{решатель задач пользовательского интерфейса ostis-систем}
\scntext{примечание}{\textit{решатель задач пользовательского интерфейса ostis-систем} состоит из некоторого коллектива sc-агентов, обеспечивающих работу пользователя с компонентами пользовательского интерфейса ostis-системы.}
\scntext{примечание}{При использовании \textit{sc-агентов} стоит помнить различия в \uline{семантической} и \uline{прагматической} составляющей любого \textit{компонента пользовательского интерфейса}. \uline{Семантическая составляющая} заключается в определении того, знаком какой сущности является отображаемый на экране компонент. \uline{Прагматическая составляющая} рассматривает прикладной аспект (аспект применения) отображаемого на экране компонента.}
\scntext{примечание}{На уровне sc-памяти имеет значение только \uline{семантическая составляющая}, однако данный факт не влияет на процесс эксплуатации системы пользователем, поскольку обе составляющие отражают разные стороны одного и того же знака некоторой сущности. Например, за каждой кнопкой скрывается знак некоторого \textit{класса действия}, инициируемого при нажатии на кнопку. Таким образом, \textit{интерфейсное действие пользователя}, как правило, инициирует некоторое \textit{внутреннее действие системы}.}

\scnheader{решатель задач пользовательского интерфейса ostis-систем}
\scntext{примечание}{С точки зрения обработки модели базы знаний пользовательского интерфейса ostis-систем должны быть решены следующие задачи:
	\begin{scnitemize}
		\item{обработка пользовательских действий;}
		\item{интерпретация модели базы знаний пользовательского интерфейса ostis-системы (построение пользовательского интерфейса);}
	\end{scnitemize}
}
\scnsuperset{интерпретатор sc-моделей пользовательских интерфейсов}
\begin{scnindent}
	\scntext{примечание}{\textit{интерпретатор sc-моделей пользовательских интерфейсов} в качестве входного параметра принимает экземпляр \textit{компонента пользовательского интерфейса для отображения}. При этом компонент может быть как атомарным, так и неатомарным (например, компонент главного окна приложения). Результатом работы интерпретатора является графическое представление указанного компонента с учетом используемой реализации \textit{платформы интерпретации семантических моделей ostis-систем}.}
	\scntext{примечание}{Алгоритм работы данного интерпретатора следующий:
		\begin{scnitemize}
			\item{Проверяется тип входного компонента пользовательского интерфейса (атомарный или неатомарный).}
			\item{Если \textit{компонент пользовательского интерфейса} является атомарным, то отобразить его графическое представление на основании указанных для него свойств. В случае, если данный компонент не входит в \textit{декомпозицию} любого другого \textit{компонента пользовательского интерфейса} --- завершить выполнение. Иначе определить компонент, в \textit{декомпозицию} которого входит рассматриваемый компонент пользовательского интерфейса, применить его свойства для текущего атомарного компонента и начать обработку найденного неатомарного компонента, перейдя к первому пункту.}
			\item{Если \textit{компонент пользовательского интерфейса} является неатомарным, то проверить, были ли отображены компоненты, на которые он был декомпозирован. Если да, то завершить выполнение, иначе определить еще не отображенный компонент из декомпозиции обрабатываемого неатомарного компонента и начать обработку найденного компонента, перейдя к первому пункту.}
		\end{scnitemize}
	}
\end{scnindent}
\scnsuperset{интерпретатор пользовательских действий}
\begin{scnindent}
	\scntext{примечание}{\textit{интерпретатор пользовательских действий} является \textit{неатомарным sc-агентом}, который включает в себя множество \textit{sc-агентов}, каждый из которых обрабатывает интерфейсные действия пользователя определенного класса (например, \textit{абстрактный sc-агент обработки действия нажатия мыши}, \textit{абстрактный sc-агент обработки действия отпускания мыши} и так далее). Интерпретатор реагирует на появление в базе знаний системы экземпляра интерфейсного действия пользователя, находит связанный с ним класс внутреннего действия и генерирует экземпляр данного внутреннего действия для последующей обработки.}
\end{scnindent}

\bigskip
\end{scnsubstruct}
\scnendcurrentsectioncomment
\end{SCn}