\begin{SCn}
\scnsectionheader{Предметная область и онтология действий и внутренних агентов пользовательского интерфейса ostis-системы}
\begin{scnsubstruct}
\scnrelfrom{соавтор}{Садовский М.Е.}

\scnheader{решатель задач пользовательского интерфейса ostis-систем}
\scnsuperset{sc-агенты}
\begin{scnindent}
	\scnidtf{обеспечивают работу пользователя с компонентами пользовательского интерфейса ostis-системы}
\end{scnindent}
\scnsuperset{семантическая составляющая}
\begin{scnindent}
	\scntext{задача}{определение того, знаком какой сущности является отображаемый на экране компонент}
\end{scnindent}
\scnsuperset{прагматическая составляющая}
\begin{scnindent}
	\scnidtf{рассматривает прикладной аспект (аспект применения) отображаемого на экране компонента}
\end{scnindent}
\scntext{примечание}{На уровне sc-памяти имеет значение только семантическая составляющая, однако данный факт не влияет на процесс эксплуатации системы пользователем, поскольку обе составляющие отражают разные стороны одного и того же знака некоторой сущности.}
\scnheader{внутреннее действие системы}
\scnsuperset{внутреннее действие ostis-системы}
\scnheader{внутреннее действие ostis-системы}
\scnidtf{действие в sc-памяти}
\scnidtf{действие, выполняемое в sc-памяти}
\scnheader{действие в sc-памяти}
\scnsuperset{действие в sc-памяти, инициируемое вопросом}
\scnsuperset{действие редактирования базы знаний ostis-системы}
\scnsuperset{действие установки режима ostis-системы}
\scnsuperset{действие редактирования файла, хранимого в sc-памяти}
\scnsuperset{ действие интерпретации программы, хранимой в sc памяти}
\scnheader{база знаний пользовательского интерфейса ostis-системы}
\begin{scnrelfromset}{задачи}
	\scnitem{обработка пользовательских действий}
	\scnitem{интерпретация модели базы знаний пользовательского интерфейса ostis-системы(построение пользовательского интерфейса)}
\end{scnrelfromset}
\scnheader{решатель задач пользовательского интерфейса ostis-систем}
\begin{scnrelfromset}{разбиение}
	\scnitem{интерпретатор sc-моделей пользовательских интерфейсов}
	\begin{scnindent}
		\scnidtf{в качестве входного параметра принимает экземпляр компонента пользовательского интерфейса для отображения}
		\scnsuperset{атомарный компонент}
		\scnsuperset{неатомарный компонент}
		\scntext{результат}{представление указанного компонента с учетом используемой реализации платформы интерпретации семантических моделей ostis-систем}
		\scnsuperset{алгоритм}
		\begin{scnindent}
			\scntext{первое действие}{проверяется тип входного компонента пользовательского интерфейса (атомарный или неатомарный)}
			\scntext{второе действие}{если компонент пользовательского интерфейса является атомарным, то отобразить его графическое представление на основании указанных для него свойств. В случае, если данный компонент не входит в декомпозициюлюбого другого компонента пользовательского интерфейса— завершить выполнение. Иначе определить компонент, в декомпозицию которого входит рассматриваемый компонент пользовательского интерфейса, применить его свойства для текущего атомарного компонента и начать обработку найденного неатомарного компонента, перейдя к первому пункту}
			\scntext{третье действие}{если компонент пользовательского интерфейса является неатомарным, то проверить, были ли отображены компоненты, на которые он был декомпозирован. Если да, то завершить выполнение, иначе определить еще не отображенный компонент из декомпозиции обрабатываемого неатомарного компонента и начать обработку найденного компонента, перейдя к первому пункту}
		\end{scnindent}
	\end{scnindent}
	\scnitem{интерпретатор пользовательских действий}
	\begin{scnindent}
		\scnidtf{является неатомарным sc-агентом, который включает в себя множество sc-агентов, каждый из которых обрабатывает интерфейсные действия пользователя определенного класса}
		\scnsuperset{алгоритм}
		\begin{scnindent}
			\scntext{первое действие}{реагирует на появление в базе знаний системы экземпляра интерфейсного действия пользователя}
			\scntext{второе действие}{находит связанный с ним класс внутреннего действия}
			\scntext{третье действие}{генерирует экземпляр данного внутреннего действия для последующей обработки}
		\end{scnindent}
	\end{scnindent}
\end{scnrelfromset}

\bigskip
\end{scnsubstruct}
\scnendcurrentsectioncomment
\end{SCn}