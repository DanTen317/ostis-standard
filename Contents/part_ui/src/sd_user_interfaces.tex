\begin{SCn}
\scnsectionheader{Предметная область и онтология интерфейсов ostis-систем}
\begin{scnsubstruct}
\scnrelfrom{соавтор}{Садовский М.Е.}
\begin{scnrelfromlist}{дочерний раздел}
    \scnitem{Предметная область и онтология интерфейсных действий пользователей ostis-систем}
    \scnitem{Предметная область и онтология естественных языков}
\end{scnrelfromlist}

\scnheader{Предметная область интерфейсов ostis-систем}
\scniselement{предметная область}
\begin{scnhaselementrole}{максимальный класс объектов исследования}
    {интерфейс}
\end{scnhaselementrole}
\begin{scnhaselementrolelist}{класс объектов исследования}
	\scnitem{пользовательский интерфейс}
	\scnitem{физический интерфейс}
	\scnitem{программный интерфейс}
	\scnitem{интерфейс ostis-систем}
	\scnitem{адаптивный интерфейс}
	\scnitem{интеллектуальный интерфейс}
	\scnitem{мультимодальный интерфейс}
    \scnitem{командный пользовательский интерфейс}
    \scnitem{графический пользовательский интерфейс}
    \scnitem{WIMP-интерфейс}
    \scnitem{SILK-интерфейс}
    \scnitem{естественно-языковой интерфейс}
    \scnitem{речевой интерфейс}
    \scnitem{пользовательский интерфейс ostis-системы}
    \scnitem{компонент пользовательского интерфейса}
    \scnitem{атомарный компонент пользовательского интерфейса}
    \scnitem{неатомарный компонент пользовательского интерфейса}
    \scnitem{визуальная часть пользовательского интерфейса ostis-системы}
    \scnitem{компонент пользовательского интерфейса для представления}
    \scnitem{компонент вывода}
    \scnitem{компонент выполнения}
    \scnitem{параграф}
    \scnitem{декоративный компонент пользовательского интерфейса}
    \scnitem{контейнер}
    \scnitem{меню}
    \scnitem{строка меню}
    \scnitem{панель инструментов}
    \scnitem{панель вкладок}
    \scnitem{окно}
    \scnitem{модальное окно}
    \scnitem{немодальное окно}
    \scnitem{интерактивный компонент пользовательского интерфейса}
    \scnitem{флаговая кнопка}
    \scnitem{радиокнопка}
    \scnitem{переключатель}
    \scnitem{кнопка-счетчик}
    \scnitem{полоса прокрутки}
    \scnitem{кнопка}
\end{scnhaselementrolelist}

\scnheader{Предметная область и онтология интерфейсов ostis-систем}
\begin{scnrelfromset}{введение}
	\scnfileitem{Организация взаимодействия пользователей с компьютерными системами (в том числе и с интеллектуальными компьютерными системами) оказывает существенное влияние на эффективность автоматизации человеческой деятельности, пользовательский опыт и уровень удовлетворенности пользователей.}
	\scnfileitem{Одним из ключевых свойств интеллектуальных компьютерных систем нового поколения является их \scnkeyword{интероперабельность} --- способность к эффективному взаимодействию. Такие системы являются автономными и самодостаточными субъектами деятельности наравне с человеком. Однако, в основе современной организации взаимодействия пользователя с компьютерной системой лежит парадигма \uline{грамотного пользователя}, который знает, как управлять системой и несет полную ответственность за качество взаимодействия с ней. Многообразие форм и видов интерфейсов приводит к необходимости пользователя адаптироваться к каждой конкретной системе, обучаться принципам взаимодействия с ней для решения необходимых ему задач.}
	\scnfileitem{На современном этапе развития Искусственного интеллекта для повышения эффективности взаимодействия необходим переход от парадигмы грамотного управления используемым инструментом к парадигме \textbf{равноправного сотрудничества, партнерскому взаимодействию} интеллектуальной компьютерной системы со своим пользователем. Дружественность пользовательского интерфейса должна заключаться в адаптивности системы к особенностям и квалификации пользователя, исключении любых проблем для пользователя в процессе диалога с интеллектуальной компьютерной системой, в перманентной заботе о совершенствовании коммуникационных навыков пользователя. Следовательно, необходимо отойти от привычной адаптации пользователя к системе (путем обучения ее использованию) в сторону адаптации самого интерфейса под цели, задачи и характеристики конкретного пользователя в режиме реального времени.}
	\begin{scnindent}
		\scnrelfrom{источник}{\scncite{Fomina2020}}
	\end{scnindent}
\end{scnrelfromset}

\scnrelfrom{анализ}{Анализ и проблемы существующих принципов организации интерфейсов}

\scnheader{Анализ и проблемы существующих принципов организации интерфейсов}
\begin{scnsubstruct}
	\scnheader{пользовательский интерфейс}
	\begin{scnrelfromlist}{проблемы текущего состояния}
		\scnfileitem{\uline{Необходимость пользователя обучаться принципам взаимодействия} с каждой конкретной системой.}
		\scnfileitem{\uline{Отсутствие партнерского взаимодействия} между пользователем и системой (система является объектом управления со стороны пользователя), что приводит к необходимости пользователя быть постоянным инициатором взаимодействия.}
		\scnfileitem{\uline{Отсутствие адаптации системы} к особенностям каждого конкретного пользователя и окружающей среды для максимально комфортного взаимодействия пользователя с системой.}
	\end{scnrelfromlist}
	\scntext{примечание}{Взаимодействие с большей частью традиционных компьютерных систем происходит с помощью клавиатуры и мыши (тачпада, стилуса). Пользовательский интерфейс таких систем, как правило, не хранит информацию о модели пользователя, историю его действий и так далее. Традиционный пользовательский интерфейс также не содержит модуль адаптации.}
	\begin{scnindent}
		\scnrelfrom{иллюстрация}{\scnfileimage[40em]{Contents/part_ui/src/images/sd_ui/traditional_ui.png}}
		\begin{scnindent}
			\scnidtf{Рисунок. Структура традиционного пользовательского интерфейса}
			\scnrelfrom{источник}{\scncite{Sadouski2022}}
		\end{scnindent}
	\end{scnindent}
	
	\scnheader{адаптивный пользовательский интерфейс}
	\scnrelfrom{иллюстрация}{\scnfileimage[40em]{Contents/part_ui/src/images/sd_ui/adaptive_ui_tools.png}}
	\begin{scnindent}
		\scnidtf{Рисунок. Существующие средства создания адаптивных пользовательских интерфейсов}
		\scnrelfrom{источник}{\scncite{Hussain2018}}
	\end{scnindent}
	
	\scnheader{интерфейс интеллектуальных компьютерных систем нового поколения}
	\scntext{примечание}{\scnkeyword{интерфейс интеллектуальных компьютерных систем нового поколения} должен воспринимать различные типы ввода информации. Для организации такого взаимодействия используются термины \textit{адаптивного}, \textit{интеллектуального} и \textit{мультимодального интерфейса}.}
	
	\scnheader{адаптивный интеллектуальный мультимодальный пользовательский интерфейс}
	\scnrelfrom{иллюстрация}{\scnfileimage[40em]{Contents/part_ui/src/images/sd_ui/adaptive_ui.png}}
	\begin{scnindent}
		\scnidtf{Рисунок. Структура адаптивного интеллектуального мультимодального пользовательского интерфейса}
	\end{scnindent}
	\scntext{примечание}{Вне зависимости от средств создания \textit{адаптивных интеллектуальных мультимодальных пользовательских интерфейсов} такие системы должны эффективно хранить и обрабатывать знания о пользователе, взаимодействии с ним и другую необходимую информацию. Большинство таких систем используют онтологическую модель для хранения информации для адаптации пользовательского интерфейса, а также декларативные языки описания \textit{пользовательского интерфейса}.}
	\begin{scnindent}
		\begin{scnrelfromlist}{источник}
			\scnitem{\scncite{Hitz2016}}
			\scnitem{\scncite{Liu2005}}
			\scnitem{\scncite{Gaulke2015}}
			\scnitem{\scncite{Gribova2005}}
			\scnitem{\scncite{Gribova2011a}}
			\scnitem{\scncite{Abrams1999}}
			\scnitem{\scncite{Paterno2009}}
			\scnitem{\scncite{Limbourg2004}}
			\scnitem{\scncite{Puerta1994}}
		\end{scnrelfromlist}
		\scntext{примечание}{Онтологический подход позволяет:
		\begin{scnitemize}
			\item создать наиболее полное \uline{унифицированное описание} различных аспектов пользовательского интерфейса;
			\item \uline{легко интегрировать} различные аспекты \textit{пользовательского интерфейса};
			\item \uline{упростить повторное использование} модели интерфейса.
		\end{scnitemize}
		}
	\end{scnindent}
	\scntext{примечание}{База знаний адаптивного интеллектуального мультимодального интерфейса должна включать как минимум следующие предметные области:
		\begin{scnitemize}
			\item Предметную область и онтологию модели пользователя;
			\item Предметную область и онтологию компонентов интерфейса;
			\item Предметную область и онтологию интерфейсных действий;
			\item Предметную область и онтологию контекста использования.
		\end{scnitemize}
	}
	
	\scnheader{онтология модели пользователя}
	\scntext{описание примера}{Среди существующих онтологий модели пользователя можно выделить онтологию GUMO, в рамках которой выделяют:
		\begin{scnitemize}
			\item физиологическое состояние - может измениться за секунды;
			\item психическое состояние - может измениться за минуты;
			\item эмоциональное состояние - может измениться за часы;
			\item характер - может измениться за месяцы;
			\item личность - может измениться за годы;
			\item демография - обычно не может измениться.
		\end{scnitemize}
	}
	\begin{scnindent}
		\scnrelfrom{источник}{\scncite{Heckmann2007}}
	\end{scnindent}
	
	\scnheader{онтология компонентов интерфейсов}
	\scntext{описание примера}{На верхнем уровне рассматриваются следующие типы компонентов:
		\begin{scnitemize}
			\item компонент пользовательского интерфейса для отображения;
			\item декоративный компонент пользовательского интерфейса;
			\item интерактивный компонент пользовательского интерфейса;
			\item компонент ввода данных;
			\item компонент для манипуляции отображением;
			\item компонент для запуска операций;
			\item контейнер;
			\item окно;
			\item модальное окно;
			\item немодальное окно;
		\end{scnitemize}
	}
	\begin{scnindent}
		\scnrelfrom{источник}{\scncite{Paulheim2013}}
	\end{scnindent}
	\scntext{примечание}{В данную онтологию также можно включить классы свойств компонента, которые определяют оформление внешнего вида интерфейсных элементов, начиная от простых, таких как шрифт, цвет, размер элементов, до составных,	содержащих наборы интерфейсных решений.}
	\begin{scnindent}
		\scnrelfrom{источник}{\scncite{Gribova2022}}
	\end{scnindent}

	\scnheader{онтология интерфейсных действий}
	\scntext{описание примера}{Классификация интерфейсных содержит следующие основные классы:
		\begin{scnitemize}
			\item действие мышью;
			\item действие речью;
			\item действие осязания;
			\item действие прикосновения.
		\end{scnitemize}
	}
	\begin{scnindent}
		\scnrelfrom{источник}{\scncite{Paulheim2013}}
	\end{scnindent}
	
	\scnheader{онтология контекста использования}
	\scntext{описание примера}{
		\begin{scnitemize}
			\item Статус пользователей:
			\begin{scnitemizeii}
				\item Движение (стояние, сидение, ходьба);
				\item Возможность слушать (да, нет);
				\item Возможность печатать (да, нет);
				\item Возможность читать (да, нет);
			\end{scnitemizeii}
			\item Естественная среда:
			\begin{scnitemizeii}
				\item Освещение (яркий, умеренно освещенный, темный);
				\item Шум (шумный, тихий);
				\item Ветер (сильный, слабый, безветрие);
				\item Погода (солнечно, облачно, дождливо);
				\item Температура (жарко, тепло, холодно);
				\item Местоположение (в офисе, в аэропорту, на улице, в библиотеке, дома, в торговом центре);
			\end{scnitemizeii}
			\item Особенности устройства:
			\begin{scnitemizeii}
				\item Размер экрана (большой, маленький);
				\item Тип экрана (монохромный, цветной);
				\item Клавиатура (большая, маленькая, виртуальная).
			\end{scnitemizeii}
		\end{scnitemize}
	}
	\begin{scnindent}
		\scnrelfrom{источник}{\scncite{Kong2011}}
	\end{scnindent}

	\scnheader{интеллектуальный агент}
	\scntext{примечание}{Для управления взаимодействием пользователя с системой принято использовать \textit{интеллектуальные агенты}.}
	\scntext{примечание}{\textit{интеллектуальный агент} способен выполнять гибкое автономное действие для достижения своих целей. Согласно данному определению, гибкость означает:
		\begin{scnitemize}
			\item реактивность (интеллектуальные агенты способны воспринимать свою среду и реагировать в своевременном режиме на изменения в ней, чтобы удовлетворить свои цели проектирования);
			\item проактивность (интеллектуальные агенты способны проявлять целенаправленное поведение, инициируя действия для достижения своих целей проектирования);
			\item социальная способность (интеллектуальные агенты способны взаимодействовать с другими агентами (и, возможно, людьми) с целью удовлетворения своих целей проектирования).
		\end{scnitemize}
	}
	\scntext{примечание}{\textit{интеллектуальные агенты} направлены на единственную цель, но обладают большим знанием о рассуждении в пределах своей деятельности. Умение использовать другие ресурсы (других агентов), предпочтения пользователя или клиента и другие способности являются признаками интеллектуального агента.}
	
	\scnheader{Анализ и проблемы существующих принципов организации интерфейсов}
	\begin{scnrelfromlist}{вывод}
		\scnfileitem{Для перехода к \uline{парадигме равноправного сотрудничества пользователя и системы} интерфейсы должны быть адаптивными, интеллектуальными и мультимодальными. Существующие решения позволяют проектировать такие интерфейсы, однако имеют ряд недостатков.}
		\scnfileitem{Структура \textit{интеллектуальных интерфейсов} включает базу знаний, модуль управления взаимодействием пользователя с системой.}
		\scnfileitem{При проектировании базы знаний активно применяется онтологический подход и уже реализованы некоторые онтологии, которые используются при проектировании интеллектуальных интерфейсов.}
		\scnfileitem{Модуль управления взаимодействием пользователя с системой, как правило, реализуется на основе многоагентного подхода.}
	\end{scnrelfromlist}
	
	\scnheader{Анализ и поблемы существующих принципов организации интерфейсов}
	\scnrelfrom{недостатки текущего состояния}{Недостатки текущего состояния принципов организации интерфейсов}
	\begin{scnindent}
		\begin{scneqtoset}
			\scnfileitem{Существующие решения, как правило, предусматривают \uline{вопросно-ответный принцип взаимодействия}.}
			\scnfileitem{Актуальной остается \uline{проблема совместимости} интеллектуального интерфейса с интеллектуальной системой, для которой он создается, в силу различий используемых средств и методов при проектировании и реализации.}
			\scnfileitem{Актуальной остается \uline{проблема совместимости} компонентов интеллектуального интерфейса (база знаний и модуль управления взаимодействием) между собой.}
		\end{scneqtoset}
		\scntext{примечание}{Для устранения недостатков существующих решений предлагается использовать онтологический подход на основе семантической модели при проектировании и реализации \textit{адаптивного интеллектуального мультимодального пользовательского интерфейса}.}
	\end{scnindent}
\end{scnsubstruct}

\scnheader{интерфейс}
\scnidtf{совокупность технических, программных и методических (протоколов, правил, соглашений) средств, обеcпечивающих обмен информацией между пользователем, устройствами и программами, а также между устройствами и другими устройствами и программами}
\scntext{пояснение}{В широком смысле слова, \scnkeyword{интерфейс} --- это способ (стандарт) взаимодействия между объектами.}
\scntext{пояснение}{Интерфейс в техническом смысле слова задает параметры, процедуры и характеристики взаимодействия объектов.}
\scntext{примечание}{\scnkeyword{интерфейсы} бывают разных видов. Они отличаются по характеру систем, которые взаимодействуют между собой, реализацией и функциями. Вне зависимости от типа интерфейса, взаимодействие компьютерной системы с окружающей средой происходит при помощи сенсоров и эффекторов. Ключевая задача интерфейса --- обеспечение эффективного взаимодействия с внешними субъектами (пользователями, другими ostis-системами, другими традиционными компьютерными системами).}
\begin{scnrelfromset}{разбиение}
	\scnitem{пользовательский интерфейс}
	\begin{scnindent}
		\scnidtf{один из наиболее важных компонентов компьютерной системы, представляющий собой совокупность аппаратных и программных средств, обеспечивающих обмен информацией между пользователем и компьютерной системой}
	\end{scnindent}
	\scnitem{программный интерфейс}
	\begin{scnindent}
		\scnidtf{система унифицированных связей, предназначенных для обмена информацией между компонентами вычислительной системы, задающая набор необходимых процедур, их параметров и способов обращения}
	\end{scnindent}
	\scnitem{физический интерфейс}
	\begin{scnindent}
		\scnidtf{устройство, преобразующее сигналы и передающее их от одного компонента оборудования к другому, определяющееся набором электрических связей и характеристиками сигналов}
	\end{scnindent}
\end{scnrelfromset}

\scnheader{адаптивный интерфейс}
\scnidtf{\textit{пользовательский интерфейс}, который изменяется на основе потребностей пользователя или контекста}
\scntext{примечание}{Как правило, контекст использования адаптивного интерфейса состоит из знаний о пользователе, платформе и среде.}
\begin{scnindent}
	\scnrelfrom{источник}{\scncite{Hussain2018}}
	\scnrelfrom{иллюстрация}{\scnfileimage[40em]{Contents/part_ui/src/images/sd_ui/user-context.png}}
	\begin{scnindent}
		\scnidtf{Рисунок. Контекст использования системы}
	\end{scnindent}
\end{scnindent}
\scntext{примечание}{Настройка функциональных возможностей и параметров интерфейса может осуществляться либо вручную самим пользователем, либо автоматически системой на основании имеющейся информации о пользователе. Таким образом, следует различать адаптивные и адаптируемые системы, эти термины не являются синонимами, хотя в литературе довольно часто можно встретить подмену данных понятий.}
\begin{scnindent}
	\scnrelfrom{источник}{\scncite{Valeev2018}}
\end{scnindent}
\scntext{примечание}{В адаптируемых системах любая адаптация является предопределенной и может изменяться пользователями перед запуском системы. В адаптивных же системах, напротив, любая адаптация является динамической, то есть происходит в то же время, когда пользователь взаимодействует с системой, и зависит от поведения пользователя. Но система также может быть адаптируемой и адаптивной одновременно. Недостаток ручного редактирования интерфейса заключается в необходимости пользователя быть достаточно хорошо знакомым, как с самой системой, так и со средствами, позволяющими изменять ее \scnkeyword{интерфейс}.}
\begin{scnindent}
	\scnrelfrom{источник}{\scncite{Montero2005}}
\end{scnindent}

\scnheader{адаптированный интерфейс}
\scntext{определение}{\scnkeyword{адаптированный интерфейс} --- это \scnkeyword{пользовательский интерфейс}, который адаптирован к конечному пользователю при проектировании и не изменяется во время эксплуатации системы.}
\begin{scnindent}
	\scnrelfrom{источник}{\scncite{Schlungbaum1997}}
\end{scnindent}

\scnheader{интеллектуальный интерфейс}
\scnidtf{\textit{пользовательский интерфейс}, который может предположить дальнейшие действия пользователей и представить информацию на основе этого предположения}
\scntext{примечание}{Можно заметить, что понятия \textit{интеллектуальный} и \textit{адаптивный интерфейс} имеют отличия. Однако, в различных статьях эти понятия рассматриваются как синонимы.}
\begin{scnindent}
	\begin{scnrelfromlist}{источник}
		\scnitem{\scncite{Brdnik2022}}
		\scnitem{\scncite{Volkel2020}}
		\scnitem{\scncite{Ehlert2003}}
	\end{scnrelfromlist}
\end{scnindent}

\scnheader{мультимодальный интерфейс}
\scnidtf{\textit{пользовательский интерфейс}, предназначенный для обработки двух или более комбинированных режимов пользовательского ввода, таких как речь, перо, касание, ручные жесты и взгляд, скоординированным образом с выводом мультимедийной системы}

\scnheader{пользовательский интерфейс}
\scnsuperset{командный пользовательский интерфейс}
\scnsuperset{графический пользовательский интерфейс}	
\begin{scnindent}
	\scnsuperset{WIMP-интерфейс}
	\begin{scnindent}
		\scnsuperset{пользовательский интерфейс ostis-системы}
		\begin{scnindent}
			\scnhaselement{Пользовательский интерфейс Метасистемы IMS.ostis}
			\scnhaselement{Пользовательский интерфейс ИСС по геометрии}
			\begin{scnindent}
				\scnidtf{Пользовательский интерфейс интеллектуальной справочной системы по геометрии}
			\end{scnindent}
			\scnhaselement{Пользовательский интерфейс ИСС по дискретной математике}
			\begin{scnindent}
				\scnidtf{Пользовательский интерфейс интеллектуальной справочной системы по дискретной математике}
			\end{scnindent}
			\scnhaselement{Пользовательский интерфейс ИСС по географии}
			\begin{scnindent}
				\scnidtf{Пользовательский интерфейс интеллектуальной справочной системы по 	географии}
			\end{scnindent}
			\scnhaselement{Пользовательский интерфейс ИСС по искусственным нейронным сетям}
			\begin{scnindent}
				\scnidtf{Пользовательский интерфейс интеллектуальной справочной системы по искусственным нейронным сетям}
			\end{scnindent}
			\scnhaselement{Пользовательский интерфейс ИСС по лингвистике}
			\begin{scnindent}
				\scnidtf{Пользовательский интерфейс интеллектуальной справочной системы по лингвистике}
			\end{scnindent}
		\end{scnindent}
	\end{scnindent}
	\scnsuperset{SILK-интерфейс}
	\begin{scnindent}
		\scnsuperset{естественно-языковой интерфейс}
		\begin{scnindent}
			\scnsuperset{речевой интерфейс}
		\end{scnindent}
	\end{scnindent}
\end{scnindent}

\scnheader{пользовательский интерфейс}
\scntext{пояснение}{\textit{пользовательский интерфейс} --- один из наиболее важных компонентов компьютерной системы. Представляет собой совокупность аппаратных и программных средств, обеспечивающих обмен информацией между пользователем и компьютерной системой.}

\scnheader{командный пользовательский интерфейс}
\scntext{пояснение}{\textit{командный пользовательский интерфейс} --- пользовательский интерфейс, при котором обмен информацией между компьютерной системой и пользователем осуществляется путем написания текстовых инструкций или команд.}

\scnheader{графический пользовательский интерфейс}
\scntext{пояснение}{\textit{графический пользовательский интерфейс} --- пользовательский интерфейс, при котором обмен информацией между компьютерной системой и пользователем осуществляется при помощи графических компонентов компьютерной системы.}

\scnheader{WIMP-интерфейс}
\scnidtf{Window, Image, Menu, Pointer --- интерфейс}
\scnidtf{Окно, Образ, Меню, Указатель --- интерфейс}
\scntext{пояснение}{\textit{WIMP-интерфейс} --- пользовательский интерфейс, при котором обмен информацией между компьютерной системой и пользователем осуществляется в форме диалога при помощью окон, меню и других элементов управления.}

\scnheader{SILK-интерфейс}
\scnidtf{Speech, Image, Language, Knowledge --- интерфейс}
\scnidtf{Речь, Образ, Язык, Знание --- интерфейс}
\scntext{пояснение}{\textit{SILK-интерфейс} --- пользовательский интерфейс, наиболее приближенный к естественной для человека форме общения. Компьютерная система находит для себя команды, анализируя человеческую речь и находя в ней ключевые фразы. Результат выполнения команд преобразуется в понятную человеку форму, например, в естественно-языковую форму или изображение.}

\scnheader{естественно-языковой интерфейс}
\scntext{пояснение}{\textit{естественно-языковой интерфейс} --- SILK-интерфейс, обмен информацией между компьютерной системой и пользователем в котором происходит за счёт диалога. Диалог ведётся на одном из естественных языков.}

\scnheader{речевой интерфейс}
\scntext{пояснение}{\textit{речевой интерфейс} --- SILK-интерфейс, обмен информацией в котором происходит за счёт диалога, в процессе которого компьютерная система и пользователь общаются с помощью речи. Данный вид интерфейса наиболее приближен к естественному общению между людьми.}

\scnheader{пользовательский интерфейс ostis-системы}
\scnsubset{ostis-система}
\scnsubset{адаптивный интеллектуальный мультимодальный пользовательский интерфейс}
\scntext{пояснение}{\textit{пользовательский интерфейс ostis-системы} представляет собой специализированную \textit{ostis-систему}, ориентированную на решение интерфейсных задач и имеющую в своем составе базу знаний и решатель задач пользовательского интерфейса ostis-системы.}
\begin{scnindent}
	\scnrelfrom{источник}{\scncite{Boriskin2017}}
\end{scnindent}
\begin{scnrelfromset}{принципы, лежащие в основе}
	\scnfileitem{Для решения задачи построения пользовательского интерфейса в базе знаний \textit{пользовательского интерфейса ostis-системы} необходимо наличие sc-модели \textit{компонентов пользовательского интерфейса}, \textit{интерфейсных действий пользователей}, а также классификации \textit{пользовательских интерфейсов} в целом.}
	\scnfileitem{\textit{решатель задач пользовательского интерфейса ostis-систем} основан на многоагентном подходе, а сами агенты имеют возможность инициирования действий и сообщений пользователю и другим агентам.}
	\begin{scnindent}
		\scnrelto{примечание}{решатель задач пользовательского интерфейса ostis-систем}
	\end{scnindent}
	\scnfileitem{При проектировании интерфейса используется компонентный подход, который предполагает представление всего интерфейса приложения в виде отдельных специфицированных компонентов, которые могут разрабатываться и совершенствоваться независимо.}
	\begin{scnindent}
		\begin{scnrelfromlist}{источник}
		\scnitem{\scncite{Koronchik2011}}
		\scnitem{\scncite{Koronchik2013interface}}
		\end{scnrelfromlist}
	\end{scnindent}
	\scnfileitem{Каждый компонент \textit{пользовательского интерфейса ostis-системы} является внешним отображением определенного элемента из базы знаний, что позволяет использовать их в качестве аргументов пользовательских команд и правильно трактовать прагматику и семантику объектов интерфейсной деятельности, легко изменять интерфейс системы даже во время ее эксплуатации, позволяет пользователю задавать системе вопросы касательно любого из компонентов интерфейса.}
	\scnfileitem{Модель \textit{пользовательского интерфейса ostis-системы} строится независимо от реализации платформы интерпретации такой модели, что позволяет легко переносить разработанную модель на разные платформы.}
	\scnfileitem{Описание модели базы знаний и решателя задач \textit{пользовательского интерфейса ostis-системы} предлагается осуществлять на основе универсального унифицированного языка представления знаний, что обеспечит совместимость между этими компонентами.}
\end{scnrelfromset}
\scnrelfrom{структура}{\scnfileimage[40em]{Contents/part_ui/src/images/sd_ui/ui_model.png}}
\begin{scnindent}
	\scnidtf{Рисунок. Структура пользовательского интерфейса ostis-системы}
\end{scnindent}

\scnheader{компонент пользовательского интерфейса}
\scntext{пояснение}{\textit{компонент пользовательского интерфейса} --- знак фрагмента базы знаний, имеющий определённую форму внешнего представления на экране.}
\begin{scnsubdividing}
	\scnitem{атомарный компонент пользовательского интерфейса}
	\scnitem{неатомарный компонент пользовательского интерфейса}
\end{scnsubdividing}

\scnheader{атомарный компонент пользовательского интерфейса}
\scntext{пояснение}{\textit{атомарный компонент пользовательского интерфейса} --- компонент пользовательского интерфейса, не содержащий в своём составе других компонентов пользовательского интерфейса.}

\scnheader{неатомарный компонент пользовательского интерфейса}
\scntext{пояснение}{\textit{неатомарный компонент пользовательского интерфейса} --- компонент пользовательского интерфейса, состоящий из других компонентов пользовательского интерфейса.}

\scnheader{визуальная часть пользовательского интерфейса ostis-системы}
\scnsubset{неатомарный компонент пользовательского интерфейса}
\scntext{пояснение}{\textit{визуальная часть пользовательского интерфейса ostis-системы} --- часть базы знаний пользовательского интерфейса ostis-системы, содержащая необходимые для отображения пользовательского интерфейса компоненты.}

\scnheader{компонент пользовательского интерфейса}
\scnidtf{user interface component}
\scnsuperset{компонент пользовательского интерфейса для отображения}
\begin{scnindent}
	\scnidtf{presentation user interface component}
	\scnsuperset{компонент вывода}
	\begin{scnindent}
		\scnidtf{output}
		\scnsuperset{компонент вывода изображения}
		\begin{scnindent}
			\scnidtf{image-output}
		\end{scnindent}
		\scnsuperset{компонент вывода графической информации}
		\begin{scnindent}
			\scnidtf{graphical-output}
			\scnsuperset{диаграмма}
			\begin{scnindent}
				\scnidtf{chart}
			\end{scnindent}
			\scnsuperset{карта}
			\begin{scnindent}
				\scnidtf{map}
			\end{scnindent}
			\scnsuperset{индикатор выполнения}
			\begin{scnindent}
				\scnidtf{progress-bar}
			\end{scnindent}
		\end{scnindent}
		\scnsuperset{компонент вывода видео}
		\begin{scnindent}
			\scnidtf{video-output}
		\end{scnindent}
		\scnsuperset{компонент вывода звука}
		\begin{scnindent}
			\scnidtf{sound-output}
		\end{scnindent}
		\scnsuperset{компонент вывода текста}
		\begin{scnindent}
			\scnidtf{text-output}
			\scnsuperset{заголовок}
			\begin{scnindent}
				\scnidtf{headline}
			\end{scnindent}
			\scnsuperset{параграф}
			\begin{scnindent}
				\scnidtf{paragraph}
			\end{scnindent}
			\scnsuperset{сообщение}
			\begin{scnindent}
				\scnidtf{message}
			\end{scnindent}
		\end{scnindent}
	\end{scnindent}
	\scnsuperset{декоративный компонент пользовательского интерфейса}
	\begin{scnindent}
		\scnidtf{decorative user interface component}
		\scnsuperset{разделитель}
		\begin{scnindent}
			\scnidtf{separator}
		\end{scnindent}
		\scnsuperset{пустое пространство}
		\begin{scnindent}
			\scnidtf{blank-space}
		\end{scnindent}
	\end{scnindent}
	\scnsuperset{контейнер}
	\begin{scnindent}
		\scnidtf{container}
		\scnsuperset{меню}
		\begin{scnindent}
			\scnidtf{menu}
		\end{scnindent}
		\scnsuperset{строка меню}
		\begin{scnindent}
			\scnidtf{menu-bar}
		\end{scnindent}
		\scnsuperset{панель инструментов}
		\begin{scnindent}
			\scnidtf{tool-bar}
		\end{scnindent}
		\scnsuperset{строка состояния}
		\begin{scnindent}
			\scnidtf{status-bar}
		\end{scnindent}
		\scnsuperset{таблично-строковый контейнер}
		\begin{scnindent}
			\scnidtf{table-row-container}
		\end{scnindent}
		\scnsuperset{списковый контейнер}
		\begin{scnindent}
			\scnidtf{list-container}
		\end{scnindent}
		\scnsuperset{таблично-клеточный контейнер}
		\begin{scnindent}
			\scnidtf{table-cell-container}
		\end{scnindent}
		\scnsuperset{древовидный контейнер}
		\begin{scnindent}
			\scnidtf{tree-container}
		\end{scnindent}
		\scnsuperset{панель вкладок}
		\begin{scnindent}
			\scnidtf{tab-pane}
		\end{scnindent}
		\scnsuperset{панель вращения}
		\begin{scnindent}
			\scnidtf{spin-pane}
		\end{scnindent}
		\scnsuperset{узловой контейнер}
		\begin{scnindent}
			\scnidtf{tree-node-container}
		\end{scnindent}
		\scnsuperset{панель прокрутки}
		\begin{scnindent}
			\scnidtf{scroll-pane}
		\end{scnindent}
		\scnsuperset{окно}
		\begin{scnindent}
			\scnidtf{window}
			\scnsuperset{модальное окно}
			\begin{scnindent}
				\scnidtf{modal-window}
			\end{scnindent}
			\scnsuperset{немодальное окно}
			\begin{scnindent}
				\scnidtf{non-modal-window}
			\end{scnindent}
		\end{scnindent}
	\end{scnindent}
\end{scnindent}	
\scnsuperset{интерактивный компонент пользовательского интерфейса}
\begin{scnindent}
	\scnidtf{interactive user interface component}
	\scnsuperset{компонент ввода данных}
	\begin{scnindent}
		\scnidtf{data-input-component}
		\scnsuperset{компонент ввода данных с прямой ответной реакцией}
		\begin{scnindent}
			\scnidtf{data-input-component-with-direct-feedback}
			\scnsuperset{компонент ввода текста с прямой ответной реакцией}
			\begin{scnindent}
                \scnidtf{text-input-component-with-direct-feedback}
				\scnsuperset{многострочное текстовое поле}
				\begin{scnindent}
					\scnidtf{multi-line-text-field}
				\end{scnindent}
				\scnsuperset{однострочное текстовое поле}
				\begin{scnindent}
					\scnidtf{single-line-text-field}
				\end{scnindent}
			\end{scnindent}
			\scnsuperset{ползунок}
			\begin{scnindent}
				\scnidtf{slider}
			\end{scnindent}
			\scnsuperset{область рисования}
			\begin{scnindent}
				\scnidtf{drawing-area}
			\end{scnindent}
			\scnsuperset{компонент выбора}
			\begin{scnindent}
				\scnidtf{selection-component}
				\scnsuperset{компонент выбора нескольких значений}
				\begin{scnindent}
					\scnidtf{selection-component-multiple-values}
				\end{scnindent}
				\scnsuperset{компонент выбора одного значения}
				\begin{scnindent}
					\scnidtf{selection-component-single-values}
				\end{scnindent}
			\end{scnindent}
			\scnsuperset{компонент выбора данных}
			\begin{scnindent}
				\scnidtf{selectable-data-representation}
				\scnsuperset{флаговая кнопка}
				\begin{scnindent}
					\scnidtf{check-box}
				\end{scnindent}
				\scnsuperset{радиокнопка}
				\begin{scnindent}
					\scnidtf{radio-button}
				\end{scnindent}
				\scnsuperset{переключатель}
				\begin{scnindent}
					\scnidtf{toggle-button}
				\end{scnindent}
				\scnsuperset{выбираемый элемент}
				\begin{scnindent}
					\scnidtf{selectable-item}
				\end{scnindent}
			\end{scnindent}
		\end{scnindent}	
		\scnsuperset{компонент ввода данных без прямой ответной реакции}
		\begin{scnindent}
			\scnidtf{data-input-component-without-direct-feedback}
			\scnsuperset{кнопка-счётчик}
			\begin{scnindent}
				\scnidtf{spin-button}
			\end{scnindent}
			\scnsuperset{компонент речевого ввода}
			\begin{scnindent}
				\scnidtf{speech-input}
			\end{scnindent}
			\scnsuperset{компонент ввода движений}
			\begin{scnindent}
				\scnidtf{motion-input}
			\end{scnindent}
		\end{scnindent}
	\end{scnindent}
	\scnsuperset{компонент для представления и взаимодействия с пользователем}
	\begin{scnindent}
		\scnidtf{presentation-manipulation-component}
		\scnsuperset{активирующий компонент}
		\begin{scnindent}
			\scnidtf{activating-component}
		\end{scnindent}
		\scnsuperset{компонент непрерывной манипуляции}
		\begin{scnindent}
			\scnidtf{continuous-manipulation-component}
			\scnsuperset{полоса прокрутки}
			\begin{scnindent}
				\scnidtf{scrollbar}
			\end{scnindent}
			\scnsuperset{компонент редактирования размера}
			\begin{scnindent}
				\scnidtf{resizer}
			\end{scnindent}
		\end{scnindent}
	\end{scnindent}
	\scnsuperset{компонент запроса действий}
	\begin{scnindent}
		\scnidtf{operation-trigger-component}
		\scnsuperset{компонент выбора команд}
		\begin{scnindent}
			\scnidtf{command-selection-component}
			\scnsuperset{кнопка}
			\begin{scnindent}
				\scnidtf{button}
			\end{scnindent}
			\scnsuperset{пункт меню}
			\begin{scnindent}
				\scnidtf{menu-item}
			\end{scnindent}
		\end{scnindent}
		\scnsuperset{компонент ввода команд}
		\begin{scnindent}
			\scnidtf{command-input-component}
		\end{scnindent}
	\end{scnindent}
\end{scnindent}

\scnheader{компонент пользовательского интерфейса для представления}
\scntext{пояснение}{\textit{компонент пользовательского интерфейса для представления} --- компонент пользовательского интерфейса, не подразумевающий взаимодействия с пользователем.}

\scnheader{компонент вывода}
\scntext{пояснение}{\textit{компонент вывода} --- компонент пользовательского интерфейса, предназначенный для представления информации.}

\scnheader{индикатор выполнения}
\scntext{пояснение}{\textit{индикатор выполнения} --- компонент пользовательского интерфейса, предназначенный для отображения процента выполнения какой-либо задачи.}

\scnheader{параграф}
\scntext{пояснение}{\textit{параграф} --- компонент пользовательского интерфейса, предназначенный для отображения блоков текста. Он отделяется от других блоков пустой строкой или первой строкой с отступом.}

\scnheader{декоративный компонент пользовательского интерфейса}
\scntext{пояснение}{\textit{декоративный компонент пользовательского интерфейса} --- компонент пользовательского интерфейса, предназначенный для стилизации интерфейса.}

\scnheader{контейнер}
\scntext{пояснение}{\textit{контейнер} --- компонент пользовательского интерфейса, задача которого состоит в размещении набора компонентов, включённых в его состав.}

\scnheader{меню}
\scntext{пояснение}{\textit{меню} --- компонент пользовательского интерфейса, содержащий несколько вариантов для выбора пользователем.}

\scnheader{строка меню}
\scntext{пояснение}{\textit{строка меню} --- горизонтальная полоса, содержащая ярлыки меню. Строка меню предоставляет пользователю место в окне, где можно найти большинство основных функций программы.}

\scnheader{панель инструментов}
\scntext{пояснение}{\textit{панель инструментов} --- компонент пользовательского интерфейса, на котором размещаются элементы ввода или вывода данных.}

\scnheader{панель вкладок}
\scntext{пояснение}{\textit{панель вкладок} --- контейнер, который может содержать несколько вкладок (секций) внутри, которые могут быть отображены, нажав на вкладке с названием в верхней части панели. Одновременно отображается только одна вкладка.}

\scnheader{окно}
\scntext{пояснение}{\textit{окно} --- обособленная область экрана, содержащая различные элементы пользовательского интерфейса. Окна могут располагаться поверх друг друга.}

\scnheader{модальное окно}
\scntext{пояснение}{\textit{модальное окно} --- окно, которое блокирует работу пользователя с системой до тех пор, пока пользователь окно не закроет.}

\scnheader{немодальное окно}
\scntext{пояснение}{\textit{немодальное окно} --- окно, которое позволяет выполнять переключение между данным окном и другим окном без необходимости закрытия окна.}

\scnheader{интерактивный компонент пользовательского интерфейса}
\scntext{пояснение}{\textit{интерактивный компонент пользовательского интерфейса} --- компонент пользовательского интерфейса, с помощью которого осуществляется взаимодействие с пользователем.}

\scnheader{флаговая кнопка}
\scntext{пояснение}{\textit{флаговая кнопка} --- компонент пользовательского интерфейса, позволяющий пользователю управлять параметром с двумя состояниями --- включено и отключено.}

\scnheader{радиокнопка}
\scntext{пояснение}{\textit{радиокнопка} --- компонент пользовательского интерфейса, который позволяет пользователю выбрать одну опцию из предопределенного набора.}

\scnheader{переключатель}
\scntext{пояснение}{\textit{переключатель} --- компонент пользовательского интерфейса, который позволяет пользователю переключаться между двумя состояниями.}

\scnheader{кнопка-счетчик}
\scntext{пояснение}{\textit{кнопка-счетчик} --- компонент пользовательского интерфейса, как правило, ориентированный вертикально, с помощью которого пользователь может изменить значение в прилегающем текстовом поле, в результате чего значение в текстовом поле увеличивается или уменьшается.}

\scnheader{полоса прокрутки}
\scntext{пояснение}{\textit{полоса прокрутки} --- компонент пользовательского интерфейса, который используется для отображения компонентов пользовательского интерфейса, больших по размеру, чем используемый для их отображения контейнер.}

\scnheader{кнопка}
\scntext{пояснение}{\textit{кнопка} --- компонент пользовательского интерфейса, при нажатии на который происходит программно связанное с этим нажатием действие либо событие.}

\scnheader{Стартовая страница пользовательского интерфейса Метасистемы IMS.ostis}
\scniselement{Визуальная часть пользовательского интерфейса Метасистемы IMS.ostis}
\begin{scnindent}
	\scnsubset{визуальная часть пользовательского интерфейса ostis-системы}
	\scnrelto{часть}{Пользовательский интерфейс Метасистемы IMS.ostis}
\end{scnindent}	
\scniselement{окно}
\scnrelfrom{иллюстрация}{\scnfileimage[40em]{Contents/part_ui/src/images/sd_ui/startPage.png}}
\begin{scnrelfromset}{декомпозиция}
    \scnitem{Панель навигации}
	\begin{scnindent}
		\scniselement{неатомарный компонент пользовательского интерфейса}
		\begin{scnrelfromset}{декомпозиция}
			\scnitem{Главное меню}
			\begin{scnindent}
				\scniselement{меню}
				\begin{scnrelfromset}{декомпозиция}
					\scnitem{Пункт меню для навигации по ключевым понятиям}
					\begin{scnindent}
						\scniselement{пункт меню}
					\end{scnindent}
					\scnitem{Пункт меню для выполнения команд просмотра базы знаний}
					\begin{scnindent}
						\scniselement{пункт меню}
					\end{scnindent}
					\scnitem{Компонент перехода в экспертный режим}
					\begin{scnindent}
						\scniselement{переключатель}
					\end{scnindent} 
				\end{scnrelfromset}
			\end{scnindent} 
			\scnitem{Компонент выбора языка}
			\begin{scnindent}
				\scniselement{компонент выбора одного значения}
			\end{scnindent} 
			\scnitem{Компонент авторизации}
			\begin{scnindent}
				\scniselement{кнопка}
			\end{scnindent} 
		\end{scnrelfromset}
	\end{scnindent}	 
    \scnitem{Блок истории запросов пользователя}
    	\begin{scnindent}
			\scniselement{неатомарный компонент пользовательского интерфейса}
		\end{scnindent}
    \scnitem{Основной блок}
	\begin{scnindent}
			\scniselement{неатомарный компонент пользовательского интерфейса}
		\begin{scnrelfromset}{декомпозиция}
			\scnitem{Главное}
			\begin{scnindent}
				\scniselement{окно}
			\end{scnindent}
			\scnitem{Панель инструментов}
			\begin{scnindent}
				\scniselement{неатомарный компонент пользовательского интерфейса}
				\begin{scnrelfromset}{декомпозиция}
					\scnitem{Кнопка отправки содержимого главного окна на печать}
					\begin{scnindent}
						\scniselement{кнопка}
					\end{scnindent} 
					\scnitem{Кнопка управления видимостью блока истории запросов пользователя}
					\begin{scnindent}
						\scniselement{кнопка}
					\end{scnindent} 
					\scnitem{Кнопка отображения ссылки на текущий запрос пользователя}
					\begin{scnindent}
						\scniselement{кнопка}
					\end{scnindent} 
					\scnitem{Поле поиска}
					\begin{scnindent}
						\scniselement{однострочное текстовое поле}
					\end{scnindent} 
				\end{scnrelfromset}
			\end{scnindent} 
		\end{scnrelfromset}
	\end{scnindent}
    \scnitem{Панель отображения информации об авторских правах}
	\begin{scnindent}
		\scniselement{неатомарный компонент пользовательского интерфейса}
	\end{scnindent}
\end{scnrelfromset}

\bigskip
\end{scnsubstruct}
\scnendcurrentsectioncomment
\end{SCn}
