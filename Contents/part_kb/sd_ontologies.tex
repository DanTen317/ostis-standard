\begin{SCn}
    \scnsectionheader{Предметная область и онтология онтологий}
    \begin{scnsubstruct}
        \begin{scnreltovector}{конкатенация сегментов}
            \scnitem{Что такое онтология}
            \scnitem{Типология онтологий предметной области}
            \scnitem{Понятие объединенной онтологии предметной области, понятие предметной области и онтологии}
            \scnitem{Отношения, заданные на множестве онтологий}
        \end{scnreltovector}
        \scnheader{Предметная область \textit{онтологий}}
        \scnidtf{Предметная область теории \textit{онтологий}}
        \scnidtf{Предметная область, объектами исследования которой являются \textit{онтологии}}
        \scniselement{предметная область}
        \begin{scnhaselementrole}{максимальный класс объектов исследования}
            {онтология}
        \end{scnhaselementrole}
        \begin{scnhaselementrolelist}{класс объектов исследования}
            \scnitem{объединенная онтология}
            \scnitem{структурная спецификация предметной области}
            \scnitem{теоретико-множественная онтология предметной области}
            \scnitem{логическая онтология предметной области}
            \scnitem{логическая иерархия понятий предметной области}
            \scnitem{логическая иерархия высказываний предметной области}
            \scnitem{терминологическая онтология предметной области}
        \end{scnhaselementrolelist}
        \begin{scnhaselementrolelist}{исследуемое отношение}
            \scnitem{онтология*}
            \scnitem{используемые константы*}
            \scnitem{используемые утверждения*}
        \end{scnhaselementrolelist}
        
        \scnsegmentheader{Что такое онтология}
        \begin{scnsubstruct}
        	
            \scnheader{онтология}
            \scnidtf{sc-онтология}
            \scnidtf{онтология, представленная в SC-коде}
            \scntext{примечание}{Поскольку термин \scnqqi{\textit{онтология}} в SC-коде соответствует множеству всевозможных онтологий, представленных в SC-коде, то для формальных онтологий, представленных на других языках, необходимо использовать sc-идентификатор, содержащий явное указание этих языков, например, \scnqqi{owl-онтология}.}
            \scnidtf{sc-текст онтологии}
            \scnidtf{sc-модель онтологии}
            \scnidtf{семантическая спецификация \textit{знаний}}
            \scnidtf{семантическая спецификация любого знания, имеющего достаточно сложную структуру, любого целостного фрагмента базы знаний --- предметной области, метода решения сложных задач некоторого класса, описания истории некоторого вида деятельности, описания области выполнения некоторого множества действий (области решения задач), языка представления методов решения задач и т.д.}
            \scnidtftext{пояснение}{\uline{семантическая} \textit{спецификация} некоторого достаточно информативного ресурса (\textit{знания})}
            \scnsubset{спецификация}
            \scntext{примечание}{Если \textit{спецификация} может специфицировать (описывать) любую \textit{сущность}, то \textit{отнология} специфицирует только различные \textit{знания}. При этом наиболее важными объектами такой спецификации являются \textit{предметные области}}
            \scnsubset{метазнание}
            \scniselement{вид знаний}
            \scntext{примечание}{\textit{онтологии} являются важнейшим \textit{видом знаний} (точнее, метазнаний), обеспечивающих семантическую систематизацию \textit{знаний}, хранимых в памяти \textit{интеллектуальных компьютерных систем} (в т.ч. \textit{ostis-систем}), и, соответственно, семантическую структуризацию \textit{баз знаний}}
            \scnidtf{важнейший вид \textit{метазнаний}, входящих в состав базы знаний}
            \scnidtf{спецификация (уточнение) системы \textit{понятий}, используемых в соответствующем (специфицируемом) \textit{знании}}
            \scntext{эпиграф}{Определив точно значения слов, вы избавите человечество от половины заблуждений}
            \begin{scnindent}
            	\scnrelfrom{автор}{Рене Декарт}
            \end{scnindent}
            \scntext{пояснение}{\textit{онтология} включает в себя:
                \begin{scnitemize}
                    \item типологию специфицируемого \textit{знания};
                    \item связи специфицируемого \textit{знания} с другими \textit{знаниями};
                    \item спецификацию ключевых \textit{понятий}, используемых в специфицируемом \textit{знании}, а также ключевых экземпляров некоторых таких \textit{понятий}.
                \end{scnitemize}}
            \scntext{пояснение}{Основная \textit{цель} построения \textit{онтологии} --- семантическое уточнение (пояснение, а в идеале --- определение) такого семейства \textit{знаков}, используемых в заданном \textit{знании}, которых достаточно для понимания смысла всего специфицируемого \textit{знания}. Как выясняется, количество \textit{знаков}, смысл которых определяет смысл всего специфицируемого \textit{знания}, \uline{не является большим}.}
            \begin{scnsubdividing}
                \scnitem{неформальная онтология}
                \scnitem{формальная онтология}
                \begin{scnindent}
                    \scnidtf{онтология, представленная на формальном языке}
                    \scnrelto{ключевой знак}{\cite{Loukashevich2011}}
                \end{scnindent}
            \end{scnsubdividing}
            
            \scnheader{формальная онтология}
            \scnidtf{формальное описание \uline{денотационной семантики} (семантической интерпретации) специфицируемого знания}
            \scntext{примечание}{Очевидно, что при отсутствии достаточно полных формальных онтологий невозможно обеспечить семантическую совместимость (интегрируемость) различных знаний, хранимых в базе знаний, а также приобретаемых извне.}
            
            \scnheader{онтология предметной области}
            \scntext{примечание}{\textit{онтология} чаще всего трактуется как спецификация концептуализации (спецификация системы \textit{понятий}) заданной \textit{предметной области}. Здесь имеется в виду описание теоретико-множественных связей (прежде всего, классификации) используемых \textit{понятий}, а также описание различных закономерностей для сущностей, принадлежащих этим \textit{понятиям}. Тем не менее, важными видами спецификации \textit{предметной области} являются также:
                \begin{scnitemize}
                    \item описание связей специфицируемой \textit{предметной области} с другими \textit{предметными областями};
                    \item описание терминологии специфицируемой \textit{предметной области}.
                \end{scnitemize}}
            \scntext{примечание}{\textit{онтологию предметной области} можно трактовать, с одной стороны, как \textit{семантическую окрестность} соответствующей \textit{предметной области}, с другой стороны, как \textit{объединение} определённого вида \textit{семантических окрестностей} всех \textit{понятий}, используемых в рамках указанной \textit{предметной области}, а также, возможно, ключевых экземпляров указанных \textit{понятий}, если таковые экземпляры имеются}
            \scntext{пояснение}{Каждая конкретная онтология заданного вида представляет собой семантическую окрестность соответствующей (специфицируемой) предметной области.Каждому \textit{виду онтологий} однозначно соответствует \textit{предметная область}, фрагментами которые являются конкретные \textit{онтологии} этого вида. Следовательно, каждому \textit{виду онтологий} соответствует свой специализированный sc-язык, обеспечивающий представление \textit{онтологий} этого вида.}
            \scnidtf{описание \textit{денотационной семантики} языка, определяемого (задаваемого) соответствующей (специфицируемой) \textit{предметной области}}
            \scnidtf{информационная надстройка (метаинформация) над соответствующей (специфицируемой) \textit{предметной областью}, описывающая различные аспекты этой \textit{предметной области} как достаточно крупного, самодостаточного и семантически целостного фрагмента \textit{база знаний}}
            \scnidtf{метаинформация (метазнание) о некоторой \textit{предметной области}}
            
            \bigskip
            
        \end{scnsubstruct}
        \scnendsegmentcomment{Что такое онтология}
        \scnsegmentheader{Типология онтологий предметной области}
        \begin{scnsubstruct}
            \scnheader{онтология предметной области}
            \begin{scnsubdividing}
                \scnitem{частная онтология предметной области}
                \begin{scnindent}
                    \scnidtf{\textit{онтология}, представляющая спецификацию соответствующей \textit{предметной области} в том или ином аспекте}
                \end{scnindent}
                \scnitem{объединённая онтология предметной области}
                \begin{scnindent}
                    \scnidtf{онтология \textit{предметной области}, являющаяся результатом объединения всех известных \textit{частных онтологий} этой предметной области}
                \end{scnindent}
            \end{scnsubdividing}
            
            \scnheader{частная онтология предметной области}
            \scntext{примечание}{Каждая \textit{частная онтология} является фрагментом \textit{предметной области}, включающей в себя \uline{все}(!) частные онтологии, принадлежащие соответствующему \textit{виду онтологии}. При этом указанная \textit{предметная область}, в свою очередь, также имеет соответствующую ей \textit{онтологию}, которая уже является не метазнанием (как любая онтология), а метаметазнанием (спецификацией метазнания).}
            \scnrelfrom{разбиение}{вид онтологий предметных областей}
            \begin{scnindent}
	            \begin{scneqtoset}
	                \scnitem{структурная спецификация предметной области}
	                \begin{scnindent}
	                    \scnidtf{sc-окрестность (sc-спецификация) заданной предметной области в рамках \textit{Предметной области предметных областей}}
	                    \scnidtf{схема предметной области}
	                \end{scnindent}
	                \scnitem{теоретико-множественная онтология предметной области}
	                \begin{scnindent}
	                	\scnidtf{sc-спецификация заданной предметной области в рамках \textit{Предметной области множеств}}
	                \end{scnindent}
	                \scnitem{логическая онтология предметной области}
	                \begin{scnindent}
	                	\scnidtf{sc-текст формальной теории заданной предметной области}
	                \end{scnindent}
	                \scnitem{терминологическая онтология предметной области}
	            \end{scneqtoset}
            \end{scnindent}
            
            \scnheader{вид онтологий предметных областей}
            \scnidtf{вид спецификаций \textit{предметных областей}}
            \scnidtf{вид \textit{метазнаний}, описывающих соответствующие этому виду \textit{метазнаний} свойства \textit{предметных областей}}
            \scntext{примечание}{Каждому виду \textit{онтологий предметных областей}, представленных в \textit{SC-коде} (здесь представленными в \textit{SC-коде} предполагаются не только сами \textit{онтологии}, но и специфицируемые ими \textit{предметные области}) ставится в соответствие \textit{sc-метаязык}, обеспечивающий представление \textit{метазнаний}, входящих в состав указанного \textit{вида онтологий}. Кроме того, обычное теоретико-множественное \textit{объединение} всех \textit{sc-текстов} указанного \textit{sc-метаязыка} означает построение \textit{предметной области} (предметной метаобласти), которая формально задаёт соответствующий \textit{вид онтологии} и которая будет иметь свою \textit{онтологию}. При этом каждую частную онтологию можно связать с той предметной областью, фрагментом которой эта онтология является.}
            
            \scnheader{логическая онтология предметной области}
            \scnrelto{семейство подмножеств}{Предметная область логических формул, высказываний и формальных теорий}
            \scnheader{теоретико-множественная онтология предметной области}
            \scnrelto{семейство подмножеств}{Предметная область множеств}
            \scnheader{структурная спецификация предметной области}
            \scnrelto{семейство подмножеств}{Предметная область предметных областей}
            
            \scnheader{структурная спецификация предметной области}
            \scnidtf{структурная онтология предметной области}
            \scnidtf{ролевая структура ключевых элементов предметной области}
            \scnidtf{схема ролей понятий предметной области и её связи со смежными предметными областями}
            \scnidtf{схема предметной области}
            \scnidtf{спецификация предметной области с точки зрения теории графов и теории \textit{алгебраических систем}}
            \scnidtf{описание внутренней (ролевой) структуры \textit{предметной области}, а также её внешних связей с другими \textit{предметными областями}}
            \scnidtf{описание ролей ключевых элементов предметной области (прежде всего, понятий --- концептов), а также место специфицируемой предметной области в множестве себе подобных}
            \scnidtf{\textit{семантическая окрестность} знака \textit{предметной области} в рамках самой этой \textit{предметной области}, включающая в себя все \textit{ключевые знаки}, входящие в состав \textit{предметной области} (ключевые понятия и ключевые объекты исследования предметной области) с указанием их ролей (свойств) в рамках этой \textit{предметной области} и \textit{семантическая окрестность} знака специфицируемой \textit{предметной области} в рамках \textit{Предметной области предметных областей}, включающая в себя связи специфицируемой \textit{предметной области} с другими семантически близкими ей \textit{предметными областями} (дочерними и родительскими, аналогичными в том или ином смысле (например, изоморфными), имеющими одинаковые \textit{классы объектов исследования} или одинаковые наборы \textit{исследуемых отношений})}
            
            \scnheader{теоретико-множественная онтология предметной области}
            \scnidtf{\textit{семантическая окрестность} специфицируемой \textit{предметной области} в рамках \textit{Предметной области множеств}, описывающая теоретико-множественные связи между \textit{понятиями} специфицируемой \textit{предметной области}, включая связи \textit{отношений} с их \textit{областями определения} и \textit{доменами}, связи используемых \textit{параметров} и классов структур их \textit{областями определения}}
            \scnidtf{онтология описывающая:
                \begin{scnitemize}
                    \item классификацию объектов исследования специфицируемой предметной области;
                    \item соотношение областей определения и доменов используемых отношений с выделенным классами объектов исследования, а также с выделенными классами вспомогательных (смежных) объектов, не являющихся объектами исследования в специфицируемой предметной области;
                    \item спецификацию используемых отношений и, в том числе, указание того, все ли связки этих отношений входят в состав специфицируемой предметной области
                \end{scnitemize}}
            \scntext{пояснение}{теоретико-множественная онтология предметной области включает в себя:
                \begin{scnitemize}
                    \item теоретико-множественные связи (в т.ч. таксономию) между всеми используемыми понятиями, входящими в состав специфицируемой предметной области;
                    \item теоретико-множественную спецификацию всех \textit{отношений}, входящих в состав специфицируемой предметной области (ориентированность, арность, область определения, домены и т.д.);
                    \item теоретико-множественную спецификацию всех параметров, используемых в предметной области (области определения параметров, шкалы, единицы измерения, точки отсчета);
                    \item теоретико-множественную спецификацию всех используемых классов структур
                \end{scnitemize}}
            
            \scnheader{логическая онтология предметной области}
            \scnidtf{формальная теория заданной (специфицируемой) предметной области, описывающая с помощью переменных, кванторов, логических связок, формул различные свойства экземпляров понятий, используемых в специфицируемой предметной области}
            \scnidtftext{пояснение}{онтология предметной области, которая включает в себя:
                \begin{scnitemize}
                    \item формальные определения всех понятий, которые в рамках специфицируемой предметной области являются определяемыми;
                    \item неформальные пояснения и некоторые формальные спецификации (как минимум, примеры) для всех понятий, которые в рамках специфицируемой предметной области являются неопределяемыми;
                    \item иерархическую систему понятий, в которой для каждого понятия, исследуемого в специфицируемой предметной области либо указывается факт неопределяемости этого понятия, либо указываются все понятия, на основе которых даётся определение данному понятию.В результате этого множество исследуемых понятий разбивается на ряд уровней:
                    \begin{scnitemizeii}
                        \item неопределяемые понятия;
                        \item понятия 1-го уровня, определяемые только на основе неопределяемых понятий;
                        \item понятия 2-го уровня, определяемые на основе понятий, изменяющих 1-й уровень и ниже;
                        \item и т.д.
                    \end{scnitemizeii}
                    \item формальную запись всех аксиом, т.е. высказываний, которые не требуют доказательств;
                    \item формальную запись высказываний, истинность которых требует обоснования (доказательства);
                    \item формальные тексты доказательства истинности высказываний, представляющие собой спецификацию последовательности шагов соответствующих рассуждений (шагов логического вывода, применения различных правил логического вывода);
                    \item иерархическую систему высказываний, в которой для каждого высказывания, истинного по отношению к специфицируемой предметной области, либо указывается аксиоматичность этого высказывания, либо перечисляются \uline{все} высказывания, на основе которых доказывается данное высказывание. В результате этого множество высказываний, истинных по отношению к специфицируемой предметной области, разбивается на ряд уровней:
                    \begin{scnitemizeii}
                        \item аксиомы;
                        \item высказывания 1-го уровня, доказываемые только на основе аксиом;
                        \item высказывания 2-го уровня, доказываемые на основе высказываний, находящихся на 1-м уровне и ниже.
                    \end{scnitemizeii}
                    \item формальная запись гипотетических высказываний;
                    \item формальное описание логико-семантической типологии высказываний --- высказываний о существовании, о несуществовании, об однозначности, высказывания определяющего типа (которые можно использовать в качестве определений соответствующих понятий);
                    \item формальное описание различного вида логико-семантических связей между высказываниями (например, между высказыванием и его обобщением);
                    \item формальное описание аналогии
                    \begin{scnitemizeii}
                        \item между определениями;
                        \item между высказываниями любого вида;
                        \item между доказательствами различных высказываний.
                    \end{scnitemizeii}
                \end{scnitemize}}
            
            \scnheader{терминологическая онтология предметной области}
            \scnidtf{онтология, описывающая \uline{правила построения} терминов (sc-идентификаторов), соответствующих \mbox{sc-элементам}, принадлежащим специфицируемой предметной области, а также описывающая различного рода терминологические связи между используемыми терминами, характеризующие происхождение этих терминов}
            \scnidtf{система терминов заданной предметной области}
            \scnidtf{тезаурус соответствующей предметной области}
            \scnidtf{словарь соответствующей (специфицируемой) предметной области}
            \scnidtf{фрагмент глобальной \textit{Предметной области sc-идентификаторов} (внешних идентификаторов sc-элементов), обеспечивающий терминологическую спецификацию некоторой предметной области}
            
            \bigskip
            
        \end{scnsubstruct}
        
        \scnendsegmentcomment{Типология онтологии предметной области}
        \scnsegmentheader{Понятие объединённой онтологии предметной области, а также понятие предметной области и онтологии}
        \begin{scnsubstruct}
        	
            \scnheader{объединённая онтология предметной области}
            \scnidtf{объединение всех частных онтологий, соответствующих одной предметной области}
            \begin{scnreltoset}{обобщённое объединение}
                \scnitem{структурная спецификация предметной области}
                \scnitem{теоретико-множественная онтология предметной области}
                \scnitem{логическая онтология предметной области}
                \scnitem{терминологическая онтология предметной области}
            \end{scnreltoset}
            
            \scnheader{предметная область и онтология}
            \scnidtf{интеграция некоторой \textit{предметной области} c соответствующей ей \textit{\uline{объединённой} онтологией}}
            \scnidtf{предметная область \& онтология}
            \begin{scnreltoset}{обобщённое объединение}
                \scnitem{предметная область}
                \scnitem{объединённая онтология предметной области}
            \end{scnreltoset}
            \scnidtf{sc-текст, являющийся объединением некоторой предметной области, представленной в SC-коде, и объединённой онтологии этой предметной области, также представленной в SC-коде}
            \scnidtf{интеграция предметной области и всех онтологий, специфицирующих эту предметную область}
            \scnidtf{совокупность различных \textit{фактов} о структуре некоторой области деятельности некоторых \textit{субъектов}, а также различного вида \textit{знаний}, специфицирующих эту область деятельности}
            \scnidtf{факты и знания о некоторой области деятельность}
            \scnidtf{sc-модель предметной области и всевозможных онтологий, специфицирующих эту предметную область (и, в первую очередь, её ключевых понятий) в разных ракурсах}
            \scnidtf{целостный с логико-семантической точки зрения фрагмент базы знаний ostis-системы, акцентирующий внимание на конкретном классе объектов исследования и на конкретном аспекте их рассмотрения}
            \scntext{примечание}{Декомпозиция \textit{базы знаний} на предметные области вместе с соответствующими им обязательными онтологиями носит в известной мере условный характер. При этом надо помнить, что для исследования и формализации межпредметных (междисциплинарных) свойств и закономерностей необходимо строить иерархию \textit{предметных областей} т.е. переходить от \textit{предметных областей} к их объединениям.}
            \scntext{примечание}{\textit{предметные области и онтологии} являются основным видом \textit{разделов баз знаний}, обладающих высокой степенью их независимости друг от друга и четкими правилами их согласования, что обеспечивает их семантическую (понятную) совместимость в рамках всей \textit{базы знаний}.}
            \scnidtf{основной вид семантических кластеров \textit{базы знаний ostis-системы}}
            \scntext{примечание}{При формализации \textit{научных знаний} в самых различных областях в рамках \textit{базы знаний ostis-системы} (в том числе в рамках \textit{объединённой виртуальной базы знаний Экосистемы OSTIS}) \uline{каждой}(!) научной дисциплине будет соответствовать своя \textit{предметная область} и соответствующая ей \textit{объединённая онтология}. Следует при этом подчеркнуть, что наличие в Технологии OSTIS развитых метаязыковых средств создает хорошие условия для активизации междисциплинарных исследований, для конвергенции различных научных дисциплин.}
            \scntext{примечание}{Все ролевые отношения, предметные области с ключевыми объектами исследования, а также с используемыми, вводимыми и исследуемыми понятиями, логичным образом применяются и в онтологиях, и в объединениях предметных областей с соответствующими им онтологиями, поскольку указанные ключевые объекты исследования и ключевые понятия входят и являются ключевыми не только для предметных областей, но и для соответствующих им онтологий}
            \bigskip
            
        \end{scnsubstruct}
        
        \scnendsegmentcomment{Понятие объединённой онтологии предметной области, а также понятие предметной области и онтологии}
        \scnsegmentheader{Отношения, заданные на множестве онтологий}
        \begin{scnsubstruct}
        	
            \scnheader{отношение, заданное на множестве онтологий}
            \scnhaselement{онтология*}
            \begin{scnindent}
	            \scnidtf{быть онтологией заданного информационного ресурса*}
	            \scnsuperset{онтология предметной области*}
	        \end{scnindent}
            \scnhaselement{часть знания*}
            \scnhaselement{декомпозиция знания*}
            \begin{scnindent}
	            \scnsuperset{декомпозиция предметной области и онтологии*}
	            \begin{scnindent}
	            	\scnidtf{декомпозиция предметной области и онтологии на предметную область и её частные онтология*}
	            \end{scnindent}
	        \end{scnindent}
            \scnhaselement{область выполнения действий*}
            
            \scnheader{онтология*}
            \scnsubset{спецификация*}
            \begin{scnindent}
            	\scnidtf{быть спецификацией заданной сущности*}
            \end{scnindent}
            \scnidtf{быть спецификацией знания (информационного ресурса)*}
            \scnidtf{метазнание*}
            
            \scnheader{онтология предметной области*}
            \scnsubset{онтология*}
            \scnidtf{быть онтологией заданной предметной области*}
            \scnidtf{Бинарное ориентированное отношение, здорово связывает некоторую предметную область с онтологией, специфицирующей эту предметную область*}
            \scnsuperset{объединённая онтология предметной области*}
            \scnsuperset{структурная спецификация предметной области*}
            \scnsuperset{теоретико множественная онтология предметной области*}
            \scnsuperset{логическая онтология предметной области*}
            \scnsuperset{терминологическая онтология предметной области*}
            \scntext{примечание}{Отношение \scnqqi{\textit{онтология предметной области*}} и его подмножества не являются обязательными, поскольку конкретный \textit{вид онтологии} и вид сущности, специфицируемой данной онтологией, однозначно задаются классами, которым принадлежат эта онтология и эта сущность. Таким образом, указанные отношения вводятся исключительно для дидактических целей и не рекомендуются к использованию, поскольку увеличивают число случаев логической эквивалентности sc-текстов.}
            
            \scnheader{часть знания*}
            \scnsubset{часть*}
            \scnidtf{\textit{\uline{знание}}, являющееся фрагментом (\textit{частью}) другого заданного \textit{знания}*}
            \scntext{примечание}{С помощью данного \textit{отношения}, также различных подмножеств отношения \scnqqi{быть \textit{онтологией}*} можно переходить от \textit{первичных знаний} к \textit{метазнаниям}, от \textit{метазнаний} к \textit{метаметазнаниям} и так далее.}
            
            \scnheader{область выполнения действий*}
            \scnidtf{быть формальной моделью области выполнения заданной системы действий (заданного сложного действия или заданной деятельности)*}
            \scntext{примечание}{\textit{предметные области} и онтологии (предметные области, интегрированные с их \textit{объединёнными онтологиями}) являются часто используемыми и весьма удобными формальными моделями, обеспечивающими качественную семантическую систематизацию областей выполнения различного вида действий, осуществляемых кибернетическими системами как в собственной памяти (путём непосредственной обработки соответствующих предметных областей и их онтологий), так и в своей внешней среде.\\
                Многим \textit{предметным областям и онтологиям} будут ставиться в соответствие следующие \textit{локальные предметные области действий и задач}, выполняемых на основе этих \textit{предметных областей и онтологий}:
                \begin{scnitemize}
                    \item история эксплуатации соответствующей предметной области и онтологии при выполнении действий и задач в рамках собственной памяти;
                    \item история эволюции соответствующей предметной области и онтологии;
                    \item история выполнения действий и задач во внешней среде на основе соответствующей предметной области и онтологии.
                \end{scnitemize}}
            
            \bigskip
        \end{scnsubstruct}
        \scnendsegmentcomment{Отношения, заданные на множестве онтологий}
        \bigskip
    \end{scnsubstruct}
    \scnendcurrentsectioncomment
\end{SCn}
