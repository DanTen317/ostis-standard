\begin{SCn}
    \scnsectionheader{Предметная область и онтология темпоральных сущностей}
    \begin{scnsubstruct}
    	
        \scnheader{Предметная область темпоральных сущностей}
        \scnidtf{Предметная область темпоральных связей и отношений}
        \scnidtf{Предметная область временных сущностей}
        \scniselement{предметная область}
        \begin{scnhaselementrole}{максимальный класс объектов исследования}
            {временная сущность}
        \end{scnhaselementrole}
        \begin{scnhaselementrolelist}{класс объектов исследования}
            \scnitem{прошлая сущность}
            \scnitem{настоящая сущность}
            \scnitem{будущая сущность}
            \scnitem{временная связь}
            \scnitem{ситуация}
            \scnitem{процесс}
            \scnitem{процесс в sc-памяти}
            \scnitem{процесс во внешней среде ostis-системы}
            \scnitem{материальная сущность}
            \scnitem{воздействие}
            \scnitem{отношение}
            \scnitem{класс временных связей}
            \scnitem{класс временных и постоянных связей}
            \scnitem{множество}
            \scnitem{ситуативное множество}
            \scnitem{неситуативное множество}
            \scnitem{частично ситуативное множество}
            \scnitem{темпоральная связь}
            \scnitem{темпоральное отношение}
            \scnitem{начало\scnsupergroupsign}
            \scnitem{завершение\scnsupergroupsign}
            \scnitem{длительность\scnsupergroupsign}
            \scnitem{тысячелетие}
            \scnitem{век}
            \scnitem{год}
            \scnitem{месяц}
            \scnitem{сутки}
            \scnitem{час}
            \scnitem{минута}
            \scnitem{секунда}
        \end{scnhaselementrolelist}
        \begin{scnhaselementrolelist}{исследуемое отношение}
            \scnitem{воздействующая сущность*}
            \scnitem{объект воздействия*}
            \scnitem{начальная ситуация*}
            \scnitem{причинная ситуация*}
            \scnitem{конечная ситуация*}
            \scnitem{событие*}
            \scnitem{последний добавленный sc-элемент\scnrolesign}
            \scnitem{темпоральное включение*}
            \scnitem{темпоральная часть*}
            \scnitem{начальный этап*}
            \scnitem{конечный этап*}
            \scnitem{промежуточный этап*}
            \scnitem{темпоральное включение без совпадения начальных и конечных моментов*}
            \scnitem{темпоральное включение с совпадением начальных моментов*}
            \scnitem{темпоральное включение с совпадением конечных моментов*}
            \scnitem{темпоральное совпадение*}
            \scnitem{темпоральное объединение*}
            \scnitem{темпоральная декомпозиция*}
            \scnitem{темпоральная смежность*}
            \scnitem{темпоральная последовательность с промежутком*}
            \scnitem{темпоральная последовательность с пересечением*}
            \scnitem{номер тысячелетия\scnrolesign}
            \scnitem{номер века\scnrolesign}
            \scnitem{номер года\scnrolesign}
            \scnitem{номер месяца в году\scnrolesign}
            \scnitem{номер суток в месяце\scnrolesign}
            \scnitem{номер часа в дне\scnrolesign}
            \scnitem{номер минуты в часе\scnrolesign}
            \scnitem{номер секунды в минуте\scnrolesign}
        \end{scnhaselementrolelist}
        
        \scnheader{временная сущность}
        \scnidtf{временно существующая сущность}
        \scnidtf{нестационарная сущность}
        \scnidtf{сущность, имеющая и/или начало, и/или конец своего существования}
        \scnidtf{sc-элемент, являющийся знаком некоторой временно существующей сущности}
        \scnidtf{сущность, обладающая темпоральными характеристиками (длительностью, начальным моментом, конечным моментом и т.д.)}
        \begin{scnsubdividing}
            \scnitem{прошлая сущность}
            \scnitem{настоящая сущность}
            \scnitem{будущая сущность}
        \end{scnsubdividing}
        \begin{scnsubdividing}
            \scnitem{временная связь}
            \scnitem{темпоральная структура}
            \begin{scnindent}
                \scnidtf{структура, содержащая хотя бы одну временную сущность}
                \scnrelfrom{включение}{структура}
                \scntext{примечание}{Следует отличать:
                    \begin{scnitemize}
                        \item временный характер самой структуры как sc-элемента;
                        \item временный характер sc-элементов, принадлежащих данной структуре, и сущностей, обозначаемых этими sc-элементами;
                        \item временный характер пар принадлежности, связывающих структуру с ее элементами.
                    \end{scnitemize}}
                \scnidtf{структура, описывающая темпоральные свойства (свойства, связанные со временем) окружающей среды, частью которой являются также и различные базы знаний кибернетических систем (в том числе и собственная база знаний).}
                \begin{scnsubdividing}
                    \scnitem{ситуация}
                    \begin{scnindent}
                        \scnidtf{статическая темпоральная структура}
                    \end{scnindent}
                    \scnitem{процесс}
                    \begin{scnindent}
                        \scnidtf{динамическая структура}
                        \scnidtf{динамическая темпоральная структура}
                    \end{scnindent}
                \end{scnsubdividing}
            \end{scnindent}
            \scnitem{материальная сущность}
        \end{scnsubdividing}
        \begin{scnsubdividing}
            \scnitem{непрерывная временная сущность}
            \begin{scnindent}
                \begin{scnsubdividing}
                    \scnitem{точечная временная сущность}
                    \begin{scnindent}
                        \scnidtf{атомарная временная сущность}
                        \scnidtf{условно мгновенная временная сущность}
                        \scnidtf{временная сущность, длительность существования которой в данном контексте считается несущественной (пренебрежительно малой)}
                    \end{scnindent}
                    \scnitem{длительная непрерывная временная сущность}
                \end{scnsubdividing}
            \end{scnindent}
            \scnitem{дискретная временная сущность}
            \begin{scnindent}
                \scnidtf{временная сущность, которая может быть декомпозирована на последовательность точечных временных сущностей}
                \scnidtf{временная сущность, которой соответствует некоторый временной ряд параметров (состояний) точечных временных сущностей, на которые декомпозируется исходная временная сущность}
            \end{scnindent}
            \scnitem{прерывистая временная сущность}
            \begin{scnindent}
                \scnidtf{временная сущность, являющаяся результатом соединения нескольких не только точечных временных сущностей}
                \scnidtf{временная сущность с прерываниями}
            \end{scnindent}
        \end{scnsubdividing}
        \scntext{примечание}{Следует отметить, что приведенная классификация \textit{временных сущностей} характеризует не столько сами \textit{временные сущности}, сколько наши знания о них и степень детализации знаний об этих сущностях, с которой они описаны в базе знаний. Так, если для решения конкретных задач не важно, как изменялась некоторая \textit{временная сущность} в рамках какого-либо периода времени, а важно только ее начальное и конечное состояние, то она может рассматриваться как \textit{точечная временная сущность}. Впоследствии же та же \textit{временная сущность} может быть рассмотрена и описана с большей степенью детализации, и таким образом, уже не будет точечной.}
        \scntext{пояснение}{Следует отличать:
            \begin{scnitemize}
                \item временный характер сущности, обозначаемой \textit{sc-элементом};
                \item временный характер существования самого \textit{sc-элемента} в рамках \textit{sc-памяти}, поскольку в ходе обработки информации каждый \textit{sc-элемент} может быть удален из \textit{sc-памяти};
                \item временный характер описываемых ситуаций, событий и самих процессов;
                \item временный характер хранения в sc-памяти тех sc-конструкций, которые являются самими описаниями соответствующих ситуаций, событий и процессов.
            \end{scnitemize}}
        
        \scnheader{следует отличать*}
        \begin{scnhaselementset}
            \scnitem{временная сущность}
            \scnitem{процесс}
        \end{scnhaselementset}
        \scntext{примечание}{Следует отличать, например, \textit{материальную сущность} (некоторый физический или, в частности, биологический объект) от различных динамических структур (\textit{процессов}), которые с той или иной степенью детализации и в том или ином ракурсе отражают (описывают) динамику изменений этой \textit{материальной сущности}.
        	 \\При этом сам \textit{процесс} как уточнение динамики некоторой последовательности ситуаций и событий, также является сущностью, принадлежащей к классу \textit{временных сущностей}.}
        
        \scnheader{прошлая сущность}
        \scnidtf{сущность, существовавшая в прошлом времени}
        \scnidtf{сущность прошлого времени}
        \scnidtf{сущность, завершившая свое существование}
        
        \scnheader{настоящая сущность}
        \scnidtf{сущность, существующая в текущий момент времени}
        \scnidtf{сущность, существующая сейчас}
        \scnidtf{сущность настоящего времени}
        
        \scnheader{будущая сущность}
        \scnidtf{возможно будущая сущность}
        \scnidtf{прогнозируемая временная сущность}
        \scnidtf{временная сущность, которая может существовать в будущем}
        \scnidtf{сущность, которая может или должна начать свое существование в будущем времени}
        \scnrelfrom{включение}{инициированное действие}
        \scntext{пояснение}{Каждой \textbf{\textit{будущей сущности}} можно поставить в соответствие вероятность ее возникновения.}
        
        \scnheader{временная связь}
        \scnidtf{нестационарная связь}
        \scnidtf{временно существующая связь}
        \scntext{пояснение}{Каждая \textbf{\textit{временная связь}} представляет собой \textit{связку}, принадлежащую множеству \textit{временных сущностей}.
        	\\Понятие \textbf{\textit{временной связи}} не следует путать с понятием \textit{темпоральной связи}, которая сама является \textit{постоянной сущностью}, описывающей то, как связаны во времени некоторые \textit{временные сущности}.}
        
        \scnheader{ситуация}
        \scnidtf{состояние}
        \scnidtf{временная структура}
        \scnidtf{временно существующая структура}
        \scnidtf{квазистационарная структура}
        \scnidtf{состояние некоторой динамической системы, описываемое с некоторой степенью детализации (подробности)}
        \scnidtf{квазистационарная структура, существующая временно (в течение некоторого отрезка времени)}
        \scntext{пояснение}{Под ситуацией понимается \textit{структура}, содержащая, по крайней мере, один элемент, который является \textit{временной сущностью}. Наличие в рамках ситуации нескольких \textit{временных сущностей} говорит о том, что существует момент времени (в прошлом, настоящем или будущем), в который все они существуют одновременно.}
        
        \scnheader{процесс}
        \scnidtf{процесс преобразования некоторой временной сущности из одного состояния в другое}
        \scnidtf{процесс перехода от одной ситуации к другой}
        \scnidtf{абстрактный процесс}
        \scnidtf{информационная модель некоторых преобразований}
        \scnidtf{динамическая sc-модель}
        \scnidtf{динамическая структура}
        \scnrelfrom{включение}{воздействие}
        \scntext{пояснение}{Каждый \textbf{\textit{процесс}} определяется (задается) либо возникновением или исчезновением связей, связывающих заданную \textit{временную сущность} с другими сущностями, либо возникновением или исчезновением связей, связывающих части указанной \textit{временной сущности} с другими сущностями.\\
            Многим \textbf{\textit{процессам}} можно поставить в соответствие \textit{ситуацию}, которая является его \textit{начальной ситуацией*} и \textit{ситуацию}, которая является его \textit{конечной ситуацией*}, то есть указать \textit{ситуации}, переход между которыми осуществляется в ходе \textbf{\textit{процесса}}.\\
            При этом знаки тех \textit{временных сущностей}, с которыми связаны изменения, описываемые некоторым \textbf{\textit{процессом}}, входят в данный \textbf{\textit{процесс}} как элементы и при необходимости уточняются дополнительными \textit{ролевыми отношениями}.}
        \begin{scnsubdividing}
            \scnitem{процесс в sc-памяти}
            \scnitem{процесс во внешней среде ostis-системы}
        \end{scnsubdividing}
        \scntext{примечание}{Каждой \textbf{\textit{материальной сущности}} можно поставить в соответствие различные \textit{процессы}, описывающие ее преобразование из одного состояния в другое.}
        \scntext{примечание}{Поскольку \textit{процесс} представляет собой изменяющуюся во времени динамическую структуру, то полностью представить процесс в базе знаний в общем случае не представляется возможным. Однако, можно ввести sc-элемент, обозначающий конкретный процесс, с необходимой степенью детализации описать его декомпозицию на более частные подпроцессы и/или описать ситуации, соответствующие состояниям процесса в разные моменты времени.
        	\\В данном случае можно провести некоторую аналогию с \textit{бесконечными множествами}, все элементы которых физически не могут быть представлены в базе знаний одновременно, тем не менее, само множество и некоторые из его элементов могут быть описаны с необходимой степенью детализации.}
        
        \scnheader{воздействие}
        \scnidtf{процесс, осуществляющийся на основе (в результате) воздействия одной сущности на другую}
        \scnrelfrom{включение}{действие}
        \scntext{пояснение}{Каждому \textbf{\textit{воздействию}} может быть поставлена в соответствие (1) некоторая \textit{воздействующая сущность*}, т.е. сущность, осуществляющая \textbf{\textit{воздействие}} (в частности, это может быть некоторое физическое поле), и (2) некоторый \textit{объект воздействия*}, т.е. сущность, на которую воздействие направлено. Если \textbf{\textit{воздействие}} связано с \textit{материальной сущностью}, то его объектом воздействия является либо сама эта \textit{материальная сущность}, либо некоторая ее пространственная часть.}
        
        \scnheader{исходная ситуация*}
        \scnidtf{начальная ситуация процесса*}
        \scnidtf{начальная ситуация*}
        \scniselement{бинарное отношение}
        \scnrelfrom{первый домен}{процесс}
        \scnrelfrom{второй домен}{ситуация}
        \scntext{пояснение}{Связки отношения \textbf{\textit{исходная ситуация*}} связывают некоторый \textit{процесс} и некоторую ситуацию, являющуюся начальной для этого \textit{процесса}, и, как правило, изменяемой в течение выполнения этого \textit{процесса}.
        	\\Первым компонентом каждой связки отношения \textbf{\textit{исходная ситуация*}} является знак \textit{процесса}, вторым --- знак начальной \textit{ситуации}.}
        
        \scnheader{причинная ситуация*}
        \scnsubset{начальная ситуация*}
        \scntext{пояснение}{Под причинной ситуацией понимается такая \textit{начальная ситуация*}, которая обладает достаточной полнотой для однозначного задания инициируемого \textit{процесса}.}
        
        \scnheader{конечная ситуация*}
        \scnidtf{конечная ситуация процесса*}
        \scnidtf{результирующая ситуация*}
        \scniselement{бинарное отношение}
        \scnrelfrom{первый домен}{процесс}
        \scnrelfrom{второй домен}{ситуация}
        \scntext{пояснение}{Связки отношения \textbf{\textit{конечная ситуация*}} связывают некоторый \textit{процесс} и некоторую \textit{ситуацию}, ставшую результатом выполнения этого \textit{процесса}, то есть его следствием.\\
            Первым компонентом каждой связки отношения \textbf{\textit{конечная ситуация*}} является знак \textit{процесса}, вторым --- знак конечной \textit{ситуации}.}
        
        \scnheader{точечный процесс}
        \scnidtf{атомарный процесс}
        \scnidtf{условно мгновенный процесс}
        \scnidtf{процесс, длительность которого в данном контексте считается несущественной (пренебрежимо малой)}
        \scnsubset{точечная временная сущность}
        
        \scnheader{элементарный процесс}
        \scnidtf{процесс, детализация которого на входящие в него подпроцессы в текущем контексте не производится}
        \scnsuperset{точечный процесс}
        \scntext{примечание}{Элементарные процессы могут иметь длительность и, следовательно, не обязательно являются атомарными процессами.}
        \scntext{примечание}{Понятия \textit{точечного процесса} и \textit{элементарного процесса}, как и понятие \textit{точечной временной сущности} в целом, характеризуют не столько характеристики самого \textit{процесса}, сколько степень наших знаний о нем и степень детализации описания процесса в базе знаний. Так, очевидно, что любой процесс, протекающий в компьютерной системе, может быть при необходимости детализирован до уровня команд процессора, затем до уровня микропрограмм и даже до уровня физических процессов (изменения физических характеристик сигналов), однако чаще всего такая детализация не требуется.}
        \begin{scnindent}
        	\scnrelto{примечание}{точечный процесс}
        \end{scnindent}
        
        \scnheader{событие}
        \scnsubset{точечная временная сущность}
        \scnidtf{точечная временная сущность, являющаяся началом и/или завершением какой-либо временной сущности (например, процесса)}
        \scnidtf{граничная точка временной сущности}
        \scnrelfrom{описание примера}{\scnfileimage[20em]{Contents/part_kb/src/images/sd_temp_entities/event.png}}
        \begin{scnindent}
        	\scntext{примечание}{Одно и то же событие может быть одновременно завершением одной временной сущности и началом другой. В приведенном примере событие $\bm{ei}$ является завершением временной сущности $\bm{si}$ и началом временной сущности $\bm{sj}$.}
        \end{scnindent}
        
        \scnheader{начало*}
        \scnidtf{быть начальным событием заданной временной сущности*}
        \scnrelfrom{первый домен}{временная сущность}
        \scnrelfrom{второй домен}{событие}
        \scnidtf{быть начальной точечной временной частью заданной временной сущности*}
        
        \scnheader{завершение*}
        \scnidtf{конец*}
        \scnidtf{быть конечным событием заданной временной сущности*}
        \scnidtf{быть конечной точечной временной частью заданной временной сущности*}
        \scnrelfrom{первый домен}{временная сущность}
        \scnrelfrom{второй домен}{событие}
        
        \scnheader{событие*}
        \scniselement{бинарное отношение}
        \scntext{пояснение}{Связки отношения \textbf{\textit{событие*}} связывают знак процесса и ориентированную пару, первым компонентом которой является знак \textit{начальной ситуации*} данного процесса, вторым компонентом --- знак \textit{конечной ситуации*} данного процесса.}
        \scnrelfrom{описание примера}{\scnfileimage[20em]{Contents/part_kb/src/images/sd_temp_entities/nrel_event.png}}
        
        \scnheader{детализация процесса*}
        \scnidtf{бинарное ориентированное отношение, каждая связка которого связывает некоторый процесс с более детальным его описанием, что предполагает представление декомпозиции этого процесса на систему взаимосвязанных его подпроцессов (в том числе элементарных).}
        \scntext{пример}{Переход от процесса, соответствующего какой-либо программе, к рассмотрению декомпозиции этого процесса (протокола) в терминах языка программирования высокого уровня, затем переход для каждого из полученных подпроцессов (операторов языка высокого уровня) к детализации выполнения этих подпроцессов на уровне машинных операций, выполняемых процессором компьютера (на уровне ассемблера), и далее к детализации выполнения подпроцессов уровня машинных операций к подпроцессам на уровне языка микропрограммирования. Таким образом, детализация процесса может быть иерархической, вплоть до уровня \textit{элементарных процессов}.}
        
        \scnheader{отношение}
        \begin{scnsubdividing}
            \scnitem{класс временных связей}
            \scnitem{класс постоянных связей}
            \scnitem{класс временных и постоянных связей}
        \end{scnsubdividing}
        
        \scnheader{класс временных связей}
        \scnidtf{отношение, все связки которого являются нестационарными}
        \scntext{пояснение}{В общем случае \textbf{\textit{класс временных связей}} не является \textit{ситуативным множеством}, поскольку факт принадлежности некоторой \textit{временной связи} такому классу следует считать постоянным, а не временным, поскольку временность/постоянство связи и ее семантический тип, задаваемый классом (отношением), это принципиально разные параметры (характеристики, признаки) любой связи.}
        
        \scnheader{класс постоянных связей}
        \scnidtf{отношение, все связки которого являются стационарными}
        
        \scnheader{класс временных и постоянных связей}
        \scnidtf{отношение, некоторые (но не все) связки которого являются нестационарными}
        
        \scnheader{множество}
        \begin{scnsubdividing}
            \scnitem{ситуативное множество}
            \scnitem{неситуативное множество}
            \scnitem{частично ситуативное множество}
        \end{scnsubdividing}
        
        \scnheader{ситуативное множество}
        \scnidtf{полностью ситуативное множество}
        \scntext{пояснение}{Под \textbf{\textit{ситуативным множеством}} понимается постоянное множество, у которого все выходящие из него связи принадлежности являются \textit{временными сущностями}. В частности, ситуативное множество может использоваться как вспомогательная динамическая структура, которая содержит элементы некоторых структур, обрабатываемые в данный момент, например, это может быть копия некоторого множества, из которой постепенно удаляются элементы по мере их просмотра и обработки. В случае, когда такая структура содержит всего один элемент, ее можно считать \underline{указателем} на данный элемент, при этом в разные моменты времени это могут быть разные элементы.}
        
        \scnheader{последний добавленный sc-элемент\scnrolesign}
        \scniselement{ролевое отношение}
        
        \scnheader{неситуативное множество}
        \scntext{пояснение}{Под \textbf{\textit{неситуативным множеством}} понимается постоянное множество, у которого все выходящие из него связи принадлежности являются \textit{постоянными сущностями}.}
        
        \scnheader{частично ситуативное множество}
        \scntext{пояснение}{Под \textbf{\textit{частично ситуативным множеством}} понимается постоянное множество, у которого некоторые (но не все) выходящие из него связи принадлежности являются \textit{временными сущностями}.}
        
        \scnheader{темпоральная связь}
        \scnidtf{связь во времени}
        \scnidtf{\uline{постоянная} связь, описывающая связь во времени между временными сущностями}
        
        \scnheader{темпоральное отношение}
        \scnrelto{семейство подмножеств}{темпоральная связь}
        \scnidtf{класс темпоральных связей}
        \scnidtf{отношение, задающее темпоральные связи между временными сущностями}
        \scnhaselement{темпоральное включение*}
        \scnhaselement{темпоральное объединение*}
        \scnhaselement{темпоральная декомпозиция*}
        \scnhaselement{темпоральная последовательность*}
        \begin{scnsubdividing}
            \scnitem{темпоральная смежность*}
            \scnitem{темпоральная последовательность с промежутком*}
            \scnitem{темпоральная последовательность с пересечением*}
        \end{scnsubdividing}
        
        \scnheader{темпоральное включение*}
        \scntext{пояснение}{Связки отношения \textbf{\textit{темпоральное включение*}} связывают две \textit{временные сущности}, период существования одной из которых полностью включается в период существования второй.\\
            Первым компонентом каждой связки отношения \textbf{\textit{темпоральное включение*}} является знак \textit{временной сущности}, \textit{длительность} существования которой больше.}
        \scnsuperset{темпоральная часть*}
        \scnsuperset{темпоральное включение без совпадения начальных и конечных моментов*}
        \scnsuperset{темпоральное совпадение*}
        \scnsuperset{темпоральное включение с совпадением начальных моментов*}
        \scnsuperset{темпоральное включение с совпадением конечных моментов*}
        
        \scnheader{темпоральная часть*}
        \scnidtf{этап (период) заданной временной сущности*}
        \scnidtf{этап процесса существования временной сущности*}
        \scnsuperset{начальный этап*}
        \scnsuperset{конечный этап*}
        \scnsuperset{промежуточный этап*}
        \scnsuperset{подпроцесс*}
        \begin{scnindent}
	        \scnrelfrom{первый домен}{процесс}
	        \scnrelfrom{второй домен}{процесс}
        \end{scnindent}
        \scnrelfrom{описание примера}{\scnfileimage[20em]{Contents/part_kb/src/images/sd_temp_entities/temporal_part.png}}
        \scnrelfrom{иллюстрация}{\scnfileimage[20em]{Contents/part_kb/src/images/sd_temp_entities/img_temporal_part.png}}
        \scntext{примечание}{Связки отношения \textbf{\textit{темпоральная часть*}} связывают две \textit{временные сущности}, одна из которых является частью другой, например, действие и одно из его поддействий. Соответственно, период существования одной из этих сущностей всегда будет включаться в период существования другой (большей).\\
            В отличие от более общего отношения \textit{темпоральное включение*}, связки которого могут связывать любые \textit{временные сущности}, связки отношения \textbf{\textit{темпоральная часть*}} связывают только \textit{временные сущности}, одна из которых является частью другой.}
        \scnheader{следует отличать*}
        \begin{scnhaselementset}
            \scnitem{темпоральная часть*}
            \begin{scnindent}
                \scnsuperset{подпроцесс*}
            \end{scnindent}
            \scnitem{темпоральное включение*}
            \begin{scnindent}
                \scntext{примечание}{Связь \textit{темпорального включения*} может связывать абсолютно разные \textit{временные сущности}, существующие в общем случае в разных местах, а не только \textit{временные сущности}, одна из которых является частью другой. Хотя формально и можно объединить любые разные \textit{временные сущности} в одну общую \textit{временную сущность}, далеко не всегда имеет смысл это делать.}
            \end{scnindent}
        \end{scnhaselementset}
        
        \scnheader{темпоральное включение без совпадения начальных и конечных моментов*}
        \scnidtf{строгое темпоральное включение*}
        \scnrelfrom{описание примера}{\scnfileimage[20em]{Contents/part_kb/src/images/sd_temp_entities/strict_temporal_inclusion.png}}
        \scnrelfrom{иллюстрация}{\scnfileimage[20em]{Contents/part_kb/src/images/sd_temp_entities/img_strict_temporal_inclusion.png}}
        
        \scnheader{темпоральное включение с совпадением начальных моментов*}
        \scnrelfrom{описание примера}{\scnfileimage[20em]{Contents/part_kb/src/images/sd_temp_entities/temporal_include_with_match_start_points.png}}
        \scnrelfrom{иллюстрация}{\scnfileimage[20em]{Contents/part_kb/src/images/sd_temp_entities/img_temporal_include_with_match_start_points.png}}
        
        \scnheader{темпоральное включение с совпадением конечных моментов*}
        \scnrelfrom{описание примера}{\scnfileimage[20em]{Contents/part_kb/src/images/sd_temp_entities/temporal_include_with_terminal_point_match.png}}
        \scnrelfrom{иллюстрация}{\scnfileimage[20em]{Contents/part_kb/src/images/sd_temp_entities/img_temporal_include_with_terminal_point_match.png}}
        
        \scnheader{темпоральное совпадение*}
        \scnidtf{совпадение начала и завершения*}
        \scniselement{отношение эквивалентности}
        
        \scnheader{темпоральное объединение*}
        \scnidtf{преобразование нескольких временных сущностей в одну общую временную сущность, которая может оказаться прерывистой или даже дискретной*}
        \scnrelboth{аналог}{объединение множеств*}
        \scntext{примечание}{С формальной точки зрения объединять можно любые временные сущности. Но делать это надо только тогда, когда это имеет смысл, точно так же, как и в случае объединения множеств.}
        \scnrelfrom{описание примера}{\scnfileimage[20em]{Contents/part_kb/src/images/sd_temp_entities/temporal_union.png}}
        \scnrelfrom{иллюстрация}{\scnfileimage[20em]{Contents/part_kb/src/images/sd_temp_entities/img_temporal_union.png}}
        
        \scnheader{темпоральная декомпозиция*}
        \scnidtf{Темпоральное отношение, связывающее временную сущность и множество смежных во времени временных сущностей, которые являются темпоральными частями исходной сущности и результатом темпорального объединения которых является исходная сущность*}
        \scnrelboth{аналог}{разбиение*}
        \scnrelfrom{описание примера}{\scnfileimage[20em]{Contents/part_kb/src/images/sd_temp_entities/temporal_decomposition.png}}
        \scnrelfrom{иллюстрация}{\scnfileimage[20em]{Contents/part_kb/src/images/sd_temp_entities/img_temporal_decomposition.png}}
        
        \scnheader{темпоральная смежность*}
        \scnidtf{сразу позже*}
        \scnidtf{смежность во времени*}
        \scnidtf{строгая темпоральная последовательность (без темпорального промежутка)*}
        \scnidtf{темпоральная последовательность без промежутка*}
        \scnrelfrom{описание примера}{\scnfileimage[20em]{Contents/part_kb/src/images/sd_temp_entities/temporal_adjacency.png}}
        \scnrelfrom{иллюстрация}{\scnfileimage[20em]{Contents/part_kb/src/images/sd_temp_entities/img_temporal_adjacency.png}}
        
        \scnheader{темпоральная последовательность с промежутком*}
        \scnidtf{позже*}
        \scnrelfrom{описание примера}{\scnfileimage[20em]{Contents/part_kb/src/images/sd_temp_entities/temporal_sequence_with_intermediate.png}}
        \scnrelfrom{иллюстрация}{\scnfileimage[20em]{Contents/part_kb/src/images/sd_temp_entities/img_temporal_sequence_with_intermediate.png}}
        
        \scnheader{темпоральная последовательность с пересечением*}
        \scnrelfrom{описание примера}{\scnfileimage[20em]{Contents/part_kb/src/images/sd_temp_entities/temporal_sequence_with_intersection.png}}
        \scnrelfrom{иллюстрация}{\scnfileimage[20em]{Contents/part_kb/src/images/sd_temp_entities/img_temporal_cross_sequence.png}}
        
        \scnheader{начало\scnsupergroupsign}
        \scnidtf{одновременность начинаний\scnsupergroupsign}
        \scnidtf{класс одновременно начавшихся сущностей\scnsupergroupsign}
        \scniselement{параметр}
        \scntext{пояснение}{Каждый элемент множества \textbf{начало} представляет собой класс \textit{временных сущностей}, у которых совпадает момент начала их существования. Конкретное значение данного \textit{параметра} может быть как \textit{точной величиной}, так и \textit{неточной величиной} или \textit{интервальной величиной}.}
        \scnrelfrom{описание примера}{\scnfileimage[20em]{Contents/part_kb/src/images/sd_temp_entities/start.png}}
        \begin{scnindent}
        	\scntext{пояснение}{В данном примере \textbf{\textbf{\textit{ki}}} обозначает класс сущностей, начавших свое существование 19 февраля 2015 года по григорианскому календарю. Конкретные примеры таких сущностей --- \textbf{\textit{bi}} и \textbf{\textit{bj}}. \textbf{\textit{ti}} обозначает временную точку григорианского календаря, соответствующую 19 февраля 2015 года.}
        \end{scnindent}
        
        \scnheader{завершение\scnsupergroupsign}
        \scnidtf{конец\scnsupergroupsign}
        \scnidtf{одновременность завершений\scnsupergroupsign}
        \scnidtf{класс одновременно завершившихся сущностей\scnsupergroupsign}
        \scniselement{параметр}
        \scntext{пояснение}{Каждый элемент множества \textbf{\textit{завершение}} представляет собой класс \textit{временных сущностей}, у которых совпадает конечный момент их существования (момент завершения существования). Конкретное значение данного \textit{параметра} может быть как \textit{точной величиной}, так и \textit{неточной величиной} или \textit{интервальной величиной}.}
        \scnrelfrom{описание примера}{\scnfileimage[20em]{Contents/part_kb/src/images/sd_temp_entities/completion.png}}
        \begin{scnindent}
        	\scntext{пояснение}{В данном примере \textbf{\textit{ki}} обозначает класс сущностей, завершивших свое существование 21 февраля 2015 года по григорианскому календарю. Конкретные примеры таких сущностей --- \textbf{\textit{bi}} и \textbf{\textit{bj}}. \textbf{\textit{ti}} обозначает временную точку григорианского календаря, соответствующую 21 февраля 2015 года.}
        \end{scnindent}
        
        \scnheader{одновременность\scnsupergroupsign}
        \scnidtf{параметр, значениями (элементами) которого являются классы либо одновременно существующих (происходящих) \textit{точечных временных сущностей}, одновременность которых рассматривается с заданной степенью точности, либо одновременно начинающихся и заканчивающихся длительных процессов}
        \scntext{пояснение}{Важно отметить, что элементами некоторого значения параметра \textit{одновременности} с заданной точностью могут быть только те временные сущности, которые и начались, и завершились в течение периода времени, заданного указанным значением этого параметра, но при этом начало и завершение этих временных сущностей не обязательно должно совпадать с началом и завершением указанного периода времени. Так, например, можно ввести значение параметра \textit{одновременности} \textit{2022 год по Григорианскому календарю}, элементами которого будут все временные сущности, начавшие и закончившие свое существовавшие в рамках 2022 года. При этом не обязательно, чтобы эти временные сущности начались именно в полночь 1 января 2022 года и закончились в полночь 1 января 2023 года, это могут быть временные сущности, существовавшие, например, в течение июля 2022 года.}
        \scnrelfrom{описание примера}{\scnfileimage[30em]{Contents/part_kb/src/images/sd_temp_entities/simultaneity.png}}
        \begin{scnindent}
        	\scntext{примечание}{Некоторые значения параметра одновременности могут быть подмножествами других значений того же параметра. Семантика такой связи будет выражаться в том, что первое из указанных значений описывает \textit{одновременность} \textit{временных сущностей} с большей точностью. Так, в приведенном примере величина \textit{2002 год} описывает одновременность временных сущностей с точностью до года, а величина \textit{июль 2022 года} описывает одновременность временных сущностей с точностью до месяца. При этом очевидно, что сущности, входящие во величину \textit{июль 2022 года} будут также входить и в величину \textit{2022 год} (как например временная сущность $\bm{sk})$. В приведенном примере для простоты предполагается, что все измерения производятся по Григорианскому календарю.}
        \end{scnindent}
        
        \scnheader{соединение значений ориентированного параметра*}
        \scnrelfrom{описание примера}{\scnfileimage[30em]{Contents/part_kb/src/images/sd_temp_entities/temporal_values_join.png}}
        \begin{scnindent}
        	\scntext{примечание}{В приведенном примере множество сущностей, существовавших 10.01.2022, и множество сущностей, существовавших 12.01.2022, при помощи отношения \textit{соединение значений ориентированного параметра*} образуют множество сущностей, существовавших в период 10-12.02.2022.}
        \end{scnindent}
        
        \scnheader{следует отличать*}
        \begin{scnhaselementset}
            \scnitem{темпоральное совпадение*}
            \begin{scnindent}
                \scniselement{отношение эквивалентности}
            \end{scnindent}
            \scnitem{одновременность\scnsupergroupsign}
            \begin{scnindent}
                \scnidtf{фактор-множество для отношения темпоральное совпадение*}
            \end{scnindent}
        \end{scnhaselementset}
        
        \scnheader{длительность\scnsupergroupsign}
        \scnidtf{класс временных сущностей, имеющих одинаковую длительность\scnsupergroupsign}
        \scniselement{параметр}
        \scnhaselement{тысячелетие}
        \scnhaselement{век}
        \scnhaselement{год}
        \scnhaselement{месяц}
        \scnhaselement{день}
        \scnhaselement{час}
        \scnhaselement{минута}
        \scnhaselement{секунда}
        \scntext{пояснение}{Каждый элемент множества \textbf{\textit{длительность}} представляет собой класс \textit{временных сущностей}, у которых совпадает длительность их существования. Конкретное значение данного \textit{параметра} может быть как \textit{точной величиной}, так и \textit{неточной величиной} или \textit{интервальной величиной}.}
        \scnrelfrom{описание примера}{\scnfileimage[30em]{Contents/part_kb/src/images/sd_temp_entities/duration.png}}
        \begin{scnindent}
        	\scntext{пояснение}{В данном примере \textbf{\textit{ki}} обозначает класс сущностей, существовавших в течение 2 месяцев. Конкретный пример такой сущности --- \textbf{\textit{bi}}.}
        \end{scnindent}
        \bigskip
    \end{scnsubstruct}
    \scnendcurrentsectioncomment
\end{SCn}
