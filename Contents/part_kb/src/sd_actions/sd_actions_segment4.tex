\scnsegmentheader{Предметная область и онтология субъектно-объектных спецификаций воздействий}
\begin{scnsubstruct}
    \begin{scnrelfromlist}{соавтор}
        \scnitem{Гордей А.Н.}
        \scnitem{Никифоров С.А.}
        \scnitem{Бобёр Е.С.}
        \scnitem{Святощик М.И.}
    \end{scnrelfromlist}
    \scniselement{предметная область и онтология}
    
    \scnheader{индивид}
    \scnidtftext{часто используемый sc-идентификатор}{субъект}
    \begin{scnindent}
    	\scnrelfrom{источник}{\cite{Hardzei2005}}
    \end{scnindent}
    
    \scnheader{участник воздействия\scnrolesign}
    \scnidtf{участник акции\scnrolesign}
    \scniselement{ролевое отношение}
    \scnrelfrom{первый домен}{индивид}
    \scnrelfrom{второй домен}{воздействие}
    \scntext{пояснение}{\textit{участник акции\scnrolesign} --- это ролевое отношение, которое связывает акцию с участвующим в ней индивидом.}
    \begin{scnindent}
	    \scnrelfrom{источник}{\cite{Hardzei2021}}
	    \scnrelfrom{источник}{\cite{Fillmore1977}}
	    \scnrelfrom{источник}{\cite{Fillmore1982}}
	\end{scnindent}
    \begin{scnsubdividing}
        \scnitem{субъект\scnrolesign}
        \begin{scnindent}
            \scntext{пояснение}{\textit{субъект\scnrolesign} --- инициатор акции.}
	        \begin{scnsubdividing}
	             \scnitem{инициатор\scnrolesign}
	             \scnitem{вдохновитель\scnrolesign}
	             \scnitem{распространитель\scnrolesign}
	             \scnitem{вершитель\scnrolesign}
	             \begin{scnindent}
	                 \scntext{пояснение}{\textit{вершитель\scnrolesign} завершает акцию производством из объекта продукта.}
	             \end{scnindent}
	        \end{scnsubdividing}
        \end{scnindent}
        \scnitem{инструмент\scnrolesign}
        \begin{scnindent}
            \scntext{пояснение}{\textit{инструмент\scnrolesign} --- исполнитель акции.}
            \begin{scnsubdividing}
                \scnitem{активатор\scnrolesign}
                \begin{scnindent}
                    \scntext{пояснение}{\textit{активатор\scnrolesign} непосредственно воздействует на медиатор.}
               	\end{scnindent}
                \scnitem{супрессор\scnrolesign}
                \begin{scnindent}
                    \scntext{пояснение}{\textit{супрессор\scnrolesign} подавляет сопротивление медиатора.}
                \end{scnindent}
                \scnitem{усилитель\scnrolesign}
               	\begin{scnindent}
                    \scntext{пояснение}{\textit{усилитель\scnrolesign} наращивает воздействие на медиатор.}
               	\end{scnindent}
                \scnitem{преобразователь\scnrolesign}
              	\begin{scnindent}
                    \scntext{пояснение}{\textit{преобразователь\scnrolesign} преобразует медиатор в инструмент.}
               	\end{scnindent}
            \end{scnsubdividing}
        \end{scnindent}
        \scnitem{медиатор\scnrolesign}
        \begin{scnindent}
            \scntext{пояснение}{\textit{медиатор\scnrolesign} --- посредник акции.}
            \begin{scnsubdividing}
                \scnitem{ориентир\scnrolesign}
                \scnitem{локус\scnrolesign}
                \begin{scnindent}
                    \scntext{пояснение}{\textit{локус\scnrolesign} частично или полностью окружает объект и тем самым локализует его в пространстве.}
                \end{scnindent}
                \scnitem{транспортёр\scnrolesign}
                \begin{scnindent}
                    \scntext{пояснение}{\textit{транспортёр\scnrolesign} перемещает объект.}
                \end{scnindent}
                \scnitem{адаптер\scnrolesign}
                \begin{scnindent}
                    \scntext{пояснение}{\textit{адаптер\scnrolesign} приспосабливает инструмент к воздействию на объект.}
                \end{scnindent}
                \scnitem{материал\scnrolesign}
                \begin{scnindent}
                    \scntext{пояснение}{\textit{материал\scnrolesign} используется в качестве объекта-сырья для производства продукта.}
                \end{scnindent}
                \scnitem{макет\scnrolesign}
                \begin{scnindent}
                    \scntext{пояснение}{\textit{макет\scnrolesign} является исходным образцом для производства из объекта продукта.}
                \end{scnindent}
                \scnitem{фиксатор\scnrolesign}
                \begin{scnindent}
                    \scntext{пояснение}{\textit{фиксатор\scnrolesign} превращает переменный локус объекта в постоянный.}
                \end{scnindent}
                \scnitem{ресурс\scnrolesign}
                \begin{scnindent}
                    \scntext{пояснение}{\textit{ресурс\scnrolesign} питает инструмент.}
                \end{scnindent}
                \scnitem{стимул\scnrolesign}
                \begin{scnindent}
                    \scntext{пояснение}{\textit{стимул\scnrolesign} проявляет параметр объекта.}
                \end{scnindent}
                \scnitem{регулятор\scnrolesign}
                \begin{scnindent}
                    \scntext{пояснение}{\textit{регулятор\scnrolesign} служит инструкцией в производстве из объекта продукта.}
                \end{scnindent}
                \scnitem{хронотоп\scnrolesign}
                \begin{scnindent}
                    \scntext{пояснение}{\textit{хронотоп\scnrolesign} локализует объект во времени.}
                \end{scnindent}
                \scnitem{источник\scnrolesign}
                \begin{scnindent}
                    \scntext{пояснение}{\textit{источник\scnrolesign} обеспечивает инструкциями инструмент.}
                \end{scnindent}
                \scnitem{индикатор\scnrolesign}
                \begin{scnindent}
                    \scntext{пояснение}{\textit{индикатор\scnrolesign} отображает параметр воздействия на объект или параметр продукта как результата воздействия на объект.}
             	\end{scnindent}
            \end{scnsubdividing}
        \end{scnindent}
        \scnitem{объект\scnrolesign}
        \begin{scnindent}
            \scntext{пояснение}{\textit{объект\scnrolesign} --- реципиент акции.}
            \begin{scnsubdividing}
                \scnitem{покрытие\scnrolesign}
                \begin{scnindent}
                    \scntext{пояснение}{\textit{покрытие\scnrolesign} --- внешняя изоляция оболочки индивида.}
                \end{scnindent}
                \scnitem{корпус\scnrolesign}
                \begin{scnindent}
                    \scntext{пояснение}{\textit{корпус\scnrolesign} --- оболочка индивида.}
                \end{scnindent}
                \scnitem{прослойка\scnrolesign}
                \begin{scnindent}
                    \scntext{пояснение}{\textit{прослойка\scnrolesign} --- внутренняя изоляция оболочки индивида.}
                \end{scnindent}
                \scnitem{сердцевина\scnrolesign}
                \begin{scnindent}
                    \scntext{пояснение}{\textit{сердцевина\scnrolesign} --- ядро индивида.}
               \end{scnindent} 
            \end{scnsubdividing}
        \end{scnindent}
        \scnitem{продукт\scnrolesign}
        \begin{scnindent}
            \scntext{пояснение}{\textit{продукт\scnrolesign} --- результат воздействия субъекта на объект.}
            \begin{scnsubdividing}
                \scnitem{заготовка\scnrolesign}
                \begin{scnindent}
                    \scntext{пояснение}{\textit{заготовка\scnrolesign} --- превращённый в сырьё объект.}
                \end{scnindent}
                \scnitem{полуфабрикат\scnrolesign}
                \begin{scnindent}
                    \scntext{пояснение}{\textit{полуфабрикат\scnrolesign} --- наполовину изготовленный из сырья продукт.}
                \end{scnindent}
                \scnitem{прототип\scnrolesign}
                \begin{scnindent}
                    \scntext{пояснение}{\textit{прототип\scnrolesign} --- опытный образец продукта.}
                \end{scnindent}
                \scnitem{изделие\scnrolesign}
                \begin{scnindent}
                    \scntext{пояснение}{\textit{изделие\scnrolesign} --- готовый продукт.}
                \end{scnindent}
            \end{scnsubdividing}
        \end{scnindent}
    \end{scnsubdividing}
    
    \scnheader{воздействие}
    \scnidtf{акция}
    \scnrelfrom{источник}{\cite{Hardzei2017}}
    \scnrelfrom{разбиение}{\scnkeyword{Типология по характеру взаимодействия участников\scnsupergroupsign}}
    \begin{scnindent}
	    \begin{scneqtoset}
	        \scnitem{воздействие активизации}
	        \begin{scnindent}
	            \scntext{пояснение}{\textit{воздействие активизации} --- воздействие, в ходе которого взаимодействие осуществляется между субъектом и инструментом.}
	        \end{scnindent}
	        \scnitem{воздействие эксплуатации}
	        \begin{scnindent}
	            \scntext{пояснение}{\textit{воздействие эксплуатации} --- воздействие, в ходе которого взаимодействие осуществляется между инструментом и медиатором.}
	        \end{scnindent}
	        \scnitem{воздействие трансформации}
	        \begin{scnindent}
	            \scntext{пояснение}{\textit{воздействие трансформации} --- воздействие, в ходе которого взаимодействие осуществляется между объектом и продуктом.}
	        \end{scnindent}
	        \scnitem{воздействие нормализации}
	        \begin{scnindent}
	            \scntext{пояснение}{\textit{воздействие нормализации} --- воздействие, в ходе которого взаимодействие осуществляется между объектом и продуктом.}
	        \end{scnindent}
	    \end{scneqtoset}
    \end{scnindent}
    \scnrelfrom{разбиение}{\scnkeyword{Типология воздействий по виду взаимодействующих подсистем\scnsupergroupsign}}
    \begin{scnindent}
	    \begin{scneqtoset}
	        \scnitem{воздействие среда-оболочка}
	        \scnitem{воздействие оболочка-ядро}
	        \scnitem{воздействие ядро-оболочка}
	        \scnitem{воздействие оболочка-среда}
	    \end{scneqtoset}
	\end{scnindent}
	
    \scnheader{воздействие}
    \scnrelfrom{разбиение}{\scnkeyword{Типология воздействий по фазам наращивания воздействия\scnsupergroupsign}}
	\begin{scnindent}
	    \begin{scneqtoset}
	        \scnitem{воздействие инициации}
	        \begin{scnindent}
	            \scntext{пояснение}{\textit{воздействие инициации} --- воздействие, в ходе которого оно начинается в каждой подсистеме.}
	        \end{scnindent} 
	        \scnitem{воздействие аккумуляции}
	        \begin{scnindent}
	            \scntext{пояснение}{\textit{воздействие аккумуляции} --- воздействие, в ходе которого происходит его накапливание в каждой подсистеме.}
	        \end{scnindent}
	        \scnitem{воздействие амплификации}
	        \begin{scnindent}
	            \scntext{пояснение}{\textit{воздействие амплификации} --- воздействие, в ходе которого происходит его усиление.}
	        \end{scnindent}
	        \scnitem{воздействие генерации}
	        \begin{scnindent}
	            \scntext{пояснение}{\textit{воздействие генерации} --- воздействие, которое представляет собой переход в каждой подсистеме с одного уровня, например, среды-оболочки, на другой, например, оболочки-ядра.}
	        \end{scnindent}
	    \end{scneqtoset}
	\end{scnindent}
	
    \scnheader{воздействие}
    \scnrelfrom{разбиение}{\scnkeyword{Типология воздействий по виду инструмента\scnsupergroupsign}}
	\begin{scnindent}
	    \begin{scneqtoset}
	        \scnitem{физическое воздействие}
	        \begin{scnindent}
	            \scntext{пояснение}{\textit{физическое воздействие} --- воздействие, в котором в роли инструмента выступает оболочка субъекта.}
	        \end{scnindent}
	        \scnitem{информационное воздействие}
	        \begin{scnindent} 
	            \scntext{пояснение}{\textit{информационное воздействие} --- воздействие, в котором в роли инструмента выступает среда субъекта.}
	       \end{scnindent}
	    \end{scneqtoset}
	\end{scnindent}

    \scnheader{Рис. Таблица воздействий}
    \scneqfile{\scnfileimage[30em]{Contents/part_kb/src/images/sd_actions/macroproc_table.png}}
    \scnrelfrom{источник}{\cite{Hardzei2017}}
    \scntext{пояснение}{На изображении представлена типология \textit{воздействий}. Любое воздействие характеризуется принадлежностью четырём классам, соответствующим признакам классификации. Заштрихованы \textit{физические воздействия}.}
   
    \scnheader{Специфицируемые классы воздействий}
    \begin{scnhassubset}
            \scnitem{формование}
            \begin{scnindent}
	            \begin{scnreltoset}{пересечение}
	                \scnitem{воздействие трансформации}
	                \scnitem{воздействие среда-оболочка}
	                \scnitem{воздействие генерации}
	                \scnitem{физическое воздействие}
	            \end{scnreltoset}
        	\end{scnindent}
            \scnitem{притягивание}
            \begin{scnindent}
	            \begin{scnreltoset}{пересечение}
	                \scnitem{воздействие активизации}
	                \scnitem{воздействие среда-оболочка}
	                \scnitem{воздействие инициации}
	                \scnitem{физическое воздействие}
	            \end{scnreltoset}
        	\end{scnindent}
            \scnitem{выхолащивание}
            \begin{scnindent}
	            \begin{scnreltoset}{пересечение}
	                \scnitem{воздействие трансформации}
	                \scnitem{воздействие ядро-оболочка}
	                \scnitem{воздействие генерации}
	                \scnitem{физическое воздействие}
	            \end{scnreltoset}
        	\end{scnindent}
            \scnitem{аннигилирование}
            \begin{scnindent}
	            \begin{scnreltoset}{пересечение}
	                \scnitem{воздействие трансформации}
	                \scnitem{воздействие оболочка-среда}
	                \scnitem{воздействие генерации}
	                \scnitem{физическое воздействие}
	            \end{scnreltoset}
        	\end{scnindent}
            \scnitem{введение}
            \begin{scnindent}
	            \begin{scnreltoset}{пересечение}
	                \scnitem{воздействие эксплуатации}
	                \scnitem{воздействие оболочка-ядро}
	                \scnitem{воздействие инициации}
	                \scnitem{физическое воздействие}
	            \end{scnreltoset}
        	\end{scnindent}
            \scnitem{распускание}
            \begin{scnindent}
	            \begin{scnreltoset}{пересечение}
	                \scnitem{воздействие трансформации}
	                \scnitem{воздействие оболочка-среда}
	                \scnitem{воздействие амплификации}
	                \scnitem{физическое воздействие}
	            \end{scnreltoset}
        	\end{scnindent}
            \scnitem{разжимание}
            \begin{scnindent}
	            \begin{scnreltoset}{пересечение}
	                \scnitem{воздействие трансформации}
	                \scnitem{воздействие ядро-оболочка}
	                \scnitem{воздействие амплификации}
	                \scnitem{физическое воздействие}
	            \end{scnreltoset}
        	\end{scnindent}
            \scnitem{разъединение}
            \begin{scnindent}
	            \begin{scnreltoset}{пересечение}
	                \scnitem{воздействие эксплуатации}
	                \scnitem{воздействие ядро-оболочка}
	                \scnitem{воздействие генерации}
	                \scnitem{физическое воздействие}
	            \end{scnreltoset}
        	\end{scnindent}
    \end{scnhassubset}
    
    \scnheader{Пример sc.g-текста, описывающего спецификацию воздействия}
    \scneq{\scnfileimage[30em]{Contents/part_kb/src/images/sd_actions/tapaz_description_example.png}}
    \scniselement{sc.g-текст}
    \scntext{пояснение}{Представленный фрагмент базы знаний содержит декомпозицию воздействия во времени, указание принадлежности данного декомпозируемого воздействия и полученных в результате данной декомпозиции воздействий определенному их классу из приведенной выше классификации, а также указание участников данных акций.}
    \scntext{пояснение}{Представленный фрагмент базы знаний можно протранслировать в следующий текст естественного языка: <<Некто принимает молоко, затем окисляет молоко, а именно: нормализует молоко до 15-процентной жирности, затем очищает молоко, затем пастеризует молоко, затем охлаждает молоко до определённой температуры, затем вносит закваску в молоко, затем сквашивает молоко, затем режет сгусток, затем подогревает сгусток, затем обрабатывает сгусток, затем отделяет сыворотку, затем охлаждает сгусток и, в итоге, производит творог>>.}
    
    \scnheader{субъект}
    \scnidtftext{часто используемый sc-идентификатор}{индивид}
    \scnidtf{активная сущность}
    \scnsubset{сущность}
    \scnidtf{сущность, способная самостоятельно выполнять некоторые виды действий}
    \scnidtf{агент деятельности}
    \scnsuperset{Собственное Я}
    \scnsuperset{внутренний субъект ostis-системы}
    \scnsuperset{внешний субъект ostis-системы, с которым осуществляется взаимодействие}
    \scnsuperset{внешний субъект ostis-системы, с которым взаимодействие не происходит}
    \scnheader{внутренний субъект ostis-системы}
    \scnidtf{субъект, входящий в состав той \textit{ostis-системы, в базе знаний} которой он описывается}
    \scnsuperset{sc-агент}
    \scntext{пояснение}{Под \textit{внутренним субъектом ostis-системы} понимается такой \textit{субъект}, который выполняет некоторые \textit{действия} в \uline{той же памяти}, в которой хранится его знак.\\
        К числу \textit{внутренних субъектов ostis-системы} относятся входящие в нее \textit{sc-агенты}, частные sc-машины, целые интеллектуальные подсистемы.}
    
    \scnheader{внешний субъект ostis-системы, с которым осуществляется взаимодействие}
    \scntext{пояснение}{К числу \textit{внешних субъектов ostis-системы, с которыми осуществляется взаимодействие}, относятся конечные пользователи \textit{ostis-системы}, ее разработчики, а также другие компьютерные системы (причем не только интеллектуальные).}
    
    \scnheader{субъект действия\scnrolesign}
    \scnsubset{субъект\scnrolesign}
    \scnidtf{сущность, воздействующая на некоторую другую сущность в процессе заданного действия\scnrolesign}
    \scnidtf{сущность, создающая \textit{причину} изменений другой сущности (объекта действия)\scnrolesign}
    \scnidtf{быть субъектом данного действия\scnrolesign}
    \scnsuperset{субъект неосознанного воздействия\scnrolesign}
    \scnsuperset{субъект осознанного воздействия\scnrolesign}
	\begin{scnindent}
	    \scnidtf{субъект целенаправленного, активного воздействия\scnrolesign}
	\end{scnindent}
    
    \scnheader{субъект\scnrolesign}
    \scnidtf{воздействующая сущность\scnrolesign}
    \scnidtf{сущность, создающая \textit{причину} изменений другой сущности (объекта)\scnrolesign}
    
    \scnheader{объект\scnrolesign}
    \scnidtf{воздействуемая сущность\scnrolesign}
    \scnidtf{сущность, являющаяся в рамках заданного воздействия исходным условием (аргументом), необходимым для выполнения этого воздействия\scnrolesign}
    
    \scnheader{исполнитель*}
    \scntext{пояснение}{Связки отношения \textit{исполнитель*} связывают \textit{sc-элементы}, обозначающие \textit{действие} и \textit{sc-элементы}, обозначающие \textit{субъекта}, который предположительно будет осуществлять, осуществляет или осуществлял выполнение указанного \textit{действия}. Данное отношение может быть использовано при назначении конкретного исполнителя для проектной задачи по развитию баз знаний.\\
        В случае, когда заранее неизвестно, какой именно \textit{субъект*} будет исполнителем данного \textit{действия}, отношение \textit{исполнитель*} может отсутствовать в первоначальной формулировке \textit{задачи} и добавляться позже, уже непосредственно при исполнении.\\
        Когда действие выполняется (является \textit{настоящей сущностью}) или уже выполнено (является \textit{прошлой сущностью}), то исполнитель этого действия в каждый момент времени уже определён. Но когда действие только инициировано, тогда важно знать:
        \begin{enumerate}
            \item кто \uline{хочет} выполнить это действие и насколько важно для него стать исполнителем данного действия;
            \item кто \uline{может} выполнить данное действие и каков уровень его квалификации и опыта;
            \item кто и кому поручает выполнить это действие и каков уровень ответственности за невыполнение (приказ, заказ, официальный договор, просьба...)
        \end{enumerate}
        При этом следует помнить, что связь отношения \textit{исполнитель*} в данном случае также является временной прогнозируемой сущностью.\\
        Первым компонентом связок отношений \textit{исполнитель*} является знак \textit{действия}, вторым --- знак \textit{субъекта} исполнителя.}
        
    \scnheader{объект воздействия\scnrolesign}
    \scnsubset{объект\scnrolesign}
    \scnidtf{сущность, на которую осуществляется воздействие в рамках заданного действия\scnrolesign}
    \scnidtf{сущность, являющаяся в рамках заданного действия исходным условием (аргументом), необходимым для выполнения этого действия\scnrolesign}
    \scntext{примечание}{Для разных действий количество объектов действий может быть различным.}
    \scntext{примечание}{Поскольку действие является процессом и, соответственно, представляет собой \textit{динамическую структуру}, то и знак \textit{субъекта действия\scnrolesign}, и знак \textit{объекта действия\scnrolesign} являются элементами данной структуры. В связи с этим можно рассматривать отношения \textit{субъект действия\scnrolesign} и \textit{объект действия\scnrolesign} как \textit{ролевые отношения}. Данный факт не  запрещает вводить аналогичные \textit{неролевые отношения}, однако это нецелесообразно.}
    
    \scnheader{продукт\scnrolesign}
    \scnidtf{быть продуктом заданного действия\scnrolesign}
    \scnsubset{продукт*}
    \scnsubset{результат*}
    \scnidtf{сухой остаток\scnrolesign}
    \scnidtf{то, ради чего может быть выполнено, выполняется или будет выполняться заданное действие\scnrolesign}
    \scntext{примечание}{Продуктом действия может быть некоторая материальная сущность, некоторое множество (тираж) одинаковых материальных сущностей, некоторая информационная конструкция.}
    
    \scnheader{результат*}
    \scntext{пояснение}{Связки отношения \textit{результат*} связывают \textit{sc-элемент}, обозначающий \textit{действие}, и \textit{sc-конструкцию}, описывающую результат выполнения рассматриваемого действия, другими словами, цель, которая должна быть достигнута при выполнении \textit{действия}.\\
    Результат может специфицироваться как атомарным высказыванием, так и неатомарным, т.е. конъюнктивным, дизъюнктивным, строго дизъюнктивным и т.д.\\
    В случае, когда успешное выполнение \textit{действия} приводит к изменению какой-либо конструкции в \textit{\mbox{sc-памяти}}, которое необходимо занести в историю изменений базы знаний или использовать для демонстрации протокола решении задачи, генерируется соответствующая связка отношения \textit{результат*}, связывающая задачу и \textit{sc-конструкцию}, описывающую данное изменение. Конкретный вид указанной \textit{\mbox{sc-конструкции}} зависит от типа действия.}
    \scnrelboth{следует отличать}{цель*}
	\begin{scnindent}
	    \scnidtf{спецификация планируемого результата*}
	    \scntext{примечание}{Следует также отмечать то, что является непосредственно результатом (продуктом) некоторого действия, и то, что является предварительной (исходной, стартовой) спецификацией этого результата. Далеко не всегда результатом действия является именно то, что планировалось (было целью) изначально.}
	\end{scnindent}
    
    \scnheader{цель*}
	\scnidtf{целевая ситуация*}
	\scnsubset{спецификация*}
	\scnidtf{описание того, что требуется получить (какая ситуация должна быть достигнута) в результате выполнения заданного (специфицируемого) действия*}

    \scnheader{класс выполняемых действий*}
    \scnidtf{класс действий, выполняемых классом субъектов*}
    \scntext{пояснение}{Связки отношения \textit{класс выполняемых действий*} связывают классы \textit{субъектов} и классы действий, при этом предполагается, что каждый субъект указанного класса способен выполнять действия указанного класса действий.}
    
    \scnheader{заказчик*}
    \scntext{пояснение}{Связки отношения \textit{заказчик*} связывают классы \textit{sc-элементов}, обозначающие \textit{действие}, и классы \textit{sc-элементов}, обозначающие \textit{субъекта}, который заинтересован в выполнении данного действия и, как правило, инициирует его выполнение. Данное отношение может быть использовано при указании того, кто поставил проектную задачу по развитию баз знаний.\\
        Первым компонентов связок отношения \textit{заказчик*} является знак \textit{действия}, вторым --- знак \textit{субъекта}.}
    
    \scnheader{инициатор*}
    \scntext{пояснение}{Связки отношения \textit{инициатор*} связывают \textit{sc-элемент}, обозначающий \textit{инициированное действие}, и знак \textit{субъекта}, который является инициатором данного \textit{действия}, то есть \textit{субъектом}, который инициировал данное \textit{действие} и, как правило, заинтересован в его успешном выполнении.}
    
    \scnheader{объект\scnrolesign}
    \scnidtf{аргумент действия\scnrolesign}
    \scntext{пояснение}{Связки отношения \textit{объект\scnrolesign} связывают \textit{sc-элемент}, обозначающий \textit{действие}, и знак той сущности, над которой (по отношению к которой) осуществляется данное \textit{действие}, и, например, знак \textit{структуры}, подлежащий верификации.}
    \scnsuperset{первый аргумент действия\scnrolesign}
    \scnsuperset{второй аргумент действия\scnrolesign}
    \scnsuperset{третий аргумент действия\scnrolesign}
    
    \scnheader{класс аргументов*}
    \scnidtf{класс аргументов класса команд*}
    \scnidtf{быть классом sc-элементов, экземпляры которого являются аргументами для заданного класса команд*}
    \scnsuperset{класс первых аргументов*}
    \scnsuperset{класс вторых аргументов*}
    \scntext{пояснение}{Связки отношения \textit{класс аргументов*} связывают \textit{классы команд} (подмножества множества \textit{команд}) и классы \textit{sc-элементов}, которые могут быть аргументами действий, соответствующих данному \textit{классу команд}. В случае, когда \textit{команды} данного класса имеют один аргумент, используется собственно отношение  \textit{класс аргументов*}, в случае, когда команды данного класса имеют более одного аргумента, используются подмножества данного отношения, такие как \textit{класс первых аргументов*}, \textit{класс вторых аргументов*} и т.д.\\
        Если для некоторого \textit{класса команд} не указан тип какого-либо из аргументов, то предполагается, что в качестве данного аргумента может выступать любой \textit{sc-элемент}.\\
        Первым компонентом связок отношения \textit{класс аргументов*} является знак \textit{класса команд}, вторым --- знак класса \textit{sc-элемента}, которые могут быть \textit{аргументами действий\scnrolesign}, соответствующих данному \textit{классу команд}.}
    
        \bigskip
\end{scnsubstruct}
\scnendsegmentcomment{Предметная область и онтология субъектно-объектных спецификаций воздействий}
