\scnsegmentheader{Использование SC-кода для формального описания собственного синтаксиса}
\begin{scnsubstruct}
    \scnheader{SC-код}
    \scntext{примечание}{В предыдущем сегменте \scnqqi{\textbf{SC-код как синтаксическое расширение Ядра SC-кода}} рассмотрен \textbf{Синтаксис SC-кода} путём:
    \begin{scnitemize}
        \item введения \textit{синтаксически выделяемых классов sc-элементов} в рамках \textit{SC-кода};
        \item описания \textit{теоретико-множественных связей} между указанными классами \textit{sc-элементов} (к такому описанию, в частности, относится \textit{Синтаксическая классификация sc-элементов в рамках SC-кода});
        \item введения двух отношений инцидентности \textit{sc-элементов} --- \textit{Отношения инцидентности sc-коннекторов*} и \textit{Отношения инцидентности входящих sc-дуг*};
        \item описания \textit{Синтаксических правил SC-кода}, которые, прежде всего, описывают формальные свойства указанных выше отношений инцидентности \textit{sc-элементов}.
    \end{scnitemize}
    Однако для того, чтобы получить возможность \uline{все} (!) \textit{Синтаксические правила SC-кода} записать средствами самого \textit{SC-кода}, необходимо иметь \uline{явное} представление \textit{пар} отношений инцидентности \textit{sc-элементов} в виде \textit{sc-дуг}, принадлежащим этим отношениям. В случае, если указанные \textit{sc-дуги} инцидентности являются \textit{sc-переменными}, логико-семантических проблем не возникнет. И этого, кстати, вполне достаточно, чтобы \textit{Синтаксические правила SC-кода}, сформулированные в виде \textit{логических высказываний}, записать средствами \textit{SC-кода}. Но, если разрешить \textit{sc-дугам} инцидентности быть \textit{sc-константами}, то, во-первых, в \textit{Синтаксические правила SC-кода} необходимо добавить Правило удаления \textit{константной sc-дуги инцидентности}, если эта инцидентность представлена неявно, а, во-вторых, в \textit{Правила синтаксической трансформации sc-элементов} необходимо добавить Правило трансформации (замены) \textit{константной sc-дуги инцидентности} на неявное представление этой инцидентности.В теоретическом и, возможно, даже в практическом плане может быть интересна такая синтаксическая модификация (синтаксическое расширение) \textit{SC-кода}, в котором: 
    \begin{scnitemize}
        \item \uline{все} неявно представленные \textit{пары инцидентности sc-элементов} заменяются на \textit{константные sc-дуги инцидентности} --- неявно представленными \textit{парами инцидентности} остаются \uline{только} \textit{пары инцидентности} константных \textit{sc-дуг} инцидентности с компонентами этих \textit{sc-дуг};
        \item В \textbf{Алфавит SC-кода} вводятся два новых \textit{синтаксически выделяемых класса sc-элементов} --- \textit{класс sc-дуг инцидентности sc-коннекторов}, а также \textit{класс sc-дуг инцидентности входящих sc-дуг}.
    \end{scnitemize}
    В результате такого преобразования конструкций \textit{SC-кода} конструкции \textit{SC-кода} перестают быть графовыми конструкциями нетрадиционного вида, в которых рёбра, гиперрёбра, дуги могут быть инцидентны другим рёбрам, гиперребрам и дугам, а становятся классическими графами с двумя типами дуг (с \textit{sc-дугами инцидентности sc-коннекторов} и с \textit{sc-дугами инцидентности входящих sc-дуг}) и с пятью типами вершин (с вершинами, представляющими \textit{sc-узлы общего вида}, с вершинами, представляющими \textit{sc-узлы}, являющиеся знаками \textit{внутренних файлов ostis-системы}, с вершинами, представляющими \textit{sc-рёбра общего вида}, с вершинами, представляющими \textit{sc-дуги общего вида}, с вершинами, представляющими \textit{базовые sc-дуги}).}
    \scntext{примечание}{Рассмотренное преобразование конструкций \textit{SC-кода} в теории графов называется поздразделением или подразбиением графа.}
    \begin{scnindent}
    	\scnrelfrom{смотрите}{\scncite{Trudeau1993}}
    \end{scnindent}
\end{scnsubstruct}
\scnsourcecommentinline{Завершили Сегмент \scnqqi{Использование SC-кода для формального описания собственного синтаксиса}}
