\begin{SCn}
\scnsectionheader{\currentname}
\begin{scnsubstruct}
\begin{scnrelfromlist}{дочерний раздел}
	%TODO: Update
	\scnitem{Предметная область и онтология действий, задач, планов, протоколов и методов, реализуемых ostis-системой, а также внутренних агентов, выполняющих эти действия}
	\scnitem{Предметная область и онтология Базового языка программирования ostis-систем}
	\scnitem{Предметная область и онтология искусственных нейронных сетей и соответствующая им предметная область и онтология действий по обработке искусственных нейронных сетей}
\end{scnrelfromlist}

\bigskip

\begin{scnrelfromlist}{ключевой знак}
	\scnitem{Язык SCP}
	\scnitem{Абстрактная scp-машина}
\end{scnrelfromlist}

\begin{scnrelfromlist}{ключевое понятие}
	\scnitem{действие в sc-памяти}
	\scnitem{действие в sc-памяти, инициируемое вопросом}
	\scnitem{действие редактирования базы знаний}
	\scnitem{задача, решаемая в sc-памяти}
	\scnitem{класс логически атомарных действий}
	\scnitem{sc-агент}
	\scnitem{абстрактный sc-агент}
	\scnitem{атомарный абстрактный sc-агент}
	\scnitem{неатомарный абстрактный sc-агент}
	\scnitem{абстрактный sc-агент, реализуемый на Языке SCP}
	\scnitem{абстрактный sc-агент, не реализуемый на Языке SCP}	
	\scnitem{тип блокировки}
	\scnitem{транзакция в sc-памяти}	
	\scnitem{scp-оператор}
	\scnitem{решатель задач ostis-системы}
	\scnitem{машина обработки знаний}
\end{scnrelfromlist}

\begin{scnrelfromlist}{ключевое отношение}
	\scnitem{блокировка*}
	\scnitem{планируемая блокировка*}
	\scnitem{приоритет блокировки*}
	\scnitem{удаляемые sc-элементы*}
	\scnitem{параметр scp-программы\scnrolesign}
	\scnitem{scp-операнд\scnrolesign}
\end{scnrelfromlist}

\bigskip

\begin{scnrelfromlist}{библиографическая ссылка}
	\scnitem{\scncite{Kolesnikov2001}}
	\scnitem{\scncite{Pratt2002}}
	\scnitem{\scncite{Gladkov2006}}
	\scnitem{\scncite{Emelyanov2003}}
	\scnitem{\scncite{Berkinblit1993}}
	\scnitem{\scncite{Golovko2001}}
	\scnitem{\scncite{Gorban1996}}
	\scnitem{\scncite{Vagin2008}}
	\scnitem{\scncite{Khulick2001}}
	\scnitem{\scncite{Polya1975}}
	\scnitem{\scncite{Batyrshin2001}}
	\scnitem{\scncite{Demenkov2005}}
	\scnitem{\scncite{Pospelov1989}}
	\scnitem{\scncite{Reiter1980}}
	\scnitem{\scncite{Eremeev1997}}
	\scnitem{\scncite{Kachro1988}}
	\scnitem{\scncite{Ephymov1982}}
	\scnitem{\scncite{Raghovsky2011}}
	\scnitem{\scncite{Podkholzyn2008}}
	\scnitem{\scncite{Khurbatov2016}}
	\scnitem{\scncite{Vladimirov2010}}
	\scnitem{\scncite{AIRefBookP11990}}
	\scnitem{\scncite{Jackson1998}}
	\scnitem{\scncite{W3C}}
	\scnitem{\scncite{RDF}}
	\scnitem{\scncite{OWL}}
	\scnitem{\scncite{SPARQL}}
	\scnitem{\scncite{Neo4j}}
	\scnitem{\scncite{OWLImplementations}}
	\scnitem{\scncite{Gribova2015a}}
	\scnitem{\scncite{Gribova2011}}
	\scnitem{\scncite{Phylyppov2016}}
	\scnitem{\scncite{Borisov2014}}
	\scnitem{\scncite{Dutta1993}}
	\scnitem{\scncite{Pau1990}}
	\scnitem{\scncite{Wooldridge2009}}
	\scnitem{\scncite{Weyns2007}}
	\scnitem{\scncite{ACL}}
	\scnitem{\scncite{Finin1994}}
	\scnitem{\scncite{KIF}}
	\scnitem{\scncite{Hartung2008}}
	\scnitem{\scncite{Sims2008}}
	\scnitem{\scncite{Excelente-Toledo2004}}
	\scnitem{\scncite{NagendraPrasad1999}}
	\scnitem{\scncite{Vasconcelos2009}}
	\scnitem{\scncite{Rumbell2012}}
	\scnitem{\scncite{Gorodetsky2015}}
	\scnitem{\scncite{Bordini2007}}
	\scnitem{\scncite{Castillo2014}}
	\scnitem{\scncite{EVE}}
	\scnitem{\scncite{GAMA}}
	\scnitem{\scncite{GOAL}}
	\scnitem{\scncite{Evertsz2004}}
	\scnitem{\scncite{JADE}}
	\scnitem{\scncite{Boissier2013}}
	\scnitem{\scncite{Omicini1999}}
	\scnitem{\scncite{Jagannathan1989}}
	\scnitem{\scncite{Pospelov1986}}
	\scnitem{\scncite{Dijkstra2002}}
	\scnitem{\scncite{Hoare1983}}
	\scnitem{\scncite{Chatterjee2022}}
	\scnitem{\scncite{Narinjani2004}}
	\scnitem{\scncite{Cao2010}}
	\scnitem{\scncite{Cao2014}}
	\scnitem{\scncite{Pavel2015}}
	\scnitem{\scncite{Altshuller2010}}
	\scnitem{\scncite{Shhedrovickij1995}}
	\scnitem{\scncite{Sapatyj1986}}
	\scnitem{\scncite{Moldovan1985}}
	\scnitem{\scncite{Letichevskij2003}}
	\scnitem{\scncite{Letichevskij2012}}
\end{scnrelfromlist}


\scntext{введение}{Одним из ключевых компонентов \textit{интеллектуальной системы}, обеспечивающим возможность решать широкий круг \textit{задач}, является \textit{решатель задач}. Их особенностью по сравнению с другими современными \textit{программными системами} является необходимость решать \textit{задачи} в условиях, когда необходимые сведения не локализованы явно в \textit{базе знаний} \textit{интеллектуальной системы} и должны быть найдены в процессе решения \textit{задачи} на основании каких-либо критериев. 

Говоря другими словами, если в традиционных системах при решении задачи всегда подразумевается, что есть некоторые локализованные исходные данные ("дано") и некоторое описание желаемого результата ("что требуется"), то в \textit{интеллектуальной системе} в качестве исходных данных при решении большого числа \textit{задач} выступает вся имеющаяся на текущий момент в системе информация, то есть вся \textit{база знаний}. Кроме того, при невозможности решения задачи в текущем состоянии базы знаний интеллектуальная система должна иметь возможность понять, чего именно не хватает для продолжения процесса решения и попытаться добыть недостающие сведения во внешней среде (например, запросить у пользователя).

К настоящему времени в рамках различных направлений \textit{Искусственного интеллекта} разработано большое количество различных \textit{моделей решения задач}, каждая из которых позволяет решать задачи определенного класса. Расширение областей применения \textit{интеллектуальных систем} требует от них возможности решать так называемые \textit{комплексные задачи}, решение каждой из которых требует комбинирования нескольких моделей решения задач, при этом априори неизвестно, в каком порядке и сколько раз будет применяться так или иная модель. \textit{решатели задач}, в рамках которых комбинируются несколько \textit{моделей решения задач}, получили название \textit{гибридных решателей задач}, а интеллектуальные системы, в рамках которых комбинируются различные \textit{виды знаний} и различные \textit{модели решения задач} -- \textit{гибридных интеллектуальных систем} (см. \scncite{Kolesnikov2001}).

Повышение эффективности разработки и эксплуатации \textit{гибридных интеллектуальных систем} требует унификации моделей представления различных \textit{видов знаний} и \textit{моделей обработки знаний}, которая бы позволила легко интегрировать на ее основе компоненты, соответствующие различным моделям решения задач.}

\scnheader{решатель задач}
\scntext{современное состояние технологий разработки}{Существующее многообразие подходов к решению \textit{задач} в \textit{компьютерных системах} можно разделить на два класса:
	\begin{textitemize}
		\item \textbf{решение задач с использованием хранимых программ.} В данном случае предполагается, что в системе заранее присутствует программа решения задачи заданного класса и решение сводится к поиску такой программы и интерпретации ее на заданных входных данных. К системам, ориентированным на такой подход к решению задач, относятся в том числе системы, использующие:
		\begin{textitemize}
			\item программы, написанные на языках программирования, относящихся как к императивной, так и к декларативной парадигме, в том числе логических и функциональных (см. \scncite{Pratt2002});
			\item реализации генетических алгоритмов (см. \scncite{Gladkov2006}, \scncite{Emelyanov2003});
			\item нейросетевые модели обработки знаний (см. \scncite{Berkinblit1993},\newline \scncite{Golovko2001}, \scncite{Gorban1996}).
		\end{textitemize}	
		\vspace{-2\parskip}
		Следует отметить, что даже в случае использования хранимой \textit{программы} решение \textit{задачи} далеко не всегда тривиально, поскольку, во-первых, требуется найти такую хранимую \textit{программу} на основе некоторой спецификации, во-вторых, обеспечить ее интерпретацию.
		\vspace{\parskip}
		\item \textbf{решение задач в условиях, когда программа решения не известна.} В этом случае предполагается, что в системе необязательно присутствует готовая \textit{программа} решения для \textit{класса задач}, которому принадлежит некоторая сформулированная задача, подлежащая решению. В связи с этим необходимо применять дополнительные методы поиска путей решения задачи, не рассчитанные на какой-либо узкий \textit{класс задач} (например, разбиение задачи на подзадачи, методы поиска решений в глубину и ширину, метод случайного поиска решения и метод проб и ошибок, метод деления пополам и другие), а также различные модели \textit{логического вывода}: классические дедуктивные (см. \scncite{Vagin2008}), индуктивные (см. \scncite{Khulick2001}, \scncite{Polya1975}), абдуктивные (см. \scncite{Vagin2008}); модели, основанные на \textit{нечетких логиках} (см. \scncite{Batyrshin2001}, \scncite{Demenkov2005}, \scncite{Pospelov1989}), \textit{логике умолчаний} (см. \scncite{Reiter1980}), \textit{темпоральной логике} (см. \scncite{Eremeev1997}), и многие другие.
	\end{textitemize}
	
	Подробный обзор \textit{решателей задач}, разработанных в период до 1982 года, таких как \textit{GPS}, \textit{STRIPS}, \textit{QA3}, \textit{ПРИЗ} (см. \scncite{Kachro1988}), \textit{ППР} приведен в книге \scncite{Ephymov1982}. Среди современных работ, исследующих вопросы применения \textit{моделей решения задач}, не ориентированных на конкретную предметную область, можно выделить \scncite{Raghovsky2011}. Среди наиболее заметных представителей класса \textit{интеллектуальных решателей задач}, разработанных в более поздний период, можно отметить \textit{Компьютерный решатель математических задач} (см. \scncite{Podkholzyn2008}), \textit{Решатель задач по планиметрии НИЦ ЭВТ} (см. \scncite{Khurbatov2016}), \textit{Программный комплекс ``УДАВ''} (см. \scncite{Vladimirov2010}). 
	
	Отдельного внимания заслуживают популярные в настоящее время \textit{системы компьютерной алгебры}, такие как \textit{Wolfram Mathematica}, \textit{Maple}, \textit{MathCAD} и другие. Указанные программные комплексы обладают мощной функциональностью как для проведения различного рода вычислений и экспериментов, так и для построения на их основе систем различного назначения, например обучающих. Более подробно возможности применения систем данного семейства для решения \textit{задач} в рамках \textit{Экосистемы OSTIS} рассмотрены в \textit{\ref{sec_integration_algebra}~\nameref{sec_integration_algebra}}.}

\scnheader{гибридный решатель задач}
\scntext{обоснование разработки}{Однако при всем многообразии решаемых рассмотренными системами \textit{задач} множество \textit{классов задач} ограничивается имеющимся в системе набором жестко заданных приемов и алгоритмов решения \textit{задач}, явно используемых при решении той или иной \textit{задачи}. В то же время построение сложных систем, например, систем комплексной автоматизации, невозможно без обеспечения согласованного использования различных \textit{видов знаний} и \textit{моделей решения задач} в рамках одной системы при решении одной и той же \textit{комплексной задачи}. Кроме того, становится актуальной \textit{задача} поддержки такой системы в состоянии, соответствующем текущему уровню развития технологий, дополнения ее более совершенными \textit{моделями} и \textit{методами решения задач}. При этом очевидно, что подобная реконфигурация системы должна осуществляться \myuline{непосредственно в процессе эксплуатации системы}, а не требовать каждый раз, например, полной остановки всего производства или отдельных его частей.}

%OLD BELOW

\scnheader{решатель задач ostis-системы}
\scnidtf{совокупность всех навыков, которыми обладает ostis-система на текущий момент времени}
\scnrelto{семейство подмножеств}{навык}
\scntext{примечание}{Предлагаемый в рамках \textit{Технологии OSTIS} подход к построению решателей задач позволяет обеспечить их модифицируемость, что, в свою очередь, позволяет \textit{ostis-системе} при необходимости легко приобретать новые \textit{навыки}, модифицировать (совершенствовать) уже имеющиеся и даже избавляться от некоторых навыков с целью повышения производительности системы. Таким образом, имеет смысл говорить не о жестко фиксированном решателе задач, который разрабатывается один раз при создании первой версии системы и далее не меняется, а о совокупности навыков, фиксированной в каждый текущий момент времени, но постоянно эволюционирующей.}
\scnsuperset{объединенный решатель задач ostis-системы}
	\begin{scnindent}
		\scnidtf{полный решатель задач ostis-системы}
		\scnidtf{интегрированный решатель задач ostis-системы}
		\scnidtf{решатель задач ostis-системы, реализующий все ее функциональные возможности, как основные, так и вспомогательные}
		\scntext{пояснение}{В общем случае \textit{объединенный решатель задач ostis-системы} решает задачи, связанные с:
		\begin{scnitemize}
			\item{обеспечением основных функциональных возможностей системы (например, решение явно сформулированных задач по требованию пользователя);}
			\item{обеспечением корректности и оптимизацией работы самой ostis-системы (перманентно на протяжении всего жизненного цикла ostis-системы);}
			\item{обеспечением повышения квалификации конечных пользователей и разработчиков ostis-системы;}
			\item{обеспечением автоматизации развития и управления развитием ostis-системы.}
		\end{scnitemize}
		}
	\end{scnindent}
\scnsuperset{гибридный решатель задач ostis-системы}
	\begin{scnindent}
		\scnidtf{решатель задач ostis-системы, реализующий две и более модели решения задач}
	\end{scnindent}
		
\scnheader{машина обработки знаний}
\scnsubset{sc-агент}
\scntext{пояснение}{Под \textit{машиной обработки знаний} будем понимать совокупность интерпретаторов всех \textit{навыков}, составляющих некоторый \textit{решатель задач}. С учетом многоагентного подхода к обработке информации, используемого в рамках Технологии OSTIS, \textit{машина обработки знаний} представляет собой \textit{sc-агент} (чаще всего --- \textit{неатомарный sc-агент}), в состав которого входят более простые sc-агенты, обеспечивающие интерпретацию соответствующего множества \textit{методов}. Таким образом, \textit{машина обработки знаний} в общем случае представляет собой иерархическую систему \textit{sc-агентов}.}

\scnheader{решатель задач ostis-системы}
\scnhaselement{Решатель задач Метасистемы IMS.ostis}
\scnsuperset{решатель задач вспомогательной ostis-системы}
	\begin{scnindent}
		\scnsuperset{решатель задач интерфейса компьютерной системы}
			\begin{scnindent}
				\begin{scnsubdividing}
					%TODO: check by human--->
					\scnitem{решатель задач пользовательского интерфейса компьютерной системы}
					\scnitem{решатель задач интерфейса компьютерной системы с другими компьютерными системами}
					\scnitem{решатель задач интерфейса компьютерной системы с окружающей средой}
					%<---TODO: check by human
				\end{scnsubdividing}
			\end{scnindent}
		\scnsuperset{решатель задач ostis-подсистемы поддержки проектирования компонентов определенного класса}
		\begin{scnindent}
			\scnsuperset{решатель задач ostis-подсистемы поддержки проектирования баз знаний}
				\begin{scnindent}
					\scnsuperset{решатель задач повышения качества базы знаний}
						\begin{scnindent}
							\scnsuperset{решатель задач верификации базы знаний}
								\begin{scnindent}
									\scnsuperset{решатель задач поиска и устранения некорректностей в базе знаний}
									\scnsuperset{решатель задач поиска и устранения неполноты}
								\end{scnindent}
							\scnsuperset{решатель задач оптимизации структуры базы знаний}
							\scnsuperset{решатель задач выявления и устранения информационного мусора}
						\end{scnindent}
				\end{scnindent}
			\scnsuperset{решатель задач ostis-подсистемы поддержки проектирования решателей задач ostis-систем}
			\begin{scnindent}
					\begin{scnsubdividing}
						%TODO: check by human--->
						\scnitem{решатель задач ostis-подсистемы поддержки проектирования программ обработки знаний}
						\scnitem{решатель задач ostis-подсистемы поддержки проектирования агентов обработки знаний}
						%<---TODO: check by human
					\end{scnsubdividing}
				\end{scnindent}
		\end{scnindent}
		\scnsuperset{решатель задач подсистемы управления проектирования компьютерных систем и их компонентов}
	\end{scnindent}
\scnsuperset{решатель задач самостоятельной ostis-системы}

\scnheader{решатель задач ostis-системы}
\scnsuperset{решатель задач с использованием хранимых методов}
	\begin{scnindent}
		\scnidtf{решатель, способный решать задачи тех классов, для которых на данный момент времени известен соответствующий метод решения}
		\scnsuperset{решатель задач на основе нейросетевых моделей}
		\scnsuperset{решатель задач на основе генетических алгоритмов}
		\scnsuperset{решатель задач на основе императивных программ}
			\begin{scnindent}
				\scnsuperset{решатель задач на основе процедурных программ}
				\scnsuperset{решатель задач на основе объектно-ориентированных программ}
			\end{scnindent}
		\scnsuperset{решатель задач на основе декларативных программ}
			\begin{scnindent}
				\scnsuperset{решатель задач на основе логических программ}
				\scnsuperset{решатель задач на основе функциональных программ}
			\end{scnindent}
	\end{scnindent}
\scnsuperset{решатель задач в условиях, когда метод решения задач данного класса в текущий момент времени не известен}
	\begin{scnindent}
		\scnidtf{решатель, реализующий стратегии решения задач, позволяющие породить метод решения задачи, который в текущий момент времени не известен ostis-системе}
		\scnidtf{решатель, использующий для решения задач метаметоды, соответствующие более общим классам задач по отношению к заданной}
		\scnidtf{решатель задач, позволяющий породить метод, который является частным по отношению к какому-либо известному ostis-системе методу и интерпретируется соответствующей машиной обработки знаний}
		\scnsuperset{решатель, реализующий стратегию поиска путей решения задачи в глубину}
		\scnsuperset{решатель, реализующий стратегию поиска путей решения задачи в ширину}
		\scnsuperset{решатель, реализующий стратегию проб и ошибок}
		\scnsuperset{решатель, реализующий стратегию разбиения задачи на подзадачи}
		\scnsuperset{решатель, реализующий стратегию решения задач по аналогии}
		\scnsuperset{решатель, реализующий концепцию интеллектуального пакета программ}
	\end{scnindent}

\scnheader{машина обработки знаний}
\scnsuperset{машина логического вывода}
\begin{scnindent}
	\scnsuperset{машина дедуктивного вывода}
		\begin{scnindent}
			\scnsuperset{машина прямого дедуктивного вывода}
			\scnsuperset{машина обратного дедуктивного вывода}
		\end{scnindent}
	\scnsuperset{машина индуктивного вывода}
	\scnsuperset{машина абдуктивного вывода}
	\scnsuperset{машина нечеткого вывода}
	\scnsuperset{машина вывода на основе логики умолчаний}
	\scnsuperset{машина логического вывода с учетом фактора времени}
\end{scnindent}

\scnheader{решатель задач ostis-системы}
\scnsuperset{решатель задач информационного поиска}
	\begin{scnindent}
		\begin{scnsubdividing}
			%TODO: check by human--->
			\scnitem{решатель задач поиска информации, удовлетворяющей заданным критериям}
			\scnitem{решатель задач поиска информации, не удовлетворяющей заданным критериям}
			%<---TODO: check by human
		\end{scnsubdividing}
	\end{scnindent}
\scnsuperset{решатель явно сформулированных задач}
	\begin{scnindent}
		\scnidtf{решатель задач, для которых явно сформулирована цель}
		\scnsuperset{решатель задач поиска или вычисления значений заданного множества величин}
		\scnsuperset{решатель задач установления истинности заданного логического высказывания в рамках заданной формальной теории}
		\scnsuperset{решатель задач формирования доказательства заданного высказывания в рамках заданной формальной теории}
		\scnsuperset{машина верификации ответа на указанную задачу}
		\scnsuperset{машина верификации решения указанной задачи}
			\begin{scnindent}
				\scnsuperset{машина верификации доказательства заданного высказывания в рамках заданной формальной теории}
			\end{scnindent}
	\end{scnindent}
\scnsuperset{решатель задач классификации сущностей}
	\begin{scnindent}
		\scnsuperset{машина соотнесения сущности с одним из заданного множества классов}
		\scnsuperset{машина разделения множества сущностей на классы по заданному множеству признаков}
	\end{scnindent}
\scnsuperset{решатель задач синтеза информационных конструкций}
	\begin{scnindent}
		\scnsuperset{решатель задач синтеза естественно-языковых текстов}
		\scnsuperset{решатель задач синтеза изображений}
		\scnsuperset{решатель задач синтеза сигналов}
		\begin{scnindent}
			\scnsuperset{решатель задач синтеза речи}
		\end{scnindent}
	\end{scnindent}
\scnsuperset{решатель задач анализа информационных конструкций}
	\begin{scnindent}
		\scnsuperset{решатель задач анализа естественно-языковых текстов}
			\begin{scnindent}
				\scnsuperset{решатель задач понимания естественно-языковых текстов}
				\scnsuperset{решатель задач верификации естественно-языковых текстов}
			\end{scnindent}
		\scnsuperset{решатель задач анализа изображений}
			\begin{scnindent}
				\scnsuperset{решатель задач сегментации изображений}
				\scnsuperset{решатель задач понимания изображений}
			\end{scnindent}
		\scnsuperset{решатель задач анализа сигналов}
			\begin{scnindent}
				\scnsuperset{решатель задач анализа речи}
					\begin{scnindent}
						\scnsuperset{решатель задач понимания речи}
					\end{scnindent}
			\end{scnindent}
	\end{scnindent}

\bigskip
\end{scnsubstruct}
\end{SCn}
