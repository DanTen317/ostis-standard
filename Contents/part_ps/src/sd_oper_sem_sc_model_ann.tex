\begin{SCn}
    \scnsectionheader{Предметная область и онтология операционной семантики sc-моделей искусственных нейронных сетей}
    \begin{scnsubstruct}

        \scnsegmentheader{Операционная семантика sc-моделей искусственных нейронных сетей, используемых в ostis-системах}
        \begin{scnsubstruct}

            \scnheader{Предметная область нейросетевых методов}
            \scnidtf{Предметная область искусственных нейронных сетей}
            \begin{scnrelfromset}{дочерняя предметная область}
                \scnitem{Предметная область нейросетевых методов SCP}
                \scnitem{Предметная область нейросетевых методов Python}
                \scnitem{Предметная область нейросетевых методов C++}
            \end{scnrelfromset}
            \scniselement{предметная область}
            \begin{scnhaselementrole}{максимальный класс объектов исследования}
            {нейросетевой метод}
            \end{scnhaselementrole}
            \begin{scnhaselementrolelist}{класс объектов исследования}
                \scnitem{нейросетевой метод}
                \scnitem{действие интерпретации нейросетевого метода}
                \scnitem{ориентированное множество чисел}
                \scnitem{матрица}
                \scnitem{действие вычисления взвешенной суммы всех нейронов слоя}
                \scnitem{действие вычисления функции активации всех нейронов слоя}
                \scnitem{действие интерпретации сверточного слоя}
                \scnitem{действие интерпретации пулинг слоя}
                \scnitem{проблема задания различных аргументов для нейронов одного слоя}
                \scnitem{агентно-ориентированная модель интерпретации искусственных нейронных сетей}
                \scnitem{задача "ИСКЛЮЧАЮЩЕЕ ИЛИ"}
                \scnitem{однослойный персептрон, решающий задачу \scnqq{ИСКЛЮЧАЮЩЕЕ ИЛИ}}
            \end{scnhaselementrolelist}

            \scnheader{нейросетевой метод}
            \scntext{примечание}{в случае описания \textbf{\textit{нейросетевого метода}} на внешнем языке, такой метод описывается в соответствующей предметной области, в рамках которой также специфицируется \textbf{\textit{действие интерпретации}} данного \textbf{\textit{нейросетевого метода}}}
            \begin{scnindent}
            \scntext{примечание}{внешний язык --- язык, который не является внутренним языком ostis-системы, т.е. не Язык SCP}
            \end{scnindent}
            \scntext{пример описания}{\textbf{\textit{задача "ИСКЛЮЧАЮЩЕЕ ИЛИ"}}}

            \scnheader{действие интерпретации нейросетевого метода}
            \scnidtf{действие интерпретации и.н.с}
            \scntext{примечание}{действию соответствует агент, реализованный на соответствующем языке программирования}
                \begin{scnrelfromset}{декомпозиция}
                    \scnitem{действие интерпретации слоя и.н.с.}
                    \begin{scnindent}
                        \begin{scnrelfromset}{декомпозиция}
                            \scnitem{\textbf{действие вычисления взвешенной суммы всех нейронов слоя}}
                            \scnitem{\textbf{действие вычисления функции активации всех нейронов слоя}}
                            \scnitem{\textbf{действие интерпретации сверточного слоя}}
                            \scnitem{\textbf{действие интерпретации пулинг слоя}}
                        \end{scnrelfromset}
                    \end{scnindent}
                \end{scnrelfromset}


            \scnheader{ориентированное множество чисел}
            \scnidtf{ормножество чисел}
            \scnrelto{включение}{число}
            \scnrelto{включение}{ориентированное множество}
            \scnrelto{первый домен}{строковое представление ормножества чисел*}

            \scnheader{матрица}
            \scndefinition{\textbf{\textit{матрица}} --- \textit{ориентированное множество} \textit{ориентированных множеств} чисел равной мощности }

            \scnheader{действие вычисления взвешенной суммы всех нейронов слоя}
            \begin{scnrelfromset}{отношения характеризующие аргументы данного действия}
                \scnitem{входной вектор\scnrolesign}
                \begin{scnindent}
                    \scniselement{ролевое отношение}
                    \scnrelfrom{первый домен}{действие интерпретации и.н.с.}
                    \scnrelfrom{второй домен}{матрица}
                \end{scnindent}
                \scnitem{матрица весовых коэффициентов нейронов слоя\scnrolesign}
                \begin{scnindent}
                    \scniselement{ролевое отношение}
                    \scnrelfrom{первый домен}{действие по обработке и.н.с.}
                    \scnrelfrom{второй домен}{матрица}
                \end{scnindent}
                \scnitem{результат\scnrolesign}
                \begin{scnindent}
                    \scniselement{ролевое отношение}
                    \scnrelfrom{первый домен}{действие интерпретации нейросетевого метода}
                    \scnrelfrom{второй домен}{взвешенная сумма нейронов соответствующего слоя}
                    \begin{scnindent}
                    \scnsubset{ориентированное множество чисел}
                    \end{scnindent}
                \end{scnindent}
            \end{scnrelfromset}
            \scnrelfrom{описание примера}{\scnfileimage[40em]{Contents/part_ps/src/images/sd_ps/sd_ann/action_weighted_sum.png}}
            \begin{scnindent}
            \scntext{примечание}{Пример спецификации действия вычисления взвешенной суммы всех нейронов слоя для слоя с двумя нейронами и входным вектором размерностью 2}
            \end{scnindent}

            \scnheader{действие вычисления функции активации всех нейронов слоя}
            \begin{scnrelfromset}{отношения характеризующие аргументы данного действия}
                \scnitem{вектор взвешенных сумм нейронов слоя\scnrolesign}
                \begin{scnindent}
                    \scniselement{ролевое отношение}
                    \scnrelfrom{первый домен}{действие по обработке и.н.с.}
                    \scnrelfrom{второй домен}{ориентированное множество чисел}
                \end{scnindent}
                \scnitem{вектор порогов нейронов слоя\scnrolesign}
                \begin{scnindent}
                    \scniselement{ролевое отношение}
                    \scnrelfrom{первый домен}{действие по обработке и.н.с.}
                    \scnrelfrom{второй домен}{ориентированное множество чисел}
                \end{scnindent}
                \scnitem{функция активации\scnrolesign}
                \begin{scnindent}
                    \scniselement{ролевое отношение}
                    \scnrelfrom{первый домен}{действие по обработке и.н.с.}
                    \scnrelfrom{второй домен}{функция}
                \end{scnindent}
                \scnitem{результат\scnrolesign}
                \begin{scnindent}
                    \scniselement{ролевое отношение}
                    \scnrelfrom{первый домен}{действие интерпретации нейросетевого метода}
                    \scnrelfrom{второй домен}{выходные значения нейронов слоя}
                    \begin{scnindent}
                        \scnsubset{ориентированное множество чисел}
                    \end{scnindent}
                \end{scnindent}
            \end{scnrelfromset}

            \scnheader{действие интерпретации сверточного слоя}
            \begin{scnrelfromset}{отношения характеризующие аргументы данного действия}
                \scnitem{входная матрица\scnrolesign}
                \begin{scnindent}
                    \scniselement{ролевое отношение}
                    \scnrelfrom{первый домен}{действие интерпретации и.н.с.}
                    \scnrelfrom{второй домен}{матрица}
                \end{scnindent}
                \scnitem{ядро свертки\scnrolesign}
                \begin{scnindent}
                    \scniselement{ролевое отношение}
                    \scnrelfrom{первый домен}{действие интерпретации сверточного слоя}
                    \scnrelfrom{второй домен}{матрица}
                \end{scnindent}
                \scnitem{шаг свертки\scnrolesign}
                \begin{scnindent}
                    \scniselement{ролевое отношение}
                    \scnrelfrom{первый домен}{действие интерпретации сверточного слоя}
                    \scnrelfrom{второй домен}{число}
                \end{scnindent}
                \scnitem{результат\scnrolesign}
                \begin{scnindent}
                    \scniselement{ролевое отношение}
                    \scnrelfrom{первый домен}{действие интерпретации нейросетевого метода}
                    \scnrelfrom{второй домен}{результат свертки входной матрицы с ядром свертки}
                    \begin{scnindent}
                        \scnsubset{матрица}
                    \end{scnindent}
                \end{scnindent}
            \end{scnrelfromset}

            \scnheader{действие интерпретации пулинг слоя}
            \begin{scnrelfromset}{отношения характеризующие аргументы данного действия}
                \scnitem{входная матрица\scnrolesign}
                \begin{scnindent}
                    \scniselement{ролевое отношение}
                    \scnrelfrom{первый домен}{действие интерпретации и.н.с.}
                    \scnrelfrom{второй домен}{матрица}
                \end{scnindent}
                \scnitem{размер окна пулинга\scnrolesign}
                \begin{scnindent}
                    \scniselement{ролевое отношение}
                    \scnrelfrom{первый домен}{действие интерпретации пулинг слоя}
                    \scnrelfrom{второй домен}{матрица}
                \end{scnindent}
                \scnitem{размер окна пулинга\scnrolesign}
                \begin{scnindent}
                    \scniselement{ролевое отношение}
                    \scnrelfrom{первый домен}{действие интерпретации пулинг слоя}
                    \scnrelfrom{второй домен}{матрица}
                \end{scnindent}
                \scnitem{шаг окна пулинга\scnrolesign}
                \begin{scnindent}
                    \scniselement{ролевое отношение}
                    \scnrelfrom{первый домен}{действие интерпретации пулинг слоя}
                    \scnrelfrom{второй домен}{число}
                \end{scnindent}
                \scnitem{результат\scnrolesign}
                \begin{scnindent}
                    \scniselement{ролевое отношение}
                    \scnrelfrom{первый домен}{действие интерпретации нейросетевого метода}
                    \scnrelfrom{второй домен}{результат пулинга входной матрицы}
                    \begin{scnindent}
                        \scnsubset{матрица}
                    \end{scnindent}
                \end{scnindent}
            \end{scnrelfromset}

            \scnheader{проблема задания различных аргументов для нейронов одного слоя}
            \scntext{пояснение}{присутствует необходимость спецификации соответствующего действия(задание различных аргументов для нейронов одного слоя)}
            \scntext{примечание}{нерешённая на данный момент}
            \begin{scnindent}
                \scntext{причина}{слабая изученность \texit{нейросетевых моделей решения задач} подобного рода}
            \end{scnindent}

            \scnheader{агентно-ориентированная модель интерпретации искусственных нейронных сетей}
            \scnsubset{спецификация агентов, соотвтствующих действиям интерпретации нейросетевых методов}
            \scntext{реализация}{интерпретатор искусственных нейронных сетей}
            \scntext{примечание}{любой агент, интерпретирующий действия с заданными с помощью отношения \textit{функция активации'} аргументами, должен использовать интерпретатор математических функций. использующихся в качестве функций активации}

            \scnheader{задача \scnqq{ИСКЛЮЧАЮЩЕЕ ИЛИ}}
            \scnidtf{нейросетевой метод, решающий задачу \scnqq{ИСКЛЮЧАЮЩЕЕ ИЛИ}}
            \scntext{формулировка}{вычислить результат логической операции \scnqq{ИСКЛЮЧАЮЩЕЕ ИЛИ} для значений двух логических переменных}
            \scnrelfrom{решение задачи с помощью сигнальных функций}{\scnfileimage[30em]{Contents/part_ps/src/images/sd_ps/sd_ann/strong_or_graphic.png}}
            \scntext{метод решения}{использование \textbf{\textit{однослойного персептрона, решающего задачу \scnqq{ИСКЛЮЧАЮЩЕЕ ИЛИ}}}}
            \begin{scnindent}
                \scnrelfrom{описание метода, представленного с помощью языка представления нейросетевых методов SCP}{\scnfileimage[35em]{Contents/part_ps/src/images/sd_ps/sd_ann/exclusive_or_ann_scp.png}}
                \scntext{пояснение}{описание метода состоит из последовательности двух обобщенных спецификаций действий --- действия вычисления взвешенной суммы всех нейронов слоя и действия вычисления функции активации для всех нейронов слоя}
            \end{scnindent}

            \scnheader{однослойный персептрон, решающий задачу \scnqq{ИСКЛЮЧАЮЩЕЕ ИЛИ}}
            \begin{scnrelfromset}{характеристика}
                \scnitem{два входных нейрона}
                \scnitem{один выходной нейрон}
                \scnitem{порог - 0,5}
                \scnitem{сигнальная функция активации}
                \begin{scnindent}
                    \scnrelfrom{формула}{
                        \begin{equation*}
                            F(S) =
                            \begin{cases}
                                1, 0 < S < 0,\\
                                0, else
                            \end{cases}
                        \end{equation*}}
                    \scnrelfrom{логическая формула}{\scnfileimage[30em]{Contents/part_ps/src/images/sd_ps/sd_ann/signal_function_def.png}}
                \end{scnindent}
                \scnitem{весовые коэффициенты синапсов входного слоя - 1}
            \end{scnrelfromset}
            \scntext{источник}{\scncite{Golovko2017}}
            \scnrelfrom{схема}{\scnfileimage[20em]{Contents/part_ps/src/images/sd_ps/sd_ann/strong_or_ann.png}}



            \bigskip
        \end{scnsubstruct}

        \scnendsegmentcomment{Операционная семантика sc-моделей искусственных нейронных сетей, используемых в ostis-системах}

        \bigskip
    \end{scnsubstruct}

    \scnendcurrentsectioncomment

\end{SCn}
